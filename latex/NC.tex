\documentclass[a4paper]{article}
\usepackage{geometry}
\usepackage{graphicx}
\usepackage{epsfig}
\usepackage{amsmath}
\usepackage{indentfirst}
\usepackage{float}
\usepackage{setspace}
\usepackage{amsfonts}
\usepackage{hyperref}
\usepackage{booktabs}
\usepackage{caption}
\usepackage{subfigure}
\usepackage{times}
\usepackage{hyperref}
\usepackage[round]{natbib}
\usepackage{verbatim}
\usepackage{colortbl}
\usepackage{lscape}
\usepackage[affil-it]{authblk}


\geometry{left=2cm,right=2cm,top=2cm,bottom=2cm}

\setlength{\parindent}{2em}

\begin{document}
	
	\title{Narrative Conservatism}
	
	\author{Beatriz Garc\'ia Osma}
	
	\author{Juan Manuel Garc\'ia Lara}
	
	\author{Fengzhi Zhu%
		\thanks{Email: \texttt{fzhu@emp.uc3m.es}}}
	
	\affil{Department of Business Administration, Universidad Carlos III de Madrid, Spain}
	
	\date{Dated: \today}
	
	\maketitle

\begin{spacing}{2}
\begin{abstract}
	I study whether firms' level of corporate social responsibility (CSR) affects their speed of response to gains and losses in financial reports, i.e. conditional conservatism. Using a natural experiment of staggered constituency statute enactments in the U.S. during 1980s, which allow managers to consider stakeholder interest in decision-making and thus improves CSR overall, I find that conservatism increases after CSR improvement. Such increase in conservatism is more significant in firms with strong debt-contracting demand and high managerial ability, consistent with debt being the main resource of conservatism and managerial style playing a role in determining CSR and conservatism. The results are robust to various alternative sample selections, variable measurements and model specifications. This study contributes to the linkage between CSR and conservatism literature by using a unique setting to document a positive causal impact of CSR on conservatism. Furthermore, this paper adds to the discussion on social influence of constituency statutes by providing a novel accounting viewpoint.\\
	\newline
	
	\textbf{Keywords}: \textit{corporate social responsibility;  conditional conservatism; constituency statutes; trust}
\end{abstract}

\clearpage

\section{Introduction}
\cite{}
%[motivation]

%[setting]

%[data]

%[robustness]

%[contribution]

The rest of the paper structures as follows: Section 2 describes theoretical framework. Section 3 explains empirical models and data construction. Section 4 presents main results. Section 5 performs robustness checks. Section 6 concludes. \cite{liTextualAnalysisCorporate2010}

\section{Theoretical Framework}

\begin{center}
	H1: Firms' conditional conservatism decreases as CSR increases.
\end{center}

\begin{center}
	H2: Firms' conditional conservatism increases as CSR increases.
\end{center}

\section{Research Design}
\subsection{News proxy: Market Returns}

\subsection{Text properties}

 \begin{equation}
 	EARN_{i,t}=\alpha_i+\omega_t+\beta_1RET_{i,t}+\beta_2NEG_{i,t}+\beta_3RET_{i,t}\times NEG_{i,t}+\epsilon_{i,t}
 \end{equation}

\subsection{Data}

\section{Results}

\subsection{Summary Statistics}

\subsection{Main Results}

\section{Robustness Checks}

\subsection{Other news proxy}

\section{Conclusions}

\end{spacing}

\newpage
\section{Appendix}
\subsection{Appendix A: 10-Q and 8-K scraping}

\subsection{Appendix B: Financial Variable Definition}
\begin{table}[H]
	\centering
	\begin{tabular}{lp{15cm}p{15cm}}
		\textbf{Variable} & \textbf{Definition} \\
		BTM   & Book to market ratio: book value of equity (Compustat CEQ) divided by market value of equity (Compustat PRCC\_F $\times$ Compustat CSHO), as of current fiscal year. \\
		CASH  & Cash holdings: cash and short-term investments (Compustat CHE) to the book value of total assets (Compustat AT), as of current fiscal year. \\
		DCD   & Debt-contracting demand defined in two ways. (1) DCD1: a dummy variable that takes 1 if a firm experience an increase in average leverage ratio between the pre-enactment and post-enactment period (LEV\_POST $>$ LEV\_PRE), and 0 otherwise. (2) DCD2: The difference between average leverage ratios of the pre-enactment and post-enactment period (LEV\_POST - LEV\_PRE). \\
		EARN  & Earnings: income before extraordinary items (Compustat IB) divided by lagged market value of equity (Compustat PRCC\_F $\times$ Compustat CSHO). \\
		LEV   & Leverage ratio: short term debt (Compustat DLC) plus long term debt (Compustat DLTT) divided by market value of equity, as of current fiscal year. \\
		MA    & Managerial ability: an index of managerial ability constructed by \cite{demerjianQuantifyingManagerialAbility2012}. \\
		MTB   & Market to book ratio: inverse of BTM. \\
		NEG   & Dummy variable for bad news, which takes 1 when market -adjusted stock return (RET) is negative and is 0 otherwise. \\
		POST  & Dummy variable for firm-year observation subject to constituency statute, which takes 1 when a firm-year observation appears after the year in which the constituency statute is enacted in the firm's state of incorporation, and is 0 otherwise. \\
		RET   & Adjusted market return: sum of monthly buy-and-hold stock return (CRSP RET) over the fiscal year (starting from the fourth month of the fiscal year) minus the sum of monthly value-weighted stock return (CRSP VWRETD) over the same period.  \\
		ROA   & Return on assets: operating income before depreciation (Compustat OIBDP) divided by total assets (Compustat AT), as of current fiscal year. \\
		SIZE  & Firm size: natural log of market value of equity (Compustat PRCC\_F $\times$ Compustat CSHO), as of current fiscal year. \\
		TOBINSQ & Tobins' q: market value of total assets, which equals to book value of total assets (Compustat AT) plus market value of equity (Compustat PRCC\_F $\times$ Compustat CSHO) minus the sum of the book value of common stock (Compustat CEQ) and balance sheet deferred taxes (Compustat TXDB), divided by book value of total assets (Compustat AT), as of current fiscal year. \\
	\end{tabular}%
\end{table}%

\subsection{Appendix C: Text Variable Definition}
\begin{table}[H]
	\centering
	\begin{tabular}{lp{15cm}p{15cm}}
		\textbf{Indexes} & \textbf{Definition} \\
		KLD\_TOTAL &  Total number of strengths minus total number of concerns (KLD\_STR - KLD\_CON) \\
		KLD\_STR &  Number of strengths across all five dimensions (env\_str + com\_str + pro\_str + emp\_str + hum\_str) \\
		KLD\_CON &  Number of concerns across all five dimensions (env\_con + com\_con + pro\_con + emp\_con + hum\_con) \\
		com\_str  &  Total number of community strengths, including 6 subcategories: Charitable Giving, Innovative Giving, Non-US Charitable Giving, Support for Housing, Support for Education, and Other Strength. \\
		com\_con  &  Total number of community concerns, including 4 subcategories: Investment Controversies, Negative Economic, Tax Disputes, and Other Concern. \\
		div\_str  &  Total number of diversity strengths, including 6 subcategories: Promotion, Work/Life Benefits, Women \& Minority Contracting, Employment of the Disabled, Gay \& Lesbian Policies, and Other Strength. \\
		div\_con  &  Total number of diversity concerns, including 2 subcategories: Controversies, and Other Concern. \\
		env\_str  &  Total number of environment strengths, including 5 subcategories: Beneficial Products and Services, Pollution Prevention, Recycling, Clean Energy, and Other Strength. \\
		env\_con  &  Total number of environment concerns, including 7 subcategories: Hazardous Waste, Regulatory Problems, Ozone Depleting Chemicals, Substantial Emissions, Agricultural Chemicals, Climate Change, and Other Concern. \\
		emp\_str  &  Total number of employee relations strengths, including 6 subcategories: Union Relations, Cash Profit Sharing, Employee Involvement, Retirement Benefits Strength, Health and Safety Strength, and Other Strength. \\
		emp\_con  &  Total number of employee relations concerns, including 4 subcategories: Union Relations, Health and Safety Concern, Work force Reductions and Retirement Benefits Concern. \\
		hum\_str  &  Total number of human rights strengths, including 2 subcategories: Indigenous Peoples Relations and Labor Rights. \\
		hum\_con  &  Total number of human rights concerns, including 4 subcategories: Burma Concern, Labor Rights, Indigenous Peoples Relations, and Other Concern. \\
		pro\_str  &  Total number of product strengths, including 4 subcategories: Quality, R\&D/Innovation, Benefits to Economically Disadvantaged, and Other \\
		pro\_con  &  Total number of product concerns, including 4 subcategories: Product Safety, Marketing/Contracting, Antitrust, and Other Concern. \\
	\end{tabular}%
\end{table}%

\subsection{Online Appendix}
Tables of untabulated results can be accessed via this website:

%\href{https://drive.google.com/file/d/1P8XRZRd68AaeMenJVAEf1wEs_eRkLUfu/view?usp=sharing}{https://drive.google.com/file/d/1P8XRZRd68AaeMenJVAEf1wEs_eRkLUfu/view?usp=sharing}.

\newpage
\bibliographystyle{plainnat}
\bibliography{NC}

\newpage
%%%%%%%%%%%%%% Table 1: Sample Selection Process
\begin{landscape}
\begin{table}[htbp] \label{T1}
  \centering
    \begin{tabular}{lcc}
    \multicolumn{3}{c}{\textbf{Table 1. Sample Selection Process}} \\ 
      & &  \\
    \begin{comment}
    \multicolumn{3}{c}{10-Q} \\
    &   \multicolumn{2}{c}{Number of observations}\\
    & &  \\
    Retrieved from EDGAR & & 575,579 \\
    After merging with COMP and CRSP data & & 302,343 \\
    (-) Number of obs. from utility and financial firms & 82,498 & \\
    (-) Number of firm-quarters with missing values in SIC, SIZE, MTB, LEV, & & \\
    \hspace{5mm}or with non-positive total assets or book value of equity or common shares outstanding, & & \\
    \hspace{5mm}or with common share price less than \$1 & 25,959 & \\
    (-) Number of obs. with total words less than 1\% percentile (1,237 words) & 1,939 & \\
    (-) Number of obs. that contain negative or larger than 99\% TLAG & 1,855 & \\
    \bottomrule
    After dropping obs. with missing values in key variables and screening & & 190,092 \\
    After merging with I\textbackslash{}B\textbackslash{}E\textbackslash{}S and segment data & & 130,750 \\
    (-) Number of obs. that contain missing EARN, STD\_EARN and AF & 14,770 & \\
    \bottomrule
    Full 10-Q sample & & \textbf{115,980} \\
    & &  \\
    \end{comment}
    
     &   \multicolumn{2}{c}{Number of observations}\\
      & &  \\
    Retrieved from EDGAR & & 1,540,911 \\
    After matching with Compustat and CRSP data  & & 442,575 \\
    (-) Number of obs. from utility and financial firms & 112,729 & \\
    (-) Number of firm-quarters with missing values in SIC, SIZE, MTB, LEV, & & \\
    \hspace{5mm}or with non-positive total assets or book value of equity or common shares outstanding, & & \\
    \hspace{5mm}or with common share price less than \$1 & 46,865 & \\
    (-) Number of obs. with total words less than 1\% percentile (133 words) & 2,785 & \\
    (-) Number of obs. that are reversals of previous news day & 5,160 & \\
    (-) Number of obs. with negative or larger than 99\% percentile TLAG  & 154,861 & \\
    \bottomrule
    After dropping obs. with missing values in key variables and screening  & & 120,175 \\
    After merging with IBES and Compustat Segment data (Full 8-K sample) & & \textbf{83,464}  \\
    \begin{comment}
    	After dropping obs. with TLAG larger than four (five) days after (before) the 8-K reform &  & \\
    	(Restricted 8-K sample) &  & \textbf{40,700} 
    \end{comment}
    \end{tabular}%
\end{table}%

\end{landscape}

\newpage
%%%%%%%%%%%%%%%%%%%%%%%%% TABLE 2 Panel A
%\begin{landscape}
% Table generated by Excel2LaTeX from sheet 'T2PA '
\begin{table}[htbp] \label{T2PA}
  \centering
    \begin{tabular}{lcccccccc}
    \multicolumn{9}{c}{\textbf{Table 2. Panel A: Summary Statistics 10-Q}} \\
    \midrule
    \midrule
      & count & mean & std & min & 25\% & 50\% & 75\% & max \\
    \midrule
    \textbf{Textual Vars.} &   &   &   &   &   &   &   &  \\
    NW & 91607 & 9.020 & 0.757 & 7.120 & 8.506 & 9.086 & 9.547 & 13.544 \\
    nw & 91607 & 10937 & 10204 & 1236 & 4942 & 8829 & 13997 & 752337 \\
    TONE & 91607 & -8.921 & 7.236 & -63.579 & -13.127 & -7.875 & -3.866 & 24.215 \\
    TLAG & 91607 & 39 & 6 & 0 & 36 & 40 & 44 & 52 \\
    READ & 91607 & 38.161 & 42.160 & 14.580 & 17.840 & 20.210 & 39.660 & 262.515 \\
    ABTONE & 91607 & 0.000 & 6.919 & -55.759 & -3.946 & 0.939 & 4.777 & 34.181 \\
    \textbf{Financial Vars.} &   &   &   &   &   &   &   &  \\
    QRET & 91607 & 0.018 & 0.253 & -1.579 & -0.113 & 0.007 & 0.130 & 4.849 \\
    NEG & 91607 & 0.483 & 0.500 & 0 & 0 & 0 & 1 & 1 \\
    SIZE & 91607 & 6.447 & 1.776 & 2.002 & 5.175 & 6.317 & 7.563 & 11.206 \\
    MTB & 91607 & 3.515 & 4.009 & 0.288 & 1.485 & 2.343 & 3.902 & 30.902 \\
    LEV & 91607 & 0.192 & 0.182 & 0.000 & 0.011 & 0.162 & 0.315 & 0.724 \\
    AF & 91607 & 0.043 & 0.066 & -0.262 & 0.023 & 0.049 & 0.073 & 0.227 \\
    AFE & 91607 & -0.021 & 0.067 & -0.445 & -0.018 & -0.002 & 0.002 & 0.078 \\
    BUSSEG & 91607 & 0.859 & 0.447 & 0.693 & 0.693 & 0.693 & 0.693 & 2.773 \\
    GEOSEG & 91607 & 0.898 & 0.532 & 0.693 & 0.693 & 0.693 & 0.693 & 3.045 \\
    AGE & 91607 & 8.312 & 1.033 & 5.811 & 7.635 & 8.420 & 9.089 & 10.288 \\
    EARN & 91607 & 0.005 & 0.042 & -0.201 & 0.001 & 0.012 & 0.023 & 0.084 \\
    $\Delta$EARN & 91607 & 0.002 & 0.031 & -0.126 & -0.006 & 0.001 & 0.008 & 0.150 \\
    STD\_EARN & 91607 & 0.020 & 0.030 & 0.001 & 0.005 & 0.009 & 0.021 & 0.188 \\
    STD\_QRET & 91607 & 0.089 & 0.070 & 0.007 & 0.040 & 0.070 & 0.115 & 0.379 \\
    LOSS & 91607 & 0.242 & 0.429 & 0 & 0 & 0 & 0 & 1 \\
    \bottomrule
    \bottomrule
    \end{tabular}%
\end{table}%

%\end{landscape}

\newpage
%%%%%%%%%%%%%%%%%%%%%%%%% + TABLE 2 Panel B
%\begin{landscape}
% Table generated by Excel2LaTeX from sheet 'T2PB'
\begin{table}[H] \label{T2PB}
  \begin{center}
  	    \begin{tabular}{lcccccccc}
  		\multicolumn{9}{c}{\textbf{Table 2. Panel B: Summary Statistics 8-K}} \\
  		\midrule
  		\midrule
  		& count & mean & std & min & 25\% & 50\% & 75\% & max \\
  		\midrule
  		\textbf{Textual Variables} &   &   &   &   &   &   &   &  \\
  		NW & 119615 & 6.093 & 0.926 & 4.898 & 5.553 & 5.846 & 6.358 & 12.486 \\
  		nw & 119615 & 1339 & 6398 & 133 & 257 & 345 & 576 & 264704 \\
  		TONE & 119615 & -0.552 & 7.424 & -97.851 & -3.049 & 0.000 & 3.677 & 45.929 \\
  		TLAG & 119615 & 15 & 17 & 0 & 2 & 9 & 21 & 93 \\
  		N8K & 119615 & 1 & 0 & 1 & 1 & 1 & 1 & 4 \\
  		NITEM & 119615 & 2 & 1 & 1 & 2 & 2 & 2 & 16 \\
  		
  		\textbf{Financial Variables} &   &   &   &   &   &   &   &  \\
  		DRET & 119615 & 0.003 & 0.097 & -0.833 & -0.039 & -0.003 & 0.041 & 5.991 \\
  		$\Delta$DRET & 119615 & -0.018 & 0.187 & -9.062 & -0.121 & -0.050 & 0.100 & 5.989 \\
  		BN & 119615 & 0.542 & 0.498 & 0 & 0 & 1 & 1 & 1 \\
  		SIZE & 119615 & 6.326 & 1.993 & 2.122 & 4.896 & 6.262 & 7.664 & 11.379 \\
  		MTB & 119615 & 3.741 & 4.784 & 0.123 & 1.366 & 2.293 & 4.055 & 33.434 \\
  		LEV & 119615 & 0.204 & 0.192 & 0.000 & 0.012 & 0.171 & 0.334 & 0.735 \\
  		\bottomrule
  		\bottomrule
  	\end{tabular}%
  \end{center}
	\begin{footnotesize}
		\noindent Table 2 Panel A and Table 2 Panel B present the summary statistics of key variables in 10-Q and 8-K sample. READ and all financial variables except returns are winsorized at 1\% and 99\% level. See \hyperref[appb]{Appendix B} for variable definitions.
	\end{footnotesize}
\end{table}%
%\end{landscape}

%%%%%%%%%%%%%%%%%%%%%%%%% TABLE 2 Panel C
\newpage
%\begin{landscape}
% Table generated by Excel2LaTeX from sheet 'T2PC'
\begin{table}[H] \label{T2PC}
	\centering
	\begin{tabular}{lrrrrrrrrr}
		\multicolumn{10}{c}{\textbf{Table 2. Panel C: Correlation Matrix 8-K}} \\
		\midrule
		\midrule
		& \multicolumn{1}{c}{(1)} & \multicolumn{1}{c}{(2)} & \multicolumn{1}{c}{(3)} & \multicolumn{1}{c}{(4)} & \multicolumn{1}{c}{(5)} & \multicolumn{1}{c}{(6)} & \multicolumn{1}{c}{(7)} & \multicolumn{1}{c}{(8)} & \multicolumn{1}{c}{(9)}\\
		\midrule
		(1) tlag &  & -0.069 & 0.103 & -0.042 & -0.055 & 0.004 & -0.057 & -0.010 & -0.035 \\
		(2) TONE & -0.105 &  & -0.232 & -0.023 & -0.092 & -0.138 & 0.026 & 0.000 & 0.008 \\
		(3) nw & 0.116 & -0.415 &  & 0.017 & 0.004 & 0.308 & -0.025 & 0.017 & -0.006 \\
		(4) n8k & -0.058 & -0.044 & 0.213 &  & 0.437 & 0.209 & 0.066 & 0.015 & 0.007 \\
		(5) nitem & -0.096 & -0.114 & 0.197 & 0.302 &  & 0.461 & 0.091 & 0.008 & 0.004 \\
		(6) nexhibit & -0.069 & -0.112 & 0.175 & 0.203 & 0.614 &  & 0.101 & 0.015 & -0.007 \\
		(7) ngraph & -0.166 & 0.123 & -0.028 & 0.102 & 0.299 & 0.314 & & 0.004 & 0.003 \\
		(8) DRET & -0.021 & 0.005 & -0.003 & 0.004 & 0.004 & 0.006 & 0.008 &  & 0.700 \\
		(9) $\Delta$DRET & -0.048 & 0.013 & -0.015 & 0.003 & 0.007 & 0.003 & 0.017 & 0.765 &  \\
		(10) BN & 0.051 & -0.009 & 0.012 & -0.002 & -0.006 & -0.003 & -0.016 & -0.774 & -0.864 \\
		(11) SIZE & -0.095 & 0.068 & 0.020 & 0.026 & 0.009 & 0.003 & 0.091 & 0.021 & 0.070 \\
		(12) MTB & -0.006 & 0.029 & 0.038 & -0.001 & -0.015 & -0.023 & 0.007 & 0.007 & 0.008 \\
		(13) LEV & -0.046 & -0.037 & 0.075 & 0.028 & 0.029 & 0.046 & 0.073 & 0.015 & 0.024 \\
		(14) AF & -0.050 & 0.013 & -0.018 & 0.004 & 0.017 & 0.030 & 0.039 & -0.024 & 0.042 \\
		(15) AFE & -0.011 & 0.032 & -0.020 & 0.008 & 0.002 & -0.012 & 0.017 & 0.032 & 0.004 \\
		(16) BUSSEG & -0.070 & 0.095 & 0.035 & 0.027 & 0.044 & -0.010 & 0.200 & 0.004 & 0.021 \\
		(17) GEOSEG & -0.076 & 0.094 & 0.041 & 0.023 & 0.041 & -0.013 & 0.194 & 0.008 & 0.031 \\
		(18) EARN & -0.021 & 0.068 & -0.069 & -0.004 & -0.001 & -0.004 & -0.019 & 0.045 & 0.059 \\
		(19) STD\_QRET & 0.018 & -0.056 & 0.056 & -0.011 & -0.010 & -0.009 & -0.013 & -0.030 & -0.058 \\
		\bottomrule
		\bottomrule
	\end{tabular}%
\end{table}%
% Table generated by Excel2LaTeX from sheet 'T2PD'
\begin{table}[H]
  \begin{center}
  	\begin{tabular}{lrrrrrrrrrr}
  		\multicolumn{11}{c}{\textbf{Table 2. Panel C: Correlation Matrix 8-K (Continued) }} \\
  		\midrule
  		\midrule
  		& \multicolumn{1}{c}{(10)} & \multicolumn{1}{c}{(11)} & \multicolumn{1}{c}{(12)} & \multicolumn{1}{c}{(13)} & \multicolumn{1}{c}{(14)} & \multicolumn{1}{c}{(15)} & \multicolumn{1}{c}{(16)} & \multicolumn{1}{c}{(17)} & \multicolumn{1}{c}{(18)} & \multicolumn{1}{c}{(19)} \\
  		\midrule
  		(1) tlag & 0.039 & -0.075 & -0.004 & -0.039 & -0.012 & 0.001 & -0.061 & -0.062 & 0.005 & 0.003 \\
  		(2) TONE & -0.008 & 0.062 & 0.012 & -0.028 & -0.013 & 0.042 & 0.061 & 0.065 & 0.033 & -0.037 \\
  		(3) nw & 0.004 & -0.055 & 0.010 & 0.039 & 0.006 & -0.016 & -0.071 & -0.073 & -0.014 & 0.026 \\
  		(4) n8k & -0.003 & 0.025 & 0.000 & 0.028 & -0.001 & 0.005 & 0.027 & 0.020 & 0.002 & -0.008 \\
  		(5) nitem & -0.003 & -0.001 & -0.007 & 0.032 & 0.001 & -0.003 & 0.036 & 0.026 & -0.005 & 0.002 \\
  		(6) nexhibit & 0.006 & -0.006 & -0.002 & 0.053 & 0.004 & -0.015 & -0.010 & -0.019 & -0.025 & 0.021 \\
  		(7) ngraph & -0.004 & 0.039 & 0.014 & 0.045 & -0.003 & 0.003 & 0.079 & 0.073 & -0.005 & -0.004 \\
  		(8) DRET & -0.587 & -0.014 & 0.004 & 0.003 & 0.006 & 0.009 & -0.007 & -0.006 & 0.017 & 0.005 \\
  		(9) $\Delta$DRET & -0.765 & 0.057 & -0.008 & 0.013 & 0.062 & 0.001 & 0.018 & 0.024 & 0.062 & -0.055 \\
  		(10) BN & & -0.032 & 0.000 & -0.011 & -0.027 & 0.002 & -0.015 & -0.020 & -0.031 & 0.027 \\
  		(11) SIZE & -0.031 &  & 0.207 & 0.170 & 0.114 & 0.188 & 0.240 & 0.283 & 0.313 & -0.259 \\
  		(12) MTB & -0.004 & 0.346 &  & 0.104 & -0.152 & 0.077 & 0.006 & 0.028 & -0.055 & 0.129 \\
  		(13) LEV & -0.014 & 0.219 & -0.035 &  & 0.144 & -0.071 & 0.089 & 0.054 & 0.070 & -0.115 \\
  		(14) AF & -0.028 & 0.030 & -0.402 & 0.226 &  & -0.184 & 0.058 & 0.080 & 0.375 & -0.203 \\
  		(15) AFE & -0.003 & 0.133 & 0.122 & -0.061 & -0.218 &  & 0.053 & 0.054 & 0.193 & -0.110 \\
  		(16) BUSSEG & -0.013 & 0.224 & 0.066 & 0.074 & 0.053 & 0.028 &  & 0.644 & 0.081 & -0.091 \\
  		(17) GEOSEG & -0.021 & 0.289 & 0.072 & 0.083 & 0.090 & 0.028 & 0.715 &  & 0.105 & -0.111 \\
  		(18) EARN & -0.030 & 0.349 & 0.226 & -0.032 & 0.113 & 0.227 & 0.027 & 0.060 &  & -0.470 \\
  		(19) STD\_EARN & 0.028 & -0.338 & 0.058 & -0.177 & -0.133 & -0.066 & -0.075 & -0.101 & -0.335 & \\
  		\bottomrule
  		\bottomrule
  	\end{tabular}%
  \end{center}
	\begin{footnotesize}
		\noindent Table 2 Panel C presents the correlation matrix of key variables in 8-K sample. Pearson (Spearman) correlations are exhibited above (below) the diagonal. See \hyperref[appc]{Appendix C} for variable definitions. All financial variables except returns are winsorized at 1\% and 99\% level. 
	\end{footnotesize}
\end{table}%
%\end{landscape}

%%%%%%%%%%%%%%%%%%%%%%%%% TABLE 3
\newpage
\begin{landscape}
% Table generated by Excel2LaTeX from sheet 'T3'
\begin{table}[H] \label{T3}
	\begin{center}
		\tabcolsep=0.11cm
		\begin{tabular}{lcccc}
			\multicolumn{5}{c}{\textbf{Table 3. Is 8-K Narrative Disclosure Conservative?}} \\
			\toprule
			\toprule
			& (1) & (2) & (3) & (4) \\
			Dep. Variables & TLAG & TLAG & TONE & TONE \\
			\midrule
			%&   &   &   &  \\
			$\Delta$DRET & 1.913*** & 2.007*** & -1.744*** & -1.171** \\
			& (11.44) & (10.83) & (-2.86) & (-2.07) \\
			BN & -0.021 & -0.026 & -0.120* & -0.125 \\
			& (-1.13) & (-1.15) & (-1.71) & (-1.64) \\
			\rowcolor[rgb]{ .906,  .902,  .902} \textit{(Pred. Sign)} & (-) & (-) & (+) & (+) \\
			\rowcolor[rgb]{ .906,  .902,  .902} $\Delta$DRET$\times$BN & -2.966*** & -3.182*** & 2.893*** & 1.849** \\
			\rowcolor[rgb]{ .906,  .902,  .902}   & (-8.42) & (-7.55) & (2.70) & (1.97) \\
			SIZE &   & 0.051*** &   & 0.115* \\
			&   & (4.56) &   & (1.76) \\
			MTB &   & 0.002 &   & -0.009 \\
			&   & (1.22) &   & (-1.08) \\
			LEV &   & -0.007 &   & -0.592 \\
			&   & (-0.11) &   & (-1.45) \\
			EARN &   & -0.231* &   & 3.059** \\
			&   & (-1.70) &   & (2.51) \\
			STD\_EARN &   & -0.165 &   & -2.705**\\
			&   & (-0.72) &   & (-2.17)\\
			BUSSEG &   & -0.028 &   & -0.015 \\
			&   & (-1.52) &   & (-0.12) \\
			GEOSEG &   & 0.016 &   & 0.131 \\
			&   & (0.91) &   & (1.18) \\
			AF &   & 0.020 &   & -0.019 \\
			&   & (0.20) &   & (-0.04)\\
			AFE &   & 0.045 &   & 1.713**  \\
			&   & (0.41) &   & (2.57) \\
			Constant & -2.816*** & -3.150*** & -5.598** & -5.921*** \\
			& (-10.16) & (-10.85) & (-2.47) & (-2.71) \\
			&   &   &   &  \\
			Observations & 83,464 & 75,360 & 83,464 & 75,360 \\
			Adjusted R-squared & 0.131 & 0.132 & 0.151 & 0.147 \\
			\bottomrule
			\bottomrule
		\end{tabular}%
	\end{center}
\end{table}%
% Table generated by Excel2LaTeX from sheet 'T3'
\begin{table}
	\begin{center}
		\tabcolsep=0.11cm
		\begin{tabular}{lcccccccccc}
			\multicolumn{11}{c}{\textbf{Table 3. Is 8-K Narrative Disclosure Conservative? (Continued)}} \\
			\toprule
			\toprule
			 & (5) & (6) & (7) & (8) & (9) & (10) & (11) & (12) & (13) & (14) \\
			Dep. Variables & NW & NW & N8K & N8K & NITEM & NITEM & NEXHIBIT & NEXHIBIT & NGRAPH & NGRAPH \\
			\midrule
			%&   &   &   &   &   &  &   &   &   &  \\
			$\Delta$DRET & -0.086* & -0.042 & -0.034*** & -0.039*** & -0.075*** & -0.079*** & -0.105*** & -0.110*** & -0.151*** & -0.212*** \\
			 & (-1.78) & (-0.71) & (-3.43) & (-3.64) & (-3.34) & (-3.71) & (-2.99) & (-3.04) & (-3.03) & (-5.02) \\
			BN & -0.015** & -0.015** & -0.002** & -0.003** & -0.004 & -0.004 & -0.003 & -0.002 & 0.001 & -0.001 \\
			& (-2.04) & (-2.19) & (-2.24) & (-2.43) & (-1.13) & (-1.05) & (-0.53) & (-0.36) & (0.16) & (-0.13) \\
			\rowcolor[rgb]{ .906,  .902,  .902} \textit{(Pred. Sign)} & (+) & (+) & (+) & (+) & (+) & (+) & (+) & (+) & (+) & (+) \\
			\rowcolor[rgb]{ .906,  .902,  .902} $\Delta$DRET$\times$BN & 0.127** & 0.033 & 0.046*** & 0.051*** & 0.099*** & 0.104*** & 0.176*** & 0.175*** & 0.221*** & 0.298*** \\
			\rowcolor[rgb]{ .906,  .902,  .902}   & (2.02) & (0.40) & (3.34) & (3.36) & (2.84) & (3.06) & (3.46) & (3.32) & (4.06) & (5.71) \\
			SIZE &   & 0.018** &   & -0.001 &   & -0.002 &   & -0.003 &   & -0.004 \\
			&   & (2.13) &   & (-0.84) &   & (-0.70) &   & (-0.58) &   & (-0.60) \\
			MTB &   & -0.002 &   & -0.000 &   & -0.000 &   & -0.002*** &   & -0.003*** \\
			&    & (-1.30) &   & (-0.43) &   & (-0.96) &   & (-2.88) &   & (-2.82) \\
			LEV &  & -0.027 &   & -0.008** &   & -0.021* &   & -0.007 &   & 0.005 \\
			&   & (-0.65) &   & (-2.43) &   & (-1.68) &   & (-0.32) &   & (0.11) \\
			EARN &    & 0.406*** &   & -0.001 &   & 0.069* &   & 0.113* &   & -0.064 \\
			&   & (3.84) &   & (-0.17) &   & (1.82) &   & (1.96) &   & (-0.87) \\
			STD\_EARN &    & -0.331*** &   & -0.004 &   & -0.098** &   & -0.112 &   & 0.243* \\
			&   & (-2.75) &   & (-0.41) &   & (-2.11) &   & (-1.29) &   & (1.71) \\
			BUSSEG &    & -0.008 &   & 0.000 &   & 0.002 &   & 0.003 &   & -0.005 \\
			&    & (-0.71) &   & (0.21) &   & (0.39) &   & (0.42) &   & (-0.31) \\
			GEOSEG &    & 0.007 &   & 0.002** &   & -0.001 &   & -0.011* &   & -0.011 \\
			&    & (0.67) &   & (2.27) &   & (-0.36) &   & (-1.82) &   & (-0.76) \\
			AF &    & -0.026 &   & 0.004 &   & 0.015 &   & 0.029 &   & -0.075 \\
			&    & (-0.47) &   & (0.52) &   & (0.74) &   & (0.66) &   & (-1.56) \\
			AFE &   & -0.044 &   & -0.009 &   & -0.022 &   & -0.091** &   & -0.164** \\
			&   & (-0.69) &   & (-1.36) &   & (-0.86) &   & (-2.44) &   & (-2.37) \\
			Constant & -7.291*** & -7.295*** & -0.688*** & -0.684*** & -0.872*** & -0.843*** & -0.506*** & -0.459*** & 0.051 & 0.096 \\
			&  (-27.57) & (-28.75) & (-190.40) & (-120.16) & (-25.72) & (-22.63) & (-4.91) & (-4.26) & (1.01) & (1.44) \\
			&   &   &   &   &   &   &   &   &   &  \\
			Observations & 83,464 & 75,360 & 83,464 & 75,360 & 83,464 & 75,360 & 83,464 & 75,360 & 83,464 & 75,360 \\
			Adjusted R-squared & 0.443 & 0.427 & 0.021 & 0.024 & 0.139 & 0.142 & 0.109 & 0.107 & 0.256 & 0.263 \\
			\bottomrule
			\bottomrule
		\end{tabular}%
	\end{center}
		\begin{footnotesize}
			\setcounter{equation}{0}
			\begin{equation}
				TEX_{i,t}=\beta_0+\beta_1\Delta DRET_{i,t-tlag}+\beta_2BN_{i,t-tlag}+\beta_3\Delta DRET_{i,t-tlag}\times 	BN_{i,t-tlag}+\sum\beta_nCONTROLS_{i,t}+\epsilon_{i,t}
			\end{equation}
			
			\noindent Table 3 presents the regression results of Equation (1). TEX represents a vector of textual properties. CONTROLS denotes a vector of control variables. See \hyperref[appc]{Appendix C} for variable definitions. All financial variables except returns are winsorized at 1\% and 99\% level. All regressions include firm and year-month fixed effects and standard errors are clustered at industry level identified by 4-digit SIC codes. ***, ** and * indicate significance at the 1\%, 5\% and 10\% levels in a two-tailed test.
		\end{footnotesize}
\end{table}%
\end{landscape}

%%%%%%%%%%%%%%%%%%%%%%%%% TABLE 4
\newpage
\begin{landscape}
% Table generated by Excel2LaTeX from sheet 'T3'
\begin{table}[H] \label{T4}
	\begin{center}
		\tabcolsep=0.11cm
		\begin{tabular}{lcccc}
			\multicolumn{5}{c}{\textbf{Table 4. Narrative Conservatism and Conditional Conservatism}} \\
			\toprule
			\toprule
			Dep. Variables & \multicolumn{2}{c}{TLAG} & \multicolumn{2}{c}{TONE} \\
			\cmidrule{2-5}
			& (1) & (2) & (3) & (4) \\
			CONS. & LOW & HIGH & LOW & HIGH \\
			\midrule
			%&   &   &   &  \\
			$\Delta$DRET & 2.647*** & 1.775*** & -2.473** & -0.206 \\
			& (9.71) & (11.56) & (-2.33) & (-0.31) \\
			BN & -0.051* & -0.009 & -0.186 & -0.079 \\
			& (-1.91) & (-0.33) & (-1.54) & (-0.80) \\
			\rowcolor[rgb]{ .906,  .902,  .902} \textit{(Pred. Sign)} & (-) & (-) & (+) & (+) \\
			\rowcolor[rgb]{ .906,  .902,  .902} $\Delta$DRET$\times$BN& -4.639*** & -2.687*** & 3.553** & 0.549 \\
			\rowcolor[rgb]{ .906,  .902,  .902} & (-8.75) & (-8.84) & (2.17) & (0.54) \\
			SIZE & 0.087*** & 0.030** & 0.092 & 0.101 \\
			& (4.69) & (2.12) & (0.92) & (1.07) \\
			MTB & -0.000 & 0.003 & 0.018 & -0.005 \\
			& (-0.09) & (1.09) & (0.81) & (-0.38) \\
			LEV & -0.002 & -0.082 & -0.937* & -0.581 \\
			& (-0.02) & (-0.94) & (-1.81) & (-0.90) \\
			EARN & 0.031 & -0.306 & 1.008 & 3.218** \\
			& (0.13) & (-1.61) & (0.46) & (2.53) \\
			STD\_EARN & -0.041 & -0.030 & -2.801 & -3.046*** \\
			& (-0.13) & (-0.10) & (-1.19) & (-2.65) \\
			BUSSEG & -0.026 & -0.025 & -0.059 & -0.046 \\
			& (-1.14) & (-0.78) & (-0.36) & (-0.23) \\
			GEOSEG & 0.034 & 0.004 & 0.031 & 0.253 \\
			& (1.55) & (0.18) & (0.22) & (1.56) \\
			AF & 0.153 & -0.028 & 0.022 & 0.067 \\
			& (1.22) & (-0.22) & (0.03) & (0.10) \\
			AFE & 0.059 & 0.032 & 2.629*** & 0.810 \\
			& (0.34) & (0.21) & (2.75) & (0.83) \\
			Constant & -2.845*** & -2.492*** & -0.198 & -0.826 \\
			& (-17.51) & (-23.87) & (-0.25) & (-1.38) \\
			&   &   &   &  \\
			Observations & 38,881 & 35,134 & 38,881 & 35,134 \\
			Adjusted R-squared & 0.139 & 0.120 & 0.133 & 0.154 \\
			\bottomrule
			\bottomrule
		\end{tabular}%
	\end{center}
\end{table}%
% Table generated by Excel2LaTeX from sheet 'T3'
\begin{table}[H]
	\begin{center}
		\tabcolsep=0.11cm
		\begin{tabular}{lcccccccccc}
			\multicolumn{11}{c}{\textbf{Table 4. Narrative Conservatism and Conditional Conservatism (Continued)}} \\
			\toprule
			\toprule
			Dep. Variables & \multicolumn{2}{c}{NW} & \multicolumn{2}{c}{N8K} & \multicolumn{2}{c}{NITEM} & \multicolumn{2}{c}{NEXHIBIT} & \multicolumn{2}{c}{NGRAPH} \\
			\cmidrule{2-11}
			& (5) & (6) & (7) & (8) & (9) & (10) & (11) & (12) & (13) & (14) \\
			CONS. & LOW & HIGH & LOW & HIGH & LOW & HIGH & LOW & HIGH& LOW & HIGH\\
			\midrule
			%&   &   &   &   &   &  &   &   &   &  \\
			$\Delta$DRET & -0.090 & -0.015 & -0.047*** & -0.042*** & -0.104*** & -0.061** & -0.171*** & -0.078* & -0.304*** & -0.168*** \\
			& (-0.89) & (-0.21) & (-4.04) & (-2.73) & (-2.93) & (-2.30) & (-3.12) & (-1.68) & (-2.93) & (-3.34) \\
			BN & -0.012 & -0.022** & -0.002 & -0.004** & -0.006 & -0.002 & -0.003 & 0.001 & -0.011 & 0.002 \\
			& (-1.00) & (-2.13) & (-1.31) & (-2.57) & (-1.15) & (-0.32) & (-0.41) & (0.08) & (-0.78) & (0.16) \\
			\rowcolor[rgb]{ .906,  .902,  .902} \textit{(Pred. Sign)} & (+) & (+) & (+) & (+) & (+) & (+) & (+) & (+) & (+) & (+) \\
			\rowcolor[rgb]{ .906,  .902,  .902} $\Delta$DRET$\times$BN & 0.095 & -0.025 & 0.066*** & 0.052** & 0.127*** & 0.085** & 0.281*** & 0.130** & 0.391*** & 0.244*** \\
			\rowcolor[rgb]{ .906,  .902,  .902} & (0.66) & (-0.25) & (4.01) & (2.51) & (2.89) & (2.00) & (3.62) & (2.14) & (3.20) & (4.30) \\
			SIZE & 0.024** & 0.013 & -0.001 & -0.000 & -0.004 & 0.003 & -0.012* & 0.011 & -0.003 & -0.002 \\
			& (2.10) & (1.29) & (-0.79) & (-0.19) & (-1.26) & (0.86) & (-1.95) & (1.54) & (-0.29) & (-0.20) \\
			MTB & -0.001 & -0.003 & -0.000 & -0.000 & -0.000 & -0.001 & -0.000 & -0.003*** & 0.001 & -0.004** \\
			& (-0.47) & (-1.62) & (-0.07) & (-0.87) & (-0.16) & (-1.54) & (-0.29) & (-3.16) & (0.41) & (-2.40) \\
			LEV & -0.074 & 0.035 & -0.006 & -0.010 & -0.014 & -0.016 & -0.019 & 0.005 & 0.054 & -0.047 \\
			& (-1.30) & (0.63) & (-1.03) & (-1.63) & (-0.66) & (-0.94) & (-0.60) & (0.16) & (0.76) & (-0.97) \\
			EARN & 0.263 & 0.486*** & 0.008 & 0.001 & 0.097 & 0.051 & 0.007 & 0.152** & 0.003 & -0.074 \\
			& (1.33) & (4.91) & (0.33) & (0.06) & (1.58) & (1.30) & (0.07) & (2.41) & (0.02) & (-0.89) \\
			STD\_EARN & -0.155 & -0.335** & 0.021 & -0.021* & 0.049 & -0.162*** & 0.095 & -0.186** & 0.544** & 0.077 \\
			& (-0.89) & (-2.32) & (0.88) & (-1.76) & (0.67) & (-3.04) & (0.61) & (-1.98) & (2.07) & (0.48) \\
			BUSSEG & -0.006 & -0.015 & -0.000 & 0.001 & -0.003 & 0.007 & -0.001 & 0.002 & -0.017 & 0.039* \\
			& (-0.45) & (-0.82) & (-0.19) & (0.75) & (-0.51) & (1.15) & (-0.12) & (0.18) & (-0.85) & (1.70) \\
			GEOSEG & 0.019 & 0.010 & 0.002 & 0.003* & 0.002 & -0.002 & -0.006 & -0.006 & 0.005 & -0.036* \\
			& (1.59) & (0.67) & (1.41) & (1.85) & (0.37) & (-0.29) & (-0.67) & (-0.63) & (0.26) & (-1.78) \\
			AF & -0.013 & -0.024 & 0.001 & 0.010 & 0.018 & 0.006 & 0.053 & -0.009 & -0.100 & -0.057 \\
			& (-0.16) & (-0.43) & (0.08) & (0.96) & (0.50) & (0.32) & (1.02) & (-0.17) & (-1.26) & (-0.86) \\
			AFE & -0.020 & -0.085 & -0.011 & -0.009 & -0.016 & -0.017 & -0.141** & -0.081 & -0.140 & -0.142* \\
			& (-0.23) & (-0.88) & (-1.09) & (-1.05) & (-0.47) & (-0.48) & (-2.53) & (-1.58) & (-1.28) & (-1.88) \\
			Constant & -6.223*** & -6.067*** & -0.699*** & -0.700*** & -1.058*** & -1.089*** & -0.563*** & -0.684*** & -0.442*** & -0.330*** \\
			& (-66.29) & (-94.80) & (-69.04) & (-110.27) & (-35.28) & (-51.80) & (-11.01) & (-14.53) & (-4.52) & (-6.45) \\
			&   &   &   &   &   &   &   &   &   &  \\
			Observations & 38,881 & 35,134 & 38,881 & 35,134 & 38,881 & 35,134 & 38,881 & 35,134 & 38,881 & 35,134 \\
			Adjusted R-squared & 0.362 & 0.437 & 0.029 & 0.029 & 0.133 & 0.164 & 0.097 & 0.117 & 0.267 & 0.272 \\
			\bottomrule
			\bottomrule
		\end{tabular}%
	\end{center}
		\begin{footnotesize}
			\setcounter{equation}{0}
			\begin{equation}
				TEX_{i,t}=\beta_0+\beta_1\Delta DRET_{i,t-tlag}+\beta_2BN_{i,t-tlag}+\beta_3\Delta DRET_{i,t-tlag}\times 	BN_{i,t-tlag}+\sum\beta_nCONTROLS_{i,t}+\epsilon_{i,t}
			\end{equation}
			
			\noindent Table 4 presents the regression results of Equation (1) across high and low conditional conservatism subsamples. TEX represents a vector of textual properties. CONTROLS denotes a vector of control variables. See \hyperref[appc]{Appendix C} for variable definitions. All financial variables except returns are winsorized at 1\% and 99\% level. All regressions include firm and year-month fixed effects and standard errors are clustered at industry level identified by 4-digit SIC codes. ***, ** and * indicate significance at the 1\%, 5\% and 10\% levels in a two-tailed test.
		\end{footnotesize}
\end{table}%
\end{landscape}

%%%%%%%%%%%%%%%%%%%%%%%%% TABLE 5
\newpage
\begin{landscape}
	% Table generated by Excel2LaTeX from sheet 'T3'
\begin{table}[H] \label{T5}
	\begin{center}
		\tabcolsep=0.11cm
		\begin{tabular}{lcccc}
			\multicolumn{5}{c}{\textbf{Table 5. Narrative Conservatism and Unconditional Conservatism}} \\
			\toprule
			\toprule
			Dep. Variables & \multicolumn{2}{c}{TLAG} & \multicolumn{2}{c}{TONE} \\
			\cmidrule{2-5}
			& (1) & (2) & (3) & (4) \\
			\midrule
			Panel A: Intangible Assets & LOW & HIGH & LOW & HIGH \\
			\midrule
			%&   &   &   &  \\
			$\Delta$DRET & 1.975*** & 3.026*** & -1.205 & -2.647** \\
			& (11.64) & (9.89) & (-1.23) & (-2.07) \\
			BN & -0.032 & -0.130*** & -0.193 & -0.060 \\
			& (-1.13) & (-4.26) & (-1.17) & (-0.38) \\
			\rowcolor[rgb]{ .906,  .902,  .902} \textit{(Pred. Sign)} & (-) & (-) & (+) & (+) \\
			\rowcolor[rgb]{ .906,  .902,  .902} $\Delta$DRET$\times$BN & -3.181*** & -6.326*** & 1.044 & 5.773** \\
			\rowcolor[rgb]{ .906,  .902,  .902} & (-10.61) & (-13.28) & (0.82) & (2.42) \\
			Constant & -3.065*** & -3.588*** & -0.478 & -3.469 \\
			& (-3.58) & (-6.35) & (-0.06) & (-1.18) \\
			&   &   &   &  \\
			Observations & 29,136 & 31,806 & 29,136 & 31,806 \\
			Adjusted R-squared & 0.118 & 0.146 & 0.132 & 0.123 \\
			\midrule
			Panel B: R\&D Expenses & LOW & HIGH & LOW & HIGH \\
			\midrule
			%&   &   &   &  \\
			$\Delta$DRET & 1.651*** & 1.946*** & -0.209 & -1.566 \\
			& (6.85) & (7.52) & (-0.30) & (-1.33) \\
			BN & 0.011 & -0.025 & -0.149 & -0.058 \\
			& (0.26) & (-0.91) & (-1.20) & (-0.50) \\
			\rowcolor[rgb]{ .906,  .902,  .902} \textit{(Pred. Sign)} & (-) & (-) & (+) & (+) \\
			\rowcolor[rgb]{ .906,  .902,  .902} $\Delta$DRET$\times$BN & -2.426*** & -2.983*** & -0.325 & 2.432* \\
			\rowcolor[rgb]{ .906,  .902,  .902} & (-5.65) & (-7.03) & (-0.39) & (1.66) \\
			Constant & -2.520*** & -2.678*** & -1.751 & -5.212 \\
			& (-4.66) & (-5.07) & (-0.25) & (-1.43) \\
			&   &   &   &  \\
			Observations & 19,740 & 22,608 & 19,740 & 22,608 \\
			Adjusted R-squared & 0.106 & 0.143 & 0.184 & 0.115 \\
			\bottomrule
			\bottomrule
		\end{tabular}%
	\end{center}
\end{table}%
	% Table generated by Excel2LaTeX from sheet 'T3'
\begin{table}[H]
	\begin{center}
		\tabcolsep=0.11cm
		\begin{tabular}{lcccccccccc}
			\multicolumn{11}{c}{\textbf{Table 5. Narrative Conservatism and Unconditional Conservatism (Continued)}} \\
			\toprule
			\toprule
			Dep. Variables & \multicolumn{2}{c}{NW} & \multicolumn{2}{c}{N8K} & \multicolumn{2}{c}{NITEM} & \multicolumn{2}{c}{NEXHIBIT} & \multicolumn{2}{c}{NGRAPH} \\
			\cmidrule{2-11}
			& (5) & (6) & (7) & (8) & (9) & (10) & (11) & (12) & (13) & (14) \\
			\midrule
			Panel A: Intangible Assets & LOW & HIGH & LOW & HIGH & LOW & HIGH & LOW & HIGH & LOW & HIGH \\
			\midrule
			%&   &   &   &   &   &  &   &   &   &  \\
			$\Delta$DRET & 0.041 & -0.142 & -0.033*** & -0.042*** & -0.098*** & -0.053 & -0.087 & -0.195*** & -0.148** & -0.467*** \\
			& (0.48) & (-1.02) & (-2.74) & (-2.90) & (-2.62) & (-1.25) & (-1.54) & (-2.95) & (-2.17) & (-3.88) \\
			BN & -0.002 & -0.029* & -0.002 & -0.000 & -0.007 & -0.003 & -0.000 & -0.007 & -0.001 & -0.018 \\
			& (-0.13) & (-1.92) & (-1.10) & (-0.19) & (-0.83) & (-0.45) & (-0.04) & (-0.80) & (-0.07) & (-1.17) \\
			\rowcolor[rgb]{ .906,  .902,  .902} \textit{(Pred. Sign)} & (+) & (+) & (+) & (+) & (+) & (+) & (+) & (+) & (+) & (+) \\
			\rowcolor[rgb]{ .906,  .902,  .902} $\Delta$DRET$\times$BN & -0.042 & 0.059 & 0.049*** & 0.076*** & 0.118* & 0.048 & 0.135 & 0.272*** & 0.219** & 0.622*** \\
			\rowcolor[rgb]{ .906,  .902,  .902} & (-0.34) & (0.32) & (3.01) & (3.48) & (1.95) & (0.76) & (1.51) & (2.72) & (2.14) & (3.18) \\
			Constant & -6.439*** & -7.127*** & -0.692*** & -0.692*** & -0.745*** & -0.890*** & -0.314 & -0.456*** & 0.156* & -0.173 \\
			& (-20.69) & (-21.31) & (-94.98) & (-64.07) & (-11.47) & (-14.91) & (-1.50) & (-2.71) & (1.79) & (-1.36) \\
			&   &   &   &   &   &   &   &   &   &  \\
			Observations & 29,136 & 31,806 & 29,136 & 31,806 & 29,136 & 31,806 & 29,136 & 31,806 & 29,136 & 31,806 \\
			Adjusted R-squared & 0.385 & 0.315 & 0.022 & 0.036 & 0.144 & 0.133 & 0.113 & 0.088 & 0.257 & 0.282 \\
			\midrule
			Panel B: R\&D Expenses & LOW & HIGH & LOW & HIGH & LOW & HIGH & LOW & HIGH & LOW & HIGH\\
			\midrule
			%&   &   &   &   &   &   &   &   &   &  \\
			$\Delta$DRET & -0.068 & 0.005 & -0.054*** & -0.031** & -0.120*** & -0.007 & -0.137** & -0.047 & -0.050 & -0.348*** \\
			& (-0.69) & (0.06) & (-2.60) & (-2.55) & (-3.08) & (-0.23) & (-1.98) & (-1.00) & (-0.63) & (-4.77) \\
			BN & -0.017 & -0.005 & -0.008*** & -0.001 & -0.006 & 0.005 & -0.003 & 0.013 & 0.011 & -0.020 \\
			& (-1.23) & (-0.44) & (-3.60) & (-0.38) & (-0.84) & (1.02) & (-0.23) & (1.59) & (0.56) & (-1.53) \\
			\rowcolor[rgb]{ .906,  .902,  .902} \textit{(Pred. Sign)} & (+) & (+) & (+) & (+) & (+) & (+) & (+) & (+) & (+) & (+) \\
			\rowcolor[rgb]{ .906,  .902,  .902} $\Delta$DRET$\times$BN & 0.032 & -0.010 & 0.054* & 0.049*** & 0.137** & 0.043 & 0.177** & 0.197*** & 0.128* & 0.388*** \\
			\rowcolor[rgb]{ .906,  .902,  .902} & (0.24) & (-0.08) & (1.95) & (4.22) & (2.03) & (1.03) & (2.22) & (3.08) & (1.71) & (4.30) \\
			Constant & -7.250*** & -7.660*** & -0.657*** & -0.676*** & -0.795*** & -0.852*** & -0.476*** & -0.400** & 0.394** & -0.109 \\
			& (-8.77) & (-18.41) & (-25.93) & (-63.28) & (-10.34) & (-12.76) & (-3.74) & (-2.21) & (2.07) & (-1.01) \\
			&   &   &   &   &   &   &   &   &   &  \\
			Observations & 19,740 & 22,608 & 19,740 & 22,608 & 19,740 & 22,608 & 19,740 & 22,608 & 19,740 & 22,608 \\
			Adjusted R-squared & 0.491 & 0.355 & 0.005 & 0.009 & 0.156 & 0.130 & 0.129 & 0.092 & 0.255 & 0.253 \\
			\bottomrule
			\bottomrule
		\end{tabular}%
	\end{center}
		\begin{footnotesize}
			\setcounter{equation}{0}
			\begin{equation}
				TEX_{i,t}=\beta_0+\beta_1\Delta DRET_{i,t-tlag}+\beta_2BN_{i,t-tlag}+\beta_3\Delta DRET_{i,t-tlag}\times 	BN_{i,t-tlag}+\sum\beta_nCONTROLS_{i,t}+\epsilon_{i,t}
			\end{equation}
			
			\noindent Table 5 presents the regression results of Equation (1) across high and low intangible assets and R\&D expenses subsamples. TEX represents a vector of textual properties. CONTROLS denotes a vector of control variables. See \hyperref[appc]{Appendix C} for variable definitions. All financial variables except returns are winsorized at 1\% and 99\% level. All regressions include full set of control variables, firm and year-month fixed effects. Standard errors are clustered at industry level identified by 4-digit SIC codes. ***, ** and * indicate significance at the 1\%, 5\% and 10\% levels in a two-tailed test.
		\end{footnotesize}
\end{table}%
\end{landscape}

%%%%%%%%%%%%%%%%%%%%%%%%% TABLE 6 Panel A
\newpage
%\begin{landscape}
% Table generated by Excel2LaTeX from sheet 'T6'
\begin{table}[H]	\label{T6PA}%
	\begin{center}
		\begin{tabular}{lcccc}
			\multicolumn{5}{c}{\textbf{Table 6. Panel A. Narrative Conservatism in Quarterly Reports}} \\
			\midrule
			\midrule
			& (1) & (2) & (3) & (4) \\
			Dep. Variables & TONE & TONE & NW & NW \\
			\midrule
			%&   &   &   &  \\
			QRET & -0.371*** & 0.095 & -0.039*** & -0.040*** \\
			& (-2.78) & (0.69) & (-3.54) & (-3.54) \\
			NEG & -0.077 & -0.075 & -0.004 & -0.005 \\
			& (-1.59) & (-1.52) & (-0.95) & (-1.08) \\
			\rowcolor[rgb]{ .906,  .902,  .902} \textit{(Pred. Sign)} & (+) & (+) & (+) & (+) \\
			\rowcolor[rgb]{ .906,  .902,  .902} QRET$\times$NEG & 2.274*** & 1.191*** & 0.140*** & 0.094*** \\
			\rowcolor[rgb]{ .906,  .902,  .902} & (8.19) & (5.20) & (6.56) & (5.12) \\
			SIZE &   & 0.540*** &   & -0.027*** \\
			&   & (6.36) &   & (-3.25) \\
			MTB &   & 0.046*** &   & 0.005*** \\
			&   & (3.79) &   & (5.18) \\
			LEV &   & -1.212** &   & -0.293*** \\
			&   & (-2.48) &   & (-10.11) \\
			EARN &   & 14.674*** &   & 0.635*** \\
			&   & (5.54) &   & (3.80) \\
			STD\_EARN &   & -7.233*** &   & -0.654*** \\
			&   & (-4.68) &   & (-6.85) \\
			BUSSEG &   & 0.468** &   & -0.019 \\
			&   & (2.22) &   & (-1.50) \\
			GEOSEG &   & 0.319* &   & 0.020* \\
			&   & (1.82) &   & (1.81) \\
			AF &   & -3.316*** &   & -0.043 \\
			&   & (-4.40) &   & (-1.07) \\
			AFE &   & 3.339*** &   & 0.168*** \\
			&   & (4.60) &   & (3.02) \\
			Constant & -18.117*** & -21.970*** & -8.224*** & -8.082*** \\
			& (-38.84) & (-36.79) & (-267.21) & (-156.81) \\
			&   &   &   &  \\
			Observations & 116,156 & 116,156 & 116,156 & 116,156 \\
			Adjusted R-squared & 0.586 & 0.597 & 0.695 & 0.698 \\
			\bottomrule
			\bottomrule
		\end{tabular}%
	\end{center}
\begin{footnotesize}
	\setcounter{equation}{0}
	\begin{equation}
		TEX_{i,t}=\beta_0+\beta_1QRET_{i,t}+\beta_2NEG_{i,t}+\beta_3QRET_{i,t}\times NEG_{i,t}+\sum\beta_nCONTROLS_{i,t}+\epsilon_{i,t}
	\end{equation}
	
	\noindent Table 6 Panel A presents the regression results of Equation (1) using subsamples of MD\&A (Column 1 and 3) and NFS (Column 2 and 4) sections. TEX represents a vector of textual properties that consists of NW\_MDA, NW\_NFS, TONE\_MDA and TONE\_NFS. CONTROLS denotes a vector of control variables. See \hyperref[appc]{Appendix C} for variable definitions. All financial variables except returns are winsorized at 1\% and 99\% level. All regressions include firm and year-quarter fixed effects and standard errors are clustered at industry level identified by 4-digit SIC codes. ***, ** and * indicate significance at the 1\%, 5\% and 10\% levels in a two-tailed test.
\end{footnotesize}
\end{table}%
%\end{landscape}

%%%%%%%%%%%%%%%%%%%%%%%%% TABLE 6 Panel B
\newpage
%\begin{landscape}
% Table generated by Excel2LaTeX from sheet 'T6'
\begin{table}[H] \label{T6PB}%
	\begin{center}
		\begin{tabular}{lcccc}
			\multicolumn{5}{c}{\textbf{Table 6. Panel B. Narrative Conservatism 10-Q Sections}} \\
			\midrule
			\midrule
			Dep. Variables & \multicolumn{2}{c}{TONE} & \multicolumn{2}{c}{NW} \\
			\cmidrule{2-5}
			& (1) & (2) & (3) & (4) \\
			Section & MDA & NFS & MDA & NFS \\
			\midrule
			%&   &   &   &  \\
			QRET & 0.109 & 0.297 & -0.055*** & -0.033* \\
			& (0.64) & (1.15) & (-4.34) & (-1.70) \\
			NEG & -0.123** & 0.014 & -0.012*** & -0.005 \\
			& (-1.98) & (0.17) & (-3.05) & (-1.01) \\
			\rowcolor[rgb]{ .906,  .902,  .902} \textit{(Pred. Sign)} & (+) & (+) & (+) & (+) \\
			\rowcolor[rgb]{ .906,  .902,  .902} QRET$\times$NEG & 1.423*** & 0.882* & 0.102*** & 0.055* \\
			\rowcolor[rgb]{ .906,  .902,  .902} & (4.54) & (1.88) & (4.18) & (1.65) \\
			SIZE & 0.626*** & 0.900*** & -0.030*** & -0.013 \\
			& (4.26) & (5.14) & (-3.36) & (-1.01) \\
			MTB & 0.021 & 0.054** & 0.003** & 0.004*** \\
			& (1.12) & (2.21) & (2.41) & (3.28) \\
			LEV & -0.213 & -0.802 & -0.189*** & -0.362*** \\
			& (-0.33) & (-0.94) & (-5.32) & (-5.88) \\
			EARN & 17.163*** & 12.079*** & 0.470** & 0.693*** \\
			& (5.26) & (5.69) & (2.16) & (3.83) \\
			STD\_EARN & -8.090*** & -6.020** & -0.547*** & -0.816*** \\
			& (-4.64) & (-2.20) & (-3.35) & (-6.19) \\
			BUSSEG & -0.065 & -0.159 & -0.057*** & -0.031 \\
			& (-0.23) & (-0.45) & (-2.93) & (-1.58) \\
			GEOSEG & 0.052 & 0.999*** & 0.063*** & 0.036** \\
			& (0.16) & (2.61) & (3.01) & (1.98) \\
			AF & 1.979* & -0.343 & 0.140 & -0.073 \\
			& (1.86) & (-0.22) & (1.61) & (-0.95) \\
			AFE & 7.938*** & 4.137*** & 0.227*** & 0.243*** \\
			& (7.81) & (3.74) & (3.20) & (3.56) \\
			Constant & -7.264* & -12.393** & -7.167*** & -7.224*** \\
			& (-1.84) & (-2.57) & (-15.46) & (-18.08) \\
			&   &   &   &  \\
			Observations & 48,089 & 48,089 & 48,089 & 48,089 \\
			Adjusted R-squared & 0.559 & 0.579 & 0.734 & 0.816 \\
			\bottomrule
			\bottomrule
		\end{tabular}%
	\end{center}
\begin{footnotesize}
	\setcounter{equation}{0}
	\begin{equation}
		TEX_{i,t}=\beta_0+\beta_1QRET_{i,t}+\beta_2NEG_{i,t}+\beta_3QRET_{i,t}\times NEG_{i,t}+\sum\beta_nCONTROLS_{i,t}+\epsilon_{i,t}
	\end{equation}
	
	\noindent Table 6 Panel B presents the regression results of Equation (1) using subsamples of MD\&A (Column 1 and 3) and NFS (Column 2 and 4) sections. TEX represents a vector of textual properties that consists of NW\_MDA, NW\_NFS, TONE\_MDA and TONE\_NFS. CONTROLS denotes a vector of control variables. See \hyperref[appc]{Appendix C} for variable definitions. All financial variables except returns are winsorized at 1\% and 99\% level. All regressions include firm and year-quarter fixed effects and standard errors are clustered at industry level identified by 4-digit SIC codes. ***, ** and * indicate significance at the 1\%, 5\% and 10\% levels in a two-tailed test.
\end{footnotesize}
\end{table}%
%\end{landscape}


%%%%%%%%%%%%%%%%%%%%%%%%% TABLE 7
\newpage
\begin{landscape}
	% Table generated by Excel2LaTeX from sheet 'T3'
\begin{table}[H] \label{T7}
	\begin{center}
		\tabcolsep=0.3cm
		\begin{tabular}{lcccc}
			\multicolumn{5}{c}{\textbf{Table 7. Narrative Conservatism and Unconditional Conservatism}} \\
			\toprule
			\toprule
			Dep. Variables & \multicolumn{2}{c}{TLAG} & \multicolumn{2}{c}{TONE} \\
			\cmidrule{2-5}
			& (1) & (2) & (3) & (4) \\
			\midrule
			Panel A: Intangible Assets & LOW & HIGH & LOW & HIGH \\
			\midrule
			%&   &   &   &  \\
			$\Delta$DRET & 1.975*** & 3.026*** & -1.205 & -2.647** \\
			& (11.64) & (9.89) & (-1.23) & (-2.07) \\
			BN & -0.032 & -0.130*** & -0.193 & -0.060 \\
			& (-1.13) & (-4.26) & (-1.17) & (-0.38) \\
			\rowcolor[rgb]{ .906,  .902,  .902} \textit{(Pred. Sign)} & (-) & (-) & (+) & (+) \\
			\rowcolor[rgb]{ .906,  .902,  .902} $\Delta$DRET$\times$BN & -3.181*** & -6.326*** & 1.044 & 5.773** \\
			\rowcolor[rgb]{ .906,  .902,  .902} & (-10.61) & (-13.28) & (0.82) & (2.42) \\
			Constant & -3.065*** & -3.588*** & -0.478 & -3.469 \\
			& (-3.58) & (-6.35) & (-0.06) & (-1.18) \\
			&   &   &   &  \\
			Observations & 29,136 & 31,806 & 29,136 & 31,806 \\
			Adjusted R-squared & 0.118 & 0.146 & 0.132 & 0.123 \\
			\midrule
			Panel B: R\&D Expenses & LOW & HIGH & LOW & HIGH \\
			\midrule
			%&   &   &   &  \\
			$\Delta$DRET & 1.651*** & 1.946*** & -0.209 & -1.566 \\
			& (6.85) & (7.52) & (-0.30) & (-1.33) \\
			BN & 0.011 & -0.025 & -0.149 & -0.058 \\
			& (0.26) & (-0.91) & (-1.20) & (-0.50) \\
			\rowcolor[rgb]{ .906,  .902,  .902} \textit{(Pred. Sign)} & (-) & (-) & (+) & (+) \\
			\rowcolor[rgb]{ .906,  .902,  .902} $\Delta$DRET$\times$BN & -2.426*** & -2.983*** & -0.325 & 2.432* \\
			\rowcolor[rgb]{ .906,  .902,  .902} & (-5.65) & (-7.03) & (-0.39) & (1.66) \\
			Constant & -2.520*** & -2.678*** & -1.751 & -5.212 \\
			& (-4.66) & (-5.07) & (-0.25) & (-1.43) \\
			&   &   &   &  \\
			Observations & 19,740 & 22,608 & 19,740 & 22,608 \\
			Adjusted R-squared & 0.106 & 0.143 & 0.184 & 0.115 \\
			\bottomrule
			\bottomrule
		\end{tabular}%
	\end{center}
		\begin{footnotesize}
			\setcounter{equation}{0}
			\begin{equation}
				TEX_{i,t}=\beta_0+\beta_1\Delta DRET_{i,t-tlag}+\beta_2BN_{i,t-tlag}+\beta_3\Delta DRET_{i,t-tlag}\times 		BN_{i,t-tlag}+\sum\beta_nCONTROLS_{i,t}+\epsilon_{i,t}
			\end{equation}
			
			\noindent Table 7 presents the regression results of Equation (1) across high and low intangible assets and R\&D expenses subsamples. TEX represents a vector of textual properties. CONTROLS denotes a vector of control variables. See \hyperref[appc]{Appendix C} for variable definitions. All financial variables except returns are winsorized at 1\% and 99\% level. All regressions include full set of control variables, firm and year-month fixed effects. Standard errors are clustered at industry level identified by 4-digit SIC codes. ***, ** and * indicate significance at the 1\%, 5\% and 10\% levels in a two-tailed test.
		\end{footnotesize}
\end{table}%
	% Table generated by Excel2LaTeX from sheet 'T3'
\begin{table}[H]
	\begin{center}
		\tabcolsep=0.11cm
		\begin{tabular}{lcccccccccc}
			\multicolumn{11}{c}{\textbf{Table 7. Narrative Conservatism in Voluntary and Mandatory Disclosure (Continued)}} \\
			\toprule
			\toprule
			Dep. Variables & \multicolumn{2}{c}{NW} & \multicolumn{2}{c}{N8K} & \multicolumn{2}{c}{NITEM} & \multicolumn{2}{c}{NEXHIBIT} & \multicolumn{2}{c}{NGRAPH} \\
			\cmidrule{2-11}
			& (5) & (6) & (7) & (8) & (9) & (10) & (11) & (12) & (13) & (14) \\
			Disclosure Type & VD & MD & VD & MD & VD & MD & VD & MD & VD & MD \\
			\midrule
			%&   &   &   &   &   &  &   &   &   &  \\
			$\Delta$DRET & -0.156** & 0.039 & -0.063*** & -0.051 & -0.048*** & -0.020* & -0.092** & -0.017 & -0.153*** & 0.030 \\
			& (-2.37) & (0.31) & (-2.72) & (-1.08) & (-4.07) & (-1.65) & (-2.25) & (-0.18) & (-2.64) & (0.47) \\
			BN & -0.018** & 0.002 & -0.004 & -0.007 & -0.002 & -0.002 & -0.003 & 0.000 & -0.017 & 0.010 \\
			& (-2.14) & (0.13) & (-0.99) & (-1.08) & (-1.59) & (-1.59) & (-0.39) & (0.01) & (-1.61) & (1.02) \\
			\rowcolor[rgb]{ .906,  .902,  .902} \textit{(Pred. Sign)} & (+) & (+) & (+) & (+) & (+) & (+) & (+) & (+) & (+) & (+) \\
			\rowcolor[rgb]{ .906,  .902,  .902} $\Delta$DRET$\times$BN & 0.210** & 0.003 & 0.093*** & 0.045 & 0.070*** & 0.026 & 0.175*** & 0.050 & 0.133 & 0.031 \\
			\rowcolor[rgb]{ .906,  .902,  .902} & (2.18) & (0.02) & (2.91) & (0.75) & (5.44) & (1.60) & (2.86) & (0.47) & (1.61) & (0.42) \\
			SIZE & 0.011 & 0.035*** & 0.003 & -0.003 & -0.001 & -0.001 & 0.000 & 0.005 & 0.006 & -0.003 \\
			& (1.18) & (2.64) & (0.90) & (-0.63) & (-0.50) & (-0.88) & (0.04) & (0.45) & (0.56) & (-0.41) \\
			MTB & 0.000 & -0.004** & -0.000 & -0.001 & -0.000 & 0.000** & -0.002*** & -0.001 & -0.003** & -0.001 \\
			& (0.02) & (-2.20) & (-0.16) & (-0.75) & (-1.01) & (1.99) & (-2.67) & (-0.93) & (-2.02) & (-0.69) \\
			LEV & -0.102** & 0.073 & -0.033** & 0.004 & -0.012** & -0.001 & -0.021 & -0.026 & -0.004 & -0.008 \\
			& (-2.42) & (1.02) & (-2.30) & (0.16) & (-2.57) & (-0.22) & (-1.00) & (-0.51) & (-0.08) & (-0.22) \\
			EARN & 0.302*** & 0.270 & 0.047 & 0.103 & -0.003 & -0.009 & 0.109* & 0.054 & -0.110 & 0.054 \\
			& (2.72) & (1.42) & (1.20) & (1.34) & (-0.23) & (-0.99) & (1.94) & (0.44) & (-1.17) & (0.58) \\
			STD\_EARN & -0.254* & -0.021 & -0.096* & -0.078 & -0.004 & -0.018 & -0.014 & -0.255 & 0.373** & -0.136 \\
			& (-1.94) & (-0.08) & (-1.69) & (-0.81) & (-0.25) & (-0.91) & (-0.17) & (-1.34) & (2.17) & (-1.10) \\
			BUSSEG & -0.004 & -0.025 & 0.006 & -0.017** & 0.000 & 0.000 & 0.012* & -0.027* & -0.015 & 0.001 \\
			& (-0.26) & (-1.11) & (1.35) & (-2.02) & (0.28) & (0.10) & (1.71) & (-1.75) & (-0.69) & (0.11) \\
			GEOSEG & 0.008 & 0.004 & -0.004 & 0.003 & 0.002 & 0.003** & -0.022*** & 0.008 & -0.018 & -0.006 \\
			& (0.67) & (0.20) & (-0.87) & (0.33) & (1.28) & (2.54) & (-3.67) & (0.55) & (-0.92) & (-0.57) \\
			AF & -0.033 & 0.013 & 0.003 & 0.005 & 0.002 & 0.001 & 0.026 & 0.031 & -0.087 & -0.073 \\
			& (-0.43) & (0.18) & (0.13) & (0.15) & (0.17) & (0.09) & (0.74) & (0.37) & (-1.08) & (-1.57) \\
			AFE & 0.013 & -0.266** & 0.034 & -0.080 & -0.019** & 0.022** & 0.005 & -0.192** & -0.170* & -0.022 \\
			& (0.16) & (-2.05) & (1.17) & (-1.64) & (-2.33) & (2.19) & (0.12) & (-2.17) & (-1.77) & (-0.35) \\
			Constant & -6.786*** & -8.541*** & -0.889*** & -0.839*** & -0.687*** & -0.693*** & -0.436*** & -0.585*** & 0.000 & -0.020 \\
			& (-28.58) & (-14.52) & (-18.87) & (-10.34) & (-96.80) & (-130.77) & (-4.01) & (-2.98) & (0.00) & (-0.44) \\
			&   &   &   &   &   &   &   &   &   &  \\
			Observations & 53,460 & 21,900 & 53,460 & 21,900 & 53,460 & 21,900 & 53,460 & 21,900 & 53,460 & 21,900 \\
			Adjusted R-squared & 0.448 & 0.505 & 0.212 & 0.073 & 0.040 & -0.023 & 0.162 & 0.139 & 0.360 & 0.141 \\
			\bottomrule
			\bottomrule
		\end{tabular}%
	\end{center}
		\begin{footnotesize}
			\setcounter{equation}{0}
			\begin{equation}
				TEX_{i,t}=\beta_0+\beta_1\Delta DRET_{i,t-tlag}+\beta_2BN_{i,t-tlag}+\beta_3\Delta DRET_{i,t-tlag}\times 	BN_{i,t-tlag}+\sum\beta_nCONTROLS_{i,t}+\epsilon_{i,t}
			\end{equation}
			
			\noindent Table 7 presents the regression results of Equation (1) across voluntary and mandatory disclosure subsamples. TEX represents a vector of textual properties. CONTROLS denotes a vector of control variables. See \hyperref[appc]{Appendix C} for variable definitions. All financial variables except returns are winsorized at 1\% and 99\% level. All regressions include firm and year-month fixed effects and standard errors are clustered at industry level identified by 4-digit SIC codes. ***, ** and * indicate significance at the 1\%, 5\% and 10\% levels in a two-tailed test.
		\end{footnotesize}
\end{table}%
\end{landscape}


%%%%%%%%%%%%%%%%%%%%%%%%%% TABLE 7
%\newpage
%%\begin{landscape}
%% Table generated by Excel2LaTeX from sheet 'T3'
\begin{table}[H] \label{T7}
	\begin{center}
		\tabcolsep=0.3cm
		\begin{tabular}{lcccc}
			\multicolumn{5}{c}{\textbf{Table 7. Narrative Conservatism and Unconditional Conservatism}} \\
			\toprule
			\toprule
			Dep. Variables & \multicolumn{2}{c}{TLAG} & \multicolumn{2}{c}{TONE} \\
			\cmidrule{2-5}
			& (1) & (2) & (3) & (4) \\
			\midrule
			Panel A: Intangible Assets & LOW & HIGH & LOW & HIGH \\
			\midrule
			%&   &   &   &  \\
			$\Delta$DRET & 1.975*** & 3.026*** & -1.205 & -2.647** \\
			& (11.64) & (9.89) & (-1.23) & (-2.07) \\
			BN & -0.032 & -0.130*** & -0.193 & -0.060 \\
			& (-1.13) & (-4.26) & (-1.17) & (-0.38) \\
			\rowcolor[rgb]{ .906,  .902,  .902} \textit{(Pred. Sign)} & (-) & (-) & (+) & (+) \\
			\rowcolor[rgb]{ .906,  .902,  .902} $\Delta$DRET$\times$BN & -3.181*** & -6.326*** & 1.044 & 5.773** \\
			\rowcolor[rgb]{ .906,  .902,  .902} & (-10.61) & (-13.28) & (0.82) & (2.42) \\
			Constant & -3.065*** & -3.588*** & -0.478 & -3.469 \\
			& (-3.58) & (-6.35) & (-0.06) & (-1.18) \\
			&   &   &   &  \\
			Observations & 29,136 & 31,806 & 29,136 & 31,806 \\
			Adjusted R-squared & 0.118 & 0.146 & 0.132 & 0.123 \\
			\midrule
			Panel B: R\&D Expenses & LOW & HIGH & LOW & HIGH \\
			\midrule
			%&   &   &   &  \\
			$\Delta$DRET & 1.651*** & 1.946*** & -0.209 & -1.566 \\
			& (6.85) & (7.52) & (-0.30) & (-1.33) \\
			BN & 0.011 & -0.025 & -0.149 & -0.058 \\
			& (0.26) & (-0.91) & (-1.20) & (-0.50) \\
			\rowcolor[rgb]{ .906,  .902,  .902} \textit{(Pred. Sign)} & (-) & (-) & (+) & (+) \\
			\rowcolor[rgb]{ .906,  .902,  .902} $\Delta$DRET$\times$BN & -2.426*** & -2.983*** & -0.325 & 2.432* \\
			\rowcolor[rgb]{ .906,  .902,  .902} & (-5.65) & (-7.03) & (-0.39) & (1.66) \\
			Constant & -2.520*** & -2.678*** & -1.751 & -5.212 \\
			& (-4.66) & (-5.07) & (-0.25) & (-1.43) \\
			&   &   &   &  \\
			Observations & 19,740 & 22,608 & 19,740 & 22,608 \\
			Adjusted R-squared & 0.106 & 0.143 & 0.184 & 0.115 \\
			\bottomrule
			\bottomrule
		\end{tabular}%
	\end{center}
		\begin{footnotesize}
			\setcounter{equation}{0}
			\begin{equation}
				TEX_{i,t}=\beta_0+\beta_1\Delta DRET_{i,t-tlag}+\beta_2BN_{i,t-tlag}+\beta_3\Delta DRET_{i,t-tlag}\times 		BN_{i,t-tlag}+\sum\beta_nCONTROLS_{i,t}+\epsilon_{i,t}
			\end{equation}
			
			\noindent Table 7 presents the regression results of Equation (1) across high and low intangible assets and R\&D expenses subsamples. TEX represents a vector of textual properties. CONTROLS denotes a vector of control variables. See \hyperref[appc]{Appendix C} for variable definitions. All financial variables except returns are winsorized at 1\% and 99\% level. All regressions include full set of control variables, firm and year-month fixed effects. Standard errors are clustered at industry level identified by 4-digit SIC codes. ***, ** and * indicate significance at the 1\%, 5\% and 10\% levels in a two-tailed test.
		\end{footnotesize}
\end{table}%
%%\end{landscape}
%
%%%%%%%%%%%%%%%%%%%%%%%%%% TABLE 8
%\newpage
%%\begin{landscape}
%% Table generated by Excel2LaTeX from sheet 'T8PA'
\begin{table}[htbp] \label{T8}
  \centering
    \begin{tabular}{lcccccc}
    \multicolumn{7}{c}{\textbf{Table 8. Managerial Incentives and Narrative Conservatism}} \\
    \midrule
    \midrule
    Dep. Vars.& \multicolumn{2}{c}{NW} & \multicolumn{2}{c}{TONE} & \multicolumn{2}{c}{TLAG}\\
    \midrule
    \textbf{Panel A}  & (1) & (2) & (3) & (4) & (5) & (6) \\
    \multicolumn{1}{l}{\textbf{Option Value}} & LOW & HIGH & LOW & HIGH & LOW & HIGH \\
    \cmidrule{2-7}
    \multicolumn{1}{l}{QRET} & 0.041 & 0.104*** & 1.164*** & 0.545 & -0.224 & -0.151 \\
      & (0.98) & (2.66) & (3.48) & (1.38) & (-0.71) & (-0.54) \\
    \multicolumn{1}{l}{NEG} & 0.023 & -0.003 & -0.082 & -0.132 & 0.095 & -0.070 \\
      & (1.38) & (-0.26) & (-0.70) & (-1.01) & (0.97) & (-0.56) \\
%    \rowcolor[rgb]{ .933,  .925,  .882} Sign Prediction & - & - & + & + & + & + \\
    \rowcolor[rgb]{ .933,  .925,  .882} \multicolumn{1}{l}{QRET$\times$NEG} & -0.177** & -0.252*** & 1.615** & 1.584** & -0.859 & -0.693 \\
    \rowcolor[rgb]{ .933,  .925,  .882}  & (-2.24) & (-4.44) & (2.57) & (2.55) & (-1.63) & (-1.51) \\
%    \multicolumn{1}{l}{SIZE} & -0.007 & 0.026 & 1.316*** & 1.158*** & -0.325** & -0.023 \\
%      & (-0.32) & (1.39) & (5.49) & (5.58) & (-2.15) & (-0.19) \\
%    \multicolumn{1}{l}{MTB} & -0.003 & -0.007*** & 0.101*** & 0.124*** & -0.083*** & -0.006 \\
%      & (-0.80) & (-3.06) & (2.63) & (4.13) & (-3.06) & (-0.35) \\
%    \multicolumn{1}{l}{LEV} & 0.508*** & 0.298*** & -1.696 & -1.473 & 1.485 & 1.000 \\
%      & (5.30) & (3.27) & (-1.59) & (-1.28) & (1.62) & (1.60) \\
%    \multicolumn{1}{l}{Constant} & 8.185*** & 8.324*** & -20.044*** & -6.415 & 46.377*** & 40.805*** \\
%      & (57.72) & (13.16) & (-3.64) & (-1.35) & (27.95) & (11.75) \\
    \rowcolor[rgb]{ .933,  .925,  .882} \multicolumn{1}{l}{Diff. QRET$\times$NEG} & \multicolumn{2}{c}{0.076$^{\star\star\star}$} & \multicolumn{2}{c}{0.030} & \multicolumn{2}{c}{-0.166} \\
    \rowcolor[rgb]{ .933,  .925,  .882}  & \multicolumn{2}{c}{(2.02)} & \multicolumn{2}{c}{(0.14)} & \multicolumn{2}{c}{(-0.86)} \\
      &   &   &   &   &   &  \\
    \multicolumn{1}{l}{Observations} & 15,229 & 15,226 & 15,229 & 15,226 & 15,229 & 15,226 \\
    \multicolumn{1}{l}{Adjusted R-squared} & 0.443 & 0.493 & 0.551 & 0.610 & 0.540 & 0.588 \\
    \midrule
    \textbf{Panel B}  & (1) & (2) & (3) & (4) & (5) & (6) \\
    \multicolumn{1}{l}{\textbf{SEO}} & NO & YES & NO & YES & NO & YES \\
    \cmidrule{2-7}
    QRET & 0.060*** & 0.027 & -0.222 & 0.060 & -0.506*** & -0.153 \\
    & (3.70) & (1.52) & (-1.32) & (0.41) & (-2.89) & (-0.79) \\
    NEG & -0.002 & -0.001 & -0.104 & -0.073 & 0.048 & 0.050 \\
    & (-0.25) & (-0.11) & (-1.56) & (-0.82) & (1.00) & (0.84) \\
    \rowcolor[rgb]{ .933,  .925,  .882} QRET$\times$NEG & -0.153*** & -0.163*** & 2.448*** & 1.357*** & -0.510* & -0.415 \\
    \rowcolor[rgb]{ .933,  .925,  .882} & (-5.33) & (-4.25) & (8.25) & (3.16) & (-1.73) & (-1.49) \\
    \rowcolor[rgb]{ .933,  .925,  .882} \multicolumn{1}{l}{Diff. QRET$\times$NEG} & \multicolumn{2}{c}{0.009} & \multicolumn{2}{c}{1.091$^{\star\star\star}$} & \multicolumn{2}{c}{-0.095} \\
    \rowcolor[rgb]{ .933,  .925,  .882}  & \multicolumn{2}{c}{(1.07)} & \multicolumn{2}{c}{(5.99)} & \multicolumn{2}{c}{(-0.59)} \\
    &   &   &   &   &   &  \\
    Observations & 45,490 & 37,054 & 45,490 & 37,054 & 45,490 & 37,054 \\
    Adjusted R-squared & 0.696 & 0.687 & 0.552 & 0.623 & 0.634 & 0.674 \\
    \midrule
    \multicolumn{1}{l}{Year-quarter FE} & YES & YES & YES & YES & YES & YES \\
    \multicolumn{1}{l}{Firm FE} & YES & YES & YES & YES & YES & YES \\
    \multicolumn{1}{l}{Industry clustered SE} & YES & YES & YES & YES & YES & YES \\
    \multicolumn{1}{l}{Controls} & YES & YES & YES & YES & YES & YES \\
    \bottomrule
    \bottomrule
    \end{tabular}%
\end{table}%

%%\end{landscape}
%
%%%%%%%%%%%%%%%%%%%%%%%%%% TABLE 9
%\newpage
%%\begin{landscape}
%% Table generated by Excel2LaTeX from sheet 'T10'
\begin{table}[H]   \label{T9}
  \begin{center}
  	    \begin{tabular}{lcccccc}
  		\multicolumn{7}{c}{\textbf{Table 9. Narrative Conservatism, Intangible Assets and R\&D Expenses}} \\
  		\midrule
  		\midrule
  		Dep. Variables & \multicolumn{2}{c}{NW} & \multicolumn{2}{c}{TONE} & \multicolumn{2}{c}{TLAG} \\
  		& (1) & \multicolumn{1}{c}{(2)} & (3) & \multicolumn{1}{c}{(4)} & (5) & \multicolumn{1}{c}{(6)} \\
  		\midrule
  		\textbf{Panel A: Intangible Assets} & LOW & HIGH & LOW & HIGH & LOW & HIGH \\
  		\cmidrule{2-7}
%  		QRET & 0.002 & 0.037* & 0.566*** & 0.466 & -0.527*** & -0.304 \\
%  		& (0.20) & (1.95) & (3.53) & (1.55) & (-3.55) & (-1.27) \\
%  		NEG & 0.006* & 0.010*** & 0.015 & -0.134** & -0.023 & 0.051 \\
%  		& (1.70) & (2.66) & (0.21) & (-2.16) & (-0.44) & (0.89) \\
  		\rowcolor[rgb]{ .933,  .925,  .882} \textit{(Pred. Sign)} & (-) & (-) & (+) & (+) & (+) & (+) \\
  		\rowcolor[rgb]{ .933,  .925,  .882} QRET$\times$NEG & -0.024 & -0.068*** & 0.469 & 0.475 & -0.109 & -0.093 \\
  		\rowcolor[rgb]{ .933,  .925,  .882} & (-1.21) & (-2.71) & (1.50) & (1.08) & (-0.44) & (-0.24) \\
  		&   &   &   &   &   &  \\
  		Observations & 29,636 & 29,634 & 29,636 & 29,634 & 29,636 & 29,634 \\
  		Adjusted R-squared & 0.831 & 0.798 & 0.708 & 0.678 & 0.654 & 0.693 \\
  		\midrule
  		\textbf{Panel B: R\&D Expenses} & LOW & HIGH & LOW & HIGH & LOW & HIGH \\
  		\cmidrule{2-7}
  
%  		QRET & 0.015 & 0.055** & 0.659** & 0.525*** & -0.413** & -0.450* \\
%  		 & (0.75) & (2.18) & (2.52) & (2.68) & (-2.05) & (-1.70) \\
%  		NEG & 0.000 & 0.015 & -0.061 & -0.120 & 0.109 & -0.025 \\
%  		 & (0.05) & (1.64) & (-0.64) & (-1.20) & (1.50) & (-0.34) \\
  		\rowcolor[rgb]{ .933,  .925,  .882} \textit{(Pred. Sign)} & (-) & (-) & (+) & (+) & (+) & (+) \\
  		\rowcolor[rgb]{ .933,  .925,  .882} QRET$\times$NEG & -0.065 & -0.075** & 0.710 & 0.048 & 0.336 & -0.029 \\
  		\rowcolor[rgb]{ .933,  .925,  .882}  & (-1.56) & (-2.45) & (1.53) & (0.10) & (1.15) & (-0.06) \\
  		&   &   &   &   &   &  \\
  		Observations & 22,899 & 22,898 & 22,899 & 22,898 & 22,899 & 22,898 \\
  		Adjusted R-squared & 0.623 & 0.682 & 0.581 & 0.635 & 0.626 & 0.619 \\
  		
  		\bottomrule
  		\bottomrule
  	\end{tabular}%
  \end{center}
\begin{footnotesize}
	\setcounter{equation}{0}
	\begin{equation}
		TEX_{i,t}=\beta_0+\beta_1QRET_{i,t}+\beta_2NEG_{i,t}+\beta_3QRET_{i,t}\times NEG_{i,t}+\sum\beta_nCONTROLS_{i,t}+\epsilon_{i,t}
	\end{equation}
	
	\noindent Table 9 presents the regression results of Equation (1) using 10-Q subsamples of intangible assets (Panel A) and R\&D expenses (Panel B). TEX represents a vector of textual properties that consists of NW, TONE and TLAG. All regressions control for SIZE, MTB, LEV, EARN, STD\_RET, STD\_EARN, AGE, BUSSEG, GEOSEG, AFE and AF (untabulated). See \hyperref[appb]{Appendix B} for variable definitions. All financial variables except returns are winsorized at 1\% and 99\% level. All regressions include firm and time fixed effects and standard errors are clustered at industry level identified by 4-digit SIC codes. ***, ** and * indicate significance at the 1\%, 5\% and 10\% levels in a two-tailed test.
\end{footnotesize}
\end{table}%

%%\end{landscape}
\end{document}

