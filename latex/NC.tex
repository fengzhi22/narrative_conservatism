\documentclass[a4paper]{article}
\usepackage{geometry}
\usepackage{graphicx}
\usepackage{epsfig}
\usepackage{amsmath}
\usepackage{indentfirst}
\usepackage{float}
\usepackage{setspace}
\usepackage{amsfonts}
\usepackage{hyperref}
\usepackage{booktabs}
\usepackage{caption}
\usepackage{subfigure}
\usepackage{times}
\usepackage{hyperref}
\usepackage[round]{natbib}
\usepackage{verbatim}
\usepackage{colortbl}
\usepackage{lscape}
\usepackage[affil-it]{authblk}
\urlstyle{same}
\usepackage{footnotebackref}

\geometry{left=2cm,right=2cm,top=2cm,bottom=2cm}

\setlength{\parindent}{2em}

\begin{document}
	
	\title{Narrative Conservatism}
	
	\author{Beatriz Garc\'ia Osma}
	
	\author{Juan Manuel Garc\'ia Lara}
	
	\author{Fengzhi Zhu%
		\thanks{Email: fzhu@emp.uc3m.es}}
	
	\affil{Department of Business Administration, Universidad Carlos III de Madrid, Spain}
	
	\date{Dated: \today}
	
	\maketitle
\thispagestyle{empty}
\begin{spacing}{2}
\begin{abstract}
	Extant accounting literature documents the existance of conditional conservatism in numerical disclosures, i.e. earnings reflecting bad news in a more timely manner than good news. However, little is known about the level of conservatism in narrative disclosure. In this paper, we study whether corporate narrative disclosure is more responsive to bad news than good news. News is proxied by market returns and narrative responsiveness is defined in terms of TEXt content and filing timeliness. Using 10-Q and 8-K filings automatically retrieved from the SEC EDGAR database with a time coverage from 1993 to 2020, we find that narrative disclosure is longer, more consistent with news, and timelier in response to bad news comparing to good news, consistent with narrative disclosure being conservative. 
	%[addtional tests and roubustness]% 
	We contribute to the literature on accounting conservatism by providing evidence on asymmetric responsiveness to good news and bad news in narrative disclosure.\\
	\newline
	
	\textbf{Keywords}: \textit{narrative disclosure;  conditional conservatism; asymmetric responsiveness; tone; textual analysis}
\end{abstract}

\clearpage

\setcounter{page}{1}
\section{Introduction}

%[motivation]

%[setting]

%[data]

%[robustness]

%[contribution]

The rest of the paper structures as follows: Section 2 develops theoretical framework. Section 3 explains empirical models and data selection. Section 4 presents main results. Section 5 performs robustness checks and Section 6 concludes.

\section{Theoretical Framework}
\subsection{Reliability/faithful representaton and Timeliness}

Why 10-Q and 8-K?

Advantage and disadvantage of 10-Q and 8-K?

1) 10-Q contains managerial discussion while 8-K does not (more neutral language)

2) 10-Q is not timely but 8-K is 

\cite{secFinalRuleAdditional2004} announced a reform in 8-K item classification which became effective on May 23rd of 2004 (inclusive), in which SEC states that:
“\textit{Under the previous Form 8-K regime, companies were required to report very few significant corporate events. The limited number of Form 8-K disclosure items permitted a public company to delay disclosure of many significant events until the due date for its next periodic report. During such a delay, the market was unable to assimilate such undisclosed information into the value of a company's securities. The revisions that we adopt today will benefit markets by increasing the number of unquestionably or presumptively material events that must be disclosed currently. They will also provide investors with better and more timely disclosure of important corporate events.}”

Nature of the two types of filings, why we have ABTONE only for 10-Q and N8K, NITEM only for 8-K?

\begin{center}
	H1: Narrative disclosure is longer in response to bad news comparing to good news.
\end{center}

\begin{center}
	H2: Narrative disclosure is more consistent with news in response to bad news comparing to good news.
\end{center}

\begin{center}
	H3: Narrative disclosure is timelier in response to bad news comparing to good news.
\end{center}

\section{Research Design}
\subsection{Model Specification}
\subsubsection{Form 10-Q}
10-Q filings are quarterly reports that are filed to SEC within 40 (for accelerated filers) or 45 days (for all other registrants) after fiscal quarter-end according to Section 13 or 15(d) of the Securities Exchange Act of 1934.  Given their stable periodicity, we design the following model to explore how 10-Q filings behave when firms have good versus bad news. 
\begin{equation} \label{eq1}
TEX_{i,t}=\beta_0+\beta_1QRET_{i,t}+\beta_2NEG_{i,t}+\beta_3QRET_{i,t}\times NEG_{i,t}+\beta_nControls_{i,t}+\epsilon_{i,t}
\end{equation}

In Equation (1), QRET denotes the quarterly market-adjusted stock return of firm i at time t. Assuming some degree of market efficiency, stock returns incorporate public and private information and therefore is indicative of good and bad news of firms. %[cite needed]
NEG is an indicator for bad news, which is set to 1 if QRET is negative and 0 otherwise. \textit{Controls} represents a vector of control variables, which includes firm size (SIZE), market-to-book ratio (MTB) and leverage ratio (LEV) (see \hyperref[appc]{Appendix C} for detailed variable definition). We aim to alleviate some of the omitted unobservable variable bias by controlling for these three firm characteristics, as the three factors can affect stock returns and firm narrative disclosure simultaneously (\cite{liInformationContentForwardLooking2010}, \cite{huangToneManagement2014}). %[specific explanation on each of the controls and cite needed]
Notice that the right-hand side of Equation (1) resembles the Basu model on conditional conservatism (\cite{basuConservatismPrincipleAsymmetric1997}), in which he studies how earnings respond differently to market returns when the returns are positive versus negative. Our model differs from the Basu model in replacing earnings with several textual variables in order to examine the responses of narrative disclosures to positive versus negative market returns. Specifically, TEX represents a vector of textual properties that consists of number of words (NW), tone (TONE) and reporting time lag (TLAG) \footnote{There are two limitations in proxying timeliness of narrative disclosure with 10-Q reporting time lag. First, 10-Q filings shall be filed once every quarter and after fiscal quarter-end, so it cannot be a very timely communication vehicle in responding to the news released during early days in a fiscal quarter, regardless of managerial reporting incentive. Through 10-Q, managers can only bunch information acquired during the fiscal quarter and respond at the quarter-end due to this quarterly report format limitation. Second, apart from narrative disclosures, 10-Q filings also contains quarterly financial statements, so the reporting time lag of 10-Q does not strictly measure the timeliness of narrative disclosure, but the timeliness of disclosure in general. We address this deficiency with 8-K reporting time lag.}. NW is calculated as the natural logarithm of one plus the count of total words. TONE is defined as number of net positive words per thousand total words, which is calculated as total number of positive words minus the sum of total number of negative words and total number of negations, and multiply the previous result by one thousand for ease of interpretation. We follow \cite{loughranWhenLiabilityNot2011} and count negations as cases where negation words \footnote{Negation words include: “no”, “not", “none", “neither", “never", “nobody" (\cite{tottieNegationEnglishSpeech1991}).} occurs within four or fewer words from a positive word. By taking negations of positive words into consideration when calculating tone, we control for the fact that it is common for firms to frame bad news using negated positive words (“did not profit”). We do not control for negations of negative words because firms rarely communicate good news with negated negative words (“did not fail”). TLAG is defined as number of days elapsed between the news release date and document filing date in EDGAR.

The coefficient of interest in Equation (1) is $\beta_3$, which can be interpreted as the difference in responsiveness of textual properties to good and bad news. If narrative disclosure is conservative, we expect it to be longer, more consistent with news and timelier when firms have bad news. In the case of NW being the dependent variable, $\beta_3^{NW}$ should be negative under H1 because QRET is always negative when NEG equals 1, and therefore the product of the interactive term $\beta_3^{NW}QRET_{i,t}\times NEG_{i,t}$ is positive, which translate into an incremental length in terms of number of words. Following the same logic, $\beta_3^{TLAG}$ of TLAG regression should be positive under H3, which translates into shorter reporting time lag. The interpretation of $\beta_3^{TONE}$ is different from those of the previous two estimates, in the sense that $\beta_3^{TONE}$ represents the incremental consistency between news and tone. We define consistency as the correspondence of positive tone to good news and negative tone to bad news. Under this definition, a positive incremental consistency, which is reflected in Equation (1) as positive $\beta_3^{TONE}$, means that on average, more negative words are used to discuss bad news than positive words are used to discuss good news, given the same magnitude of news impact.

Additionally, we construct an abnormal tone measure (ABTONE) following the expected tone model in \cite{huangToneManagement2014}. ABTONE is calculated as the residual of the following model \footnote{Our expected tone model differs from \cite{huangToneManagement2014} in replacing book-to-market ratio with market-to-book ratio}:
\begin{equation} \label{eq2}
\begin{split}
TONE_{i,t}=\beta_0&+\beta_1EARN_{i,t}+\beta_2RET_{i,t}+\beta_3SIZE_{i,t}+\beta_4MTB_{i,t}+\beta_5STD\_EARN_{i,t}\\
&+\beta_6STD\_RET_{i,t}+\beta_7AGE_{i,t}+\beta_8BUSSEG_{i,t}+\beta_9GEOSEG_{i,t}+\beta_{10}LOSS_{i,t}\\
&+\beta_{11}\Delta EARN_{i,t}+\beta_{12}AFE_{i,t}+\beta_{13}AF_{i,t}+\epsilon_{i,t}
\end{split}
\end{equation}
Where TONE is the number of net positive words per thousand total words. Other financial variables are defined in \hyperref[appc]{Appendix C}. As residuals of Equation (2), ABTONE captures the portion in tone that is orthogonal to firm fundamentals such as business complexity, growth opportunities and risk, and therefore is the portion subject to managerial discretion. Our regression result of Equation (2) is consistent with \cite{huangToneManagement2014} \footnote{See result comparison in \hyperref[oat1]{Table 1 of Online Appendix}.}.
%[explain ABTONE interpretation]

\subsubsection{Form 8-K}
8-K filings are required to be filed upon the occurrence of any one or more events pertaining to a wide set of pre-specified corporate events, where each type of event is classified as an 8-K \textit{item} (see list of 8-K items in \hyperref[appd]{Appendix D}). 
%%%% 8-K filing data structure
8-K filings in EDGAR database have a unique data structure: although most companies only file one 8-K report in one day and each 8-K report usually contains only one or two 8-K items, some firms report more than one 8-K per day and each 8-K may contain more than two items. As we want to analyze the responsiveness of 8-K filings to good and bad news, and our news proxy daily stock return is at firm-day level, we aggregate the raw 8-K data at individual event level into 8-K data at firm-day level by summing up all raw count variables over each firm-day. For instance, the count variable $nw_{i,t}$ in 8-K dataset stands for number of total words in all 8-K filings reported in day t for firm i, instead of number of total words of one specific 8-K filing. We further construct two new variables N8K and NITEM, which are defined as number of 8-K filings reported in one day and number of 8-K items reported in one day, respectively. We call a firm-day “8-K day” if there is at least one 8-K reported for that day.

%%%% 8-K news proxy
Next, we build our proxy for news under 8-K context. We obtain the daily market-adjusted stock returns (DRET) based on raw data from CRSP and calculate the change in daily returns ($\Delta$DRET). Then, we define a firm-day as a “bad (good) news day” if the negative (positive) change in daily market-adjusted stock return ($\Delta$DRET) is three times larger than the firm's average decrease (increase) in daily return over the calendar year. BN is an indicator for bad news day, which is set to 1 if this firm-day is a bad news day, and 0 if this firm-day is a good news day \footnote{We code BN to missing if the firm-day does not have any news. Therefore, in our final 8-K sample for regression analysis, all observations are either good or bad news firm-days.}. Notice that we define good and bad news differently under 8-K context comparing to the 10-Q context, because daily returns are more volatile than quarterly returns and the sign of daily returns can change constantly merely due to trading noises. Therefore, we only focus on firm-days with sizable changes (three times than annual average) in daily returns, which is more likely to result from significant events happened in the firm and is more likely to reflect fundamental information about the firm.

%%%% 8-K event matching process
At last, we construct 8-K sample for regression analysis through a matching process as illustrated in \hyperref[fig1]{Figure 1}. The idea of matching is to pair the news releases to firms' responses to the precedent news in form of 8-K filings. Specifically, we match every news day to its first posterior 8-K day, ignoring the successive 8-K days (if any) between two news days (Match-1), or in some cases the 8-K day coincides with news day (Match-2). After matching, we calculate TLAG of 8-K sample as the number of days elapsed between the news release date and document filing date \footnote{All filings in EDGAR have two dates: filing date and reporting period date. Filing date is the date when the document is filed to EDGAR, and reporting period date is the end date of reporting period of the filing. We match by 8-K \textit{reporting period date} because we want to make sure that the 8-K filings reported on a specific date are indeed responses to the news revealed just before, although we acknowledge that the fact that some 8-K filings are reported immediately after a certain news does not guarantee that the 8-K filings are meant to address that news, i.e. time sequence does not necessarily imply association. We calculate TLAG using 8-K \textit{filing date} because we are interested in whether 8-K filings respond to good and bad news in distinct timely manner, allowing for managerial discretion in reporting speed.}. 

After 8-K sample construction, we design the following model to explore how 8-K filings behave when firms have good versus bad news.
\begin{equation} \label{eq3}
TEX_{i,t}=\beta_0+\beta_1\Delta DRET_{i,t-tlag}+\beta_2BN_{i,t-tlag}+\beta_3\Delta DRET_{i,t-tlag}\times BN_{i,t-tlag}+\beta_nControls_{i,t}+\epsilon_{i,t}
\end{equation}
Where $\Delta$DRET and BN are changes in daily returns and bad news indicator for firm i \textit{at news release date}, respectively. We deploy $\Delta$DRET rather than DRET in our model because under 8-K context, our bad news indicator BN is defined based on $\Delta$DRET, as opposed to DRET. In Equation (3), \textit{Controls} denotes a vector of control variables \textit{at 8-K filing date}\footnote{Because our measures of firm fundamentals are calculated based on Compustat quarterly data, the variation in firm fundamental measures is very small (if any) either we control for them at news release date (t-tlag) or at 8-K filing date (t), as the average reporting time lag of 8-K is only 23 days.}, which includes firm size (SIZE), market-to-book ratio (MTB) and leverage ratio (LEV). We aim to control for these fundamental characteristics that could affect firms reporting policy in order to address some of the omitted unobservable variable bias. TEX represents a vector of textual properties that consists of number of words (NW), tone (TONE) and reporting time lag (TLAG), which share the same definition as in 10-Q context. The coefficient of interest in Equation 3 is still $\beta_3$, and its interpretation is the same as that in the context of 10-Q. If narrative disclosure is conservative, we expect it to be longer, more consistent with news and timelier when firms have bad news, which is reflected as negative $\beta_3^{NW}$, positive  $\beta_3^{TONE}$ and positive $\beta_3^{TLAG}$.

\subsection{Data}
We obtain historical financial data and segment data from Compustat, stock returns from Center for Research in Security Prices (CRSP), analysts’ earnings forecasts data from I/B/E/S. We retrieve 10-Q and 8-K data from EDGAR through a self-developed Python program (see \hyperref[appa]{Appendix A} for detailed description of EDGAR data collection process). First, we successfully parsed and retrieved 575,579 (1,489,626) unique 10-Q (8-K) filings out of 594,017 (1,628,467) existing filings in EDGAR from 1993-Q1 to 2020-Q1. Next, we merge 10-Q and 8-K dataset with other datasets of firm characteristics and market performance. Finally, we screen the merged 10-Q and 8-K dataset according to the following criteria. We eliminate observations with missing value in key accounting and financial-market variables or with beginning-of-quarter stock prices below \$1. In 10-Q sample, we further delete observations without analyst coverage variables. We exclude financial (SIC code between 6000 and 6999) and utility (SIC code between 4900 and 4999) firms because the accounting policy for the former is different from that of other industries, and they are both highly regulated which makes them incomparable to other industries in general. Observations with non-positive total assets or book value of equity, or with negative or longer-than-99\% reporting time lag \footnote{Before truncation, the average reporting time lag for 10-Q is 40 days, but the maximum lag is 4,069 days, which is filed by CPI Corp in 2007-06-21 to report a quarterly result as of 1996-04-27 (see \url{https://www.sec.gov/Archives/edgar/containers/fix041/25354/0001140361-07-012753.txt}). We read some of the 10-Q filings with such extremely long reporting lag but do not find an explanation for the unusual delay. In theory SEC requires 10-Q filings to be filed within 40 or 45 days after fiscal quarter-end (Section 13 or 15(d) of the Securities Exchange Act of 1934), so it remains a puzzle as to why in practice there exists a few accepted filings with such a big delay in EDGAR database. However, for the purpose of this study we eliminate observations with unusual delay. We also perform the 99\% truncation on TLAG for 8-K filings.}, or with less-than-1\% total number of words are dropped. All financial variables except returns are winsorized at 1\% and 99\% level in order to minimize the impact of outliers. Our final 10-Q sample contains 91,606 firm-quarter observations which constitues of observations from 5,250 unique firms from 1993 to 2016. Final 8-K sample contains 244,401 firm-day observations which constitues of observations from 8,876 unique firms from 1993 to 2019. \hyperref[fig2]{Figure 2} illustrates the sample selection process of 10-Q and 8-K filings. Sample size can vary across different test specifications and is stated in each table. 

\section{Results}
\subsection{Summary Statistics}
\hyperref[T1PA]{Table 1 Panel A} presents summary statistics for key variables in 10-Q sample. The summary statistics of raw word count for positive, negative, uncertainty, litigation and modal words in 10-Q narratives (untabulated) are consistent with LM 10-Q dataset \footnote{Bill McDonald and Tim Loughran created a dataset containing summary data for each individual 10-X (e.g., 10-K, 10-K/A, 10-Q405, etc.) filing, available at \url{https://sraf.nd.edu/textual-analysis/resources/\#LM_10X_Summaries}}. On average, each 10-Q filing contains 10,215 words, with considerable variation across filings. TONE is negative in general and we propose two possible explanations for this. First, the LM sentiment word list contains more negative (2,355) than positive (354) words by construction, so the range of words being classified as negative words is broader than the range of positive words, resulting in larger negative word count than positive word count in general. Second, since optimistic language increases litigation risk (\cite{rogersDisclosureToneShareholder2011}), firms may avoid positive words in 10-Q filings in order to reduce litigation risk. On average, 10-Qs are filed 39 days after fiscal quarter-end, and 75\% of 10-Qs are filed within 44 days after fiscal quarter-end, which are one day before the filing deadline for accelerated filers and all other filers, respectively. This shows that firms do have discretion in reporting speed \footnote{One concern is that the length of reporting time lag may not be fully controlled by firms, as prior auditing literature suggests that a set of auditor characteristics contributes to unexpected audit report lag (\cite{knechelAdditionalEvidenceAudit2001}, \cite{bamberAuditStructureOther1993}), which consequently leads to filing delay in audited financial reports. However, because audit for quarterly filings is not mandated by law, and due to the expensive auditing cost, most 10-Q filings are not audited.}. ABTONE is normally distributed around zero by construction, and its summary statistics is consistent with that in \cite{huangToneManagement2014}.

\hyperref[T1PB]{Table 1 Panel B} presents summary statistics for key variables in 8-K sample. 8-K filings are more neutral in tone comparing to 10-Q filings, with average TONE being almost zero. Also, 8-K filings are more timely responses to news events, with average TLAG being 23 days, which is 16 days sooner than average 10-Q filings. In more than 75\% of our 8-K firm-day observatiosns, there is only one reported 8-K filing per day, and the maximum number of 8-Ks a firm has reported in one day is five. On average, all reported 8-Ks in one day contains 1,258 words in total, which is significantly less than the number of words per 10-Q. Finally, on average firms report two 8-K items per day, with the maximum number being sixteen. \hyperref[fig3]{Figure 3} illustrates the 8-K item distribution before (left) and after (right) May 23rd of 2004, respectively. Each share of pie chart shows the percentage of corporate events reported under each 8-K items. The most commonly reported 8-K items before reform are Item 7: financial statements and exhibits (36.4\%), Item 5: other events (29.6\%) and Item 2: acquisition or disposition of assets (13.8\%), whereas after reform the most frequent ones are Item 9.01: financial statements and exhibits (37.7\%), Item 2.02: results of operations and financial condition (18.9\%) and Item 8.01: other events (9.4\%). Despite of a sharp decline in reporting frequency from 29.6\% to 9.4\%, voluntary disclosure item, i.e. other events, still makes up for a large proportion in total 8-K filings. This indicates that firms indeed use 8-K filings to report events that are not explicitly required but the firms consider important to shareholders, and it further suggests that firms do have discretion in choosing whether, when and how to communicate with the general public via 8-K form. 

\hyperref[T1PC]{Panel C} and \hyperref[T1PD]{Panel D} of Table 1 present correlation matrix of key variables in 10-Q and 8-K sample, respectively. In Panel C, the correlations between ABTONE and other financial variables are close to zero, which verifies the fact that ABTONE captures the portion of discretionary tone that is orthogonal to firm fundamentals. 

\subsection{Is 10-Q narrative disclosure more responsive to bad news than good news?}
\hyperref[T2PA]{Table 2 Panel A} presents the regression result of \hyperref[eq1]{Equation 1}. Column 2, 4 and 6 include firm and time fixed effects in order to control for unobservable firm characteristics or time trends that may bias our estimation. Furthurmore, given that reporting policy of firms within a same industry may be similar, which could lead to high correlations among observations in textual variables such as NW, TLAG and TONE, we cluster standard errors in Column 2, 4 and 6 at 4-digit SIC code industry level to correct the potential existance of serial correlation in dependent variables (\cite{petersenEstimatingStandardErrors2009}). Our clustering approach yields 375 clusters in 10-Q sample (approximately 244 observations per cluster on average). As predicted by H1, the coefficient of QRET$\times$NEG is significantly negative for NW, consistent with 10-Q narratives being longer in response to bad news comparing to good news. Also, consistent with H2, the coefficient of QRET$\times$NEG is significantly positive for TONE, which suggests that 10-Q narratives are more consistent with news in response to bad news comparing to good news. However, in contrast to H3, the coefficient of QRET$\times$NEG is significantly negative for TLAG, which suggests that 10-Q reporting time lag is incrementally longer in response to bad news comparing to good news, i.e. 10-Q fillings respond to good news in a timlier manner than bad news. This delay in bad news response may appear because firms invest more resource and time on preparing the 10-Q filings in order to analyze and explain the causes of bad news. Due to the two limitations discussed in previous section about proxying timeliness of narrative disclosure with 10-Q reporting time lag, we interpret the TLAG result obtained in 10-Q sample only as supplemental evidence on timeliness of narrative disclosure in general.

In addition to the main hypotheses, we are interested in whether firms use different tone management strategy to influence investors' perception in response to good news versus bad news. We extend \hyperref[eq1]{Equation (1)} to the abnormal tone (ABTONE) proposed by \cite{huangToneManagement2014}, and estimate the following model:
\begin{equation} \label{eq4}
ABTONE_{i,t}=\beta_0+\beta_1QRET_{i,t}+\beta_2NEG_{i,t}+\beta_3QRET_{i,t}\times NEG_{i,t}+\beta_nControls_{i,t}+\epsilon_{i,t}
\end{equation}
Where ABTONE measures the discretionary portion of tone that is uncorrelated with firm fundamentals such as business complexity, growth opportunities and risk. Positive (negative) ABTONE indicates that the sentiment of 10-Q filing is more positive (negative) than it should be conditional on firm fundamentals. In Equation (4), positive $\beta_1$ can be obtained only when the signs of returns (QRET) and abnormal tone (ABTONE) agree, suggesting that firms deploy more positive (negative) sentiment than they should in 10-Q filings in response to good (bad) news. Vice versa, negative $\beta_1$ suggests that firms deploy more positive (negative) sentiment than they should in 10-Q filings in response to bad (good) news. We label the phenomenon behind positive $\beta_1$ as \textit{tone exaggeration} and that behind negative $\beta_1$ as \textit{tone attenuation}, which are two different forms of tone management. If none of the two types of tone management is present in 10-Q filings, then $\beta_1$ should not be significantly different from zero. The coefficient of interest in Equation (4) is $\beta_3$, which can be interpreted as the \textit{incremental} tone exaggeration or attenuation in case of bad news comparing to good news, depending on the sign of $\beta_3$.

One key research design issue in estimating \hyperref[eq4]{Equation 4} is that the dependent variable ABTONE is calculated as residuals from \hyperref[eq2]{Equation 2}. \cite{chenIncorrectInferencesWhen2018} points out that using residuals as dependent variables may cause incorrect inferences. Therefore, we apply the following two remedies as suggested in \cite{chenIncorrectInferencesWhen2018}. First, we include all regressors in Equation 2 as control variables in Equation 4. Second, we combine all the model regressors in Equation 2 and Equation 4 into one single-, as opposed to two-step regression, i.e. we estimate the following single-step regression:
\begin{equation} \label{eq5}
\begin{split}
TONE_{i,t}=\beta_0+\beta_1QRET_{i,t}+\beta_2NEG_{i,t}+\beta_3QRET_{i,t}\times NEG_{i,t}+\beta_nControls_{i,t}+\epsilon_{i,t}
\end{split}
\end{equation}
Where TONE is number of net positive words per thousand total words and \textit{Controls} denotes a vector of control variables including firm size (SIZE), market-to-book ratio (MTB), leverage ratio (LEV) and all other regressors in \hyperref[eq2]{Equation 2}.

\hyperref[T2PB]{Table 2 Panel B} presents the regression result of Equation 4 (Column 1 and 2) and 5 (Column 3 and 4). Column 2 and 4 include firm and time fixed effects and standard errors are clustered at industry level identified by 4-digit SIC codes. Regression results are very similar (if not identical) between Column 1 and 3 and Column 2 and 4. In both senarios, $\beta_3$ is significantly positive, which suggests that firms tend to exaggerate more the impact of bad news comparing to good news, potentially due to litigation pressure. The significance of $\beta_1$ confirms the existance of tone management in response to good news, although it is not clear whether the commonly applied strategy is tone exaggeration or tone attenuation, as the sign of $\beta_1$ remains indeterminate.

Overall, the results demonstrate that 10-Q filings are generally longer, more consistent with news directionlly and less timelier in response to bad news comparing to good news. Moreover, 10-Q filings tend to exaggerate more the impact of bad news in comparison with good news. 

\subsection{Is 8-K narrative disclosure more responsive to bad news than good news?}
\hyperref[T3PA]{Table 3 Panel A} presents the regression result of \hyperref[eq3]{Equation 3}. Column 2, 4 and 6 include firm and time fixed effects and standard errors are clustered at 4-digit SIC code industry level. Our clustering approach yields 383 clusters in 8-K sample (approximately 638 observations per cluster on average). As predicted by H1, the coefficient of $\Delta$DRET$\times$NEG is significantly negative for NW, consistent with 8-K narratives being longer in response to bad news comparing to good news. Also, consistent with H2, the coefficient of $\Delta$DRET$\times$NEG is significantly positive for TONE, which suggests that 8-K narratives are more consistent with news in response to bad news comparing to good news. Notice that because 8-K filings are generally shorter and more standardized than 10-Q filings, the tone results obtained in 8-K sample may serve only as supplemental evidence on the faithful representation of narrative disclosure in general. Finally, in line with H3, the coefficient of QRET$\times$NEG is significantly positive for TLAG, which suggests that 8-K reporting time lag is shorter in response to bad news comparing to good news, i.e. 8-K filings respond to bad news in a timelier manner comparing to good news.

We perform three additional tests to assess the responsiveness of 8-K to good versus bad news from three other aspects, making use of the unique data structure of 8-K filings. First, we test whether firms report more 8-K items per day in response to bad news comparing to good news by taking NITEM as dependent variable in \hyperref[eq3]{Equation 3}. Second, we analyze whether firms are more likely to report more 8-K filings per day in response to bad news by estimating an ordered logistics model on N8K (N8K = 1, 2, 3, 4, 5). Last but not least, we restrict our 8-K sample to observations with reporting time lag less than or equal to four calendar days (TLAG = 0, 1, 2, 3, 4)\footnote{We construct this restricted 8-K sample because issuers that are subject to the reporting requirements of Section 13(a) and Section 15(d) of the Exchange Act must file required current reports on Form 8-K within four business days of a triggering event (\cite{secFinalRuleAdditional2004}). Therefore, 8-K filings reported within four days of news release are more likely to be related to the precedent news, as is regulated by law. Our sample selection criterion is more strict than the regulation for two reasons. First, while the regulation requires firms to file 8-K within four business days of a triggering event, we reduce this reporting deadline to four calendar days, which is always shorter or at most equal to four business day. Second, the regulation exempt 8-K filings subjet to item 7.01 and item 8.01 from the four business day reporting deadline, but our restricted sample still apply this reporting deadline to these two items. This more stringent sample selection criterion further ensures that 8-K filings in our restricted sample are indeed responding to precedent news. We repeat our main analysis of 8-K using the restricted sample, and the results (see \hyperref[oat2]{Table 2 of Online Appendix}) remain unchanged. }, and examine whether firms are more likely to report more promptly via 8-K in response to bad news by estimating an ordered logistics model on TLAG using the restricted sample. If the narrative disclosure 8-K is conservative, we expect firm to report more 8-K items and 8-K filings per day in response to bad news comparing to good news, which is reflected as significantly negative $\beta_3^{NITEM}$ and $\beta_3^{N8K}$ \footnote{As $\Delta$DRET is always negative when BN equals to 1, a negative $\beta_3$ makes the interaction term positive, which translates into more 8-K items or filings. Similar reasoning applies to the sign expectation for $\beta_3^{TLAG}$.}. Also, we expect 8-K filings to respond more promptly to bad news, which is reflected as significantly positive $\beta_3^{TLAG}$.

\hyperref[T3PB]{Table 3 Panel B} presents the regression results for three additional tests. Column 1 presents results of NITEM using an ordinary least square (OLS) regression with firm and time fixed effects and clustered standard errors at industry level identified by 4-digit SIC codes. Column 2 and 3 present results of ordered logistics models for N8K and TLAG respectively. Aligned with previous predictions, the coefficients of $\Delta DRET_{i,t-tlag}\times BN_{i,t-tlag}$ are significantly negative for NITEM and N8K, and is significantly positive for TLAG.

Overall, the results demonstrate that 8-K filings are generally longer, more consistent with news directionlly and more timelier in response to bad news comparing to good news. Moreover, firms tend to report more 8-K items and filings per day in response to bad news comparing to good news. 
\section{Robustness Checks}

\subsection{Alternative news proxy}
\section{Conclusions}

\end{spacing}

\newpage
\bibliographystyle{apa-good}
\bibliography{NC}

\newpage
%%%%%%%%%%%%%% Figure 1: 8-K Merging Process
\begin{figure}
	\caption{8-K Merging Process} \label{fig1}
	\begin{center}
		\includegraphics[scale=0.6]{../output/fig/fig1_matching.png}
	\end{center}
\end{figure}

Figuer 1 illustrates the 8-K sample matching process. We match every news day to its first posterior 8-K day, ignoring the successive 8-K days (if any) between two news days (Match-1), or in some cases the 8-K day coincides with news day (Match-2).

%%%%%%%%%%%%%% Figure 2: Sample Selection Process
% Table generated by Excel2LaTeX from sheet 'Fig2'
\begin{table}[htbp] \label{fig2}
  \centering
    \begin{tabular}{lr}
    \multicolumn{2}{c}{Figure 2: Sample Selection Process} \\
    \multicolumn{2}{c}{10-Q} \\
    Numer of observations: &  \\
    Retrieved from EDGAR & 575,579 \\
    After merging with COMP and CRSP data & 190,341 \\
    After merging with I\textbackslash{}B\textbackslash{}E\textbackslash{}S and segment data & 110,114 \\
    After dropping obs. with missing values in key variables and screening & \textbf{91,606} \\
      &  \\
    \multicolumn{2}{c}{8-K} \\
    Numer of observations: &  \\
    Retrieved from EDGAR & 1,489,626 \\
    After merging and matching with COMP and CRSP data & 390,698 \\
    After dropping obs. with missing values in key variables and screening & \textbf{244,401} \\
    After filtering obs. with TLAG smaller or equal to 4 (8-K restricted sample) & \textbf{62,300} \\
    \end{tabular}%
\end{table}%
 \label{fig2}

%%%%%%%%%%%%%% Figure 3: 8-K Item Distribution
\setcounter{figure}{2}
\begin{figure}[htbp]
	\begin{center}
		\caption{8-K Item Distribution} \label{fig3}
		\includegraphics[scale=0.5]{../output/fig/fig3_8-K_before.png}
		\includegraphics[scale=0.5]{../output/fig/fig3_8-K_after.png}
	\end{center}
\end{figure}

Figuer 3 illustrates the 8-K item distribution before (left) and after (right) May 23rd of 2004, respectively. Each share of pie chart shows the percentage of corporate events reported under each 8-K items. See 8-K item list in \hyperref[appd]{Appendix D}.

\newpage
%%%%%%%%%%%%%% Table 1: Sample Selection Process
\begin{landscape}
\begin{table}[htbp] \label{T1}
  \centering
    \begin{tabular}{lcc}
    \multicolumn{3}{c}{\textbf{Table 1. Sample Selection Process}} \\ 
      & &  \\
    \begin{comment}
    \multicolumn{3}{c}{10-Q} \\
    &   \multicolumn{2}{c}{Number of observations}\\
    & &  \\
    Retrieved from EDGAR & & 575,579 \\
    After merging with COMP and CRSP data & & 302,343 \\
    (-) Number of obs. from utility and financial firms & 82,498 & \\
    (-) Number of firm-quarters with missing values in SIC, SIZE, MTB, LEV, & & \\
    \hspace{5mm}or with non-positive total assets or book value of equity or common shares outstanding, & & \\
    \hspace{5mm}or with common share price less than \$1 & 25,959 & \\
    (-) Number of obs. with total words less than 1\% percentile (1,237 words) & 1,939 & \\
    (-) Number of obs. that contain negative or larger than 99\% TLAG & 1,855 & \\
    \bottomrule
    After dropping obs. with missing values in key variables and screening & & 190,092 \\
    After merging with I\textbackslash{}B\textbackslash{}E\textbackslash{}S and segment data & & 130,750 \\
    (-) Number of obs. that contain missing EARN, STD\_EARN and AF & 14,770 & \\
    \bottomrule
    Full 10-Q sample & & \textbf{115,980} \\
    & &  \\
    \end{comment}
    
     &   \multicolumn{2}{c}{Number of observations}\\
      & &  \\
    Retrieved from EDGAR & & 1,540,911 \\
    After matching with Compustat and CRSP data  & & 442,575 \\
    (-) Number of obs. from utility and financial firms & 112,729 & \\
    (-) Number of firm-quarters with missing values in SIC, SIZE, MTB, LEV, & & \\
    \hspace{5mm}or with non-positive total assets or book value of equity or common shares outstanding, & & \\
    \hspace{5mm}or with common share price less than \$1 & 46,865 & \\
    (-) Number of obs. with total words less than 1\% percentile (133 words) & 2,785 & \\
    (-) Number of obs. that are reversals of previous news day & 5,160 & \\
    (-) Number of obs. with negative or larger than 99\% percentile TLAG  & 154,861 & \\
    \bottomrule
    After dropping obs. with missing values in key variables and screening  & & 120,175 \\
    After merging with IBES and Compustat Segment data (Full 8-K sample) & & \textbf{83,464}  \\
    \begin{comment}
    	After dropping obs. with TLAG larger than four (five) days after (before) the 8-K reform &  & \\
    	(Restricted 8-K sample) &  & \textbf{40,700} 
    \end{comment}
    \end{tabular}%
\end{table}%

\end{landscape}

\newpage
%%%%%%%%%%%%%%%%%%%%%%%%% TABLE 2 Panel A
%\begin{landscape}
% Table generated by Excel2LaTeX from sheet 'T2PA '
\begin{table}[htbp] \label{T2PA}
  \centering
    \begin{tabular}{lcccccccc}
    \multicolumn{9}{c}{\textbf{Table 2. Panel A: Summary Statistics 10-Q}} \\
    \midrule
    \midrule
      & count & mean & std & min & 25\% & 50\% & 75\% & max \\
    \midrule
    \textbf{Textual Vars.} &   &   &   &   &   &   &   &  \\
    NW & 91607 & 9.020 & 0.757 & 7.120 & 8.506 & 9.086 & 9.547 & 13.544 \\
    nw & 91607 & 10937 & 10204 & 1236 & 4942 & 8829 & 13997 & 752337 \\
    TONE & 91607 & -8.921 & 7.236 & -63.579 & -13.127 & -7.875 & -3.866 & 24.215 \\
    TLAG & 91607 & 39 & 6 & 0 & 36 & 40 & 44 & 52 \\
    READ & 91607 & 38.161 & 42.160 & 14.580 & 17.840 & 20.210 & 39.660 & 262.515 \\
    ABTONE & 91607 & 0.000 & 6.919 & -55.759 & -3.946 & 0.939 & 4.777 & 34.181 \\
    \textbf{Financial Vars.} &   &   &   &   &   &   &   &  \\
    QRET & 91607 & 0.018 & 0.253 & -1.579 & -0.113 & 0.007 & 0.130 & 4.849 \\
    NEG & 91607 & 0.483 & 0.500 & 0 & 0 & 0 & 1 & 1 \\
    SIZE & 91607 & 6.447 & 1.776 & 2.002 & 5.175 & 6.317 & 7.563 & 11.206 \\
    MTB & 91607 & 3.515 & 4.009 & 0.288 & 1.485 & 2.343 & 3.902 & 30.902 \\
    LEV & 91607 & 0.192 & 0.182 & 0.000 & 0.011 & 0.162 & 0.315 & 0.724 \\
    AF & 91607 & 0.043 & 0.066 & -0.262 & 0.023 & 0.049 & 0.073 & 0.227 \\
    AFE & 91607 & -0.021 & 0.067 & -0.445 & -0.018 & -0.002 & 0.002 & 0.078 \\
    BUSSEG & 91607 & 0.859 & 0.447 & 0.693 & 0.693 & 0.693 & 0.693 & 2.773 \\
    GEOSEG & 91607 & 0.898 & 0.532 & 0.693 & 0.693 & 0.693 & 0.693 & 3.045 \\
    AGE & 91607 & 8.312 & 1.033 & 5.811 & 7.635 & 8.420 & 9.089 & 10.288 \\
    EARN & 91607 & 0.005 & 0.042 & -0.201 & 0.001 & 0.012 & 0.023 & 0.084 \\
    $\Delta$EARN & 91607 & 0.002 & 0.031 & -0.126 & -0.006 & 0.001 & 0.008 & 0.150 \\
    STD\_EARN & 91607 & 0.020 & 0.030 & 0.001 & 0.005 & 0.009 & 0.021 & 0.188 \\
    STD\_QRET & 91607 & 0.089 & 0.070 & 0.007 & 0.040 & 0.070 & 0.115 & 0.379 \\
    LOSS & 91607 & 0.242 & 0.429 & 0 & 0 & 0 & 0 & 1 \\
    \bottomrule
    \bottomrule
    \end{tabular}%
\end{table}%

%\end{landscape}

\newpage
%%%%%%%%%%%%%%%%%%%%%%%%% + TABLE 2 Panel B
%\begin{landscape}
% Table generated by Excel2LaTeX from sheet 'T2PB'
\begin{table}[H] \label{T2PB}
  \begin{center}
  	    \begin{tabular}{lcccccccc}
  		\multicolumn{9}{c}{\textbf{Table 2. Panel B: Summary Statistics 8-K}} \\
  		\midrule
  		\midrule
  		& count & mean & std & min & 25\% & 50\% & 75\% & max \\
  		\midrule
  		\textbf{Textual Variables} &   &   &   &   &   &   &   &  \\
  		NW & 119615 & 6.093 & 0.926 & 4.898 & 5.553 & 5.846 & 6.358 & 12.486 \\
  		nw & 119615 & 1339 & 6398 & 133 & 257 & 345 & 576 & 264704 \\
  		TONE & 119615 & -0.552 & 7.424 & -97.851 & -3.049 & 0.000 & 3.677 & 45.929 \\
  		TLAG & 119615 & 15 & 17 & 0 & 2 & 9 & 21 & 93 \\
  		N8K & 119615 & 1 & 0 & 1 & 1 & 1 & 1 & 4 \\
  		NITEM & 119615 & 2 & 1 & 1 & 2 & 2 & 2 & 16 \\
  		
  		\textbf{Financial Variables} &   &   &   &   &   &   &   &  \\
  		DRET & 119615 & 0.003 & 0.097 & -0.833 & -0.039 & -0.003 & 0.041 & 5.991 \\
  		$\Delta$DRET & 119615 & -0.018 & 0.187 & -9.062 & -0.121 & -0.050 & 0.100 & 5.989 \\
  		BN & 119615 & 0.542 & 0.498 & 0 & 0 & 1 & 1 & 1 \\
  		SIZE & 119615 & 6.326 & 1.993 & 2.122 & 4.896 & 6.262 & 7.664 & 11.379 \\
  		MTB & 119615 & 3.741 & 4.784 & 0.123 & 1.366 & 2.293 & 4.055 & 33.434 \\
  		LEV & 119615 & 0.204 & 0.192 & 0.000 & 0.012 & 0.171 & 0.334 & 0.735 \\
  		\bottomrule
  		\bottomrule
  	\end{tabular}%
  \end{center}
	\begin{footnotesize}
		\noindent Table 2 Panel A and Table 2 Panel B present the summary statistics of key variables in 10-Q and 8-K sample. READ and all financial variables except returns are winsorized at 1\% and 99\% level. See \hyperref[appb]{Appendix B} for variable definitions.
	\end{footnotesize}
\end{table}%
%\end{landscape}

%%%%%%%%%%%%%%%%%%%%%%%%% TABLE 2 Panel C
\newpage
%\begin{landscape}
% Table generated by Excel2LaTeX from sheet 'T2PC'
\begin{table}[H] \label{T2PC}
	\centering
	\begin{tabular}{lrrrrrrrrr}
		\multicolumn{10}{c}{\textbf{Table 2. Panel C: Correlation Matrix 8-K}} \\
		\midrule
		\midrule
		& \multicolumn{1}{c}{(1)} & \multicolumn{1}{c}{(2)} & \multicolumn{1}{c}{(3)} & \multicolumn{1}{c}{(4)} & \multicolumn{1}{c}{(5)} & \multicolumn{1}{c}{(6)} & \multicolumn{1}{c}{(7)} & \multicolumn{1}{c}{(8)} & \multicolumn{1}{c}{(9)}\\
		\midrule
		(1) tlag &  & -0.069 & 0.103 & -0.042 & -0.055 & 0.004 & -0.057 & -0.010 & -0.035 \\
		(2) TONE & -0.105 &  & -0.232 & -0.023 & -0.092 & -0.138 & 0.026 & 0.000 & 0.008 \\
		(3) nw & 0.116 & -0.415 &  & 0.017 & 0.004 & 0.308 & -0.025 & 0.017 & -0.006 \\
		(4) n8k & -0.058 & -0.044 & 0.213 &  & 0.437 & 0.209 & 0.066 & 0.015 & 0.007 \\
		(5) nitem & -0.096 & -0.114 & 0.197 & 0.302 &  & 0.461 & 0.091 & 0.008 & 0.004 \\
		(6) nexhibit & -0.069 & -0.112 & 0.175 & 0.203 & 0.614 &  & 0.101 & 0.015 & -0.007 \\
		(7) ngraph & -0.166 & 0.123 & -0.028 & 0.102 & 0.299 & 0.314 & & 0.004 & 0.003 \\
		(8) DRET & -0.021 & 0.005 & -0.003 & 0.004 & 0.004 & 0.006 & 0.008 &  & 0.700 \\
		(9) $\Delta$DRET & -0.048 & 0.013 & -0.015 & 0.003 & 0.007 & 0.003 & 0.017 & 0.765 &  \\
		(10) BN & 0.051 & -0.009 & 0.012 & -0.002 & -0.006 & -0.003 & -0.016 & -0.774 & -0.864 \\
		(11) SIZE & -0.095 & 0.068 & 0.020 & 0.026 & 0.009 & 0.003 & 0.091 & 0.021 & 0.070 \\
		(12) MTB & -0.006 & 0.029 & 0.038 & -0.001 & -0.015 & -0.023 & 0.007 & 0.007 & 0.008 \\
		(13) LEV & -0.046 & -0.037 & 0.075 & 0.028 & 0.029 & 0.046 & 0.073 & 0.015 & 0.024 \\
		(14) AF & -0.050 & 0.013 & -0.018 & 0.004 & 0.017 & 0.030 & 0.039 & -0.024 & 0.042 \\
		(15) AFE & -0.011 & 0.032 & -0.020 & 0.008 & 0.002 & -0.012 & 0.017 & 0.032 & 0.004 \\
		(16) BUSSEG & -0.070 & 0.095 & 0.035 & 0.027 & 0.044 & -0.010 & 0.200 & 0.004 & 0.021 \\
		(17) GEOSEG & -0.076 & 0.094 & 0.041 & 0.023 & 0.041 & -0.013 & 0.194 & 0.008 & 0.031 \\
		(18) EARN & -0.021 & 0.068 & -0.069 & -0.004 & -0.001 & -0.004 & -0.019 & 0.045 & 0.059 \\
		(19) STD\_QRET & 0.018 & -0.056 & 0.056 & -0.011 & -0.010 & -0.009 & -0.013 & -0.030 & -0.058 \\
		\bottomrule
		\bottomrule
	\end{tabular}%
\end{table}%
% Table generated by Excel2LaTeX from sheet 'T2PD'
\begin{table}[H]
  \begin{center}
  	\begin{tabular}{lrrrrrrrrrr}
  		\multicolumn{11}{c}{\textbf{Table 2. Panel C: Correlation Matrix 8-K (Continued) }} \\
  		\midrule
  		\midrule
  		& \multicolumn{1}{c}{(10)} & \multicolumn{1}{c}{(11)} & \multicolumn{1}{c}{(12)} & \multicolumn{1}{c}{(13)} & \multicolumn{1}{c}{(14)} & \multicolumn{1}{c}{(15)} & \multicolumn{1}{c}{(16)} & \multicolumn{1}{c}{(17)} & \multicolumn{1}{c}{(18)} & \multicolumn{1}{c}{(19)} \\
  		\midrule
  		(1) tlag & 0.039 & -0.075 & -0.004 & -0.039 & -0.012 & 0.001 & -0.061 & -0.062 & 0.005 & 0.003 \\
  		(2) TONE & -0.008 & 0.062 & 0.012 & -0.028 & -0.013 & 0.042 & 0.061 & 0.065 & 0.033 & -0.037 \\
  		(3) nw & 0.004 & -0.055 & 0.010 & 0.039 & 0.006 & -0.016 & -0.071 & -0.073 & -0.014 & 0.026 \\
  		(4) n8k & -0.003 & 0.025 & 0.000 & 0.028 & -0.001 & 0.005 & 0.027 & 0.020 & 0.002 & -0.008 \\
  		(5) nitem & -0.003 & -0.001 & -0.007 & 0.032 & 0.001 & -0.003 & 0.036 & 0.026 & -0.005 & 0.002 \\
  		(6) nexhibit & 0.006 & -0.006 & -0.002 & 0.053 & 0.004 & -0.015 & -0.010 & -0.019 & -0.025 & 0.021 \\
  		(7) ngraph & -0.004 & 0.039 & 0.014 & 0.045 & -0.003 & 0.003 & 0.079 & 0.073 & -0.005 & -0.004 \\
  		(8) DRET & -0.587 & -0.014 & 0.004 & 0.003 & 0.006 & 0.009 & -0.007 & -0.006 & 0.017 & 0.005 \\
  		(9) $\Delta$DRET & -0.765 & 0.057 & -0.008 & 0.013 & 0.062 & 0.001 & 0.018 & 0.024 & 0.062 & -0.055 \\
  		(10) BN & & -0.032 & 0.000 & -0.011 & -0.027 & 0.002 & -0.015 & -0.020 & -0.031 & 0.027 \\
  		(11) SIZE & -0.031 &  & 0.207 & 0.170 & 0.114 & 0.188 & 0.240 & 0.283 & 0.313 & -0.259 \\
  		(12) MTB & -0.004 & 0.346 &  & 0.104 & -0.152 & 0.077 & 0.006 & 0.028 & -0.055 & 0.129 \\
  		(13) LEV & -0.014 & 0.219 & -0.035 &  & 0.144 & -0.071 & 0.089 & 0.054 & 0.070 & -0.115 \\
  		(14) AF & -0.028 & 0.030 & -0.402 & 0.226 &  & -0.184 & 0.058 & 0.080 & 0.375 & -0.203 \\
  		(15) AFE & -0.003 & 0.133 & 0.122 & -0.061 & -0.218 &  & 0.053 & 0.054 & 0.193 & -0.110 \\
  		(16) BUSSEG & -0.013 & 0.224 & 0.066 & 0.074 & 0.053 & 0.028 &  & 0.644 & 0.081 & -0.091 \\
  		(17) GEOSEG & -0.021 & 0.289 & 0.072 & 0.083 & 0.090 & 0.028 & 0.715 &  & 0.105 & -0.111 \\
  		(18) EARN & -0.030 & 0.349 & 0.226 & -0.032 & 0.113 & 0.227 & 0.027 & 0.060 &  & -0.470 \\
  		(19) STD\_EARN & 0.028 & -0.338 & 0.058 & -0.177 & -0.133 & -0.066 & -0.075 & -0.101 & -0.335 & \\
  		\bottomrule
  		\bottomrule
  	\end{tabular}%
  \end{center}
	\begin{footnotesize}
		\noindent Table 2 Panel C presents the correlation matrix of key variables in 8-K sample. Pearson (Spearman) correlations are exhibited above (below) the diagonal. See \hyperref[appc]{Appendix C} for variable definitions. All financial variables except returns are winsorized at 1\% and 99\% level. 
	\end{footnotesize}
\end{table}%
%\end{landscape}

%%%%%%%%%%%%%%%%%%%%%%%%% TABLE 3
\newpage
\begin{landscape}
% Table generated by Excel2LaTeX from sheet 'T3'
\begin{table}[H] \label{T3}
	\begin{center}
		\tabcolsep=0.11cm
		\begin{tabular}{lcccc}
			\multicolumn{5}{c}{\textbf{Table 3. Is 8-K Narrative Disclosure Conservative?}} \\
			\toprule
			\toprule
			& (1) & (2) & (3) & (4) \\
			Dep. Variables & TLAG & TLAG & TONE & TONE \\
			\midrule
			%&   &   &   &  \\
			$\Delta$DRET & 1.913*** & 2.007*** & -1.744*** & -1.171** \\
			& (11.44) & (10.83) & (-2.86) & (-2.07) \\
			BN & -0.021 & -0.026 & -0.120* & -0.125 \\
			& (-1.13) & (-1.15) & (-1.71) & (-1.64) \\
			\rowcolor[rgb]{ .906,  .902,  .902} \textit{(Pred. Sign)} & (-) & (-) & (+) & (+) \\
			\rowcolor[rgb]{ .906,  .902,  .902} $\Delta$DRET$\times$BN & -2.966*** & -3.182*** & 2.893*** & 1.849** \\
			\rowcolor[rgb]{ .906,  .902,  .902}   & (-8.42) & (-7.55) & (2.70) & (1.97) \\
			SIZE &   & 0.051*** &   & 0.115* \\
			&   & (4.56) &   & (1.76) \\
			MTB &   & 0.002 &   & -0.009 \\
			&   & (1.22) &   & (-1.08) \\
			LEV &   & -0.007 &   & -0.592 \\
			&   & (-0.11) &   & (-1.45) \\
			EARN &   & -0.231* &   & 3.059** \\
			&   & (-1.70) &   & (2.51) \\
			STD\_EARN &   & -0.165 &   & -2.705**\\
			&   & (-0.72) &   & (-2.17)\\
			BUSSEG &   & -0.028 &   & -0.015 \\
			&   & (-1.52) &   & (-0.12) \\
			GEOSEG &   & 0.016 &   & 0.131 \\
			&   & (0.91) &   & (1.18) \\
			AF &   & 0.020 &   & -0.019 \\
			&   & (0.20) &   & (-0.04)\\
			AFE &   & 0.045 &   & 1.713**  \\
			&   & (0.41) &   & (2.57) \\
			Constant & -2.816*** & -3.150*** & -5.598** & -5.921*** \\
			& (-10.16) & (-10.85) & (-2.47) & (-2.71) \\
			&   &   &   &  \\
			Observations & 83,464 & 75,360 & 83,464 & 75,360 \\
			Adjusted R-squared & 0.131 & 0.132 & 0.151 & 0.147 \\
			\bottomrule
			\bottomrule
		\end{tabular}%
	\end{center}
\end{table}%
% Table generated by Excel2LaTeX from sheet 'T3'
\begin{table}
	\begin{center}
		\tabcolsep=0.11cm
		\begin{tabular}{lcccccccccc}
			\multicolumn{11}{c}{\textbf{Table 3. Is 8-K Narrative Disclosure Conservative? (Continued)}} \\
			\toprule
			\toprule
			 & (5) & (6) & (7) & (8) & (9) & (10) & (11) & (12) & (13) & (14) \\
			Dep. Variables & NW & NW & N8K & N8K & NITEM & NITEM & NEXHIBIT & NEXHIBIT & NGRAPH & NGRAPH \\
			\midrule
			%&   &   &   &   &   &  &   &   &   &  \\
			$\Delta$DRET & -0.086* & -0.042 & -0.034*** & -0.039*** & -0.075*** & -0.079*** & -0.105*** & -0.110*** & -0.151*** & -0.212*** \\
			 & (-1.78) & (-0.71) & (-3.43) & (-3.64) & (-3.34) & (-3.71) & (-2.99) & (-3.04) & (-3.03) & (-5.02) \\
			BN & -0.015** & -0.015** & -0.002** & -0.003** & -0.004 & -0.004 & -0.003 & -0.002 & 0.001 & -0.001 \\
			& (-2.04) & (-2.19) & (-2.24) & (-2.43) & (-1.13) & (-1.05) & (-0.53) & (-0.36) & (0.16) & (-0.13) \\
			\rowcolor[rgb]{ .906,  .902,  .902} \textit{(Pred. Sign)} & (+) & (+) & (+) & (+) & (+) & (+) & (+) & (+) & (+) & (+) \\
			\rowcolor[rgb]{ .906,  .902,  .902} $\Delta$DRET$\times$BN & 0.127** & 0.033 & 0.046*** & 0.051*** & 0.099*** & 0.104*** & 0.176*** & 0.175*** & 0.221*** & 0.298*** \\
			\rowcolor[rgb]{ .906,  .902,  .902}   & (2.02) & (0.40) & (3.34) & (3.36) & (2.84) & (3.06) & (3.46) & (3.32) & (4.06) & (5.71) \\
			SIZE &   & 0.018** &   & -0.001 &   & -0.002 &   & -0.003 &   & -0.004 \\
			&   & (2.13) &   & (-0.84) &   & (-0.70) &   & (-0.58) &   & (-0.60) \\
			MTB &   & -0.002 &   & -0.000 &   & -0.000 &   & -0.002*** &   & -0.003*** \\
			&    & (-1.30) &   & (-0.43) &   & (-0.96) &   & (-2.88) &   & (-2.82) \\
			LEV &  & -0.027 &   & -0.008** &   & -0.021* &   & -0.007 &   & 0.005 \\
			&   & (-0.65) &   & (-2.43) &   & (-1.68) &   & (-0.32) &   & (0.11) \\
			EARN &    & 0.406*** &   & -0.001 &   & 0.069* &   & 0.113* &   & -0.064 \\
			&   & (3.84) &   & (-0.17) &   & (1.82) &   & (1.96) &   & (-0.87) \\
			STD\_EARN &    & -0.331*** &   & -0.004 &   & -0.098** &   & -0.112 &   & 0.243* \\
			&   & (-2.75) &   & (-0.41) &   & (-2.11) &   & (-1.29) &   & (1.71) \\
			BUSSEG &    & -0.008 &   & 0.000 &   & 0.002 &   & 0.003 &   & -0.005 \\
			&    & (-0.71) &   & (0.21) &   & (0.39) &   & (0.42) &   & (-0.31) \\
			GEOSEG &    & 0.007 &   & 0.002** &   & -0.001 &   & -0.011* &   & -0.011 \\
			&    & (0.67) &   & (2.27) &   & (-0.36) &   & (-1.82) &   & (-0.76) \\
			AF &    & -0.026 &   & 0.004 &   & 0.015 &   & 0.029 &   & -0.075 \\
			&    & (-0.47) &   & (0.52) &   & (0.74) &   & (0.66) &   & (-1.56) \\
			AFE &   & -0.044 &   & -0.009 &   & -0.022 &   & -0.091** &   & -0.164** \\
			&   & (-0.69) &   & (-1.36) &   & (-0.86) &   & (-2.44) &   & (-2.37) \\
			Constant & -7.291*** & -7.295*** & -0.688*** & -0.684*** & -0.872*** & -0.843*** & -0.506*** & -0.459*** & 0.051 & 0.096 \\
			&  (-27.57) & (-28.75) & (-190.40) & (-120.16) & (-25.72) & (-22.63) & (-4.91) & (-4.26) & (1.01) & (1.44) \\
			&   &   &   &   &   &   &   &   &   &  \\
			Observations & 83,464 & 75,360 & 83,464 & 75,360 & 83,464 & 75,360 & 83,464 & 75,360 & 83,464 & 75,360 \\
			Adjusted R-squared & 0.443 & 0.427 & 0.021 & 0.024 & 0.139 & 0.142 & 0.109 & 0.107 & 0.256 & 0.263 \\
			\bottomrule
			\bottomrule
		\end{tabular}%
	\end{center}
		\begin{footnotesize}
			\setcounter{equation}{0}
			\begin{equation}
				TEX_{i,t}=\beta_0+\beta_1\Delta DRET_{i,t-tlag}+\beta_2BN_{i,t-tlag}+\beta_3\Delta DRET_{i,t-tlag}\times 	BN_{i,t-tlag}+\sum\beta_nCONTROLS_{i,t}+\epsilon_{i,t}
			\end{equation}
			
			\noindent Table 3 presents the regression results of Equation (1). TEX represents a vector of textual properties. CONTROLS denotes a vector of control variables. See \hyperref[appc]{Appendix C} for variable definitions. All financial variables except returns are winsorized at 1\% and 99\% level. All regressions include firm and year-month fixed effects and standard errors are clustered at industry level identified by 4-digit SIC codes. ***, ** and * indicate significance at the 1\%, 5\% and 10\% levels in a two-tailed test.
		\end{footnotesize}
\end{table}%
\end{landscape}

%%%%%%%%%%%%%%%%%%%%%%%%% TABLE 4
\newpage
\begin{landscape}
% Table generated by Excel2LaTeX from sheet 'T3'
\begin{table}[H] \label{T4}
	\begin{center}
		\tabcolsep=0.11cm
		\begin{tabular}{lcccc}
			\multicolumn{5}{c}{\textbf{Table 4. Narrative Conservatism and Conditional Conservatism}} \\
			\toprule
			\toprule
			Dep. Variables & \multicolumn{2}{c}{TLAG} & \multicolumn{2}{c}{TONE} \\
			\cmidrule{2-5}
			& (1) & (2) & (3) & (4) \\
			CONS. & LOW & HIGH & LOW & HIGH \\
			\midrule
			%&   &   &   &  \\
			$\Delta$DRET & 2.647*** & 1.775*** & -2.473** & -0.206 \\
			& (9.71) & (11.56) & (-2.33) & (-0.31) \\
			BN & -0.051* & -0.009 & -0.186 & -0.079 \\
			& (-1.91) & (-0.33) & (-1.54) & (-0.80) \\
			\rowcolor[rgb]{ .906,  .902,  .902} \textit{(Pred. Sign)} & (-) & (-) & (+) & (+) \\
			\rowcolor[rgb]{ .906,  .902,  .902} $\Delta$DRET$\times$BN& -4.639*** & -2.687*** & 3.553** & 0.549 \\
			\rowcolor[rgb]{ .906,  .902,  .902} & (-8.75) & (-8.84) & (2.17) & (0.54) \\
			SIZE & 0.087*** & 0.030** & 0.092 & 0.101 \\
			& (4.69) & (2.12) & (0.92) & (1.07) \\
			MTB & -0.000 & 0.003 & 0.018 & -0.005 \\
			& (-0.09) & (1.09) & (0.81) & (-0.38) \\
			LEV & -0.002 & -0.082 & -0.937* & -0.581 \\
			& (-0.02) & (-0.94) & (-1.81) & (-0.90) \\
			EARN & 0.031 & -0.306 & 1.008 & 3.218** \\
			& (0.13) & (-1.61) & (0.46) & (2.53) \\
			STD\_EARN & -0.041 & -0.030 & -2.801 & -3.046*** \\
			& (-0.13) & (-0.10) & (-1.19) & (-2.65) \\
			BUSSEG & -0.026 & -0.025 & -0.059 & -0.046 \\
			& (-1.14) & (-0.78) & (-0.36) & (-0.23) \\
			GEOSEG & 0.034 & 0.004 & 0.031 & 0.253 \\
			& (1.55) & (0.18) & (0.22) & (1.56) \\
			AF & 0.153 & -0.028 & 0.022 & 0.067 \\
			& (1.22) & (-0.22) & (0.03) & (0.10) \\
			AFE & 0.059 & 0.032 & 2.629*** & 0.810 \\
			& (0.34) & (0.21) & (2.75) & (0.83) \\
			Constant & -2.845*** & -2.492*** & -0.198 & -0.826 \\
			& (-17.51) & (-23.87) & (-0.25) & (-1.38) \\
			&   &   &   &  \\
			Observations & 38,881 & 35,134 & 38,881 & 35,134 \\
			Adjusted R-squared & 0.139 & 0.120 & 0.133 & 0.154 \\
			\bottomrule
			\bottomrule
		\end{tabular}%
	\end{center}
\end{table}%
% Table generated by Excel2LaTeX from sheet 'T3'
\begin{table}[H]
	\begin{center}
		\tabcolsep=0.11cm
		\begin{tabular}{lcccccccccc}
			\multicolumn{11}{c}{\textbf{Table 4. Narrative Conservatism and Conditional Conservatism (Continued)}} \\
			\toprule
			\toprule
			Dep. Variables & \multicolumn{2}{c}{NW} & \multicolumn{2}{c}{N8K} & \multicolumn{2}{c}{NITEM} & \multicolumn{2}{c}{NEXHIBIT} & \multicolumn{2}{c}{NGRAPH} \\
			\cmidrule{2-11}
			& (5) & (6) & (7) & (8) & (9) & (10) & (11) & (12) & (13) & (14) \\
			CONS. & LOW & HIGH & LOW & HIGH & LOW & HIGH & LOW & HIGH& LOW & HIGH\\
			\midrule
			%&   &   &   &   &   &  &   &   &   &  \\
			$\Delta$DRET & -0.090 & -0.015 & -0.047*** & -0.042*** & -0.104*** & -0.061** & -0.171*** & -0.078* & -0.304*** & -0.168*** \\
			& (-0.89) & (-0.21) & (-4.04) & (-2.73) & (-2.93) & (-2.30) & (-3.12) & (-1.68) & (-2.93) & (-3.34) \\
			BN & -0.012 & -0.022** & -0.002 & -0.004** & -0.006 & -0.002 & -0.003 & 0.001 & -0.011 & 0.002 \\
			& (-1.00) & (-2.13) & (-1.31) & (-2.57) & (-1.15) & (-0.32) & (-0.41) & (0.08) & (-0.78) & (0.16) \\
			\rowcolor[rgb]{ .906,  .902,  .902} \textit{(Pred. Sign)} & (+) & (+) & (+) & (+) & (+) & (+) & (+) & (+) & (+) & (+) \\
			\rowcolor[rgb]{ .906,  .902,  .902} $\Delta$DRET$\times$BN & 0.095 & -0.025 & 0.066*** & 0.052** & 0.127*** & 0.085** & 0.281*** & 0.130** & 0.391*** & 0.244*** \\
			\rowcolor[rgb]{ .906,  .902,  .902} & (0.66) & (-0.25) & (4.01) & (2.51) & (2.89) & (2.00) & (3.62) & (2.14) & (3.20) & (4.30) \\
			SIZE & 0.024** & 0.013 & -0.001 & -0.000 & -0.004 & 0.003 & -0.012* & 0.011 & -0.003 & -0.002 \\
			& (2.10) & (1.29) & (-0.79) & (-0.19) & (-1.26) & (0.86) & (-1.95) & (1.54) & (-0.29) & (-0.20) \\
			MTB & -0.001 & -0.003 & -0.000 & -0.000 & -0.000 & -0.001 & -0.000 & -0.003*** & 0.001 & -0.004** \\
			& (-0.47) & (-1.62) & (-0.07) & (-0.87) & (-0.16) & (-1.54) & (-0.29) & (-3.16) & (0.41) & (-2.40) \\
			LEV & -0.074 & 0.035 & -0.006 & -0.010 & -0.014 & -0.016 & -0.019 & 0.005 & 0.054 & -0.047 \\
			& (-1.30) & (0.63) & (-1.03) & (-1.63) & (-0.66) & (-0.94) & (-0.60) & (0.16) & (0.76) & (-0.97) \\
			EARN & 0.263 & 0.486*** & 0.008 & 0.001 & 0.097 & 0.051 & 0.007 & 0.152** & 0.003 & -0.074 \\
			& (1.33) & (4.91) & (0.33) & (0.06) & (1.58) & (1.30) & (0.07) & (2.41) & (0.02) & (-0.89) \\
			STD\_EARN & -0.155 & -0.335** & 0.021 & -0.021* & 0.049 & -0.162*** & 0.095 & -0.186** & 0.544** & 0.077 \\
			& (-0.89) & (-2.32) & (0.88) & (-1.76) & (0.67) & (-3.04) & (0.61) & (-1.98) & (2.07) & (0.48) \\
			BUSSEG & -0.006 & -0.015 & -0.000 & 0.001 & -0.003 & 0.007 & -0.001 & 0.002 & -0.017 & 0.039* \\
			& (-0.45) & (-0.82) & (-0.19) & (0.75) & (-0.51) & (1.15) & (-0.12) & (0.18) & (-0.85) & (1.70) \\
			GEOSEG & 0.019 & 0.010 & 0.002 & 0.003* & 0.002 & -0.002 & -0.006 & -0.006 & 0.005 & -0.036* \\
			& (1.59) & (0.67) & (1.41) & (1.85) & (0.37) & (-0.29) & (-0.67) & (-0.63) & (0.26) & (-1.78) \\
			AF & -0.013 & -0.024 & 0.001 & 0.010 & 0.018 & 0.006 & 0.053 & -0.009 & -0.100 & -0.057 \\
			& (-0.16) & (-0.43) & (0.08) & (0.96) & (0.50) & (0.32) & (1.02) & (-0.17) & (-1.26) & (-0.86) \\
			AFE & -0.020 & -0.085 & -0.011 & -0.009 & -0.016 & -0.017 & -0.141** & -0.081 & -0.140 & -0.142* \\
			& (-0.23) & (-0.88) & (-1.09) & (-1.05) & (-0.47) & (-0.48) & (-2.53) & (-1.58) & (-1.28) & (-1.88) \\
			Constant & -6.223*** & -6.067*** & -0.699*** & -0.700*** & -1.058*** & -1.089*** & -0.563*** & -0.684*** & -0.442*** & -0.330*** \\
			& (-66.29) & (-94.80) & (-69.04) & (-110.27) & (-35.28) & (-51.80) & (-11.01) & (-14.53) & (-4.52) & (-6.45) \\
			&   &   &   &   &   &   &   &   &   &  \\
			Observations & 38,881 & 35,134 & 38,881 & 35,134 & 38,881 & 35,134 & 38,881 & 35,134 & 38,881 & 35,134 \\
			Adjusted R-squared & 0.362 & 0.437 & 0.029 & 0.029 & 0.133 & 0.164 & 0.097 & 0.117 & 0.267 & 0.272 \\
			\bottomrule
			\bottomrule
		\end{tabular}%
	\end{center}
		\begin{footnotesize}
			\setcounter{equation}{0}
			\begin{equation}
				TEX_{i,t}=\beta_0+\beta_1\Delta DRET_{i,t-tlag}+\beta_2BN_{i,t-tlag}+\beta_3\Delta DRET_{i,t-tlag}\times 	BN_{i,t-tlag}+\sum\beta_nCONTROLS_{i,t}+\epsilon_{i,t}
			\end{equation}
			
			\noindent Table 4 presents the regression results of Equation (1) across high and low conditional conservatism subsamples. TEX represents a vector of textual properties. CONTROLS denotes a vector of control variables. See \hyperref[appc]{Appendix C} for variable definitions. All financial variables except returns are winsorized at 1\% and 99\% level. All regressions include firm and year-month fixed effects and standard errors are clustered at industry level identified by 4-digit SIC codes. ***, ** and * indicate significance at the 1\%, 5\% and 10\% levels in a two-tailed test.
		\end{footnotesize}
\end{table}%
\end{landscape}

%%%%%%%%%%%%%%%%%%%%%%%%% TABLE 5
\newpage
\begin{landscape}
	% Table generated by Excel2LaTeX from sheet 'T3'
\begin{table}[H] \label{T5}
	\begin{center}
		\tabcolsep=0.11cm
		\begin{tabular}{lcccc}
			\multicolumn{5}{c}{\textbf{Table 5. Narrative Conservatism and Unconditional Conservatism}} \\
			\toprule
			\toprule
			Dep. Variables & \multicolumn{2}{c}{TLAG} & \multicolumn{2}{c}{TONE} \\
			\cmidrule{2-5}
			& (1) & (2) & (3) & (4) \\
			\midrule
			Panel A: Intangible Assets & LOW & HIGH & LOW & HIGH \\
			\midrule
			%&   &   &   &  \\
			$\Delta$DRET & 1.975*** & 3.026*** & -1.205 & -2.647** \\
			& (11.64) & (9.89) & (-1.23) & (-2.07) \\
			BN & -0.032 & -0.130*** & -0.193 & -0.060 \\
			& (-1.13) & (-4.26) & (-1.17) & (-0.38) \\
			\rowcolor[rgb]{ .906,  .902,  .902} \textit{(Pred. Sign)} & (-) & (-) & (+) & (+) \\
			\rowcolor[rgb]{ .906,  .902,  .902} $\Delta$DRET$\times$BN & -3.181*** & -6.326*** & 1.044 & 5.773** \\
			\rowcolor[rgb]{ .906,  .902,  .902} & (-10.61) & (-13.28) & (0.82) & (2.42) \\
			Constant & -3.065*** & -3.588*** & -0.478 & -3.469 \\
			& (-3.58) & (-6.35) & (-0.06) & (-1.18) \\
			&   &   &   &  \\
			Observations & 29,136 & 31,806 & 29,136 & 31,806 \\
			Adjusted R-squared & 0.118 & 0.146 & 0.132 & 0.123 \\
			\midrule
			Panel B: R\&D Expenses & LOW & HIGH & LOW & HIGH \\
			\midrule
			%&   &   &   &  \\
			$\Delta$DRET & 1.651*** & 1.946*** & -0.209 & -1.566 \\
			& (6.85) & (7.52) & (-0.30) & (-1.33) \\
			BN & 0.011 & -0.025 & -0.149 & -0.058 \\
			& (0.26) & (-0.91) & (-1.20) & (-0.50) \\
			\rowcolor[rgb]{ .906,  .902,  .902} \textit{(Pred. Sign)} & (-) & (-) & (+) & (+) \\
			\rowcolor[rgb]{ .906,  .902,  .902} $\Delta$DRET$\times$BN & -2.426*** & -2.983*** & -0.325 & 2.432* \\
			\rowcolor[rgb]{ .906,  .902,  .902} & (-5.65) & (-7.03) & (-0.39) & (1.66) \\
			Constant & -2.520*** & -2.678*** & -1.751 & -5.212 \\
			& (-4.66) & (-5.07) & (-0.25) & (-1.43) \\
			&   &   &   &  \\
			Observations & 19,740 & 22,608 & 19,740 & 22,608 \\
			Adjusted R-squared & 0.106 & 0.143 & 0.184 & 0.115 \\
			\bottomrule
			\bottomrule
		\end{tabular}%
	\end{center}
\end{table}%
	% Table generated by Excel2LaTeX from sheet 'T3'
\begin{table}[H]
	\begin{center}
		\tabcolsep=0.11cm
		\begin{tabular}{lcccccccccc}
			\multicolumn{11}{c}{\textbf{Table 5. Narrative Conservatism and Unconditional Conservatism (Continued)}} \\
			\toprule
			\toprule
			Dep. Variables & \multicolumn{2}{c}{NW} & \multicolumn{2}{c}{N8K} & \multicolumn{2}{c}{NITEM} & \multicolumn{2}{c}{NEXHIBIT} & \multicolumn{2}{c}{NGRAPH} \\
			\cmidrule{2-11}
			& (5) & (6) & (7) & (8) & (9) & (10) & (11) & (12) & (13) & (14) \\
			\midrule
			Panel A: Intangible Assets & LOW & HIGH & LOW & HIGH & LOW & HIGH & LOW & HIGH & LOW & HIGH \\
			\midrule
			%&   &   &   &   &   &  &   &   &   &  \\
			$\Delta$DRET & 0.041 & -0.142 & -0.033*** & -0.042*** & -0.098*** & -0.053 & -0.087 & -0.195*** & -0.148** & -0.467*** \\
			& (0.48) & (-1.02) & (-2.74) & (-2.90) & (-2.62) & (-1.25) & (-1.54) & (-2.95) & (-2.17) & (-3.88) \\
			BN & -0.002 & -0.029* & -0.002 & -0.000 & -0.007 & -0.003 & -0.000 & -0.007 & -0.001 & -0.018 \\
			& (-0.13) & (-1.92) & (-1.10) & (-0.19) & (-0.83) & (-0.45) & (-0.04) & (-0.80) & (-0.07) & (-1.17) \\
			\rowcolor[rgb]{ .906,  .902,  .902} \textit{(Pred. Sign)} & (+) & (+) & (+) & (+) & (+) & (+) & (+) & (+) & (+) & (+) \\
			\rowcolor[rgb]{ .906,  .902,  .902} $\Delta$DRET$\times$BN & -0.042 & 0.059 & 0.049*** & 0.076*** & 0.118* & 0.048 & 0.135 & 0.272*** & 0.219** & 0.622*** \\
			\rowcolor[rgb]{ .906,  .902,  .902} & (-0.34) & (0.32) & (3.01) & (3.48) & (1.95) & (0.76) & (1.51) & (2.72) & (2.14) & (3.18) \\
			Constant & -6.439*** & -7.127*** & -0.692*** & -0.692*** & -0.745*** & -0.890*** & -0.314 & -0.456*** & 0.156* & -0.173 \\
			& (-20.69) & (-21.31) & (-94.98) & (-64.07) & (-11.47) & (-14.91) & (-1.50) & (-2.71) & (1.79) & (-1.36) \\
			&   &   &   &   &   &   &   &   &   &  \\
			Observations & 29,136 & 31,806 & 29,136 & 31,806 & 29,136 & 31,806 & 29,136 & 31,806 & 29,136 & 31,806 \\
			Adjusted R-squared & 0.385 & 0.315 & 0.022 & 0.036 & 0.144 & 0.133 & 0.113 & 0.088 & 0.257 & 0.282 \\
			\midrule
			Panel B: R\&D Expenses & LOW & HIGH & LOW & HIGH & LOW & HIGH & LOW & HIGH & LOW & HIGH\\
			\midrule
			%&   &   &   &   &   &   &   &   &   &  \\
			$\Delta$DRET & -0.068 & 0.005 & -0.054*** & -0.031** & -0.120*** & -0.007 & -0.137** & -0.047 & -0.050 & -0.348*** \\
			& (-0.69) & (0.06) & (-2.60) & (-2.55) & (-3.08) & (-0.23) & (-1.98) & (-1.00) & (-0.63) & (-4.77) \\
			BN & -0.017 & -0.005 & -0.008*** & -0.001 & -0.006 & 0.005 & -0.003 & 0.013 & 0.011 & -0.020 \\
			& (-1.23) & (-0.44) & (-3.60) & (-0.38) & (-0.84) & (1.02) & (-0.23) & (1.59) & (0.56) & (-1.53) \\
			\rowcolor[rgb]{ .906,  .902,  .902} \textit{(Pred. Sign)} & (+) & (+) & (+) & (+) & (+) & (+) & (+) & (+) & (+) & (+) \\
			\rowcolor[rgb]{ .906,  .902,  .902} $\Delta$DRET$\times$BN & 0.032 & -0.010 & 0.054* & 0.049*** & 0.137** & 0.043 & 0.177** & 0.197*** & 0.128* & 0.388*** \\
			\rowcolor[rgb]{ .906,  .902,  .902} & (0.24) & (-0.08) & (1.95) & (4.22) & (2.03) & (1.03) & (2.22) & (3.08) & (1.71) & (4.30) \\
			Constant & -7.250*** & -7.660*** & -0.657*** & -0.676*** & -0.795*** & -0.852*** & -0.476*** & -0.400** & 0.394** & -0.109 \\
			& (-8.77) & (-18.41) & (-25.93) & (-63.28) & (-10.34) & (-12.76) & (-3.74) & (-2.21) & (2.07) & (-1.01) \\
			&   &   &   &   &   &   &   &   &   &  \\
			Observations & 19,740 & 22,608 & 19,740 & 22,608 & 19,740 & 22,608 & 19,740 & 22,608 & 19,740 & 22,608 \\
			Adjusted R-squared & 0.491 & 0.355 & 0.005 & 0.009 & 0.156 & 0.130 & 0.129 & 0.092 & 0.255 & 0.253 \\
			\bottomrule
			\bottomrule
		\end{tabular}%
	\end{center}
		\begin{footnotesize}
			\setcounter{equation}{0}
			\begin{equation}
				TEX_{i,t}=\beta_0+\beta_1\Delta DRET_{i,t-tlag}+\beta_2BN_{i,t-tlag}+\beta_3\Delta DRET_{i,t-tlag}\times 	BN_{i,t-tlag}+\sum\beta_nCONTROLS_{i,t}+\epsilon_{i,t}
			\end{equation}
			
			\noindent Table 5 presents the regression results of Equation (1) across high and low intangible assets and R\&D expenses subsamples. TEX represents a vector of textual properties. CONTROLS denotes a vector of control variables. See \hyperref[appc]{Appendix C} for variable definitions. All financial variables except returns are winsorized at 1\% and 99\% level. All regressions include full set of control variables, firm and year-month fixed effects. Standard errors are clustered at industry level identified by 4-digit SIC codes. ***, ** and * indicate significance at the 1\%, 5\% and 10\% levels in a two-tailed test.
		\end{footnotesize}
\end{table}%
\end{landscape}

%%%%%%%%%%%%%%%%%%%%%%%%% TABLE 6 Panel A
\newpage
%\begin{landscape}
% Table generated by Excel2LaTeX from sheet 'T6'
\begin{table}[H]	\label{T6PA}%
	\begin{center}
		\begin{tabular}{lcccc}
			\multicolumn{5}{c}{\textbf{Table 6. Panel A. Narrative Conservatism in Quarterly Reports}} \\
			\midrule
			\midrule
			& (1) & (2) & (3) & (4) \\
			Dep. Variables & TONE & TONE & NW & NW \\
			\midrule
			%&   &   &   &  \\
			QRET & -0.371*** & 0.095 & -0.039*** & -0.040*** \\
			& (-2.78) & (0.69) & (-3.54) & (-3.54) \\
			NEG & -0.077 & -0.075 & -0.004 & -0.005 \\
			& (-1.59) & (-1.52) & (-0.95) & (-1.08) \\
			\rowcolor[rgb]{ .906,  .902,  .902} \textit{(Pred. Sign)} & (+) & (+) & (+) & (+) \\
			\rowcolor[rgb]{ .906,  .902,  .902} QRET$\times$NEG & 2.274*** & 1.191*** & 0.140*** & 0.094*** \\
			\rowcolor[rgb]{ .906,  .902,  .902} & (8.19) & (5.20) & (6.56) & (5.12) \\
			SIZE &   & 0.540*** &   & -0.027*** \\
			&   & (6.36) &   & (-3.25) \\
			MTB &   & 0.046*** &   & 0.005*** \\
			&   & (3.79) &   & (5.18) \\
			LEV &   & -1.212** &   & -0.293*** \\
			&   & (-2.48) &   & (-10.11) \\
			EARN &   & 14.674*** &   & 0.635*** \\
			&   & (5.54) &   & (3.80) \\
			STD\_EARN &   & -7.233*** &   & -0.654*** \\
			&   & (-4.68) &   & (-6.85) \\
			BUSSEG &   & 0.468** &   & -0.019 \\
			&   & (2.22) &   & (-1.50) \\
			GEOSEG &   & 0.319* &   & 0.020* \\
			&   & (1.82) &   & (1.81) \\
			AF &   & -3.316*** &   & -0.043 \\
			&   & (-4.40) &   & (-1.07) \\
			AFE &   & 3.339*** &   & 0.168*** \\
			&   & (4.60) &   & (3.02) \\
			Constant & -18.117*** & -21.970*** & -8.224*** & -8.082*** \\
			& (-38.84) & (-36.79) & (-267.21) & (-156.81) \\
			&   &   &   &  \\
			Observations & 116,156 & 116,156 & 116,156 & 116,156 \\
			Adjusted R-squared & 0.586 & 0.597 & 0.695 & 0.698 \\
			\bottomrule
			\bottomrule
		\end{tabular}%
	\end{center}
\begin{footnotesize}
	\setcounter{equation}{0}
	\begin{equation}
		TEX_{i,t}=\beta_0+\beta_1QRET_{i,t}+\beta_2NEG_{i,t}+\beta_3QRET_{i,t}\times NEG_{i,t}+\sum\beta_nCONTROLS_{i,t}+\epsilon_{i,t}
	\end{equation}
	
	\noindent Table 6 Panel A presents the regression results of Equation (1) using subsamples of MD\&A (Column 1 and 3) and NFS (Column 2 and 4) sections. TEX represents a vector of textual properties that consists of NW\_MDA, NW\_NFS, TONE\_MDA and TONE\_NFS. CONTROLS denotes a vector of control variables. See \hyperref[appc]{Appendix C} for variable definitions. All financial variables except returns are winsorized at 1\% and 99\% level. All regressions include firm and year-quarter fixed effects and standard errors are clustered at industry level identified by 4-digit SIC codes. ***, ** and * indicate significance at the 1\%, 5\% and 10\% levels in a two-tailed test.
\end{footnotesize}
\end{table}%
%\end{landscape}

%%%%%%%%%%%%%%%%%%%%%%%%% TABLE 6 Panel B
\newpage
%\begin{landscape}
% Table generated by Excel2LaTeX from sheet 'T6'
\begin{table}[H] \label{T6PB}%
	\begin{center}
		\begin{tabular}{lcccc}
			\multicolumn{5}{c}{\textbf{Table 6. Panel B. Narrative Conservatism 10-Q Sections}} \\
			\midrule
			\midrule
			Dep. Variables & \multicolumn{2}{c}{TONE} & \multicolumn{2}{c}{NW} \\
			\cmidrule{2-5}
			& (1) & (2) & (3) & (4) \\
			Section & MDA & NFS & MDA & NFS \\
			\midrule
			%&   &   &   &  \\
			QRET & 0.109 & 0.297 & -0.055*** & -0.033* \\
			& (0.64) & (1.15) & (-4.34) & (-1.70) \\
			NEG & -0.123** & 0.014 & -0.012*** & -0.005 \\
			& (-1.98) & (0.17) & (-3.05) & (-1.01) \\
			\rowcolor[rgb]{ .906,  .902,  .902} \textit{(Pred. Sign)} & (+) & (+) & (+) & (+) \\
			\rowcolor[rgb]{ .906,  .902,  .902} QRET$\times$NEG & 1.423*** & 0.882* & 0.102*** & 0.055* \\
			\rowcolor[rgb]{ .906,  .902,  .902} & (4.54) & (1.88) & (4.18) & (1.65) \\
			SIZE & 0.626*** & 0.900*** & -0.030*** & -0.013 \\
			& (4.26) & (5.14) & (-3.36) & (-1.01) \\
			MTB & 0.021 & 0.054** & 0.003** & 0.004*** \\
			& (1.12) & (2.21) & (2.41) & (3.28) \\
			LEV & -0.213 & -0.802 & -0.189*** & -0.362*** \\
			& (-0.33) & (-0.94) & (-5.32) & (-5.88) \\
			EARN & 17.163*** & 12.079*** & 0.470** & 0.693*** \\
			& (5.26) & (5.69) & (2.16) & (3.83) \\
			STD\_EARN & -8.090*** & -6.020** & -0.547*** & -0.816*** \\
			& (-4.64) & (-2.20) & (-3.35) & (-6.19) \\
			BUSSEG & -0.065 & -0.159 & -0.057*** & -0.031 \\
			& (-0.23) & (-0.45) & (-2.93) & (-1.58) \\
			GEOSEG & 0.052 & 0.999*** & 0.063*** & 0.036** \\
			& (0.16) & (2.61) & (3.01) & (1.98) \\
			AF & 1.979* & -0.343 & 0.140 & -0.073 \\
			& (1.86) & (-0.22) & (1.61) & (-0.95) \\
			AFE & 7.938*** & 4.137*** & 0.227*** & 0.243*** \\
			& (7.81) & (3.74) & (3.20) & (3.56) \\
			Constant & -7.264* & -12.393** & -7.167*** & -7.224*** \\
			& (-1.84) & (-2.57) & (-15.46) & (-18.08) \\
			&   &   &   &  \\
			Observations & 48,089 & 48,089 & 48,089 & 48,089 \\
			Adjusted R-squared & 0.559 & 0.579 & 0.734 & 0.816 \\
			\bottomrule
			\bottomrule
		\end{tabular}%
	\end{center}
\begin{footnotesize}
	\setcounter{equation}{0}
	\begin{equation}
		TEX_{i,t}=\beta_0+\beta_1QRET_{i,t}+\beta_2NEG_{i,t}+\beta_3QRET_{i,t}\times NEG_{i,t}+\sum\beta_nCONTROLS_{i,t}+\epsilon_{i,t}
	\end{equation}
	
	\noindent Table 6 Panel B presents the regression results of Equation (1) using subsamples of MD\&A (Column 1 and 3) and NFS (Column 2 and 4) sections. TEX represents a vector of textual properties that consists of NW\_MDA, NW\_NFS, TONE\_MDA and TONE\_NFS. CONTROLS denotes a vector of control variables. See \hyperref[appc]{Appendix C} for variable definitions. All financial variables except returns are winsorized at 1\% and 99\% level. All regressions include firm and year-quarter fixed effects and standard errors are clustered at industry level identified by 4-digit SIC codes. ***, ** and * indicate significance at the 1\%, 5\% and 10\% levels in a two-tailed test.
\end{footnotesize}
\end{table}%
%\end{landscape}


%%%%%%%%%%%%%%%%%%%%%%%%% TABLE 7
\newpage
\begin{landscape}
	% Table generated by Excel2LaTeX from sheet 'T3'
\begin{table}[H] \label{T7}
	\begin{center}
		\tabcolsep=0.3cm
		\begin{tabular}{lcccc}
			\multicolumn{5}{c}{\textbf{Table 7. Narrative Conservatism and Unconditional Conservatism}} \\
			\toprule
			\toprule
			Dep. Variables & \multicolumn{2}{c}{TLAG} & \multicolumn{2}{c}{TONE} \\
			\cmidrule{2-5}
			& (1) & (2) & (3) & (4) \\
			\midrule
			Panel A: Intangible Assets & LOW & HIGH & LOW & HIGH \\
			\midrule
			%&   &   &   &  \\
			$\Delta$DRET & 1.975*** & 3.026*** & -1.205 & -2.647** \\
			& (11.64) & (9.89) & (-1.23) & (-2.07) \\
			BN & -0.032 & -0.130*** & -0.193 & -0.060 \\
			& (-1.13) & (-4.26) & (-1.17) & (-0.38) \\
			\rowcolor[rgb]{ .906,  .902,  .902} \textit{(Pred. Sign)} & (-) & (-) & (+) & (+) \\
			\rowcolor[rgb]{ .906,  .902,  .902} $\Delta$DRET$\times$BN & -3.181*** & -6.326*** & 1.044 & 5.773** \\
			\rowcolor[rgb]{ .906,  .902,  .902} & (-10.61) & (-13.28) & (0.82) & (2.42) \\
			Constant & -3.065*** & -3.588*** & -0.478 & -3.469 \\
			& (-3.58) & (-6.35) & (-0.06) & (-1.18) \\
			&   &   &   &  \\
			Observations & 29,136 & 31,806 & 29,136 & 31,806 \\
			Adjusted R-squared & 0.118 & 0.146 & 0.132 & 0.123 \\
			\midrule
			Panel B: R\&D Expenses & LOW & HIGH & LOW & HIGH \\
			\midrule
			%&   &   &   &  \\
			$\Delta$DRET & 1.651*** & 1.946*** & -0.209 & -1.566 \\
			& (6.85) & (7.52) & (-0.30) & (-1.33) \\
			BN & 0.011 & -0.025 & -0.149 & -0.058 \\
			& (0.26) & (-0.91) & (-1.20) & (-0.50) \\
			\rowcolor[rgb]{ .906,  .902,  .902} \textit{(Pred. Sign)} & (-) & (-) & (+) & (+) \\
			\rowcolor[rgb]{ .906,  .902,  .902} $\Delta$DRET$\times$BN & -2.426*** & -2.983*** & -0.325 & 2.432* \\
			\rowcolor[rgb]{ .906,  .902,  .902} & (-5.65) & (-7.03) & (-0.39) & (1.66) \\
			Constant & -2.520*** & -2.678*** & -1.751 & -5.212 \\
			& (-4.66) & (-5.07) & (-0.25) & (-1.43) \\
			&   &   &   &  \\
			Observations & 19,740 & 22,608 & 19,740 & 22,608 \\
			Adjusted R-squared & 0.106 & 0.143 & 0.184 & 0.115 \\
			\bottomrule
			\bottomrule
		\end{tabular}%
	\end{center}
		\begin{footnotesize}
			\setcounter{equation}{0}
			\begin{equation}
				TEX_{i,t}=\beta_0+\beta_1\Delta DRET_{i,t-tlag}+\beta_2BN_{i,t-tlag}+\beta_3\Delta DRET_{i,t-tlag}\times 		BN_{i,t-tlag}+\sum\beta_nCONTROLS_{i,t}+\epsilon_{i,t}
			\end{equation}
			
			\noindent Table 7 presents the regression results of Equation (1) across high and low intangible assets and R\&D expenses subsamples. TEX represents a vector of textual properties. CONTROLS denotes a vector of control variables. See \hyperref[appc]{Appendix C} for variable definitions. All financial variables except returns are winsorized at 1\% and 99\% level. All regressions include full set of control variables, firm and year-month fixed effects. Standard errors are clustered at industry level identified by 4-digit SIC codes. ***, ** and * indicate significance at the 1\%, 5\% and 10\% levels in a two-tailed test.
		\end{footnotesize}
\end{table}%
	% Table generated by Excel2LaTeX from sheet 'T3'
\begin{table}[H]
	\begin{center}
		\tabcolsep=0.11cm
		\begin{tabular}{lcccccccccc}
			\multicolumn{11}{c}{\textbf{Table 7. Narrative Conservatism in Voluntary and Mandatory Disclosure (Continued)}} \\
			\toprule
			\toprule
			Dep. Variables & \multicolumn{2}{c}{NW} & \multicolumn{2}{c}{N8K} & \multicolumn{2}{c}{NITEM} & \multicolumn{2}{c}{NEXHIBIT} & \multicolumn{2}{c}{NGRAPH} \\
			\cmidrule{2-11}
			& (5) & (6) & (7) & (8) & (9) & (10) & (11) & (12) & (13) & (14) \\
			Disclosure Type & VD & MD & VD & MD & VD & MD & VD & MD & VD & MD \\
			\midrule
			%&   &   &   &   &   &  &   &   &   &  \\
			$\Delta$DRET & -0.156** & 0.039 & -0.063*** & -0.051 & -0.048*** & -0.020* & -0.092** & -0.017 & -0.153*** & 0.030 \\
			& (-2.37) & (0.31) & (-2.72) & (-1.08) & (-4.07) & (-1.65) & (-2.25) & (-0.18) & (-2.64) & (0.47) \\
			BN & -0.018** & 0.002 & -0.004 & -0.007 & -0.002 & -0.002 & -0.003 & 0.000 & -0.017 & 0.010 \\
			& (-2.14) & (0.13) & (-0.99) & (-1.08) & (-1.59) & (-1.59) & (-0.39) & (0.01) & (-1.61) & (1.02) \\
			\rowcolor[rgb]{ .906,  .902,  .902} \textit{(Pred. Sign)} & (+) & (+) & (+) & (+) & (+) & (+) & (+) & (+) & (+) & (+) \\
			\rowcolor[rgb]{ .906,  .902,  .902} $\Delta$DRET$\times$BN & 0.210** & 0.003 & 0.093*** & 0.045 & 0.070*** & 0.026 & 0.175*** & 0.050 & 0.133 & 0.031 \\
			\rowcolor[rgb]{ .906,  .902,  .902} & (2.18) & (0.02) & (2.91) & (0.75) & (5.44) & (1.60) & (2.86) & (0.47) & (1.61) & (0.42) \\
			SIZE & 0.011 & 0.035*** & 0.003 & -0.003 & -0.001 & -0.001 & 0.000 & 0.005 & 0.006 & -0.003 \\
			& (1.18) & (2.64) & (0.90) & (-0.63) & (-0.50) & (-0.88) & (0.04) & (0.45) & (0.56) & (-0.41) \\
			MTB & 0.000 & -0.004** & -0.000 & -0.001 & -0.000 & 0.000** & -0.002*** & -0.001 & -0.003** & -0.001 \\
			& (0.02) & (-2.20) & (-0.16) & (-0.75) & (-1.01) & (1.99) & (-2.67) & (-0.93) & (-2.02) & (-0.69) \\
			LEV & -0.102** & 0.073 & -0.033** & 0.004 & -0.012** & -0.001 & -0.021 & -0.026 & -0.004 & -0.008 \\
			& (-2.42) & (1.02) & (-2.30) & (0.16) & (-2.57) & (-0.22) & (-1.00) & (-0.51) & (-0.08) & (-0.22) \\
			EARN & 0.302*** & 0.270 & 0.047 & 0.103 & -0.003 & -0.009 & 0.109* & 0.054 & -0.110 & 0.054 \\
			& (2.72) & (1.42) & (1.20) & (1.34) & (-0.23) & (-0.99) & (1.94) & (0.44) & (-1.17) & (0.58) \\
			STD\_EARN & -0.254* & -0.021 & -0.096* & -0.078 & -0.004 & -0.018 & -0.014 & -0.255 & 0.373** & -0.136 \\
			& (-1.94) & (-0.08) & (-1.69) & (-0.81) & (-0.25) & (-0.91) & (-0.17) & (-1.34) & (2.17) & (-1.10) \\
			BUSSEG & -0.004 & -0.025 & 0.006 & -0.017** & 0.000 & 0.000 & 0.012* & -0.027* & -0.015 & 0.001 \\
			& (-0.26) & (-1.11) & (1.35) & (-2.02) & (0.28) & (0.10) & (1.71) & (-1.75) & (-0.69) & (0.11) \\
			GEOSEG & 0.008 & 0.004 & -0.004 & 0.003 & 0.002 & 0.003** & -0.022*** & 0.008 & -0.018 & -0.006 \\
			& (0.67) & (0.20) & (-0.87) & (0.33) & (1.28) & (2.54) & (-3.67) & (0.55) & (-0.92) & (-0.57) \\
			AF & -0.033 & 0.013 & 0.003 & 0.005 & 0.002 & 0.001 & 0.026 & 0.031 & -0.087 & -0.073 \\
			& (-0.43) & (0.18) & (0.13) & (0.15) & (0.17) & (0.09) & (0.74) & (0.37) & (-1.08) & (-1.57) \\
			AFE & 0.013 & -0.266** & 0.034 & -0.080 & -0.019** & 0.022** & 0.005 & -0.192** & -0.170* & -0.022 \\
			& (0.16) & (-2.05) & (1.17) & (-1.64) & (-2.33) & (2.19) & (0.12) & (-2.17) & (-1.77) & (-0.35) \\
			Constant & -6.786*** & -8.541*** & -0.889*** & -0.839*** & -0.687*** & -0.693*** & -0.436*** & -0.585*** & 0.000 & -0.020 \\
			& (-28.58) & (-14.52) & (-18.87) & (-10.34) & (-96.80) & (-130.77) & (-4.01) & (-2.98) & (0.00) & (-0.44) \\
			&   &   &   &   &   &   &   &   &   &  \\
			Observations & 53,460 & 21,900 & 53,460 & 21,900 & 53,460 & 21,900 & 53,460 & 21,900 & 53,460 & 21,900 \\
			Adjusted R-squared & 0.448 & 0.505 & 0.212 & 0.073 & 0.040 & -0.023 & 0.162 & 0.139 & 0.360 & 0.141 \\
			\bottomrule
			\bottomrule
		\end{tabular}%
	\end{center}
		\begin{footnotesize}
			\setcounter{equation}{0}
			\begin{equation}
				TEX_{i,t}=\beta_0+\beta_1\Delta DRET_{i,t-tlag}+\beta_2BN_{i,t-tlag}+\beta_3\Delta DRET_{i,t-tlag}\times 	BN_{i,t-tlag}+\sum\beta_nCONTROLS_{i,t}+\epsilon_{i,t}
			\end{equation}
			
			\noindent Table 7 presents the regression results of Equation (1) across voluntary and mandatory disclosure subsamples. TEX represents a vector of textual properties. CONTROLS denotes a vector of control variables. See \hyperref[appc]{Appendix C} for variable definitions. All financial variables except returns are winsorized at 1\% and 99\% level. All regressions include firm and year-month fixed effects and standard errors are clustered at industry level identified by 4-digit SIC codes. ***, ** and * indicate significance at the 1\%, 5\% and 10\% levels in a two-tailed test.
		\end{footnotesize}
\end{table}%
\end{landscape}


%%%%%%%%%%%%%%%%%%%%%%%%%% TABLE 7
%\newpage
%%\begin{landscape}
%% Table generated by Excel2LaTeX from sheet 'T3'
\begin{table}[H] \label{T7}
	\begin{center}
		\tabcolsep=0.3cm
		\begin{tabular}{lcccc}
			\multicolumn{5}{c}{\textbf{Table 7. Narrative Conservatism and Unconditional Conservatism}} \\
			\toprule
			\toprule
			Dep. Variables & \multicolumn{2}{c}{TLAG} & \multicolumn{2}{c}{TONE} \\
			\cmidrule{2-5}
			& (1) & (2) & (3) & (4) \\
			\midrule
			Panel A: Intangible Assets & LOW & HIGH & LOW & HIGH \\
			\midrule
			%&   &   &   &  \\
			$\Delta$DRET & 1.975*** & 3.026*** & -1.205 & -2.647** \\
			& (11.64) & (9.89) & (-1.23) & (-2.07) \\
			BN & -0.032 & -0.130*** & -0.193 & -0.060 \\
			& (-1.13) & (-4.26) & (-1.17) & (-0.38) \\
			\rowcolor[rgb]{ .906,  .902,  .902} \textit{(Pred. Sign)} & (-) & (-) & (+) & (+) \\
			\rowcolor[rgb]{ .906,  .902,  .902} $\Delta$DRET$\times$BN & -3.181*** & -6.326*** & 1.044 & 5.773** \\
			\rowcolor[rgb]{ .906,  .902,  .902} & (-10.61) & (-13.28) & (0.82) & (2.42) \\
			Constant & -3.065*** & -3.588*** & -0.478 & -3.469 \\
			& (-3.58) & (-6.35) & (-0.06) & (-1.18) \\
			&   &   &   &  \\
			Observations & 29,136 & 31,806 & 29,136 & 31,806 \\
			Adjusted R-squared & 0.118 & 0.146 & 0.132 & 0.123 \\
			\midrule
			Panel B: R\&D Expenses & LOW & HIGH & LOW & HIGH \\
			\midrule
			%&   &   &   &  \\
			$\Delta$DRET & 1.651*** & 1.946*** & -0.209 & -1.566 \\
			& (6.85) & (7.52) & (-0.30) & (-1.33) \\
			BN & 0.011 & -0.025 & -0.149 & -0.058 \\
			& (0.26) & (-0.91) & (-1.20) & (-0.50) \\
			\rowcolor[rgb]{ .906,  .902,  .902} \textit{(Pred. Sign)} & (-) & (-) & (+) & (+) \\
			\rowcolor[rgb]{ .906,  .902,  .902} $\Delta$DRET$\times$BN & -2.426*** & -2.983*** & -0.325 & 2.432* \\
			\rowcolor[rgb]{ .906,  .902,  .902} & (-5.65) & (-7.03) & (-0.39) & (1.66) \\
			Constant & -2.520*** & -2.678*** & -1.751 & -5.212 \\
			& (-4.66) & (-5.07) & (-0.25) & (-1.43) \\
			&   &   &   &  \\
			Observations & 19,740 & 22,608 & 19,740 & 22,608 \\
			Adjusted R-squared & 0.106 & 0.143 & 0.184 & 0.115 \\
			\bottomrule
			\bottomrule
		\end{tabular}%
	\end{center}
		\begin{footnotesize}
			\setcounter{equation}{0}
			\begin{equation}
				TEX_{i,t}=\beta_0+\beta_1\Delta DRET_{i,t-tlag}+\beta_2BN_{i,t-tlag}+\beta_3\Delta DRET_{i,t-tlag}\times 		BN_{i,t-tlag}+\sum\beta_nCONTROLS_{i,t}+\epsilon_{i,t}
			\end{equation}
			
			\noindent Table 7 presents the regression results of Equation (1) across high and low intangible assets and R\&D expenses subsamples. TEX represents a vector of textual properties. CONTROLS denotes a vector of control variables. See \hyperref[appc]{Appendix C} for variable definitions. All financial variables except returns are winsorized at 1\% and 99\% level. All regressions include full set of control variables, firm and year-month fixed effects. Standard errors are clustered at industry level identified by 4-digit SIC codes. ***, ** and * indicate significance at the 1\%, 5\% and 10\% levels in a two-tailed test.
		\end{footnotesize}
\end{table}%
%%\end{landscape}
%
%%%%%%%%%%%%%%%%%%%%%%%%%% TABLE 8
%\newpage
%%\begin{landscape}
%% Table generated by Excel2LaTeX from sheet 'T8PA'
\begin{table}[htbp] \label{T8}
  \centering
    \begin{tabular}{lcccccc}
    \multicolumn{7}{c}{\textbf{Table 8. Managerial Incentives and Narrative Conservatism}} \\
    \midrule
    \midrule
    Dep. Vars.& \multicolumn{2}{c}{NW} & \multicolumn{2}{c}{TONE} & \multicolumn{2}{c}{TLAG}\\
    \midrule
    \textbf{Panel A}  & (1) & (2) & (3) & (4) & (5) & (6) \\
    \multicolumn{1}{l}{\textbf{Option Value}} & LOW & HIGH & LOW & HIGH & LOW & HIGH \\
    \cmidrule{2-7}
    \multicolumn{1}{l}{QRET} & 0.041 & 0.104*** & 1.164*** & 0.545 & -0.224 & -0.151 \\
      & (0.98) & (2.66) & (3.48) & (1.38) & (-0.71) & (-0.54) \\
    \multicolumn{1}{l}{NEG} & 0.023 & -0.003 & -0.082 & -0.132 & 0.095 & -0.070 \\
      & (1.38) & (-0.26) & (-0.70) & (-1.01) & (0.97) & (-0.56) \\
%    \rowcolor[rgb]{ .933,  .925,  .882} Sign Prediction & - & - & + & + & + & + \\
    \rowcolor[rgb]{ .933,  .925,  .882} \multicolumn{1}{l}{QRET$\times$NEG} & -0.177** & -0.252*** & 1.615** & 1.584** & -0.859 & -0.693 \\
    \rowcolor[rgb]{ .933,  .925,  .882}  & (-2.24) & (-4.44) & (2.57) & (2.55) & (-1.63) & (-1.51) \\
%    \multicolumn{1}{l}{SIZE} & -0.007 & 0.026 & 1.316*** & 1.158*** & -0.325** & -0.023 \\
%      & (-0.32) & (1.39) & (5.49) & (5.58) & (-2.15) & (-0.19) \\
%    \multicolumn{1}{l}{MTB} & -0.003 & -0.007*** & 0.101*** & 0.124*** & -0.083*** & -0.006 \\
%      & (-0.80) & (-3.06) & (2.63) & (4.13) & (-3.06) & (-0.35) \\
%    \multicolumn{1}{l}{LEV} & 0.508*** & 0.298*** & -1.696 & -1.473 & 1.485 & 1.000 \\
%      & (5.30) & (3.27) & (-1.59) & (-1.28) & (1.62) & (1.60) \\
%    \multicolumn{1}{l}{Constant} & 8.185*** & 8.324*** & -20.044*** & -6.415 & 46.377*** & 40.805*** \\
%      & (57.72) & (13.16) & (-3.64) & (-1.35) & (27.95) & (11.75) \\
    \rowcolor[rgb]{ .933,  .925,  .882} \multicolumn{1}{l}{Diff. QRET$\times$NEG} & \multicolumn{2}{c}{0.076$^{\star\star\star}$} & \multicolumn{2}{c}{0.030} & \multicolumn{2}{c}{-0.166} \\
    \rowcolor[rgb]{ .933,  .925,  .882}  & \multicolumn{2}{c}{(2.02)} & \multicolumn{2}{c}{(0.14)} & \multicolumn{2}{c}{(-0.86)} \\
      &   &   &   &   &   &  \\
    \multicolumn{1}{l}{Observations} & 15,229 & 15,226 & 15,229 & 15,226 & 15,229 & 15,226 \\
    \multicolumn{1}{l}{Adjusted R-squared} & 0.443 & 0.493 & 0.551 & 0.610 & 0.540 & 0.588 \\
    \midrule
    \textbf{Panel B}  & (1) & (2) & (3) & (4) & (5) & (6) \\
    \multicolumn{1}{l}{\textbf{SEO}} & NO & YES & NO & YES & NO & YES \\
    \cmidrule{2-7}
    QRET & 0.060*** & 0.027 & -0.222 & 0.060 & -0.506*** & -0.153 \\
    & (3.70) & (1.52) & (-1.32) & (0.41) & (-2.89) & (-0.79) \\
    NEG & -0.002 & -0.001 & -0.104 & -0.073 & 0.048 & 0.050 \\
    & (-0.25) & (-0.11) & (-1.56) & (-0.82) & (1.00) & (0.84) \\
    \rowcolor[rgb]{ .933,  .925,  .882} QRET$\times$NEG & -0.153*** & -0.163*** & 2.448*** & 1.357*** & -0.510* & -0.415 \\
    \rowcolor[rgb]{ .933,  .925,  .882} & (-5.33) & (-4.25) & (8.25) & (3.16) & (-1.73) & (-1.49) \\
    \rowcolor[rgb]{ .933,  .925,  .882} \multicolumn{1}{l}{Diff. QRET$\times$NEG} & \multicolumn{2}{c}{0.009} & \multicolumn{2}{c}{1.091$^{\star\star\star}$} & \multicolumn{2}{c}{-0.095} \\
    \rowcolor[rgb]{ .933,  .925,  .882}  & \multicolumn{2}{c}{(1.07)} & \multicolumn{2}{c}{(5.99)} & \multicolumn{2}{c}{(-0.59)} \\
    &   &   &   &   &   &  \\
    Observations & 45,490 & 37,054 & 45,490 & 37,054 & 45,490 & 37,054 \\
    Adjusted R-squared & 0.696 & 0.687 & 0.552 & 0.623 & 0.634 & 0.674 \\
    \midrule
    \multicolumn{1}{l}{Year-quarter FE} & YES & YES & YES & YES & YES & YES \\
    \multicolumn{1}{l}{Firm FE} & YES & YES & YES & YES & YES & YES \\
    \multicolumn{1}{l}{Industry clustered SE} & YES & YES & YES & YES & YES & YES \\
    \multicolumn{1}{l}{Controls} & YES & YES & YES & YES & YES & YES \\
    \bottomrule
    \bottomrule
    \end{tabular}%
\end{table}%

%%\end{landscape}
%
%%%%%%%%%%%%%%%%%%%%%%%%%% TABLE 9
%\newpage
%%\begin{landscape}
%% Table generated by Excel2LaTeX from sheet 'T10'
\begin{table}[H]   \label{T9}
  \begin{center}
  	    \begin{tabular}{lcccccc}
  		\multicolumn{7}{c}{\textbf{Table 9. Narrative Conservatism, Intangible Assets and R\&D Expenses}} \\
  		\midrule
  		\midrule
  		Dep. Variables & \multicolumn{2}{c}{NW} & \multicolumn{2}{c}{TONE} & \multicolumn{2}{c}{TLAG} \\
  		& (1) & \multicolumn{1}{c}{(2)} & (3) & \multicolumn{1}{c}{(4)} & (5) & \multicolumn{1}{c}{(6)} \\
  		\midrule
  		\textbf{Panel A: Intangible Assets} & LOW & HIGH & LOW & HIGH & LOW & HIGH \\
  		\cmidrule{2-7}
%  		QRET & 0.002 & 0.037* & 0.566*** & 0.466 & -0.527*** & -0.304 \\
%  		& (0.20) & (1.95) & (3.53) & (1.55) & (-3.55) & (-1.27) \\
%  		NEG & 0.006* & 0.010*** & 0.015 & -0.134** & -0.023 & 0.051 \\
%  		& (1.70) & (2.66) & (0.21) & (-2.16) & (-0.44) & (0.89) \\
  		\rowcolor[rgb]{ .933,  .925,  .882} \textit{(Pred. Sign)} & (-) & (-) & (+) & (+) & (+) & (+) \\
  		\rowcolor[rgb]{ .933,  .925,  .882} QRET$\times$NEG & -0.024 & -0.068*** & 0.469 & 0.475 & -0.109 & -0.093 \\
  		\rowcolor[rgb]{ .933,  .925,  .882} & (-1.21) & (-2.71) & (1.50) & (1.08) & (-0.44) & (-0.24) \\
  		&   &   &   &   &   &  \\
  		Observations & 29,636 & 29,634 & 29,636 & 29,634 & 29,636 & 29,634 \\
  		Adjusted R-squared & 0.831 & 0.798 & 0.708 & 0.678 & 0.654 & 0.693 \\
  		\midrule
  		\textbf{Panel B: R\&D Expenses} & LOW & HIGH & LOW & HIGH & LOW & HIGH \\
  		\cmidrule{2-7}
  
%  		QRET & 0.015 & 0.055** & 0.659** & 0.525*** & -0.413** & -0.450* \\
%  		 & (0.75) & (2.18) & (2.52) & (2.68) & (-2.05) & (-1.70) \\
%  		NEG & 0.000 & 0.015 & -0.061 & -0.120 & 0.109 & -0.025 \\
%  		 & (0.05) & (1.64) & (-0.64) & (-1.20) & (1.50) & (-0.34) \\
  		\rowcolor[rgb]{ .933,  .925,  .882} \textit{(Pred. Sign)} & (-) & (-) & (+) & (+) & (+) & (+) \\
  		\rowcolor[rgb]{ .933,  .925,  .882} QRET$\times$NEG & -0.065 & -0.075** & 0.710 & 0.048 & 0.336 & -0.029 \\
  		\rowcolor[rgb]{ .933,  .925,  .882}  & (-1.56) & (-2.45) & (1.53) & (0.10) & (1.15) & (-0.06) \\
  		&   &   &   &   &   &  \\
  		Observations & 22,899 & 22,898 & 22,899 & 22,898 & 22,899 & 22,898 \\
  		Adjusted R-squared & 0.623 & 0.682 & 0.581 & 0.635 & 0.626 & 0.619 \\
  		
  		\bottomrule
  		\bottomrule
  	\end{tabular}%
  \end{center}
\begin{footnotesize}
	\setcounter{equation}{0}
	\begin{equation}
		TEX_{i,t}=\beta_0+\beta_1QRET_{i,t}+\beta_2NEG_{i,t}+\beta_3QRET_{i,t}\times NEG_{i,t}+\sum\beta_nCONTROLS_{i,t}+\epsilon_{i,t}
	\end{equation}
	
	\noindent Table 9 presents the regression results of Equation (1) using 10-Q subsamples of intangible assets (Panel A) and R\&D expenses (Panel B). TEX represents a vector of textual properties that consists of NW, TONE and TLAG. All regressions control for SIZE, MTB, LEV, EARN, STD\_RET, STD\_EARN, AGE, BUSSEG, GEOSEG, AFE and AF (untabulated). See \hyperref[appb]{Appendix B} for variable definitions. All financial variables except returns are winsorized at 1\% and 99\% level. All regressions include firm and time fixed effects and standard errors are clustered at industry level identified by 4-digit SIC codes. ***, ** and * indicate significance at the 1\%, 5\% and 10\% levels in a two-tailed test.
\end{footnotesize}
\end{table}%

%%\end{landscape}

\newpage
\section*{Appendix}
\subsection*{Appendix A: 10-Q and 8-K parsing}
\label{appa}
We develop a Python program to automatically parse, process and retrieve 10-K and 8-K filings from EDGAR database. Our algorithm consists of the following steps:

1. Download all quarterly master indexes from EDGAR using \textit{python-edgar}\footnote{Python-edgar package documentation available at \url{https://github.com/edouardswiac/python-edgar/blob/master/README.md}} package.

2. Filter all 10-Q and 8-K filings\footnote{Our analysis exclude amendments such as 10-Q/A and 8-K/A} from EDGAR master index files and obtain url of the \textit{filing detail} webpage\footnote{One example of filing detail webpage is available at \url{https://www.sec.gov/Archives/edgar/data/320193/000032019320000050/0000320193-20-000050-index.html}} for each of the 10-Q and 8-K filings. 

3. Extract (a) identification information\footnote{For example cik, accession number, reporting period, filing date and 8-K items etc.} and (b) url of report in HTM/TXT format\footnote{One example of report in HTM format is available at \url{https://www.sec.gov/Archives/edgar/data/320193/000032019320000050/a8-kq220203282020.htm}. We first search for url of main report in HTM format. If HTM format main report is not available, then we extract the url of TXT format full report. Each EDGAR filing can be accessed in three formats at maximum: regular text (*.txt), web pages (*.htm) and eXtensible Business Reporting Language, also known as XBRL (*.xml). Early filings in EDGAR are only in TXT format. Later filings extend to HTM format, and in 2009 SEC adopted the XBRL for all corporate filings \cite{secFinalRuleInteractive2009}. Therefore, current existing EDGAR filings all contain a TXT file, and depending on their filing date and company reporting policy they may or may not contain HTM or XML files. Normally all filings in XML format are also available in HTM format. We manually checked 100 random filings that are in XML format, and all of them are also available in HTM format with the same content. The TXT files usually contain not only the main report, but also all other additional filing materials (if any) such as graphics, exhibits and press release etc. However, the HTM files only contain the main report. We mainly focus on HTM files other than TXT files because the former naturally filters out less relevant information, and provides a cleaner textual content of the essential information. XML files are not parsed due to low tractability. } from the \textit{filing detail} webpage for each of the 10-Q and 8-K filings. 

4. Parse and cleanse\footnote{Cleansing steps are: (a) delete nondisplay section; (b) delete all tables that contains more than 4 numbers; and (c) delete all HTML tags} all 10-Q and 8-K filings with url of HTM/TXT format report, using \textit{beautiful soup}\footnote{Beautiful soup package documentation available at \url{https://www.crummy.com/software/BeautifulSoup/bs4/doc/}} package. 

5. Save all clean 10-Q and 8-K filings to local device. 

6. Perform word count on clean 10-Q and 8-K filings using LM dictionary\footnote{LM dictionary available at \url{https://sraf.nd.edu/textual-analysis/resources/\#LM\%20Sentiment\%20Word\%20Lists}}. 

All Python scripts and data are available online via \url{https://github.com/fengzhi22/narrative_conservatism}.

\subsection*{Appendix B: Textual Variable Definition}
\label{appb}
\begin{table}[H]
	\centering
	\begin{tabular}{lp{15cm}p{15cm}}
		\textbf{Variable} & \textbf{Definition} \\
		NW & Number of words, defined as the natural logarithm of one plus the count of total words (nw)\\
		nw & Raw count of total words\\
		TONE & Tone, defined as number of net positive words per thousand total words, calculated as total number of positive words minus the sum of total number of negative words and total number of negations, and multiply the previous result by one thousand\\
		TLAG & Time lag, defined as number of days elapsed between the news release date (CRSP entry date) and document filing date (EDGAR filing date)\\
		ABTONE & Abnormal tone, calculated as the residual of the cross-sectional expected tone model (Equation 3) in \citet*{huangToneManagement2014}\\
		N8K & Number of 8-Ks reported in one day\\
		NITEM & Number of 8-K items reported in one day\\
		
	\end{tabular}%
\end{table}%

\subsection*{Appendix C: Financial Variable Definition}
\label{appc}
\begin{table}[H]
	\centering
	\begin{tabular}{lp{15cm}p{15cm}}
		\textbf{Variable} & \textbf{Definition} \\
		
		EARN & Quarterly earnings, defined as quarterly earnings before extraordinary items (Compustat data item IBQ) scaled by beginning-of-quarter total assets (Compustat data item ATQ) \\
		$\Delta$EARN & Change in quarterly earnings, defined as current quarterly earnings minus one-quarter-lagged quarterly earnings \\
		LEV & Leverage ratio, defined as beginning-of-quarter short term debt (Compustat data item DLCQ) plus beginning-of-quarter long term debt (Compustat data item DLTTQ) scaled by beginning-of-quarter total assets (Compustat data item ATQ) \\
		MTB & Market-to-book ratio, defined as beginning-of-quarter market value of equity, calculated as common share price (Compustat data item PRCCQ) times common shares outstanding (Compustat data item CSHOQ) divided by beginning-of-quarter book value of equity (Compustat data item CEQQ) \\
		SIZE & Firm size, defined as the natural logarithm of market value of equity, calculated as natural logarithm of common share price (Compustat data item PRCCQ) times common shares outstanding (Compustat data item CSHOQ) \\
		QRET & Quarterly market-adjusted stock return, defined as buy-and-hold stock return (CRSP data item RET) over the fiscal quarter adjusted by the value-weighted stock return (CRSP data item VWRETD) over the same period \\
		DRET & Daily market-adjusted stock return, defined as daily buy-and-hold stock return (CRSP data item RET) adjusted by the daily value-weighted stock return (CRSP data item VWRETD)\\
		$\Delta$DRET & Change in daily market-adjusted stock return (DRET), defined as current daily market-adjusted stock return minus one-day-lagged daily market-adjusted stock return \\
		NEG & Indicator for negative quarterly return, which is set to 1 when market-adjusted stock return (QRET) is negative and 0 otherwise \\
		BN & Indicator for daily bad news, which is set to 1 (0) if the negative (positive) change in daily market-adjusted stock return ($\Delta$DRET) is three times larger than the firm’s average decrease (increase) in daily return over the calendar year.\\
		AF & Analyst forecast, defined as analysts' mean consensus forecast for one-year-ahead earnings per share, scaled by stock price per share at the end of the fiscal quarter (Compustat data item PRCCQ)\\
		AFE & Analyst forecast error, defined as I/B/E/S earnings per share minus the median of the most recent analysts' forecasts, deflated by stock price per share at the end of the fiscal quarter (Compustat data item PRCCQ)\\
		BUSSEG & Business segment, defined as the natural logarithm of one plus number of business segments, or one if item is missing from Compustat\\
		GEOSEG & Geographical segment, defined as the natural logarithm of one plus number of geographical segments, or one if item is missing from Compustat\\
		AGE & Firm age, defined as the natural logarithm of one plus number of days elapsed since the firm's first entry date in CRSP\\
		STD\_EARN & Standard deviation of quarterly earnings (EARN) over the last five quarters\\
		STD\_QRET & Standard deviation of monthly market-adjusted stock return over all months in the fiscal quarter\\
		LOSS & Indicator for loss, which is set to 1 when quaterly earnings (EARN) is negative and 0 otherwise\\
	\end{tabular}%
\end{table}%
\newpage
\subsection*{Appendix D: 8-K Item List}
\label{appd}
\input{../output/table/appd}
\newpage
\subsection*{Appendix E: 8-K Matching Cases}
We check whether the SEC filings and the market returns movements are related to the same corporate events, assuming market efficiency. First, we identify the firm-day with top ten largest changes in daily returns ($\Delta$DRET) upwards and downwards. Next, we read the 8-Ks matched to the news and see if the corporate events depicted in the 8-Ks are in line with the market movements both in terms of direction and magnitude. We find that the 8-K matching cases make economic sense overall. See selected 8-K matching cases below.
\label{appe}
\begin{center}
	\textbf{Good News}
\end{center}
\subsubsection*{Case 1}
Differential Brands Group Inc. (CIK = 844143) experienced a significant rise in market-adjusted daily stock returns ($\Delta$DRET = 5.14) on June 27 of 2018. On June 27 of 2018, the company filed an 8-K with ending reporting period on the same day, which contained Item 8.01: Other Events and Item 9.01: Financial Statements and Exhibits. This 8-K stated that ``On June 27, 2018, Differential Brands Group Inc. issued a press release announcing that it has entered into a definitive purchase agreement with Global Brands Group Holding Limited, a Hong Kong listed company (`GBG'), to acquire a significant part of GBG’s North American licensing business".
\subsubsection*{Case 2}
Karuna Therapeutics, Inc. (CIK = 1771917) experienced a significant rise in market-adjusted daily stock returns ($\Delta$DRET = 4.42) on November 18 of 2019. On November 18 of 2019, the company filed an 8-K with ending reporting period on the same day, which contained Item 8.01: Other Events and Item 9.01: Financial Statements and Exhibits. This 8-K contained a press release, which stated that ``Karuna Therapeutics, Inc. (Nasdaq: KRTX), a clinical-stage biopharmaceutical company committed to developing novel therapies with the potential to transform the lives of people with disabling and potentially fatal neuropsychiatric disorders and pain, today announced results from its Phase 2 clinical trial of KarXT for the treatment of acute psychosis in patients with schizophrenia. In the clinical trial, KarXT demonstrated a statistically significant and clinically meaningful 11.6 point mean reduction in total Positive and Negative Syndrome Scale (PANSS) score compared to placebo (p$<$0.0001) and also demonstrated good overall tolerability. A statistically significant reduction in the secondary endpoints of PANSS-Positive and PANSS-Negative scores were also observed (p$<$0.001)". 
\subsubsection*{Case 3}
Opexa Therapeutics, Inc. (CIK = 1069308) experienced a significant rise in market-adjusted daily stock returns ($\Delta$DRET = 3.34) on August 7 of 2009. August 7 of 2009, the company filed an 8-K with ending reporting period on the same day, which contained Item 1.01: Entry into a Material Definitive Agreement, Item 1.02: Termination of a Material Definitive Agreement and Item 9.01: Financial Statements and Exhibits. This 8-K stated that ``Effective August 6, 2009, Opexa Therapeutics, Inc., a company developing a novel T-cell immunotherapy for multiple sclerosis (MS), entered into an exclusive agreement with Novartis for the further development of Opexa’s novel stem cell technology. This technology, which has generated preliminary data showing the potential to generate monocyte derived islet cells from peripheral blood mononuclear cells, was in early preclinical development at Opexa". 
\subsubsection*{Case 4}
Amarin Corporation plc (CIK = 897448) experienced a significant rise in market-adjusted daily stock returns ($\Delta$DRET = 3.13) on September 24 of 2018. On September 24 of 2018, the company filed an 8-K with ending reporting period on the same day, which contained Item 8.01: Other Events and Item 9.01: Financial Statements and Exhibits. This 8-K only contained a press release, which stated that ``September 24, 2018 - Amarin Corporation plc (NASDAQ:AMRN), announced today topline results from the Vascepa® cardiovascular (CV) outcomes trial, REDUCE-IT™, a global study of 8,179 statin-treated adults with elevated CV risk. REDUCE-IT met its primary endpoint demonstrating an approximately 25\% relative risk reduction, to a high degree of statistical significance (p$<$0.001), in major adverse CV events (MACE) in the intent-to-treat patient population with use of Vascepa 4 grams/day as compared to placebo".
\subsubsection*{Case 5}
Avanir Pharmaceuticals (CIK = 858803) experienced a significant rise in market-adjusted daily stock returns ($\Delta$DRET = 3.06) on April 18 of 2007. On April 18 of 2007, the company filed an 8-K with ending reporting period on the same day, which contained Item 8.01: Other Events. This 8-K stated that ``On April 18, 2007, Avanir Pharmaceuticals (the `Company') announced top-line results from the Company's Phase III clinical trial evaluating the investigational drug Zenvia(TM) (dextromethorphan hydrobromide/quinidine sulfate (`DMQ')), an NMDA antagonist and sigma-1 agonist, in diabetic neuropathic pain".
\begin{center}
	\textbf{Bad News}
\end{center}
\subsubsection*{Case 1}
NovaBay Pharmaceuticals, Inc. (CIK = 1389545) experienced a significant drop in market-adjusted daily stock returns ($\Delta$DRET = -9.06) on June 11 of 2019. On June 17 of 2019, the company filed an 8-K with ending reporting period on June 11 of 2019, which contained Item 5.02: Departure of Directors or Certain Officers; Election of Directors; Appointment of Certain Officers: Compensatory Arrangements of Certain Officers and Item 9.01: Financial Statements and Exhibits. This 8-K stated that ``On June 17, 2019, NovaBay Pharmaceuticals, Inc. (the `Company') announced, effective as of June 11, 2019, the Board of Directors (the `Board') of the Company appointed Justin Hall and Jason Raleigh to the permanent positions of President and Chief Executive Officer and Chief Financial Officer, respectively".
\subsubsection*{Case 2}
Dynavax Technologies Corporation (CIK = 1029142) experienced a significant drop in market-adjusted daily stock returns ($\Delta$DRET = -5.79) on December 18 of 2008. On December 19 of 2008, the company filed an 8-K with ending reporting period on December 18 of 2008, which contained Item 1.02: Termination of a Material Definitive Agreement, Item 8.01: Other Events and Item 9.01: Financial Statements and Exhibits. This 8-K stated that ``On December 19, 2008, Dynavax Technologies Corporation (the `Company') announced the termination of an exclusive license and development collaboration agreement and a related manufacturing agreement (the `Collaboration Arrangement') with Merck \& Co., Inc. (`Merck') for HEPLISAV(TM), a Phase 3 hepatitis B virus vaccine". 
\subsubsection*{Case 3}
Proteostasis Therapeutics, Inc. (CIK = 1445283) experienced a significant drop in market-adjusted daily stock returns ($\Delta$DRET = -4.72) on October 19 of 2018. On October 24 of 2018, the company filed an 8-K with ending reporting period on October 22 of 2018, which contained Item 1.01: Entry into a Material Definitive Agreement, Item 8.01: Other Events and Item 9.01: Financial Statements and Exhibits. This 8-K stated that ``On October 23, 2018, Proteostasis Therapeutics, Inc. (the `Company') entered into an underwriting agreement (the `Underwriting Agreement') with Leerink Partners LLC and Piper Jaffray \& Co. as representatives of the several underwriters named therein (the `Underwriters'), relating to the underwritten public offering of 11,000,000 shares of the Company’s common stock, par value \$0.001 per share (the `Offering'). The price to the public in the Offering was \$6.75 per share". 
\subsubsection*{Case 4}
7th Level, Inc. (CIK = 920038) experienced a significant drop in market-adjusted daily stock returns ($\Delta$DRET = -4.27) on April 22 of 1998. On April 23 of 1998, the company filed an 8-K with ending reporting period on April 23 of 1998, which contained Item 5: Other events and Item 7: Financial statements and exhibits. This 8-K only contained a press release, which stated that ``...7th Level, Inc. (NASDAQ: SEVL) announced that the Company and privately-held Pulse Entertainment, Inc. of Los Angeles have decided not to proceed with their proposed merger.  Separately, 7th Level announced it has obtained commitments for a \$4.5 million bridge loan and a \$10 million private placement to finance the ramp up and rollout of 7th Level's revolutionary new line of technology products".
\subsubsection*{Case 5}
Atlantic Alliance Partnership Corp. (CIK = 1630940) experienced a significant drop in market-adjusted daily stock returns ($\Delta$DRET = -3.65) on November 3 of 2016. On November 8 of 2016, the company filed an 8-K with ending reporting period on November 7 of 2016, which contained Item 5.02: Departure of Directors or Certain Officers; Election of Directors; Appointment of Certain Officers: Compensatory Arrangements of Certain Officers. This 8-K stated that ``On November 7, 2016, Mr. Jonathan Goodwin resigned as the Chief Executive Officer and a director of Atlantic Alliance Partnership Corp. (the `Company'), and Mr. Waheed Alli resigned as the Chairman of the Company, each to pursue other professional interests. Such resignations were not the result of any disagreement with the Company. On November 7, 2016, the board of directors of the Company (the `Board') appointed Mr. Iain Abrahams as the Chief Executive Officer of the Company and Mr. Mark Klein (a director of the Company prior to such date) as the Chairman of the Company. Mr. Abrahams will continue to serve as a director of the Company. Mr. Daniel Winston has been appointed to serve on the audit committee of the Board in lieu of Mr. Abrahams".

\newpage
\setcounter{page}{1}
\section*{Online Appendix}

%%%%%%%%%%%%%%%%%%%%%%%%% Online Appendix TABLE 1
% Table generated by Excel2LaTeX from sheet 'T3'
\begin{table}[H] \label{OAT1}
	\begin{center}
		\tabcolsep=0.11cm
		\begin{tabular}{lcccc}
			\multicolumn{5}{c}{\textbf{Online Appendix. Table 1. }} \\
			\multicolumn{5}{c}{\textbf{Is 8-K Narrative Disclosure Conservative? (Restricted Sample)}} \\
			\toprule
			\toprule
			& (1) & (2) & (3) & (4) \\
			Dep. Variables & TLAG & TLAG & TONE & TONE \\
			\midrule
			&   &   &   &  \\
			$\Delta$DRET & 0.668*** & 0.708*** & -0.874 & -0.655 \\
			& (6.94) & (6.41) & (-0.76) & (-0.52) \\
			BN & -0.047*** & -0.049*** & -0.061 & -0.051 \\
			& (-3.06) & (-3.02) & (-0.52) & (-0.42) \\
			\rowcolor[rgb]{ .906,  .902,  .902} \textit{(Pred. Sign)} & (-) & (-) & (+) & (+) \\
			\rowcolor[rgb]{ .906,  .902,  .902} $\Delta$DRET$\times$BN & -1.303*** & -1.421*** & 2.460 & 2.096 \\
			\rowcolor[rgb]{ .906,  .902,  .902} & (-7.18) & (-5.87) & (1.22) & (1.05) \\
			SIZE &   & 0.006 &   & 0.100 \\
			&   & (0.59) &   & (0.98) \\
			MTB &   & 0.001 &   & -0.020 \\
			&   & (0.91) &   & (-1.10) \\
			LEV &   & -0.057 &   & -0.502 \\
			&   & (-1.30) &   & (-0.98) \\
			EARN &   & 0.377*** &   & 2.647* \\
			&   & (2.78) &   & (1.88) \\
			STD\_EARN &   & -0.132 &   & -2.315 \\
			&   & (-0.80) &   & (-1.29) \\
			BUSSEG &   & -0.008 &   & 0.070 \\
			&   & (-0.52) &   & (0.40) \\
			GEOSEG &   & 0.012 &   & 0.116 \\
			&   & (0.85) &   & (0.77) \\
			AF &   & 0.061 &   & -0.775 \\
			&   & (0.72) &   & (-1.15) \\
			AFE &   & 0.041 &   & 2.393** \\
			&   & (0.35) &   & (2.59) \\
			Constant & -0.695*** & -0.761*** & -2.929 & -2.473 \\
			& (-4.64) & (-5.08) & (-0.46) & (-0.39) \\
			&   &   &   &  \\
			Observations & 28,814 & 26,370 & 28,814 & 26,370 \\
			Adjusted R-squared & 0.174 & 0.174 & 0.219 & 0.223 \\
			\bottomrule
			\bottomrule
		\end{tabular}%
	\end{center}
\end{table}%
\begin{equation*}
\begin{split}
tone_{i,t}=\beta_0&+\beta_1EARN_{i,t}+\beta_2RET_{i,t}+\beta_3SIZE_{i,t}+\beta_4MTB_{i,t}+\beta_5STD\_EARN_{i,t}\\
&+\beta_6STD\_RET_{i,t}+\beta_7AGE_{i,t}+\beta_8BUSSEG_{i,t}+\beta_9GEOSEG_{i,t}+\beta_{10}LOSS_{i,t}\\
&+\beta_{11}\Delta EARN_{i,t}+\beta_{12}AFE_{i,t}+\beta_{13}AF_{i,t}+\epsilon_{i,t}
\end{split}
\end{equation*}

Online Appendix Table 1 presents regression results of the above Equation (Column 1) in comparison with the expected tone model results in \cite{huangToneManagement2014} (Column 2). Dependent variable $tone_{i,t}$ is defined as number of net positive words, calculated as total number of positive words minus the sum of total number of negative words and total number of negations. Independent variables are defined in \hyperref[appc]{Appendix C}. All financial variables except returns are winsorized at 1\% and 99\% level. ***, ** and * indicate significance at the 1\%, 5\% and 10\% levels in a two-tailed test. The coefficient of MTB in Column 1 is consistent with that in Column 2 in terms of sign, because in the expected tone model of \cite{huangToneManagement2014} the authors use book-to-market ratio instead of market-to-book ratio. 


%%%%%%%%%%%%%%%%%%%%%%%%% Online Appendix TABLE 2
\newpage
%\begin{landscape}
% Table generated by Excel2LaTeX from sheet 'OAT3'
\begin{table}[H] \label{oat2}
  \begin{center}
  	    \begin{tabular}{lcrrr}
  		\multicolumn{5}{c}{\textbf{Online Appendix. Table 2.}} \\
  		\multicolumn{5}{c}{\textbf{Panel A: Summary of Fiscal Yearly Regressions}} \\
  		\midrule
  		\midrule
  		Indep. Vars. & Prediction & Coeff. & S.E. & t-stats \\
  		\midrule
  		Intercept &   & -0.016 & 0.006 & -2.50 \\
  		NEG &   & 0.007 & 0.013 & 0.55 \\
  		RET & (+) & -0.051 & 0.031 & -1.66 \\
  		RET$\times$SIZE & (+) & 0.012 & 0.006 & 2.20 \\
  		RET$\times$MTB & (-) & -0.016 & 0.004 & -4.05 \\
  		RET$\times$LEV & (-) & 0.033 & 0.040 & 0.81 \\
  		RET$\times$NEG & (+) & 0.235 & 0.067 & 3.53 \\
  		RET$\times$NEG$\times$SIZE & (-) & -0.034 & 0.011 & -3.09 \\
  		RET$\times$NEG$\times$MTB & (+) & 0.020 & 0.006 & 3.25 \\
  		RET$\times$NEG$\times$LEV & (+) & -0.099 & 0.076 & -1.30 \\
  		SIZE &   & 0.004 & 0.001 & 4.20 \\
  		MTB &   & 0.001 & 0.001 & 1.00 \\
  		LEV &   & -0.013 & 0.008 & -1.61 \\
  		NEG$\times$SIZE &   & -0.001 & 0.002 & -0.48 \\
  		NEG$\times$MTB &   & 0.000 & 0.001 & -0.27 \\
  		NEG$\times$LEV &   & 0.002 & 0.015 & 0.10 \\
  		\bottomrule
  		\bottomrule
  	\end{tabular}%
  \end{center}
\end{table}%

\begin{table}[htbp]
	\begin{center}
		\begin{tabular}{lrrrrrrrrr}
			\multicolumn{10}{l}{\textbf{Panel B: Summary Statistics of C\_SCORE and G\_SCORE}}  \\
			\midrule
			\midrule
			& \multicolumn{1}{c}{mean} & \multicolumn{1}{c}{median} & \multicolumn{1}{c}{std. dev} & \multicolumn{1}{c}{max} & \multicolumn{1}{c}{min} & \multicolumn{1}{c}{p1} & \multicolumn{1}{c}{p25} & \multicolumn{1}{c}{p75} & \multicolumn{1}{c}{p99} \\
			\midrule
			C\_SCORE & 0.053 & 0.054 & 0.063 & 1.143 & -0.228 & -0.103 & 0.015 & 0.092 & 0.212 \\
			G\_SCORE & -0.008 & -0.008 & 0.031 & 0.188 & -1.314 & -0.091 & -0.023 & 0.009 & 0.067 \\
			\bottomrule
			\bottomrule
		\end{tabular}%
	\end{center}
		\begin{footnotesize}
			\setcounter{equation}{2}
			\begin{equation}
				\begin{split}
					EARN_{i,t} = \beta_0&+\beta_1NEG_{i,t}+\beta_2RET_{i,t}\\
					&+\beta_3RET_{i,t}\times SIZE_{i,t}+\beta_4RET_{i,t}\times MTB_{i,t}+\beta_5RET_{i,t}\times LEV_{i,t}+\beta_6RET_{i,t}\times NEG_{i,t}\\
					&+\beta_7RET_{i,t}\times NEG_{i,t}\times SIZE_{i,t}+\beta_8RET_{i,t}\times NEG_{i,t}\times MTB_{i,t}+\beta_9RET_{i,t}\times NEG_{i,t}\times 	LEV_{i,t}\\
					&+\beta_{10}SIZE_{i,t}+\beta_{11}MTB_{i,t}+\beta_{12}LEV_{i,t}\\
					&+\beta_{13}NEG_{i,t}\times SIZE_{i,t}+\beta_{14}NEG_{i,t}\times MTB_{i,t}+\beta_{15}NEG_{i,t}\times LEV_{i,t}+ \epsilon_{i,t}
				\end{split}
			\end{equation}
		
			\begin{equation}
				C\_SCORE_{i,t} = \beta_6+\beta_7SIZE_{i,t}+\beta_8MTB_{i,t}+\beta_9LEV_{i,t}
			\end{equation}
		
			\begin{equation}
				G\_SCORE_{i,t} = \beta_2+\beta_3SIZE_{i,t}+\beta_4MTB_{i,t}+\beta_5LEV_{i,t}
			\end{equation}
			
			\noindent Online Appendix Table 2 presents the key statistics in constructing C\_SCORE and G\_SCORE. Panel A presents the mean of coefficients, the mean of standard errors and the t-statistics obtained from 23 fiscal yearly regressions (Equation 3) using 10-Q sample from 1993 to 2015. C\_SCORE and G\_SCORE are calculated following Equation 4 and Equation 5 respectively. Panel B presents the summary statistics of C\_SCORE and G\_SCORE. See \hyperref[appb]{Appendix B} for variable definitions. All financial variables except returns are winsorized at 1\% and 99\% level. The mean and standard errors of coefficients and the summary statistics of C\_SCORE and G\_SCORE are consistent with \citeA{khanEstimationEmpiricalProperties2009} overall. 
		\end{footnotesize}
\end{table}%

\setcounter{equation}{2}
\begin{equation}
TEX_{i,t}=\beta_0+\beta_1\Delta DRET_{i,t-tlag}+\beta_2BN_{i,t-tlag}+\beta_3\Delta DRET_{i,t-tlag}\times BN_{i,t-tlag}+\beta_nControls_{i,t}+\epsilon_{i,t}
\end{equation}

Online Appendix Table 2 presents regression results of Equation (3) using restricted 8-K sample. All observations in restricted 8-K sample are subject to four (five) business day 8-K reporting deadline after (before) May 23rd 2004. TEX represents a vector of textual properties that consists of number of words (NW), tone (TONE) and reporting time lag (TLAG). \textit{Controls} denotes a vector of control variables including firm size (SIZE), market-to-book ratio (MTB) and leverage ratio (LEV). See \hyperref[appb]{Appendix B} and \hyperref[appc]{Appendix C} for textual and financial variable definitions. All financial variables except returns are winsorized at 1\% and 99\% level. Column 2, 4 and 6 include firm and time fixed effects and standard errors are clustered at industry level identified by 4-digit SIC codes. ***, ** and * indicate significance at the 1\%, 5\% and 10\% levels in a two-tailed test.
%\end{landscape} 

\end{document}

