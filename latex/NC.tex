\documentclass[a4paper]{article}
\usepackage{geometry}
\usepackage{graphicx}
\usepackage{epsfig}
\usepackage{amsmath}
\usepackage{indentfirst}
\usepackage{float}
\usepackage{setspace}
\usepackage{amsfonts}
\usepackage[hidelinks]{hyperref} 
\usepackage{booktabs}
\usepackage{caption}
\usepackage{subfigure}
%\usepackage{times}
\usepackage{hyperref}
\usepackage{verbatim}
\usepackage{colortbl}
\usepackage{lscape}
\usepackage[affil-it]{authblk}
\usepackage{footnotebackref}
\usepackage[nodoi]{apacite}
\AtBeginDocument{\urlstyle{APACsame}}
\AtBeginDocument{\renewcommand{\APACrefYearMonthDay}[3]{\BBOP{#1}\BBCP}}

\geometry{left=2cm,right=2cm,top=2cm,bottom=2cm}

\setlength{\parindent}{2em}

\begin{document}
	
	\title{Narrative Conservatism}

	\author{\vspace{1cm}Juan Manuel Garc\'ia Lara}
	
	\author{Beatriz Garc\'ia Osma}
	
	\author{Fengzhi Zhu%
		\thanks{Email: fzhu@emp.uc3m.es}}
	
	\affil{Department of Business Administration, Universidad Carlos III de Madrid, Spain}
	
	\date{\small Dated: \today}
	
	\maketitle
	
\thispagestyle{empty}
\begin{spacing}{2}

\begin{abstract}
	\begin{normalsize}
	\noindent
	Prior literature documents the existence of conditional and unconditional conservatism, which mainly manifest in the form of accounting numbers through the recognized line items in financial statements. However, little is known about conservatism in narrative disclosure. We study whether narrative disclosure is conservative, i.e., whether narratives respond to bad news in a more complete, news-consistent and timely manner than good news. We proxy news by the market returns and measure completeness by the number of words, news-consistency by the marginal change of narrative tone in response to news, and timeliness by the reporting time lag between news release date and disclosure filing date. Using 10-Q and 8-K filings from 1993 to 2020, we find that on average narratives have more number of words, greater marginal change of tone and shorter reporting time lag in response to bad news relative to good news, consistent with narratives being conservative. Moreover, we show that firms emphasize bad news more than good news via 10-Q filings, and are more likely to report larger number of 8-K filings and 8-K items per day in response to bad news relative to good news. Finally, we document that narrative conservatism is more (less) pronounced in firms with lower (higher) conditional conservatism, in explanatory (supplementary) narratives, in voluntary (mandatory) disclosure, and in settings where managers have more (less) incentives to disclose bad news.
	%[auxiliary analysis]% 
	%\cite{acharyaEndogenousInformationFlows2011, guttmanNotOnlyWhat2014, marinovicNoNewsGood2016}
	\\

	\noindent
	\textbf{Keywords}: \textit{narrative disclosure; conservatism; tone; timeliness; news-consistency; textual analysis}
	\end{normalsize}
\end{abstract}

\clearpage

\setcounter{page}{1}
\section{Introduction}
% motivation
Prior literature documents the existence of conditional and unconditional conservatism, which are measured by the recognized line items in financial statements. %recognition conservatism.\footnote{In this paper, we use the term ``recognition conservatism" to denote the union of conditional and unconditional conservatism, whose measurements are both derived from recognized line items in financial statements.} 
In this paper, we add to this prior work by defining and providing evidence of narrative conservatism. We define narrative conservatism as \textit{narratives responding to bad news in a more complete, news-consistent and timely manner than good news}. This definition builds on the work of \fullciteA{basuConservatismPrincipleAsymmetric1997}, extending the notion of accounting conservatism to narratives. Narrative conservatism is of interest for at least two reasons. First, narrative disclosure takes up a dominant space in the corporate filings.\footnote{For example, Apple Inc.'s 2019 Annual Report contains only 3 pages of numerical summary in the financial statements and around 15 pages of other tables and figures, among a total of 64 pages. The rest of the report is devoted to narratives including risk factors, management discussion and analysis (MD\&A), notes to financial statements, among other things. Also, over the past 20 years, the average number of pages in annual reports devoted to footnotes and MD\&A has quadrupled \cite{eyPointNowTime2012}.} Investors' perceptions of firm performance and their subsequent decision-making processes are likely to be shaped by narrative disclosure \fullcite{liTextualAnalysisCorporate2010}. Therefore, understanding the properties of narrative disclosure and their economic implications is essential for market participants and regulators. Second, studying narrative conservatism complements our current understanding of accounting conservatism. If recognition is merely one of the presentation formats of financial reporting, then our extant knowledge of conditional and unconditional conservatism is a partial view of accounting conservatism. Yet, we know little about whether narrative disclosure is conservative, or whether and how narrative conservatism interacts with the other two forms of conservatism.

%[Theory]
Prior literature distinguishes between recognition and disclosure. Recognition is the depictions in numbers on the face of the financial statements, and disclosure is commonly viewed as the display in the notes and supporting schedules that accompany financial statements \fullcite{schipperRequiredDisclosuresFinancial2007}. The two forms of financial reporting are subject to different reporting requirements. The Financial Accounting Standards Board (FASB) explicitly specifies a set of recognition criteria while allowing for more flexibility in disclosure \fullcite{fasbStatementFinancialAccounting1984}. This flexibility paves a way for the supplementary role of disclosure---disclosing information that cannot be recognized due to the failure to meet one or more of the recognition criteria. The other explanatory role of disclosure is to provide detailed background information of recognized numbers in financial statements \cite[par. 7]{fasbStatementFinancialAccounting1984}. 

Extensive research has been conducted on conditional and unconditional conservatism, focusing on the properties of the income statement and balance sheet items respectively. Conditional conservatism captures the asymmetric response of \textit{earnings} to positive and negative economic news, and unconditional conservatism manifests as a systematic understatement of \textit{net book value of assets} due to predetermined aspects of the accounting process \cite<e.g.,>[]{beaverConditionalUnconditionalConservatism2005}. However, conservatism in narrative disclosure is less studied. We interpret narrative conservatism as narratives responding to bad news in a more complete, news-consistent and timely manner than good news. We focus on these three properties of narratives because they correspond to the following three independent dimensions of disclosure: quantity, content and timeliness. Quantity refers to how much information is provided. Content relates to what and how information is provided via disclosure. Timeliness associates with how much time it takes to provide the information.

In theory, narrative disclosure may or may not be conservative. Prior studies outline several incentives for managers to disclose or withhold bad news \cite{skinnerWhyFirmsVoluntarily1994, skinnerEarningsDisclosuresStockholder1997, kothariManagersWithholdBad2009, baoManagersDiscloseWithhold2019}, which also influence managers' decisions on whether, and to what extent they respond to good news and bad news asymmetrically in narrative disclosure. The three major motives for managers to disclose bad news are: (a) to obtain lower financing costs resulting from reduced information asymmetry, (b) to reduce litigation risk due to the failure to disclose bad news in a timely manner and (c) to manipulate firm performance downwards prior to stock option grant. The two major motives for managers to withhold bad news are: (a) to prevent reputation loss that damages managers' future career and (b) to avoid personal wealth loss linked to performance-based compensation. In summary, given various managerial incentives, whether on average managers tend to disclose or withhold bad news relative to good news remains an empirical question.

%[research design]
To empirically test whether narrative disclosure is conservative, we adopt the following three measurements for completeness, news-consistency and timeliness respectively. We proxy disclosure completeness by the number of total words in corporate filings. Prior literature documents that managers use lengthier reports to disclose more information, which reduces information asymmetry and lowers cost of capital \fullcite{leuzDisclosureCostCapital2009}. We interpret news-consistency as deploying a positive narrative tone in response to good news and a negative narrative tone in response to bad news, and we measure the degree of news-consistency by the marginal change of narrative tone in response to good news and bad news. If narrative disclosure is conservative, the marginal change of tone in narrative disclosure should be greater in response to bad news than good news. We evaluate timeliness by the reporting time lag between news release and disclosure filing dates. The smaller the reporting time lag is, the timelier the narrative disclosure is. Overall, we posit that if narrative disclosure is conservative, then in response to bad news relative to good news, it should have more number of words, greater marginal change of tone and shorter reporting time lag. In terms of news measurement, we follow \citeA{basuConservatismPrincipleAsymmetric1997} and use stock returns as a proxy for news.

%[data and findings]
We use two types of filings required by the U.S. Securities and Exchange Commission (SEC) for all public companies---10-Q and 8-K filings---as our narrative disclosure corpora. To begin with, we retrieve 10-Q and 8-K filings from the Electronic Data Gathering, Analysis, and Retrieval system (EDGAR) from 1993 to 2020.\footnote{Since the SEC adopted the rule of electronical submission for corporate filings in 1993, data coverage in the first year of EDGAR implementation is low \fullcite{gaoInformingMarketEffect2020}. We repeat our main analyses using data from 1994 onward, and our main results sustain.} Next, we apply the financial sentiment word list developed by \fullciteA{loughranWhenLiabilityNot2011} (LM hereafter) to count the number of positive, negative, uncertainty, litigious and modal words in each corporate filing extracted from EDGAR. Finally, we construct the tone measure as the number of net positive words per thousand total words and the reporting time lag as the number of days elapsed between news release date and document reporting date. Our final 10-Q (8-K) sample consists of 91,607 (119,616) firm-quarter (firm-day) observations from 5,250 (8,261) unique firms. Our empirical results suggest that both 10-Q and 8-K filings have more number of words and greater marginal change in tone in response to bad news relative to good news, consistent with narrative disclosure being conservative. In terms of reporting time lag, we find that 8-K (10-Q) filings respond more (less) timely to bad news than good news. We argue that because 10-Qs are not as timely as 8-Ks, and because 10-Qs not only contains narrative disclosure but also financial statements, the reporting time lag of 10-Qs does not strictly proxy for narrative timeliness. Therefore, we interpret the 8-Ks results regarding narrative timeliness as evidence that narrative disclosure responds more timely to bad news relative to good news, consistent with narrative disclosure being conservative.

%[auxiliary analysis]
We perform four sets of auxiliary analyses to examine the interaction between narrative and conditional conservatism and the determinants of narrative conservatism. First, we explore the interaction between narrative and conditional conservatism. We construct a firm-quarter measure of conditional conservatism following \citeA{khanEstimationEmpiricalProperties2009} and we divide the 10-Q sample into low and high conditional conservatism subsamples. We find that firms in low conditional conservatism subsample demonstrate high narrative conservatism, consistent with managers viewing narrative conservatism and conditional conservatism as substitutes in financial reporting. Second, we study how the role of narratives affects narrative conservatism. We conjecture that supplementary narratives may be more conservative than explanatory narratives because managers have more discretion on the format and the content of the former. We choose the MD\&A (notes to financial statements) section as the representative of the explanatory (supplementary) narratives and we repeat our main analysis using the textual measures constructed based on the MD\&A and the note sections. We find that narratives in the MD\&A section are more conservative than those in the note section, consistent with supplementary narratives being more conservative than explanatory narratives. Third, we investigate how narrative conservatism varies in voluntary and mandatory disclosure. We posit that if managers choose to be conservative in narrative disclosure voluntarily, then the voluntary disclosure should be more conservative than the mandatory disclosure, as managers have more freedom to express in terms of the content and the rhetoric in voluntary disclosure. We follow \citeA{heMeasuringDisclosureUsing2020} to measure voluntary and mandatory disclosure using 8-K filings that contain voluntary and mandatory 8-K items, and we find that voluntary disclosure is more conservative than mandatory disclosure. Fourth, we inspect how managerial incentives affect narrative conservatism in three settings where managers have strong incentives to disclose or withhold bad news. Prior studies document that when executives anticipate large stock option grants and when firms are under high litigation risk, managers have more incentive to disclose bad news \cite{aboodyCEOStockOption2000, skinnerWhyFirmsVoluntarily1994, skinnerEarningsDisclosuresStockholder1997}. However, when firms are undergoing seasoned equity offering, managers have more incentive to withhold bad news \cite{teohEarningsManagementUnderperformance1998}. Similarly, we hypothesize and find evidence that the narrative disclosure of firms in the first two settings is more conservative, whereas that of firms in the third setting is less conservative.

%[contribution]
Our study contributes to the accounting literature in four aspects. First, we fill the gap in conservatism literature by documenting the existence of narrative conservatism. Second, we add to the literature on the distinction and the interaction between recognition and disclosure \fullcite{schipperRequiredDisclosuresFinancial2007, barthMarketEffectsRecognition2003, aboodyRecognitionDisclosureOil1996}. We find that firms with low (high) conditional conservatism tend to be more (less) conservative in narratives, suggesting that managers view recognition and disclosure as substitutes to some extent. Third, we provide novel evidence to the debate regarding whether managers withhold bad news. Prior research uses a wide variety of disclosure proxies to study managers' tendency to disclose or withhold bad news \cite{skinnerWhyFirmsVoluntarily1994, skinnerEarningsDisclosuresStockholder1997, kothariManagersWithholdBad2009, segalAreManagersStrategic2016, baoManagersDiscloseWithhold2019}. We use the textual properties of SEC filings as novel proxies for disclosure, and our results support the idea that firms on average disclose bad news voluntarily. Fourth, we relate to the broader literature on the informativeness of SEC filings. A stream of literature studies the market reactions to 8-Ks \fullcite{carterRelevanceForm8K1999, pinskerHasFirmsForm2006, lermanNewForm8K2010} and 10-K/Qs \fullcite{alfordExtensionsViolationsStatutory1994, liAnnualReportReadability2008, liInformationContentForwardLooking2010}. Instead, we use the market returns as indication of good and bad news for the firms and study the behavior of corporate narrative disclosure.

The rest of the study structures as follows. Section 2 reviews prior literature on recognition, disclosure, conditional and unconditional conservatism, and develops the main hypotheses. Section 3 outlines the empirical design and data selection process. Section 4 presents the main results of 10-Q and 8-K samples. Section 5 reports auxiliary analyses and Section 6 concludes.

\section{Theoretical Framework}
\subsection{Recognition and Disclosure}
% What are recognition and disclosure?
A stream of literature studies the distinctions between \textit{recognition} and \textit{disclosure} and their respective or combined effectiveness in financial reporting \fullcite{schipperRequiredDisclosuresFinancial2007, barthMarketEffectsRecognition2003, aboodyRecognitionDisclosureOil1996}. \citeA[p. 301]{schipperRequiredDisclosuresFinancial2007} defines recognition as ``depictions in numbers with captions on the face of the financial statements", and disclosure as ``display in the notes and supporting schedules that accompany financial statements".\footnote{Statement of Financial Accounting Concepts No. 5---\textit{Recognition and Measurement in Financial Statements of Business Enterprises} formally defines \textit{recognition} as ``the process of formally recording or incorporating an item into the financial statements of an entity as an asset, liability, revenue, expense, or the like. Recognition includes depiction of an item in both words and numbers, with the amount included in the totals of the financial statements" \fullcite[par. 6]{fasbStatementFinancialAccounting1984}, but does not define \textit{disclosure}. Due to the absence of a conceptual definition of disclosure, prior literature on disclosure commonly interpret disclosure as any display that is not in numbers. However, this interpretation may partially overlap with the FASB definition of recognition, which states that recognition also includes words. As \citeA[p. 302]{schipperRequiredDisclosuresFinancial2007} notes: ``...both in analytical modeling and in developing financial reporting concepts, it is difficult to distinguish between recognized and disclosed information".} In this study, we adopt the same notion of recognition as in \citeA{schipperRequiredDisclosuresFinancial2007}, and we use the terms \textit{narratives, narrative disclosure} or \textit{disclosure} interchangeably to denote all textual disclosures presented in SEC filings, including notes to financial statements, supplementary information and other means of financial reporting such as the MD\&A section. Examples of recognition are revenue, expense, asset and liability expressed in currency units on the face of financial statements, which are also known as line items in financial statements. 

% What are recognition criteria and the role of narratives
Disclosure and recognition are subject to different reporting requirements. For an economic item to be recognized in financial statements, a set of recognition criteria needs to be satisfied. First, the item must meet the definition of an element of financial statements (definition criterion). Second, the item must have a relevant attribute measurable with sufficient reliability (measurability criterion). Third, the information about the item must be capable of making a difference in user decisions (relevance criterion). Fourth, the information must be representationally faithful, verifiable, and neutral (reliability criterion) \cite{fasbStatementFinancialAccounting1984}. However, disclosure is more flexible because it can be deployed to disclose information that fails to meet certain recognition criteria \cite[par. 7b]{fasbStatementFinancialAccounting1984}. 

Narrative disclosure plays an essential role in financial reporting, as \fullciteA[par. 7, CON5-7]{fasbStatementFinancialAccounting1984} states:
\begin{singlespace}
	\indent \textit{Although financial statements have essentially the same objectives as financial reporting, some useful information is better provided by financial statements and some is better provided, or can only be provided, by notes to financial statements or by supplementary information or other means of financial reporting.}
\end{singlespace}
Concretely, narrative disclosure has two fundamental functions. First, narratives can be supplementary to recognition, conveying information about corporate events that cannot be recognized due to the inability to meet one or more of the four recognition criteria. In terms of good news, because the good news requires higher verification in recognition, firms may convey good news via disclosure rather than recognition. For instance, under U.S. General Accepted Accounting Principle (GAAP), long-lived tangible and intangible assets cannot be revaluated upwards. So when the market price of the firms' long-lived tangible and intangible assets goes up, the firms cannot recognize the gain in balance sheet, but they may discuss the price movement in SEC filings. In terms of bad news, despite that the bad news already requires lower verification than good news, it may still not be fully recognized. For example, although firms could create a provision for the expected payments that results from a potential lawsuit in the future, they cannot recognize the associated reputation losses since it is extremely difficult to obtain a reliable estimate that can be verified subsequently. However, firms may discuss the likelihood and the expected impact of entering into a lawsuit in the risk factors or the MD\&A section of the SEC filings. Another example is that internally developed intangible assets cannot be capitalized in the balance sheet, so they cannot be impaired when bad news arrives. However, firms may discuss the impact of news associated with these intangible assets in SEC filings. In sum, firms may use narrative disclosure to inform investors about the immeasurable, and thus irrecognizable impact of various corporate events and fulfill their obligation of providing relevant financial information to investors. Second, narratives can be explanatory to recognition, explaining the line items in financial statements. \fullciteA[footnote 4, CON5-7]{fasbStatementFinancialAccounting1984} gives several examples on the explanatory role of notes to financial statements:

\begin{singlespace}
	\indent \textit{For example, notes provide essential descriptive information for long-term obligations, including when amounts are due, what interest they bear, and whether important restrictions are imposed by related covenants. For inventory, the notes provide information on the measurement method used---FIFO cost, LIFO cost, current market value, etc. For an estimated litigation liability, an extended discussion of the circumstances, counsel's opinions, and the basis for management's judgment may all be provided in the notes. For sales, useful information about revenue recognition policies may appear only in the notes (FASB Statement No. 47, Disclosure of Long-Term Obligations; ARB No. 43, Chapter 4, ``Inventory Pricing", statement 8; FASB Statement No. 5, Accounting for Contingencies, par. 10; and APB Statement 4, par. 199)}.
\end{singlespace}

\subsection{Conditional, Unconditional and Narrative Conservatism: Definition}
% What are conditional, unconditional and narrative conservatism.
Extant literature studies accounting conservatism in two forms: conditional and unconditional conservatism \cite{beaverConditionalUnconditionalConservatism2005}. Conditional conservatism manifests as ``accountants' tendency to require a higher degree of verification to recognize good news as gains than to recognize bad news as losses" \fullcite[p. 7]{basuConservatismPrincipleAsymmetric1997},\footnote{\citeA{basuConservatismPrincipleAsymmetric1997} does not use the terms conditional or unconditional conservatism. Here we quote \citeA{basuConservatismPrincipleAsymmetric1997} only to describe the manifestation of the two forms of conservatism, which are now labeled as conditional and unconditional conservatism.} and is typically measured by the asymmetric response of earnings to positive and negative stock returns. Examples of conditional conservatism include allowing for \textit{impairment}, i.e., writing down by the amount of loss incurred, but not \textit{revaluation}, i.e., writing up by the difference between market price and carrying amount, for long-lived tangible and intangible assets under U.S. GAAP, and lower of cost or market accounting (LCM) for inventory under U.S. GAAP or lower of cost or net realizable value accounting (LCNRV) under International Financial Reporting Standards (IFRS). Unconditional conservatism manifests as ``accountants' preference for accounting methods that lead to lower reported values for shareholders' equity" \fullcite[p. 8]{basuConservatismPrincipleAsymmetric1997}. Examples of unconditional conservatism include immediate expensing, rather than capitalizing, of research and development (R\&D hereafter) costs, and the use of accelerated depreciation for property, plant and equipment \fullcite{beaverConditionalUnconditionalConservatism2005}. The measurements of the two types of conservatism---earnings and shareholders' equity, are both recognized line items in financial statements. %Thus, we label the union of the two forms of conservatism as \textit{recognition conservatism}. 

% What is narrative conservatism and why focus on the three constructs?
Comparing to the extensive research on conditional and unconditional conservatism, little is known about conservatism in narratives. We define narrative conservatism as \textit{narratives responding to bad news in a more complete, news-consistent and timely manner than good news}. We focus on these three properties of narratives because they correspond to the following three independent dimensions of disclosure: quantity, content and timeliness. Quantity refers to how much disclosure is provided, and we interpret completeness as firms providing more disclosure. Content relates to what and how information is provided via disclosure, and we interpret news-consistency as firms using positive tone in response to good news and negative tone in response to bad news in narrative disclosure. Timeliness associates with how much time it takes to provide the disclosure, and we interpret timely disclosure as information issued after a short period of time since the realization of the underlying event. Each firm can be conservative or not independently along each dimension. In this study, we assess the overall level of narrative conservatism along each dimension independently for our sample firms during the sample periods. 

% Why firms may or may not be conservative?
In theory, whether narrative conservatism exists is not clear. Narrative conservatism requires firms to disclose bad news in a more complete, news-consistent and timely manner than good news. However, managers may or may not disclose bad news when facing different incentives. On the one hand, managers may disclose bad news for three motives. First, managers may disclose more complete information, including bad news, to reduce financing costs. Extant theoretical work establishes that complete disclosure reduces information asymmetry and lowers cost of capital \cite<e.g.,>[]{diamondDisclosureLiquidityCost1991, baimanRelationCapitalMarkets1996}. \fullciteA{leuzEconomicConsequencesIncreased2000} show that after German firms switched from the German reporting regime to an international reporting regime which requires an increased level of disclosure, their information asymmetry is reduced leading to lower cost of capital. \citeA{leuzDisclosureCostCapital2009} find that firms respond to the adverse shock created by the Enron scandal by increasing the length of disclosures in 10-K filings, which reduces firms' cost of capital subsequently. Second, litigation pressure induces managers to disclose bad news more promptly than good news \fullcite{skinnerWhyFirmsVoluntarily1994, kasznikWarnNotWarn1995, skinnerEarningsDisclosuresStockholder1997}. Financial information users have greater incentives to sue the manager when bad news is not disclosed than when good news is not disclosed. This asymmetric litigation pressure potentially stems from users' asymmetric preference for unexpected gains and losses. Therefore, firms may predisclose bad news to avoid being sued or to minimize the costs of resolving the litigation that inevitably follows the disclosure of bad news. Third, the personal career and compensation incentives also play a role in managers' decisions to disclose bad news. \citeA{skinnerWhyFirmsVoluntarily1994} argues that managers may face reputational costs if they fail to disclose bad news. \fullciteA{yermackGoodTimingCEO1997} and \fullciteA{aboodyCEOStockOption2000} document that managers release bad news immediately prior to stock option grant dates to lower the option strike price. On the other hand, managers may withhold bad news for at least two reasons. First, managers may avoid disclosing bad news for career concerns, in expectation to bury bad news with subsequent corporate events. Significant bad news affects managerial career negatively by deterring promotion, limiting employment opportunity in the outside job market and potentially leading to termination. Second, performance-based managerial compensation also demotivates managers to disclose bad news. Bad news disclosure may lead to bonus shrink and stock price decline, reducing managers' personal wealth especially when they are compensated with shares or options \cite{kothariManagersWithholdBad2009}. In sum, while managers have a natural tendency to disclose good news, they face different incentives when it comes to the decision of disclosing or withholding bad news. Given the various managerial incentives, whether narratives on average responds to bad news in a more complete, news-consistent and timely manner than good news remains an empirical question. 

% Empirical proxy choices
To investigate this question, we construct three measurements for disclosure completeness, news-consistency and timeliness. We measure disclosure completeness by the total number of words of SEC filings. The Conceptual Framework requires complete disclosures to include ``...all information necessary for a user to understand the phenomenon being depicted, including all necessary descriptions and explanations" \fullcite[QC12]{fasbConceptualFrameworkFinancial2018}. Hence, more complete disclosure should be lengthier, which allows managers to elaborate on detailed explanations of firm performance \cite{leuzDisclosureCostCapital2009}.\footnote{We use number of words instead of number of pages, which is used in \citeA{leuzDisclosureCostCapital2009}, as the proxy for disclosure completeness for two reasons. First, for pure texts, these two measures are almost equivalent, or at least are monotonic transformations of each other, given a roughly constant number of words per page. Second, for financial reports with graphs and tables, the number of words is a more precise measure for narrative disclosure, because it counts the length of narratives only. However, the number of pages may be enlarged mechanically by graphs, tables, and even space lines embedded in the tables, which are not the focus of this study.} However, we are aware of a strand of literature suggesting that narrative disclosure is less informative when it is less readable \fullcite{liAnnualReportReadability2008, loEarningsManagementAnnual2017, loughranMeasuringReadabilityFinancial2014}, and because lengthier document is often less readable, it may appear counter-intuitive to proxy completeness with the document length. We provide two explanations for this measurement. First, some studies point out that instead of managers' intentional obfuscation, lower readability may result from the fact that bad news is inherently more complex and therefore needs more explanations \fullcite{bloomfieldDiscussionAnnualReport2008}, and that there is incremental information content embedded in complex narratives \fullcite{busheeLinguisticComplexityFirm2018}. Therefore, lower readability does not necessarily imply lower narrative disclosure quality. Second, although somewhat correlated, the document length and the readability are essentially two different constructs. In a binary classification context, texts can be long or short, readable or irreadable independently. Specifically in measuring information completeness, the document length is an appropriate construct because ``including all necessary descriptions and explanations" \fullcite[QC12]{fasbConceptualFrameworkFinancial2018} in narrative disclosure will inevitably increases the document length. Thus, if narrative disclosure is conservative, we expect it to be lengthier, i.e., contains more number of words, in response to bad news. We formulate our first hypothesis as follows:

\begin{center}
	\textbf{H1:} The total number of words in narrative disclosure is greater in response to bad news than good news.
\end{center}

We proxy the sentiment spectrum in narrative disclosure by linguistic tone and measure the degree of news-consistency by the marginal change of tone in response to unit increase (good news) or decrease (bad news) in stock market returns. News-consistency requires the marginal change to be positive, so that the positive tone responds to good news and the negative tone to bad news. Furthermore, if the narrative tone is more consistent for bad news, it implies greater marginal change of tone in response to bad news than good news. That is, the change in narrative tone should be more negative in response to bad news than it should be positive in response to good news, given the same magnitude of news impact. Narrative conservatism creates a downward bias in narrative disclosure conditional on the nature of news: either bad news is emphasized or good news is attenuated, or both.\footnote{We provide the following numerical example to illustrate the concept of marginal change of tone. Suppose that there is 1\% increase (good news) and 1\% decrease (bad news) in stock return, which should move tone upwards and downwards by 1\% in theory if the response of narrative tone is neutral, i.e., tone equally responding to good and bad news. However, in the presence of narrative conservatism, three situations may happen: (1) bad news emphasis: in response to bad news, tone decrease by 1.2\% and in response to good news, tone increase by 1\%; (2) good news attenuation: in response to bad news, tone decrease by 1\% and in response to good news, tone increase by 0.8\%; (3) a mix of both: in response to bad news, tone decrease by 1.2\% and in response to good news, tone increase by 0.8\%. In all cases, the marginal change of tone in response to bad news is greater in magnitude than that in response to good news. Therefore, under narrative conservatism, the marginal change of tone in narrative disclosure is greater in response to bad news than good news.} Thus, we formulate our second hypothesis as follows:
\begin{center}
	\textbf{H2}: The marginal change of tone in narrative disclosure is greater in response to bad news than good news.
\end{center}

We measure timeliness by the reporting time lag, defined as the number of days elapsed between the news release date and the filing date of the narrative disclosure. In line with the interpretation of timeliness in the Conceptual Framework that ``Timeliness means having information available to decision makers \textit{in time} to be capable of influencing their decisions" \fullcite[QC29, emphasis added]{fasbConceptualFrameworkFinancial2018}, the shorter is the reporting time lag, the timelier is the narrative disclosure. If narrative disclosure is conservative, we expect it to be timelier, i.e., has shorter reporting time lag, in response to bad news. Thus, we formulate our third hypothesis as follows:

\begin{center}
	\textbf{H3}: The reporting time lag of narrative disclosure is shorter in response to bad news than good news.
\end{center}

\begin{comment}
\subsection{Conditional, Unconditional and Narrative Conservatism: Usefulness}
% conservatism role: stewardship or neutrality
The controversy regarding whether conservatism is a desirable property that enhances the usefulness of financial reporting persists. Traditionally, the usefulness of accounting information can be assessed in terms of how well it serves each of the two objectives of accounting---valuation and stewardship \fullcite{cascinoUsefulnessFinancialAccounting2017}. The valuation objective is to ``provide financial information about the reporting entity that is useful to existing and potential investors, lenders, and other creditors in making decisions about providing resources to the entity \cite[OB2]{fasbConceptualFrameworkFinancial2018b}". For financial information to be useful for valuation objective, it must be relevant and faithfully represent what it purports to represent. Faithful representation further requires neutrality, which means that a depiction must be ``not slanted, weighted, emphasized, deemphasized, or otherwise manipulated to increase the probability that financial information will be received favorably or unfavorably by users". The stewardship objective is to assess ``how efficiently and effectively the entity's management and governing board have discharged their responsibilities to use the entity's economic resources \cite[OB4]{fasbConceptualFrameworkFinancial2018b}". Although not separately stated in the Conceptual Framework as one primary purpose of financial reporting, the stewardship role of accounting dates back several millennia and has been one of the main reasons for the existence of accounting \cite{lennardStewardshipObjectivesFinancial2007, murphyDiscoursesSurroundingEvolution2013, pelgerPracticesStandardsettingAnalysis2016}. On the one hand, conservatism contradicts the valuation role of accounting by introducing downward bias in financial reporting and thus weakening its ability to faithfully represent firm performance. For example, unconditional conservatism encourages firms to anticipate and recognize losses before their realization, resulting in a systematic downward bias in asset valuation \fullcite<e.g.,>[]{wattsPositiveAccountingTheory1986}. Conditional conservatism requires higher verification for good news to be recognized than bad news, leading to asymmetric timeliness in gain and loss recognition in earnings \fullcite<e.g.,>[]{basuConservatismPrincipleAsymmetric1997}. On the other hand, conditional conservatism enhances the stewardship role of accounting by improving contract efficiency. For example, in compensation contracting conditional conservatism limits managers' ability to overstate earnings and thus maximizing personal wealth at the expense of other claimholders \fullcite<e.g.,>[]{wattsConservatismAccountingPart2003}.

Aligned with the prior literature on the usefulness of conservatism, we argue that more complete, news-consistent and timely disclosure of bad news relative to good news enhances contract efficiency [specific hypotheses to be developed]. However, we do not make claims about the valuation role of narrative conservatism.

\end{comment}

\section{Research Design}
\subsection{Narrative Disclosure Corpora and News Proxy} \label{sec3.1}
% What are 10-Q and 8-K?
In this paper, we study narrative disclosure using 10-Q and 8-K filings from EDGAR database as our corpora. The form 10-Q is a comprehensive report that depicts quarterly firm performance, and it must be filed by all public companies to the SEC within 40 (for accelerated filers) or 45 days (for all other registrants) after fiscal quarter-end, according to Section 13 or 15(d) of the Securities Exchange Act of 1934. The form 8-K is a report that all public firms must file to the SEC to notify investors about material events or changes in the company, where each type of event is classified as an \textit{8-K item}. \hyperref[appd]{Appendix D} provides a full list of 8-K items. The 8-K items are listed in two distinct formats before and after August 23 of 2004 because on that date the SEC adopted an 8-K reform, which made three amendments to prior 8-K guidance: expanding the scope of the events subject to Form 8-K disclosure, creating a new topical format, and shortening the filing deadline \cite{secFinalRuleAdditional2004, lermanNewForm8K2010}. All 8-K items except the Other Events, the Regulation FD Disclosure and the Results of Operations and Financial Condition are mandatory items.\footnote{We follow \citeA{heMeasuringDisclosureUsing2020} and classify the Results of Operations and Financial Condition and the Regulation FD Disclosure as voluntary disclosure items, because the triggering event of these two items is the firm's voluntary disclosure of material events. \citeA{lermanNewForm8K2010} classify the two items as ``semi-voluntary" based on the same reason. The Other Events is voluntary following the filing requirement in \citeA{secFinalRuleAdditional2004}. } For mandatory items, the 8-K filings must be filed upon the occurrence of any one or more events pertaining to the mandatory 8-K items within a specific filing deadline. The filing deadline for mandatory items ranged from five to fifteen days after the occurrence of the event in the late 1980s, and was shortened to four business days following the occurrence of the event after August 23 of 2004 \cite{lermanNewForm8K2010}.

% 10-Q and 8-K comparing to other disclosure channels
Firms can issue narrative disclosure via multiple channels, such as social media and press, conference calls and annual reports etc. We focus on the 10-Q and 8-K filings in this study for three motives. First, the 10-Q and 8-K filings are both firm-issued filings that are required for all public companies. Their contents are under SEC scrutiny and biased reporting increases litigation risk \fullcite{rogersDisclosureToneShareholder2011}. Therefore, 10-Q and 8-K filings provide higher credibility comparing to firm-issued disclosures via social media and press. Second, the 10-Q and 8-K filings are highly scripted and have higher reporting threshold comparing to conference calls, meaning that corporate events need to have a moderate impact on firm operations in order to be discussed in the 10-Q and 8-K filings \fullcite{hassanFirmLevelPoliticalRisk2019}. Hence, we filter out less relevant events and concentrate on the ones with material impact by using 10-Q and 8-K reports. Third, the 10-Q and 8-K filings are timelier than the 10-K filings, i.e., annual reports. Using the 10-K filings, managers can only bundle information acquired during the whole fiscal year and make summarized responses to all events in one single report at year-end. Given that one of our goals is to examine the timeliness of narrative disclosure, the 10-K filings cannot provide sufficient time variation in good and bad news responses, and thus they are not appropriate text source for the purpose of this study.

% Advantage and disadvantage of 10-Q and 8-K?
There is heterogeneity between 10-Qs and 8-Ks as well. First, the 10-Qs provide more variation and diversity than 8-Ks in terms of content. The 10-Qs contain sections such as notes to financial statements and MD\&A, where managers can discuss the economic implications of significant corporate events and issue forward-looking statements, whereas the 8-Ks only offer event descriptions in a relatively standardized format. Moreover, the 8-Ks are shorter, i.e. contain fewer words than the 10-Qs on average. These features imply that the 10-Qs are more flexible in content, in the sense that managers have more discretion on what and how to disclose in the 10-Qs, which provides us with more variation in linguistic tone than 8-K filings. Thus, our analyses and conclusions regarding linguistic tone are mainly conducted on and drawn upon the 10-Q sample. 

Second, the 10-Qs are not as timely as the 8-Ks. The 10-Qs are filed only once every quarter, so regardless of managerial reporting incentive, the 10-Qs cannot be as timely as the 8-Ks in responding to unexpected corporate events, especially for those events that happen during early days in a fiscal quarter. This is testified by the following excerpt extracted from SEC's announcement of the 8-K reform in 2004:

\begin{singlespace}
	\indent \textit{Under the previous Form 8-K regime, companies were required to report very few significant corporate events. The limited number of Form 8-K disclosure items permitted a public company to delay disclosure of many significant events until the due date for its next periodic report. During such a delay, the market was unable to assimilate such undisclosed information into the value of a company's securities. The revisions that we adopt today will benefit markets by increasing the number of unquestionably or presumptively material events that must be disclosed currently. They will also provide investors with better and more timely disclosure of important corporate events.}
	\begin{flushright}
		\cite<Final Rule: Additional Form 8-K Disclosure Requirements and Acceleration of Filing Date,>[]{secFinalRuleAdditional2004}
	\end{flushright}
\end{singlespace}

Furthermore, besides narrative disclosure, the 10-Qs also contain quarterly financial statements, so the reporting time lag of the 10-Qs does not strictly measure the timeliness of narrative disclosure solely, but the timeliness of recognition and disclosure in aggregation. Considering these features, our analyses and conclusions regarding timeliness are mainly conducted on and drawn upon the 8-K sample.

% News measure and why stock returns
Following \citeA{basuConservatismPrincipleAsymmetric1997}, we measure good and bad news with stock returns. In efficient market, stock returns incorporate public and private information in a timely manner and therefore the positive and negative returns are indicative of good and bad news of firms. %[cite and development needed]
Firms respond to the news by disclosing detailed information of the events that cause changes in stock returns via the 10-Q or 8-K filings.
%\footnote{We use stock returns measured at date \textit{prior or equal to} the 10-Q and 8-K report filing date to classify the underlying corporate events as good or bad news. 
%This empirical design ensures that any changes in stock returns are not caused by the filing of corporate report, but by other corporate events that happen before the report filing date. 
%However, we do not require that the filing of report is caused by the change in market returns or vice versa. In fact, our dataset contains both cases. The core assumption here is that the SEC filings and the market returns movements are related to the same corporate event, which enables us to distinguish the nature of the event and investigate how narrative disclosure responds to such event.}

\subsection{Model Specification}
\subsubsection{Form 10-Q}
The 10-Q filings are quarterly reports that are filed to SEC within 40 or 45 days after fiscal quarter-end. Given their stable periodicity, we design the following model to explore how the 10-Q filings respond to good versus bad news. 
\begin{equation} \label{eq1}
TEX_{i,t}=\beta_0+\beta_1QRET_{i,t}+\beta_2NEG_{i,t}+\beta_3QRET_{i,t}\times NEG_{i,t}+\beta_nCONTROLS_{i,t}+\epsilon_{i,t}
\end{equation}

In Equation (1), QRET denotes the quarterly market-adjusted stock returns. NEG is an indicator for bad news, which is set to 1 if QRET is negative and 0 otherwise. CONTROLS represents a vector of control variables, which includes firm size (SIZE), market-to-book ratio (MTB) and leverage ratio (LEV). Detailed variable definitions are provided in \hyperref[appb]{Appendix B} and \hyperref[appc]{Appendix C}. We control for these three firm characteristics to alleviate the omitted variable bias, as these three factors can affect stock returns and firm narrative disclosure simultaneously \fullcite{liInformationContentForwardLooking2010, huangToneManagement2014}. %[specific explanation on each of the controls and cite needed]
Notice that the right-hand side of Equation (1) resembles the conditional conservatism model in \citeA{basuConservatismPrincipleAsymmetric1997}. Our model differs from the Basu model in replacing earnings with three textual variables to examine the responses of narrative disclosure to positive versus negative market returns. Specifically, TEX represents a vector of textual properties that consists of number of words (NW), tone (TONE) and reporting time lag (TLAG). NW is calculated as the natural logarithm of one plus the count of total words. TONE is defined as number of net positive words per thousand total words, and is calculated as total number of positive words minus the sum of total number of negative words and total number of negations, and multiply the previous result by one thousand for the ease of interpretation. We follow LM and count negations as cases where negation words\footnote{Negation words include: no, not, none, neither, never, nobody \fullcite{tottieNegationEnglishSpeech1991}.} occur within four or fewer words from a positive word. By taking negations of positive words into consideration in calculating tone, we control for the fact that it is common for firms to frame bad news using negated positive words, i.e., use ``did not increase" instead of ``decrease". We do not control for negations of negative words because firms rarely communicate good news with negated negative words, i.e, use ``did not fail" instead of ``succeeded". TLAG is defined as number of days elapsed between the fiscal quarter-end and 10-Q filing date in EDGAR. One concern of the TLAG measurement for reporting timeliness is that the length of reporting time lag may not be fully controlled by managers, and thus cannot accurately capture the discretionary reporting timeliness of firms. Prior auditing literature suggests that a set of auditor characteristics contributes to unexpected audit report lag \fullcite{knechelAdditionalEvidenceAudit2001, bamberAuditStructureOther1993}, which consequently leads to filing delay in audited financial reports. However, because audit for quarterly filings is not mandated by law, and due to the high auditing cost, most 10-Q filings are not audited.

The coefficient of interest in Equation (1) is $\beta_3$, which we interpret as the difference in responsiveness of textual properties to good versus bad news. If 10-Q narrative disclosure is conservative, we expect it to have greater number of words, greater marginal change of tone and shorter reporting time lag in response to bad news relative to good news. In the case of NW being the dependent variable, $\beta_3^{NW}$ should be negative under H1. Because QRET is always negative when NEG equals 1, the interaction term $\beta_3^{NW}QRET_{i,t}\times NEG_{i,t}$ is positive only when $\beta_3^{NW}$ is negative, which translate into greater number of words. Following the same logic, $\beta_3^{TLAG}$ of TLAG regression should be positive under H3, which translates into shorter reporting time lag. The interpretation of $\beta_3^{TONE}$ is different from those of the previous two models, in the sense that $\beta_3^{TONE}$ represents the difference in marginal change of tone in response to good versus bad news. An incremental marginal change of tone in response to bad news relative to good news is reflected as positive $\beta_3^{TONE}$.

Additionally, we construct an abnormal tone measure (ABTONE) following the expected tone model in \citeA{huangToneManagement2014}. ABTONE is calculated as the residual of the following model:\footnote{Our expected tone model differs from \citeA{huangToneManagement2014} in replacing book-to-market ratio with market-to-book ratio.}
\begin{equation} \label{eq2}
\begin{split}
TONE_{i,t}=\beta_0&+\beta_1EARN_{i,t}+\beta_2RET_{i,t}+\beta_3SIZE_{i,t}+\beta_4MTB_{i,t}+\beta_5STD\_EARN_{i,t}\\
&+\beta_6STD\_RET_{i,t}+\beta_7AGE_{i,t}+\beta_8BUSSEG_{i,t}+\beta_9GEOSEG_{i,t}+\beta_{10}LOSS_{i,t}\\
&+\beta_{11}\Delta EARN_{i,t}+\beta_{12}AFE_{i,t}+\beta_{13}AF_{i,t}+\epsilon_{i,t}
\end{split}
\end{equation}
Where TONE is the number of net positive words per thousand total words. As residuals of Equation (2), ABTONE captures the portion in tone that is orthogonal to firm fundamentals such as business complexity, growth opportunities and risk, and represents the portion of tone subject to managerial discretion. Our results of the expected tone model is consistent with \citeA{huangToneManagement2014}.\footnote{See results comparison for expected tone model in \hyperref[oat1]{Table 1 of Online Appendix}.} Finally, we measure the readability of 10-Qs using the Gunning fog index following \citeA{liAnnualReportReadability2008}. The fog index is positively correlated with the average number of words per sentence and the percentage of complex words in a document. The higher the index is, the less readable the document is.

\subsubsection{Form 8-K}
%%%% 8-K filing data structure
Due to the irregularity and unpredictability of the 8-K triggering events, the 8-K filings in EDGAR database have a unique data structure: though most companies only report one 8-K filing in one day and each 8-K filing usually contains only one or two 8-K items, some firms report more than one 8-K filings per day and each 8-K filing may contain more than two items. So we construct 8-K sample in three steps. First, as we want to analyze the responsiveness of 8-K filings to good and bad news, and our news proxy---daily stock return---is measured at firm-day level, we aggregate the raw 8-K data at individual filing level into 8-K data at firm-day level by summing up all raw count variables over each firm-day. For instance, the count variable $nw_{i,t}$ in 8-K dataset stands for number of total words in all 8-K filings reported in day \textit{t} for firm \textit{i}, instead of number of total words of one specific 8-K. To keep track of the unique data structure of 8-K filings, we further construct two new variables---N8K and NITEM, which are defined as number of 8-K filings in one reporting period date and number of 8-K items in one reporting period date, respectively. We label a firm-day as ``8-K day” if at least one 8-K's reporting period date coincides with that day.

%%%% 8-K news proxy
Next, we build our proxy for news under 8-K context. We obtain the daily market-adjusted stock returns (DRET) and calculate the change in daily returns ($\Delta$DRET). Then, we define a firm-day as a ``bad (good) news day” if the negative (positive) change in daily market-adjusted stock return ($\Delta$DRET) is three times larger than the firm's average decrease (increase) in daily return over the calendar year. BN is an indicator for bad news day, which is set to 1 if the firm-day is a bad news day, and 0 if the firm-day is a good news day.\footnote{We code BN to missing if the firm-day does not have any news. Therefore, all observations in our final 8-K sample are either good or bad news firm-days.} Notice that we define good and bad news differently under 8-K and 10-Q context. This is because that the daily returns are more volatile than quarterly returns and the sign of daily returns can change constantly merely due to trading noises. Therefore, we only focus on firm-days with sizable changes in daily returns, i.e., changes that are three times larger than the annual average change, which is more likely to result from significant corporate events and reflect fundamental firm information.

%%%% 8-K event matching process
Then, we conduct a matching process as illustrated in \hyperref[fig1]{Figure 1}. The idea of matching is to pair the firms' news releases and 8-K filings. Specifically, we match every 8-K day to its nearest news day. The matched news day can be earlier than (Match-1), the same as (Match-2) or later than (Match-3) the 8-K day. After matching, we calculate TLAG of 8-K sample as the number of days elapsed between the 8-K filing date and its nearest news day.\footnote{All filings in EDGAR have two dates: filing date and reporting period date. Filing date is the date when the report is filed to EDGAR, and reporting period date is the end date of reporting period of the filing, as defined by the SEC (https://www.sec.gov/about/webmaster-faq.htm). We match 8-Ks to news by \textit{reporting period date} because the reporting period date and the news release date are both about the actual date when the underlying event takes place. However, we calculate TLAG using 8-K \textit{filing date} because we are interested in whether 8-Ks are filed in response to good and bad news with different timeliness, allowing for managerial discretion in reporting speed.} We eliminate all Match-3 cases in which the market return movements occur after the filing of 8-Ks, because for this type of match we cannot accurately identify the date when the underlying event actually occurs.\footnote{We do not use the lag between the 8-K reporting period date and the 8-K filing date to measure reporting time lag because of two reasons. First, the reporting period date is the latest possible date, but not necessarily the exact date at which the underlying event occurs. Second, this reporting period date is self-reported by managers. Managers may have incentives to prolong or shorten the lag, which creates endogeneity concerns \cite{chapmanInformationOverloadDisclosure2019}.} Our final 8-K sample consists of Match-1 and Match-2 with non-negative TLAG.

The underlying assumption behind this matching process is that the 8-K filing and its nearest news release are triggered by the same underlying event. One concern of this assumption is that the 8-K filing and the news release may not be about the same event even if they are close in time. We provide two validity checks for this assumption from different aspects. As a first check, we construct a \textit{restricted 8-K sample} by limiting the full 8-K sample to observations with reporting time lag less than or equal to four (five) calendar days for observations with reporting period date after (before) August 23 of 2004 (TLAG = 0, 1, 2, 3, 4, 5). Because firms must file required 8-Ks within four (five) business days of a triggering event after (before) August 23 of 2004 \fullcite{secFinalRuleAdditional2004}, 8-K filings reported within four (five) days of the news release date are more likely to be related to the precedent news, as is regulated by the SEC rule. Our restricted sample selection criterion is more restrictive than the SEC rule for three reasons. First, while the regulation requires firms to file 8-K within four (five) business days of a triggering event, we reduce this reporting deadline to four (five) calendar days, which is always shorter or at most equal to four (five) business days. Second, the regulation exempts the voluntary disclosure items from the four (five) business day reporting deadline \cite{heMeasuringDisclosureUsing2020}, but we still apply this reporting deadline to these items. Third, prior to the 8-K reform, 8-Ks must be filed within five to fifteen days depending on the nature of the event occurred, but we uniformly apply the five days deadline to all items before the reform. This more stringent sample selection criterion further ensures that 8-K filings in our restricted sample are indeed responding to the precedent news. Our main results of 8-K hold using both the full and restricted 8-K samples. As a second check, we conduct a manual audit for 20 matched 8-K cases, and the results support the matching assumption that the 8-Ks are responses to their matched news releases. See \hyperref[appe]{Appendix E} for details of the manual audit process.

Once the 8-K sample is constructed, we design the following model to explore how 8-K filings respond to good versus bad news.
\begin{equation} \label{eq3}
TEX_{i,t}=\beta_0+\beta_1\Delta DRET_{i,t-tlag}+\beta_2BN_{i,t-tlag}+\beta_3\Delta DRET_{i,t-tlag}\times BN_{i,t-tlag}+\beta_nCONTROLS_{i,t}+\epsilon_{i,t}
\end{equation}
Where $\Delta$DRET and BN are changes in daily returns and bad news indicator \textit{at news release date}. We deploy $\Delta$DRET rather than DRET in this model because under 8-K context, the bad news indicator BN is defined based on $\Delta$DRET instead of DRET. In Equation (3), CONTROLS denotes a vector of control variables at 8-K \textit{filing date},\footnote{Because our measures of firm fundamentals are calculated based on Compustat quarterly data, the variation in firm fundamental measures is very small (if any) either we control for them at news release date (t-tlag) or at 8-K filing date (t), as the average reporting time lag of 8-K is only 15 days.} which includes firm size (SIZE), market-to-book ratio (MTB) and leverage ratio (LEV). We control for these three fundamental characteristics that may affect firms reporting policy to address the omitted variable bias. TEX represents a vector of textual properties that consists of number of words (NW), tone (TONE) and reporting time lag (TLAG). The coefficient of interest in Equation 3 is still $\beta_3$, and its interpretation is the same as that in the context of 10-Q. If 8-K narrative disclosure is conservative, we expect it to have greater number of words, greater marginal change of tone and shorter reporting time lag in response to bad news relative to good news, which manifests as negative $\beta_3^{NW}$, positive  $\beta_3^{TONE}$ and positive $\beta_3^{TLAG}$.

\subsection{Data}
We obtain historical financial and segment data from Compustat, stock returns from the Center for Research in Security Prices (CRSP) and analyst earnings forecasts data from I/B/E/S. We retrieve 10-Q and 8-K data from EDGAR through a self-developed Python program (see \hyperref[appa]{Appendix A} for detailed description of EDGAR data collection process). \hyperref[T1]{Table 1} illustrates the sample selection process of 10-Q and 8-K filings. First, we successfully parsed and retrieved 575,579 (1,489,626) unique 10-Q (8-K) filings out of 594,017 (1,628,467) existing filings in EDGAR from 1993-Q1 to 2020-Q1. Next, we merge 10-Q and 8-K dataset with other datasets of firm characteristics and market performance. Finally, we screen the merged 10-Q and 8-K dataset according to the following criteria. We eliminate observations with missing value in key accounting and financial variables or with beginning-of-quarter stock prices below \$1. In 10-Q sample, we further delete observations with missing values in analyst coverage variables. We exclude financial (SIC code between 6000 and 6999) and utility (SIC code between 4900 and 4999) firms because the accounting policy for the former is different from that of other industries, and they are both highly regulated industries which are incomparable to other industries in general. Observations (a) with non-positive total assets or book value of equity or common shares outstanding, or (b) with negative or above 99\% percentile reporting time lag,\footnote{Before truncation, the average reporting time lag for 10-Q is 40 days, but the maximum lag is 4,069 days, which is filed by CPI Corp in 2007-06-21 to report a quarterly result as of 1996-04-27 (see \url{https://www.sec.gov/Archives/edgar/containers/fix041/25354/0001140361-07-012753.txt}). We read some of the 10-Q filings with such extremely long reporting lag but do not find an explanation for the unusual delay. In theory 10-Q filings should be filed within 40 or 45 days after fiscal quarter-end, so it remains a puzzle as to why in practice there exists a few accepted filings with such a big delay in EDGAR database. For the purpose of this study we eliminate observations with unusual delay. We also truncate TLAG at 99\% percentile in 8-K sample.} or (c) with below 1\% percentile total number of words are dropped. The Gunning fog index is winsorized at 1\% and 99\% level. In 8-K sample, we further delete observations that are matched to news days that immediately follow another news day, because such news days may merely reflect the reversal in market returns after a dramatic change in the previous day. All financial variables except returns are winsorized at 1\% and 99\% level to minimize the impact of outliers. Our final 10-Q sample contains 91,607 firm-quarter observations which constitutes of observations from 5,250 unique firms from 1993 to 2016. Our final 8-K sample contains 119,616 firm-day observations which constitutes of observations from 8,261 unique firms from 1993 to 2020. On average, each firm in 8-K sample has two significant news days in a year. Sample size can vary across different model specifications and is noted in each table. 

\section{Main Results}
\subsection{Summary Statistics}
\hyperref[T2PA]{Table 2 Panel A} presents summary statistics for key variables in 10-Q sample. The summary statistics of the raw word count for positive, negative, uncertainty, litigation and modal words in 10-Q narratives (untabulated) are consistent with the LM 10-Q dataset.\footnote{Bill McDonald and Tim Loughran created a dataset containing summary data for each individual 10-X (e.g., 10-K, 10-K/A, 10-Q405, etc.) filing, available at \url{https://sraf.nd.edu/textual-analysis/resources/\#LM_10X_Summaries}.} On average, each 10-Q filing contains 10,937 words, with considerable variation across filings. TONE is negative in general and we propose two possible explanations for this. First, the LM sentiment word list contains more negative (2,355) than positive (354) words by construction, so the likelihood of words being classified as negative is higher than that of positive words. Second, since optimistic language increases litigation risk \cite{rogersDisclosureToneShareholder2011, cazierWhenAreFirms2020}, firms may avoid positive words in 10-Q filings to reduce litigation risk. On average, the 10-Q filings are filed 39 days after fiscal quarter-end, which is one day before the filing deadline for accelerated filers. Moreover, 75\% of the 10-Q filings are filed within 44 days after fiscal quarter-end, which is one day before the filing deadline for all other filers. This shows that firms do have discretion in reporting timeliness. ABTONE is normally distributed around zero by construction, and its quantiles are consistent with \citeA{huangToneManagement2014}. The mean of READ is 38, which is higher than the average fog readability score for 10-K in \citeA{liAnnualReportReadability2008}, and this is mainly due to some extremely high values even after winsorizing at 99\% level. The median of READ is consistent with \citeA{liAnnualReportReadability2008}. Since all financial variables but the QRET are winsorized, QRET contains some extremely high and low values. Our main results of 10-Q sustain if we winsorize QRET.

\hyperref[T2PB]{Table 2 Panel B} presents summary statistics for key variables in 8-K sample. 8-K filings are more neutral in terms of tone comparing to 10-Q filings, with average TONE being almost zero. Also, 8-K filings are more timely responses to news releases, with average TLAG being 15 days, which is 24 days sooner than average 10-Q filings. In more than 75\% of our 8-K firm-day observations, there is only one reported 8-K filing per day, and the maximum number of 8-K filings a firm has reported in one day is four. On average, all reported 8-Ks in one day contains 1,339 words in total, which is significantly less than the number of words per 10-Q. Firms report two 8-K items per day on average, with the maximum number being sixteen. \hyperref[fig2]{Figure 2} illustrates the 8-K item distribution before (left) and after (right) August 23 of 2004. Each share of pie chart shows the percentage of corporate events reported under each 8-K items. The most commonly reported 8-K items before reform are Item 7: financial statements and exhibits (34.0\%), Item 5: other events (27.8\%) and Item 2: acquisition or disposition of assets (12.7\%), whereas after reform the most frequent ones are Item 9.01: financial statements and exhibits (37.5\%), Item 2.02: results of operations and financial condition (18.2\%) and Item 8.01: other events (9.5\%). Voluntary disclosure, which consists of the Item Results of Operations and Financial Condition, the Item Regulation FD Disclosure and the Item Other Events, makes up for 38.7\% (35.7\%) of total 8-K items before (after) the 8-K reform. These statistics are consistent with \citeA{heMeasuringDisclosureUsing2020} and indicate that firms indeed use voluntary 8-K filings to report events that are not explicitly required but the firms consider important to the public. Regarding the financial variables, all but the DRET and the $\Delta$DRET are winsorized, so these two variables contain some extremely high and low values. Our main results of 8-K sustain if we winsorize DRET and $\Delta$DRET.

\hyperref[T2PC]{Panel C} and \hyperref[T2PD]{Panel D} of Table 2 present correlation matrix of key variables in 10-Q and 8-K sample, respectively. In Panel C, the correlations between ABTONE and other financial variables are close to zero, which verifies that ABTONE captures the portion of discretionary tone that is orthogonal to firm fundamentals. READ is negatively correlated with NW, suggesting that lengthier 10-Qs are actually more readable, potentially because firms try to break down abstruse concepts into more sentences and more simple words, leading to increases in document length but decreases in number of words per sentence and percentage of complex words.

\subsection{Is 10-Q narrative disclosure more responsive to bad news than good news?}
\hyperref[T3PA]{Table 3 Panel A} presents the regression result of \hyperref[eq1]{Equation 1}. Column 2, 4 and 6 include firm and time fixed effects to control for unobservable firm characteristics or time trends that may bias our estimation. Furthermore, given that reporting policy of firms within a same industry may be similar, which may lead to high correlations among observations in textual variables such as NW, TLAG and TONE, we cluster standard errors in Column 2, 4 and 6 at 4-digit SIC code industry level to correct the potential existence of serial correlation in dependent variables \fullcite{petersenEstimatingStandardErrors2009}. Our clustering approach yields 375 clusters in 10-Q sample (approximately 244 observations per cluster on average). As predicted by H1, the coefficient of QRET$\times$NEG is significantly negative for NW, consistent with 10-Q narratives being lengthier in response to bad news comparing to good news. Also, consistent with H2, the coefficient of QRET$\times$NEG is significantly positive for TONE, which suggests that the tone of 10-Q narratives are more news-consistent in response to bad news comparing to good news. However, in contrast to H3, the coefficient of QRET$\times$NEG is significantly negative for TLAG, which suggests that 10-Q reporting time lag is longer in response to bad news comparing to good news---that is, 10-Q filings respond to bad news in a less timely manner than good news. This delay in bad news response may appear because firms invest more resource and time on preparing the 10-Q filings to analyze and explain the causes of bad news. Due to the limitations discussed in \hyperref[sec3.1]{Section 3.1} about proxying timeliness of narrative disclosure with 10-Q reporting time lag, we interpret the TLAG result obtained in 10-Q sample only as supplemental evidence on timeliness of narrative disclosure.

In addition to the main hypotheses, we study whether firms use different tone management strategy to influence investors' perception in response to good versus bad news. We replace the dependent variable in \hyperref[eq1]{Equation (1)} with the abnormal tone (ABTONE) proposed by \citeA{huangToneManagement2014}, and reestimate the model. ABTONE measures the discretionary portion of tone that is uncorrelated with firm fundamentals such as business complexity, growth opportunities and risk. Positive (negative) ABTONE indicates that the tone of 10-Q filing is more positive (negative) than it should be conditional on firm fundamentals. In this model, the coefficient of QRET can be positive only when the signs of returns (QRET) and abnormal tone (ABTONE) agree, meaning that managers deploy more positive (negative) tone than they should in 10-Q filings in response to good (bad) news. Vice versa, negative coefficient of QRET suggests that firms deploy more positive (negative) tone than they should in 10-Q filings in response to bad (good) news. The two phenomena are different forms of tone management, and we label the former with positive coefficient of QRET as \textit{tone emphasis} and the latter with negative coefficient of QRET as \textit{tone attenuation}. If none of the two types of tone management is present in 10-Q filings for good news, then the coefficient of QRET should not be significantly different from zero. The coefficient of interest is the coefficient of QRET$\times$NEG, which represents the incremental tone emphasis or attenuation in response to bad news relative to good news, depending on the sign of the coefficient. If narrative disclosure is conservative, we expect incremental tone emphasis for bad news, translating into a positive coefficient of QRET$\times$NEG. One key research design issue in estimating the ABTONE model is that the dependent variable ABTONE is calculated as residuals from \hyperref[eq2]{Equation 2}. As \fullciteA{chenIncorrectInferencesWhen2018} has pointed out, using residuals as dependent variables may lead to incorrect inferences, so we apply the following two remedies as suggested by the authors. First, we include all regressors in Equation 2 as control variables in the ABTONE. Second, we combine all the regressors in Equation 2 and Equation 1 into one single-, as opposed to two-step regression.

Furthermore, we study whether and how the 10-Q readability varies in response to good versus bad news. This analysis is motivated by the concern that firms strategically respond to bad news with lengthier 10-Qs relative to good news to obfuscate the bad news, leading to information overload and lower market efficiency \cite{chapmanInformationOverloadDisclosure2019}. We replace the dependent variable in \hyperref[eq1]{Equation (1)} with the Gunning fog readability index (READ) and reestimate the model. If managers indeed intentionally obfuscate bad news with lengthier 10-Q filings, then we expect READ to be higher (less readable) for bad news than for good news on average. The obfuscation hypothesis predicts a negative coefficient of QRET$\times$NEG. 

\hyperref[T3PB]{Table 3 Panel B} presents the regression results of Equation (1), with TEX being ABTONE (Column 1), TONE (Column 2) and READ (Column 3 and 4) respectively. All regressions include firm and time fixed effects and standard errors are clustered at industry level identified by 4-digit SIC codes. The coefficients of QRET, NEG and QRET$\times$NEG are similar between Column 1 and 2. In both columns, the coefficient of QRET$\times$NEG is significantly positive, which suggests that firms tend to emphasize more the impact of bad news comparing to good news. Emphasizing bad news more than good news introduces a downward bias but provides warnings to financial information users, enhancing the stewardship role of financial reporting. The significance of QRET suggests that firms also emphasize the positive impact of good news, but to a lesser degree comparing to bad news. In Column 3 and 4, the coefficient of QRET$\times$NEG is not significant, indicating that the readability of 10-Qs does not vary between good and bad news. This result suggests that the incremental length in 10-Q in response to bad news relative to good news does not add information processing costs to investors in terms of readability.

Overall, the results demonstrate that 10-Qs are generally lengthier, more news-consistent and less timelier in response to bad news comparing to good news. In addition, 10-Qs tend to emphasize more the impact of bad news in comparison with good news. The incremental length in 10-Qs in response to bad news does not make them less readable relative to 10-Qs in response to good news.

\subsection{Is 8-K narrative disclosure more responsive to bad news than good news?}
\hyperref[T4PA]{Table 4 Panel A} presents the regression result of \hyperref[eq3]{Equation 3}. Column 2, 4, 6 and 7 include firm and time fixed effects and standard errors are clustered at 4-digit SIC code industry level. Our clustering approach yields 382 clusters in 8-K sample (approximately 313 observations per cluster on average). As predicted by H1, the coefficient of $\Delta$DRET$\times$NEG is significantly negative for NW, consistent with 8-K narratives being lengthier in response to bad news comparing to good news. Also, consistent with H2, the coefficient of $\Delta$DRET$\times$NEG is significantly positive for TONE, which suggests that 8-K narratives are more news-consistent in response to bad news comparing to good news. Notice that due to the limitations discussed in \hyperref[sec3.1]{Section 3.1} regarding using 8-K corpora to study the linguistic tone, the tone results obtained in 8-K sample may serve only as supplemental evidence on the news-consistency of narrative disclosure. Finally, in line with H3, the coefficient of QRET$\times$NEG is significantly positive for TLAG, which suggests that 8-K reporting time lag is shorter in response to bad news comparing to good news---that is, 8-K filings respond to bad news in a more timely manner comparing to good news. Column 7 presents the results of the same model using a subsample with strictly positive TLAG to address the concern that the market return movements may be triggered by the filing of 8-K report on the same day (TLAG = 0), as opposed to that the 8-Ks are responding to the prior news releases. Our results on asymmetric timeliness of narrative disclosure sustain in Column 7. We repeat the above main analyses using the restricted 8-K sample, and the results remain unchanged (see \hyperref[oat2]{Table 2 of Online Appendix}).

We perform three additional tests to assess the responsiveness of 8-K to good versus bad news, making use of the unique data structure of 8-K filings. First, we test whether firms report more 8-K items per day in response to bad news comparing to good news by replacing NITEM as the dependent variable in \hyperref[eq3]{Equation 3}. Second, we analyze whether firms are more likely to report more 8-K filings per day in response to bad news by estimating an ordered logistics version of \hyperref[eq3]{Equation 3} on N8K (N8K = 1, 2, 3, 4). Last but not least, we examine whether firms are more likely to report more promptly via 8-K in response to bad news by estimating an ordered logistics version of \hyperref[eq3]{Equation 3} on TLAG using the restricted 8-K sample. If the 8-K narrative disclosure is conservative, we expect firm to report more 8-K items and 8-K filings per day in response to bad news comparing to good news, which is reflected as significantly negative $\beta_3^{NITEM}$ and $\beta_3^{N8K}$.

\hyperref[T4PB]{Table 4 Panel B} presents the regression results for three additional tests. In line with previous predictions, the coefficients of $\Delta DRET\times BN$ are significantly negative for NITEM and N8K, and is significantly positive for TLAG. Column 1 presents the result of NITEM using an ordinary least square (OLS) regression\footnote{We choose OLS model for NITEM because the value of NITEM ranges from 1 to 16, which creates too many cutoffs for the ordered logistic model.} with firm and time fixed effects and clustered standard errors at industry level identified by 4-digit SIC codes. The significantly positive coefficient (0.232) of $\Delta$DRET shows that for good news, the number of 8-K items reported is positively associated with the magnitude of change in stock returns. Furthermore, $\beta_3^{NITEM}$ suggests that controlling for the size of daily changes in stock returns, a negative change in returns leads to 0.552 more reported 8-K items than a positive change, which is equivalent to 17\% increase based on the average number of 8-K items reported. Column 2 and 3 present results of ordered logistics models for N8K and TLAG. The baseline group of N8K regression is 1. The significantly positive coefficient (1.076) of $\Delta$DRET shows that for good news, the likelihood of reporting more number of 8-K filings is positively associated with the magnitude of change in stock returns. Moreover, $\beta_3^{N8K}$ suggests that controlling for the size of daily changes in stock returns, a negative change in returns leads to a 1.358 increase in the log odds of reporting more number of 8-K filings than a positive change. The baseline group of TLAG regression using restricted 8-K sample is 0. Similarly, the significantly negative coefficient (-0.944) of $\Delta$DRET shows that for good news, the likelihood of reporting in more days (reporting time lag being longer) is negatively associated with the magnitude of change in stock returns. Also, $\beta_3^{TLAG}$ suggests that controlling for the size of daily changes in stock returns, a negative change in returns leads to a 1.436 decrease in the log odds of reporting time lag being longer than a positive change, consistent with 8-K filings respond more timely to bad news relative to good news. 

Overall, the results demonstrate that 8-K filings are on average lengthier, more news-consistent and timelier in response to bad news comparing to good news. Moreover, firms are more likely to report larger number of 8-K items and 8-K filings per day in response to bad news comparing to good news. All results are consistent with 8-K narrative disclosure being conservative.

\section{Auxiliary Analyses}
\subsection{Narrative and Conditional Conservatism}
% motivation
In this section, we study whether and how narrative conservatism interacts with conditional conservatism. Narrative conservatism may complement conditional conservatism if managers apply consistent reporting policies to recognition and disclosure. In this case, we expect to find narrative conservatism being more pronounced in high conditional conservatism subsamples. However, if managers view the two as substitutes, they may choose to be less conservative in recognition but more conservative in narratives, or vise versa. In this case, we expect to find narrative conservatism being more pronounced in low conditional conservatism subsamples. 

% method
We construct a firm-quarter measure of conditional conservatism using the 10-Q sample following \citeA{khanEstimationEmpiricalProperties2009}. Specifically, we run the following cross-sectional model for each fiscal year from 1993 to 2020:

\begin{equation} \label{eq4}
\begin{split}
EARN_{i,t} = \beta_0&+\beta_1NEG_{i,t}+\beta_2RET_{i,t}\\
&+\beta_3RET_{i,t}\times SIZE_{i,t}+\beta_4RET_{i,t}\times MTB_{i,t}+\beta_5RET_{i,t}\times LEV_{i,t}+\beta_6RET_{i,t}\times NEG_{i,t}\\
&+\beta_7RET_{i,t}\times NEG_{i,t}\times SIZE_{i,t}+\beta_8RET_{i,t}\times NEG_{i,t}\times MTB_{i,t}+\beta_9RET_{i,t}\times NEG_{i,t}\times LEV_{i,t}\\
&+\beta_{10}SIZE_{i,t}+\beta_{11}MTB_{i,t}+\beta_{12}LEV_{i,t}\\
&+\beta_{13}NEG_{i,t}\times SIZE_{i,t}+\beta_{14}NEG_{i,t}\times MTB_{i,t}+\beta_{15}NEG_{i,t}\times LEV_{i,t}+ \epsilon_{i,t}
\end{split}
\end{equation}

We obtain the estimates from Equation 4 and use them to calculate C\_SCORE and G\_SCORE following Equation 5 and Equation 6 respectively. C\_SCORE captures the incremental timeliness of bad news and measures conditional conservatism, with more positive value being more conditionally conservative. G\_SCORE captures the timeliness of good news.
\begin{equation}\label{eq5}
C\_SCORE_{i,t} = \beta_6+\beta_7SIZE_{i,t}+\beta_8MTB_{i,t}+\beta_9LEV_{i,t}
\end{equation}
\begin{equation}\label{eq6}
G\_SCORE_{i,t} = \beta_2+\beta_3SIZE_{i,t}+\beta_4MTB_{i,t}+\beta_5LEV_{i,t}
\end{equation}

The mean of coefficients and standard errors obtained in \hyperref[eq4]{Equation 4} and the summary statistics of C\_SCORE and G\_SCORE (see \hyperref[oat3]{Table 3 of Online Appendix}) are consistent with \citeA{khanEstimationEmpiricalProperties2009} in general. Then we divide the full 10-Q sample into five subsamples according to the C\_SCORE, and we label the subsample with below 20\% (above 80\%) percentile C\_SCORE as low (high) conditional conservatism subsample. Finally we repeat our main analyses as specified by \hyperref[eq1]{Equation 1}.

% results
\hyperref[T5]{Table 5} presents the results of Equation 1 using the low and high conditional conservatism subsamples. Column 1, 3 and 5 (Column 2, 4 and 6) show the results of low (high) conditional conservatism subsample, and the row Diff. QRET$\times$NEG illustrates the difference in the coefficient QRET$\times$NEG between the low and high subsamples. First, the signs of the coefficients QRET$\times$NEG are consistent with the those in the main 10-Q results, confirming that 10-Q filings are lengthier, more news-consistent and less timely in response to bad news relative to good news. Second, in terms of number of words, the coefficient QRET$\times$NEG of low conditional conservatism subsample (-0.179) is significantly more negative than that of high conditional conservatism subsample (-0.095). Third, in terms of tone, the coefficient QRET$\times$NEG of low conditional conservatism subsample (2.982) is significantly more positive than that of high conditional conservatism subsample (1.635). However, we do not find evidence that the two subsamples differ significantly in terms of reporting time lag. Overall, the results suggest that firms with low conditional conservatism tend to be more conservative in narratives, supporting the substitution theory between narrative and conditional conservatism.

\subsection{The Role of Narratives and Narrative Conservatism}
% motivation
In this section, we investigate whether and how the role of narratives affects narrative conservatism. We hypothesize that supplementary narratives may be more conservative than explanatory narratives, because managers have more discretion over the contents and the format of the former. We focus on two sections in 10-Q filings---MD\&A and notes to financial statements. We consider narratives in the notes as explanatory narratives that ``amplify or explain information recognized in the financial statements'', and those in the MD\&A as supplementary narratives that ``add information to that in the financial statements or notes, including information that may be relevant but that does not meet all recognition criteria''\cite[CON5-7]{fasbStatementFinancialAccounting1984}. We expect to find narrative conservatism being more pronounced in the MD\&A subsamples. 

% method
We extract the MD\&A and the notes from 79,547 10-Q filings and count the number of total words, number of positive and negative words and number of negations in each section for each filing. Then we calculate the logarithm of number of total words (NW\_MDA and NW\_NOTE) and net positive tone per thousand words (TONE\_MDA and TONE\_NOTE) for each filing. Finally, we regress \hyperref[eq1]{Equation 1} replacing TEX by NW\_MDA, NW\_NOTE, TONE\_MDA and TONE\_NOTE and compare their differences. We do not repeat the main analysis for narrative timeliness because the MD\&A and the notes do not differ in timeliness by construction, since they are extracted from the same document.

% results
\hyperref[T6]{Table 6} presents the results of Equation 1 using the MD\&A and note sections. Column 1, 3 (Column 2, 4) show the results of the MD\&A (note) section, and the row Diff. QRET$\times$NEG illustrates the difference in the coefficient QRET$\times$NEG between the MD\&A and note sections. First, the signs of the coefficients QRET$\times$NEG are consistent with H1 and H2, confirming that both sections are lengthier and more news-consistent in response to bad news comparing to good news. Second, in terms of number of words, the coefficient QRET$\times$NEG of MD\&A section (-0.189) is significantly more negative than that of the note section (-0.159). Third, in terms of tone, the coefficient QRET$\times$NEG of MD\&A section (3.279) is significantly more positive than that of the note section (1.976). Overall, the results suggest that the narrative disclosure in the MD\&A section is more conservative than that in the note section, consistent with managers having more discretion in supplementary narratives and choosing to be conservative in narratives.

\subsection{Narrative Conservatism in Voluntary and Mandatory Disclosure}
% motivation
In this section, we examine whether and how narrative conservatism varies across voluntary and mandatory disclosure, using 8-K sample to measure the two types of disclosure following \citeA{heMeasuringDisclosureUsing2020}. If managers on average choose to be conservative in narratives voluntarily, then voluntary disclosure should be more conservative than mandatory disclosure in narratives. 

% method
We divide the full 8-K sample into voluntary and mandatory disclosure subsamples by the 8-K items reported in each observation. We classify an 8-K observation into voluntary disclosure subsample if it contains at least one of the voluntary 8-K items (Items 5, 9, 12 before and Items 8.01, 7.01, 2.02 after the 8-K reform) identified by prior literature \cite{lermanNewForm8K2010, heMeasuringDisclosureUsing2020}. Otherwise, we classify it into mandatory disclosure subsample. Then, we regress \hyperref[eq3]{Equation 3} using the two subsamples. Furthermore, to inspect if the types of events reported in 8-K filings also affect narrative conservatism, we create one indicator variable for every 8-K item,\footnote{For brevity, we aggregate the subcategories of 8-K items into one item for observations after August 23 of 2004. For example, Item 1 subsumes Item 1.01, 1.02, 1.03 and 1.04. If the observation contains any one of the four subcategories, the indicator variable Item 1 is set to 1, and otherwise 0. The same method applies to the rest of 8-K items.} where the indicator variable takes 1 if the 8-K observation contains that specific item, and 0 otherwise. Finally, we regress the textual metrics on the 8-K item indicators.

% results
\hyperref[T7PA]{Table 7 Panel A} presents the results of Equation 3 using the 8-K voluntary and mandatory disclosure subsamples. Column 1, 3 and 5 (Column 2, 4 and 6) show the results of the voluntary (mandatory) disclosure subsample, and the row Diff. $\Delta$DRET$\times$NEG illustrates the difference in the coefficient $\Delta$DRET$\times$NEG between voluntary and mandatory disclosure. First, the signs of the coefficients $\Delta$DRET$\times$NEG in Column 1, 3 and 5 are consistent with H1, H2 and H3, suggesting that voluntary disclosure is lengthier, more news-consistent and more timely in response to bad news comparing to good news. However, for mandatory disclosure the coefficients of $\Delta$DRET$\times$NEG in Column 2 and 4 are not significantly different from zero, and only the coefficient in Column 6 is significantly positive. This result suggests that mandatory disclosure is only conservative in terms of narrative timeliness. Second, given the difference between the two types of disclosure, voluntary disclosure is more conservative than mandatory disclosure in narratives, consistent with managers being conservative in narrative disclosure voluntarily.

\hyperref[T7PB]{Table 7 Panel B} presents the results of regressing the textual metrics on the 8-K item indicators. Column 1, 3 and 5 (Column 2, 4 and 6) show the results of the subsamples before (after) the 8-K reform. The items in bold are voluntary items. In terms of narrative length, the voluntary items Other Events (Item 5 before and Item 8 after) and Regulation FD (Item 9 before and Item 7 after) are associated with lengthier 8-Ks. The other voluntary item Results of Operations (Item 12 before and Item 2 after) is associated with shorter 8-Ks before the reform (-1.255), but becomes lengthier after the reform (0.201). In terms of narrative tone, before the reform Item 9 and Item 12 are significantly associated with positive tone, whereas after the reform all voluntary items become associated with negative tone. In terms of reporting time lag, all voluntary 8-K items are associated with shorter reporting lag. This result suggests that overall, voluntary 8-K filings have been transitioning towards lengthier, more negative, and more timely filings in recent years.

\subsection{Managerial Incentives and Narrative Conservatism}
% motivation
In this section, we inspect how narrative conservatism varies in three setting where managers have strong incentives to disclose or withhold bad news. First, when executives anticipate a stock option grant, they have incentives to disclose bad news prior to the grant to reduce the stock price to ensure a lower option exercise price at the grant date \cite{aboodyCEOStockOption2000,bakerStockOptionCompensation2003,mcanallyExecutiveStockOptions2008}. Therefore, we expect narrative disclosure to be more responsive to bad news relative to good news when firms grant stock option of high value to their executives. Second, when firms announce seasoned equity offering (SEO), managers have incentives to manipulate investor perceptions upward through earnings numbers \cite{teohEarningsManagementUnderperformance1998} and through tone management in earnings press \cite{huangToneManagement2014}. Similarly, we expect managers to be less responsive to bad news relative to good news in their narrative disclosure under SEO. Third, when firms face high litigation risk, managers have incentives to predisclose bad news voluntarily to avoid being sued or to reduce the cost of resolving litigation that inevitably follows in bad news quarters \cite{skinnerWhyFirmsVoluntarily1994, skinnerEarningsDisclosuresStockholder1997}. Therefore, we expect narrative disclosure to be more responsive to bad news relative to good news under high litigation risk.

% method
To test the stock option grants setting, we collect stock option grant data from ExecuComp from 1993 to 2006. We sum up the Black Scholes fair value of stock options granted to each individual executive (BLKSHVAL) by firm-year to create a measure of the value of stock options granted to executives for given firm in a given year (BLKSHVALSUM). Then we match the stock option grant data to our 10-Q data by GVKEY and fiscal year and we drop all unmatched observations and matched observations with non-positive or missing BLKSHVALSUM. Finally we divide the full 10-Q sample into five subsamples according to the BLKSHVALSUM of each observation, and we label the subsample with below 20\% (above 80\%) percentile BLKSHVALSUM as low (high) stock option value subsample. We reestimate \hyperref[eq1]{Equation 1} using the low and high stock option value subsamples and compare their results. To test the SEO setting, we use Sale of Common and Preferred Stock (SSTKY) data from Compustat from 1993 to 2020. We drop all observations with missing or negative SSTKY and we classify the observations with zero SSTKY into non-SEO subsample. Then we divide the full 10-Q sample into five subsamples according to the SSTKY of each observation, and we label the subsample with above 80\% percentile SSTKY as SEO subsample. We reestimate \hyperref[eq1]{Equation 1} using the non-SEO and SEO subsamples and compare their results. [LITIGATION SETTING TO BE TESTED]

% results
\hyperref[T8]{Table 8 Panel A} presents the results of Equation 1 using high and low value stock option grant subsamples. First, the coefficients of QRET$\times$NEG indicate that the narrative disclosures in 10-Qs from firms that issue stock option grants are lengthier and more news-consistent in response to bad news relative to good news. Comparing to the 10-Q main results, these 10-Qs from firms that issue stock option grants are no longer significantly less timely in response to bad news relative to good news. Second, comparing the results between the low and high value stock option grant subsamples, only the difference in number of words is significant (0.076). This shows that when managers anticipate high value stock option grant, they respond to bad news with lengthier reports relative to good news, consistent with managers being more conservative when they have strong incentives to disclose bad news. \hyperref[T8]{Table 8 Panel B} presents the results of Equation 1 using non-SEO and SEO subsamples. First, the coefficients of QRET$\times$NEG are consistent with the 10-Q main results. Second, comparing the results between the non-SEO and SEO subsamples, only the difference in tone is significant (1.091). This shows that when firms are undergoing SEO, managers respond to bad news with less new-consistent tone relative to good news, consistent with managers being less conservative when they have strong incentives to withhold bad news.

\subsection{Untabulated Robustness Checks}

\section{Conclusions}

%%% Future research: Economic Implications of Narrative Conservatism, Alternative News Proxy

\end{spacing}

\newpage
\bibliography{NC}
\bibliographystyle{apacite}

\newpage
%%%%%%%%%%%%%% Figure 1: 8-K Merging Process
\begin{figure}
	\caption{8-K Merging Process} \label{fig1}
	\begin{center}
		\includegraphics[scale=0.6]{../output/fig/fig1_matching.png}
	\end{center}
\end{figure}

Figuer 1 illustrates the 8-K sample matching process. We match every news day to its first posterior 8-K day, ignoring the successive 8-K days (if any) between two news days (Match-1), or in some cases the 8-K day coincides with news day (Match-2).

%%%%%%%%%%%%%% Figure 2: Sample Selection Process
% Table generated by Excel2LaTeX from sheet 'Fig2'
\begin{table}[htbp] \label{fig2}
  \centering
    \begin{tabular}{lr}
    \multicolumn{2}{c}{Figure 2: Sample Selection Process} \\
    \multicolumn{2}{c}{10-Q} \\
    Numer of observations: &  \\
    Retrieved from EDGAR & 575,579 \\
    After merging with COMP and CRSP data & 190,341 \\
    After merging with I\textbackslash{}B\textbackslash{}E\textbackslash{}S and segment data & 110,114 \\
    After dropping obs. with missing values in key variables and screening & \textbf{91,606} \\
      &  \\
    \multicolumn{2}{c}{8-K} \\
    Numer of observations: &  \\
    Retrieved from EDGAR & 1,489,626 \\
    After merging and matching with COMP and CRSP data & 390,698 \\
    After dropping obs. with missing values in key variables and screening & \textbf{244,401} \\
    After filtering obs. with TLAG smaller or equal to 4 (8-K restricted sample) & \textbf{62,300} \\
    \end{tabular}%
\end{table}%
 \label{fig2}

%%%%%%%%%%%%%% Figure 3: 8-K Item Distribution
\setcounter{figure}{2}
\begin{figure}[htbp]
	\begin{center}
		\caption{8-K Item Distribution} \label{fig3}
		\includegraphics[scale=0.5]{../output/fig/fig3_8-K_before.png}
		\includegraphics[scale=0.5]{../output/fig/fig3_8-K_after.png}
	\end{center}
\end{figure}

Figuer 3 illustrates the 8-K item distribution before (left) and after (right) May 23rd of 2004, respectively. Each share of pie chart shows the percentage of corporate events reported under each 8-K items. See 8-K item list in \hyperref[appd]{Appendix D}.

\newpage
%%%%%%%%%%%%%% Table 1: Sample Selection Process
\begin{landscape}
\begin{table}[htbp] \label{T1}
  \centering
    \begin{tabular}{lcc}
    \multicolumn{3}{c}{\textbf{Table 1. Sample Selection Process}} \\ 
      & &  \\
    \begin{comment}
    \multicolumn{3}{c}{10-Q} \\
    &   \multicolumn{2}{c}{Number of observations}\\
    & &  \\
    Retrieved from EDGAR & & 575,579 \\
    After merging with COMP and CRSP data & & 302,343 \\
    (-) Number of obs. from utility and financial firms & 82,498 & \\
    (-) Number of firm-quarters with missing values in SIC, SIZE, MTB, LEV, & & \\
    \hspace{5mm}or with non-positive total assets or book value of equity or common shares outstanding, & & \\
    \hspace{5mm}or with common share price less than \$1 & 25,959 & \\
    (-) Number of obs. with total words less than 1\% percentile (1,237 words) & 1,939 & \\
    (-) Number of obs. that contain negative or larger than 99\% TLAG & 1,855 & \\
    \bottomrule
    After dropping obs. with missing values in key variables and screening & & 190,092 \\
    After merging with I\textbackslash{}B\textbackslash{}E\textbackslash{}S and segment data & & 130,750 \\
    (-) Number of obs. that contain missing EARN, STD\_EARN and AF & 14,770 & \\
    \bottomrule
    Full 10-Q sample & & \textbf{115,980} \\
    & &  \\
    \end{comment}
    
     &   \multicolumn{2}{c}{Number of observations}\\
      & &  \\
    Retrieved from EDGAR & & 1,540,911 \\
    After matching with Compustat and CRSP data  & & 442,575 \\
    (-) Number of obs. from utility and financial firms & 112,729 & \\
    (-) Number of firm-quarters with missing values in SIC, SIZE, MTB, LEV, & & \\
    \hspace{5mm}or with non-positive total assets or book value of equity or common shares outstanding, & & \\
    \hspace{5mm}or with common share price less than \$1 & 46,865 & \\
    (-) Number of obs. with total words less than 1\% percentile (133 words) & 2,785 & \\
    (-) Number of obs. that are reversals of previous news day & 5,160 & \\
    (-) Number of obs. with negative or larger than 99\% percentile TLAG  & 154,861 & \\
    \bottomrule
    After dropping obs. with missing values in key variables and screening  & & 120,175 \\
    After merging with IBES and Compustat Segment data (Full 8-K sample) & & \textbf{83,464}  \\
    \begin{comment}
    	After dropping obs. with TLAG larger than four (five) days after (before) the 8-K reform &  & \\
    	(Restricted 8-K sample) &  & \textbf{40,700} 
    \end{comment}
    \end{tabular}%
\end{table}%

\end{landscape}

\newpage
%%%%%%%%%%%%%%%%%%%%%%%%% TABLE 2 Panel A
%\begin{landscape}
% Table generated by Excel2LaTeX from sheet 'T2PA '
\begin{table}[htbp] \label{T2PA}
  \centering
    \begin{tabular}{lcccccccc}
    \multicolumn{9}{c}{\textbf{Table 2. Panel A: Summary Statistics 10-Q}} \\
    \midrule
    \midrule
      & count & mean & std & min & 25\% & 50\% & 75\% & max \\
    \midrule
    \textbf{Textual Vars.} &   &   &   &   &   &   &   &  \\
    NW & 91607 & 9.020 & 0.757 & 7.120 & 8.506 & 9.086 & 9.547 & 13.544 \\
    nw & 91607 & 10937 & 10204 & 1236 & 4942 & 8829 & 13997 & 752337 \\
    TONE & 91607 & -8.921 & 7.236 & -63.579 & -13.127 & -7.875 & -3.866 & 24.215 \\
    TLAG & 91607 & 39 & 6 & 0 & 36 & 40 & 44 & 52 \\
    READ & 91607 & 38.161 & 42.160 & 14.580 & 17.840 & 20.210 & 39.660 & 262.515 \\
    ABTONE & 91607 & 0.000 & 6.919 & -55.759 & -3.946 & 0.939 & 4.777 & 34.181 \\
    \textbf{Financial Vars.} &   &   &   &   &   &   &   &  \\
    QRET & 91607 & 0.018 & 0.253 & -1.579 & -0.113 & 0.007 & 0.130 & 4.849 \\
    NEG & 91607 & 0.483 & 0.500 & 0 & 0 & 0 & 1 & 1 \\
    SIZE & 91607 & 6.447 & 1.776 & 2.002 & 5.175 & 6.317 & 7.563 & 11.206 \\
    MTB & 91607 & 3.515 & 4.009 & 0.288 & 1.485 & 2.343 & 3.902 & 30.902 \\
    LEV & 91607 & 0.192 & 0.182 & 0.000 & 0.011 & 0.162 & 0.315 & 0.724 \\
    AF & 91607 & 0.043 & 0.066 & -0.262 & 0.023 & 0.049 & 0.073 & 0.227 \\
    AFE & 91607 & -0.021 & 0.067 & -0.445 & -0.018 & -0.002 & 0.002 & 0.078 \\
    BUSSEG & 91607 & 0.859 & 0.447 & 0.693 & 0.693 & 0.693 & 0.693 & 2.773 \\
    GEOSEG & 91607 & 0.898 & 0.532 & 0.693 & 0.693 & 0.693 & 0.693 & 3.045 \\
    AGE & 91607 & 8.312 & 1.033 & 5.811 & 7.635 & 8.420 & 9.089 & 10.288 \\
    EARN & 91607 & 0.005 & 0.042 & -0.201 & 0.001 & 0.012 & 0.023 & 0.084 \\
    $\Delta$EARN & 91607 & 0.002 & 0.031 & -0.126 & -0.006 & 0.001 & 0.008 & 0.150 \\
    STD\_EARN & 91607 & 0.020 & 0.030 & 0.001 & 0.005 & 0.009 & 0.021 & 0.188 \\
    STD\_QRET & 91607 & 0.089 & 0.070 & 0.007 & 0.040 & 0.070 & 0.115 & 0.379 \\
    LOSS & 91607 & 0.242 & 0.429 & 0 & 0 & 0 & 0 & 1 \\
    \bottomrule
    \bottomrule
    \end{tabular}%
\end{table}%

%\end{landscape}

\newpage
%%%%%%%%%%%%%%%%%%%%%%%%% + TABLE 2 Panel B
%\begin{landscape}
% Table generated by Excel2LaTeX from sheet 'T2PB'
\begin{table}[H] \label{T2PB}
  \begin{center}
  	    \begin{tabular}{lcccccccc}
  		\multicolumn{9}{c}{\textbf{Table 2. Panel B: Summary Statistics 8-K}} \\
  		\midrule
  		\midrule
  		& count & mean & std & min & 25\% & 50\% & 75\% & max \\
  		\midrule
  		\textbf{Textual Variables} &   &   &   &   &   &   &   &  \\
  		NW & 119615 & 6.093 & 0.926 & 4.898 & 5.553 & 5.846 & 6.358 & 12.486 \\
  		nw & 119615 & 1339 & 6398 & 133 & 257 & 345 & 576 & 264704 \\
  		TONE & 119615 & -0.552 & 7.424 & -97.851 & -3.049 & 0.000 & 3.677 & 45.929 \\
  		TLAG & 119615 & 15 & 17 & 0 & 2 & 9 & 21 & 93 \\
  		N8K & 119615 & 1 & 0 & 1 & 1 & 1 & 1 & 4 \\
  		NITEM & 119615 & 2 & 1 & 1 & 2 & 2 & 2 & 16 \\
  		
  		\textbf{Financial Variables} &   &   &   &   &   &   &   &  \\
  		DRET & 119615 & 0.003 & 0.097 & -0.833 & -0.039 & -0.003 & 0.041 & 5.991 \\
  		$\Delta$DRET & 119615 & -0.018 & 0.187 & -9.062 & -0.121 & -0.050 & 0.100 & 5.989 \\
  		BN & 119615 & 0.542 & 0.498 & 0 & 0 & 1 & 1 & 1 \\
  		SIZE & 119615 & 6.326 & 1.993 & 2.122 & 4.896 & 6.262 & 7.664 & 11.379 \\
  		MTB & 119615 & 3.741 & 4.784 & 0.123 & 1.366 & 2.293 & 4.055 & 33.434 \\
  		LEV & 119615 & 0.204 & 0.192 & 0.000 & 0.012 & 0.171 & 0.334 & 0.735 \\
  		\bottomrule
  		\bottomrule
  	\end{tabular}%
  \end{center}
	\begin{footnotesize}
		\noindent Table 2 Panel A and Table 2 Panel B present the summary statistics of key variables in 10-Q and 8-K sample. READ and all financial variables except returns are winsorized at 1\% and 99\% level. See \hyperref[appb]{Appendix B} for variable definitions.
	\end{footnotesize}
\end{table}%
%\end{landscape}

%%%%%%%%%%%%%%%%%%%%%%%%% TABLE 2 Panel C
\newpage
%\begin{landscape}
% Table generated by Excel2LaTeX from sheet 'T2PC'
\begin{table}[H] \label{T2PC}
	\centering
	\begin{tabular}{lrrrrrrrrr}
		\multicolumn{10}{c}{\textbf{Table 2. Panel C: Correlation Matrix 8-K}} \\
		\midrule
		\midrule
		& \multicolumn{1}{c}{(1)} & \multicolumn{1}{c}{(2)} & \multicolumn{1}{c}{(3)} & \multicolumn{1}{c}{(4)} & \multicolumn{1}{c}{(5)} & \multicolumn{1}{c}{(6)} & \multicolumn{1}{c}{(7)} & \multicolumn{1}{c}{(8)} & \multicolumn{1}{c}{(9)}\\
		\midrule
		(1) tlag &  & -0.069 & 0.103 & -0.042 & -0.055 & 0.004 & -0.057 & -0.010 & -0.035 \\
		(2) TONE & -0.105 &  & -0.232 & -0.023 & -0.092 & -0.138 & 0.026 & 0.000 & 0.008 \\
		(3) nw & 0.116 & -0.415 &  & 0.017 & 0.004 & 0.308 & -0.025 & 0.017 & -0.006 \\
		(4) n8k & -0.058 & -0.044 & 0.213 &  & 0.437 & 0.209 & 0.066 & 0.015 & 0.007 \\
		(5) nitem & -0.096 & -0.114 & 0.197 & 0.302 &  & 0.461 & 0.091 & 0.008 & 0.004 \\
		(6) nexhibit & -0.069 & -0.112 & 0.175 & 0.203 & 0.614 &  & 0.101 & 0.015 & -0.007 \\
		(7) ngraph & -0.166 & 0.123 & -0.028 & 0.102 & 0.299 & 0.314 & & 0.004 & 0.003 \\
		(8) DRET & -0.021 & 0.005 & -0.003 & 0.004 & 0.004 & 0.006 & 0.008 &  & 0.700 \\
		(9) $\Delta$DRET & -0.048 & 0.013 & -0.015 & 0.003 & 0.007 & 0.003 & 0.017 & 0.765 &  \\
		(10) BN & 0.051 & -0.009 & 0.012 & -0.002 & -0.006 & -0.003 & -0.016 & -0.774 & -0.864 \\
		(11) SIZE & -0.095 & 0.068 & 0.020 & 0.026 & 0.009 & 0.003 & 0.091 & 0.021 & 0.070 \\
		(12) MTB & -0.006 & 0.029 & 0.038 & -0.001 & -0.015 & -0.023 & 0.007 & 0.007 & 0.008 \\
		(13) LEV & -0.046 & -0.037 & 0.075 & 0.028 & 0.029 & 0.046 & 0.073 & 0.015 & 0.024 \\
		(14) AF & -0.050 & 0.013 & -0.018 & 0.004 & 0.017 & 0.030 & 0.039 & -0.024 & 0.042 \\
		(15) AFE & -0.011 & 0.032 & -0.020 & 0.008 & 0.002 & -0.012 & 0.017 & 0.032 & 0.004 \\
		(16) BUSSEG & -0.070 & 0.095 & 0.035 & 0.027 & 0.044 & -0.010 & 0.200 & 0.004 & 0.021 \\
		(17) GEOSEG & -0.076 & 0.094 & 0.041 & 0.023 & 0.041 & -0.013 & 0.194 & 0.008 & 0.031 \\
		(18) EARN & -0.021 & 0.068 & -0.069 & -0.004 & -0.001 & -0.004 & -0.019 & 0.045 & 0.059 \\
		(19) STD\_QRET & 0.018 & -0.056 & 0.056 & -0.011 & -0.010 & -0.009 & -0.013 & -0.030 & -0.058 \\
		\bottomrule
		\bottomrule
	\end{tabular}%
\end{table}%
% Table generated by Excel2LaTeX from sheet 'T2PD'
\begin{table}[H]
  \begin{center}
  	\begin{tabular}{lrrrrrrrrrr}
  		\multicolumn{11}{c}{\textbf{Table 2. Panel C: Correlation Matrix 8-K (Continued) }} \\
  		\midrule
  		\midrule
  		& \multicolumn{1}{c}{(10)} & \multicolumn{1}{c}{(11)} & \multicolumn{1}{c}{(12)} & \multicolumn{1}{c}{(13)} & \multicolumn{1}{c}{(14)} & \multicolumn{1}{c}{(15)} & \multicolumn{1}{c}{(16)} & \multicolumn{1}{c}{(17)} & \multicolumn{1}{c}{(18)} & \multicolumn{1}{c}{(19)} \\
  		\midrule
  		(1) tlag & 0.039 & -0.075 & -0.004 & -0.039 & -0.012 & 0.001 & -0.061 & -0.062 & 0.005 & 0.003 \\
  		(2) TONE & -0.008 & 0.062 & 0.012 & -0.028 & -0.013 & 0.042 & 0.061 & 0.065 & 0.033 & -0.037 \\
  		(3) nw & 0.004 & -0.055 & 0.010 & 0.039 & 0.006 & -0.016 & -0.071 & -0.073 & -0.014 & 0.026 \\
  		(4) n8k & -0.003 & 0.025 & 0.000 & 0.028 & -0.001 & 0.005 & 0.027 & 0.020 & 0.002 & -0.008 \\
  		(5) nitem & -0.003 & -0.001 & -0.007 & 0.032 & 0.001 & -0.003 & 0.036 & 0.026 & -0.005 & 0.002 \\
  		(6) nexhibit & 0.006 & -0.006 & -0.002 & 0.053 & 0.004 & -0.015 & -0.010 & -0.019 & -0.025 & 0.021 \\
  		(7) ngraph & -0.004 & 0.039 & 0.014 & 0.045 & -0.003 & 0.003 & 0.079 & 0.073 & -0.005 & -0.004 \\
  		(8) DRET & -0.587 & -0.014 & 0.004 & 0.003 & 0.006 & 0.009 & -0.007 & -0.006 & 0.017 & 0.005 \\
  		(9) $\Delta$DRET & -0.765 & 0.057 & -0.008 & 0.013 & 0.062 & 0.001 & 0.018 & 0.024 & 0.062 & -0.055 \\
  		(10) BN & & -0.032 & 0.000 & -0.011 & -0.027 & 0.002 & -0.015 & -0.020 & -0.031 & 0.027 \\
  		(11) SIZE & -0.031 &  & 0.207 & 0.170 & 0.114 & 0.188 & 0.240 & 0.283 & 0.313 & -0.259 \\
  		(12) MTB & -0.004 & 0.346 &  & 0.104 & -0.152 & 0.077 & 0.006 & 0.028 & -0.055 & 0.129 \\
  		(13) LEV & -0.014 & 0.219 & -0.035 &  & 0.144 & -0.071 & 0.089 & 0.054 & 0.070 & -0.115 \\
  		(14) AF & -0.028 & 0.030 & -0.402 & 0.226 &  & -0.184 & 0.058 & 0.080 & 0.375 & -0.203 \\
  		(15) AFE & -0.003 & 0.133 & 0.122 & -0.061 & -0.218 &  & 0.053 & 0.054 & 0.193 & -0.110 \\
  		(16) BUSSEG & -0.013 & 0.224 & 0.066 & 0.074 & 0.053 & 0.028 &  & 0.644 & 0.081 & -0.091 \\
  		(17) GEOSEG & -0.021 & 0.289 & 0.072 & 0.083 & 0.090 & 0.028 & 0.715 &  & 0.105 & -0.111 \\
  		(18) EARN & -0.030 & 0.349 & 0.226 & -0.032 & 0.113 & 0.227 & 0.027 & 0.060 &  & -0.470 \\
  		(19) STD\_EARN & 0.028 & -0.338 & 0.058 & -0.177 & -0.133 & -0.066 & -0.075 & -0.101 & -0.335 & \\
  		\bottomrule
  		\bottomrule
  	\end{tabular}%
  \end{center}
	\begin{footnotesize}
		\noindent Table 2 Panel C presents the correlation matrix of key variables in 8-K sample. Pearson (Spearman) correlations are exhibited above (below) the diagonal. See \hyperref[appc]{Appendix C} for variable definitions. All financial variables except returns are winsorized at 1\% and 99\% level. 
	\end{footnotesize}
\end{table}%
%\end{landscape}

%%%%%%%%%%%%%%%%%%%%%%%%% TABLE 3
\newpage
\begin{landscape}
% Table generated by Excel2LaTeX from sheet 'T3'
\begin{table}[H] \label{T3}
	\begin{center}
		\tabcolsep=0.11cm
		\begin{tabular}{lcccc}
			\multicolumn{5}{c}{\textbf{Table 3. Is 8-K Narrative Disclosure Conservative?}} \\
			\toprule
			\toprule
			& (1) & (2) & (3) & (4) \\
			Dep. Variables & TLAG & TLAG & TONE & TONE \\
			\midrule
			%&   &   &   &  \\
			$\Delta$DRET & 1.913*** & 2.007*** & -1.744*** & -1.171** \\
			& (11.44) & (10.83) & (-2.86) & (-2.07) \\
			BN & -0.021 & -0.026 & -0.120* & -0.125 \\
			& (-1.13) & (-1.15) & (-1.71) & (-1.64) \\
			\rowcolor[rgb]{ .906,  .902,  .902} \textit{(Pred. Sign)} & (-) & (-) & (+) & (+) \\
			\rowcolor[rgb]{ .906,  .902,  .902} $\Delta$DRET$\times$BN & -2.966*** & -3.182*** & 2.893*** & 1.849** \\
			\rowcolor[rgb]{ .906,  .902,  .902}   & (-8.42) & (-7.55) & (2.70) & (1.97) \\
			SIZE &   & 0.051*** &   & 0.115* \\
			&   & (4.56) &   & (1.76) \\
			MTB &   & 0.002 &   & -0.009 \\
			&   & (1.22) &   & (-1.08) \\
			LEV &   & -0.007 &   & -0.592 \\
			&   & (-0.11) &   & (-1.45) \\
			EARN &   & -0.231* &   & 3.059** \\
			&   & (-1.70) &   & (2.51) \\
			STD\_EARN &   & -0.165 &   & -2.705**\\
			&   & (-0.72) &   & (-2.17)\\
			BUSSEG &   & -0.028 &   & -0.015 \\
			&   & (-1.52) &   & (-0.12) \\
			GEOSEG &   & 0.016 &   & 0.131 \\
			&   & (0.91) &   & (1.18) \\
			AF &   & 0.020 &   & -0.019 \\
			&   & (0.20) &   & (-0.04)\\
			AFE &   & 0.045 &   & 1.713**  \\
			&   & (0.41) &   & (2.57) \\
			Constant & -2.816*** & -3.150*** & -5.598** & -5.921*** \\
			& (-10.16) & (-10.85) & (-2.47) & (-2.71) \\
			&   &   &   &  \\
			Observations & 83,464 & 75,360 & 83,464 & 75,360 \\
			Adjusted R-squared & 0.131 & 0.132 & 0.151 & 0.147 \\
			\bottomrule
			\bottomrule
		\end{tabular}%
	\end{center}
\end{table}%
% Table generated by Excel2LaTeX from sheet 'T3'
\begin{table}
	\begin{center}
		\tabcolsep=0.11cm
		\begin{tabular}{lcccccccccc}
			\multicolumn{11}{c}{\textbf{Table 3. Is 8-K Narrative Disclosure Conservative? (Continued)}} \\
			\toprule
			\toprule
			 & (5) & (6) & (7) & (8) & (9) & (10) & (11) & (12) & (13) & (14) \\
			Dep. Variables & NW & NW & N8K & N8K & NITEM & NITEM & NEXHIBIT & NEXHIBIT & NGRAPH & NGRAPH \\
			\midrule
			%&   &   &   &   &   &  &   &   &   &  \\
			$\Delta$DRET & -0.086* & -0.042 & -0.034*** & -0.039*** & -0.075*** & -0.079*** & -0.105*** & -0.110*** & -0.151*** & -0.212*** \\
			 & (-1.78) & (-0.71) & (-3.43) & (-3.64) & (-3.34) & (-3.71) & (-2.99) & (-3.04) & (-3.03) & (-5.02) \\
			BN & -0.015** & -0.015** & -0.002** & -0.003** & -0.004 & -0.004 & -0.003 & -0.002 & 0.001 & -0.001 \\
			& (-2.04) & (-2.19) & (-2.24) & (-2.43) & (-1.13) & (-1.05) & (-0.53) & (-0.36) & (0.16) & (-0.13) \\
			\rowcolor[rgb]{ .906,  .902,  .902} \textit{(Pred. Sign)} & (+) & (+) & (+) & (+) & (+) & (+) & (+) & (+) & (+) & (+) \\
			\rowcolor[rgb]{ .906,  .902,  .902} $\Delta$DRET$\times$BN & 0.127** & 0.033 & 0.046*** & 0.051*** & 0.099*** & 0.104*** & 0.176*** & 0.175*** & 0.221*** & 0.298*** \\
			\rowcolor[rgb]{ .906,  .902,  .902}   & (2.02) & (0.40) & (3.34) & (3.36) & (2.84) & (3.06) & (3.46) & (3.32) & (4.06) & (5.71) \\
			SIZE &   & 0.018** &   & -0.001 &   & -0.002 &   & -0.003 &   & -0.004 \\
			&   & (2.13) &   & (-0.84) &   & (-0.70) &   & (-0.58) &   & (-0.60) \\
			MTB &   & -0.002 &   & -0.000 &   & -0.000 &   & -0.002*** &   & -0.003*** \\
			&    & (-1.30) &   & (-0.43) &   & (-0.96) &   & (-2.88) &   & (-2.82) \\
			LEV &  & -0.027 &   & -0.008** &   & -0.021* &   & -0.007 &   & 0.005 \\
			&   & (-0.65) &   & (-2.43) &   & (-1.68) &   & (-0.32) &   & (0.11) \\
			EARN &    & 0.406*** &   & -0.001 &   & 0.069* &   & 0.113* &   & -0.064 \\
			&   & (3.84) &   & (-0.17) &   & (1.82) &   & (1.96) &   & (-0.87) \\
			STD\_EARN &    & -0.331*** &   & -0.004 &   & -0.098** &   & -0.112 &   & 0.243* \\
			&   & (-2.75) &   & (-0.41) &   & (-2.11) &   & (-1.29) &   & (1.71) \\
			BUSSEG &    & -0.008 &   & 0.000 &   & 0.002 &   & 0.003 &   & -0.005 \\
			&    & (-0.71) &   & (0.21) &   & (0.39) &   & (0.42) &   & (-0.31) \\
			GEOSEG &    & 0.007 &   & 0.002** &   & -0.001 &   & -0.011* &   & -0.011 \\
			&    & (0.67) &   & (2.27) &   & (-0.36) &   & (-1.82) &   & (-0.76) \\
			AF &    & -0.026 &   & 0.004 &   & 0.015 &   & 0.029 &   & -0.075 \\
			&    & (-0.47) &   & (0.52) &   & (0.74) &   & (0.66) &   & (-1.56) \\
			AFE &   & -0.044 &   & -0.009 &   & -0.022 &   & -0.091** &   & -0.164** \\
			&   & (-0.69) &   & (-1.36) &   & (-0.86) &   & (-2.44) &   & (-2.37) \\
			Constant & -7.291*** & -7.295*** & -0.688*** & -0.684*** & -0.872*** & -0.843*** & -0.506*** & -0.459*** & 0.051 & 0.096 \\
			&  (-27.57) & (-28.75) & (-190.40) & (-120.16) & (-25.72) & (-22.63) & (-4.91) & (-4.26) & (1.01) & (1.44) \\
			&   &   &   &   &   &   &   &   &   &  \\
			Observations & 83,464 & 75,360 & 83,464 & 75,360 & 83,464 & 75,360 & 83,464 & 75,360 & 83,464 & 75,360 \\
			Adjusted R-squared & 0.443 & 0.427 & 0.021 & 0.024 & 0.139 & 0.142 & 0.109 & 0.107 & 0.256 & 0.263 \\
			\bottomrule
			\bottomrule
		\end{tabular}%
	\end{center}
		\begin{footnotesize}
			\setcounter{equation}{0}
			\begin{equation}
				TEX_{i,t}=\beta_0+\beta_1\Delta DRET_{i,t-tlag}+\beta_2BN_{i,t-tlag}+\beta_3\Delta DRET_{i,t-tlag}\times 	BN_{i,t-tlag}+\sum\beta_nCONTROLS_{i,t}+\epsilon_{i,t}
			\end{equation}
			
			\noindent Table 3 presents the regression results of Equation (1). TEX represents a vector of textual properties. CONTROLS denotes a vector of control variables. See \hyperref[appc]{Appendix C} for variable definitions. All financial variables except returns are winsorized at 1\% and 99\% level. All regressions include firm and year-month fixed effects and standard errors are clustered at industry level identified by 4-digit SIC codes. ***, ** and * indicate significance at the 1\%, 5\% and 10\% levels in a two-tailed test.
		\end{footnotesize}
\end{table}%
\end{landscape}

%%%%%%%%%%%%%%%%%%%%%%%%% TABLE 4
\newpage
\begin{landscape}
% Table generated by Excel2LaTeX from sheet 'T3'
\begin{table}[H] \label{T4}
	\begin{center}
		\tabcolsep=0.11cm
		\begin{tabular}{lcccc}
			\multicolumn{5}{c}{\textbf{Table 4. Narrative Conservatism and Conditional Conservatism}} \\
			\toprule
			\toprule
			Dep. Variables & \multicolumn{2}{c}{TLAG} & \multicolumn{2}{c}{TONE} \\
			\cmidrule{2-5}
			& (1) & (2) & (3) & (4) \\
			CONS. & LOW & HIGH & LOW & HIGH \\
			\midrule
			%&   &   &   &  \\
			$\Delta$DRET & 2.647*** & 1.775*** & -2.473** & -0.206 \\
			& (9.71) & (11.56) & (-2.33) & (-0.31) \\
			BN & -0.051* & -0.009 & -0.186 & -0.079 \\
			& (-1.91) & (-0.33) & (-1.54) & (-0.80) \\
			\rowcolor[rgb]{ .906,  .902,  .902} \textit{(Pred. Sign)} & (-) & (-) & (+) & (+) \\
			\rowcolor[rgb]{ .906,  .902,  .902} $\Delta$DRET$\times$BN& -4.639*** & -2.687*** & 3.553** & 0.549 \\
			\rowcolor[rgb]{ .906,  .902,  .902} & (-8.75) & (-8.84) & (2.17) & (0.54) \\
			SIZE & 0.087*** & 0.030** & 0.092 & 0.101 \\
			& (4.69) & (2.12) & (0.92) & (1.07) \\
			MTB & -0.000 & 0.003 & 0.018 & -0.005 \\
			& (-0.09) & (1.09) & (0.81) & (-0.38) \\
			LEV & -0.002 & -0.082 & -0.937* & -0.581 \\
			& (-0.02) & (-0.94) & (-1.81) & (-0.90) \\
			EARN & 0.031 & -0.306 & 1.008 & 3.218** \\
			& (0.13) & (-1.61) & (0.46) & (2.53) \\
			STD\_EARN & -0.041 & -0.030 & -2.801 & -3.046*** \\
			& (-0.13) & (-0.10) & (-1.19) & (-2.65) \\
			BUSSEG & -0.026 & -0.025 & -0.059 & -0.046 \\
			& (-1.14) & (-0.78) & (-0.36) & (-0.23) \\
			GEOSEG & 0.034 & 0.004 & 0.031 & 0.253 \\
			& (1.55) & (0.18) & (0.22) & (1.56) \\
			AF & 0.153 & -0.028 & 0.022 & 0.067 \\
			& (1.22) & (-0.22) & (0.03) & (0.10) \\
			AFE & 0.059 & 0.032 & 2.629*** & 0.810 \\
			& (0.34) & (0.21) & (2.75) & (0.83) \\
			Constant & -2.845*** & -2.492*** & -0.198 & -0.826 \\
			& (-17.51) & (-23.87) & (-0.25) & (-1.38) \\
			&   &   &   &  \\
			Observations & 38,881 & 35,134 & 38,881 & 35,134 \\
			Adjusted R-squared & 0.139 & 0.120 & 0.133 & 0.154 \\
			\bottomrule
			\bottomrule
		\end{tabular}%
	\end{center}
\end{table}%
% Table generated by Excel2LaTeX from sheet 'T3'
\begin{table}[H]
	\begin{center}
		\tabcolsep=0.11cm
		\begin{tabular}{lcccccccccc}
			\multicolumn{11}{c}{\textbf{Table 4. Narrative Conservatism and Conditional Conservatism (Continued)}} \\
			\toprule
			\toprule
			Dep. Variables & \multicolumn{2}{c}{NW} & \multicolumn{2}{c}{N8K} & \multicolumn{2}{c}{NITEM} & \multicolumn{2}{c}{NEXHIBIT} & \multicolumn{2}{c}{NGRAPH} \\
			\cmidrule{2-11}
			& (5) & (6) & (7) & (8) & (9) & (10) & (11) & (12) & (13) & (14) \\
			CONS. & LOW & HIGH & LOW & HIGH & LOW & HIGH & LOW & HIGH& LOW & HIGH\\
			\midrule
			%&   &   &   &   &   &  &   &   &   &  \\
			$\Delta$DRET & -0.090 & -0.015 & -0.047*** & -0.042*** & -0.104*** & -0.061** & -0.171*** & -0.078* & -0.304*** & -0.168*** \\
			& (-0.89) & (-0.21) & (-4.04) & (-2.73) & (-2.93) & (-2.30) & (-3.12) & (-1.68) & (-2.93) & (-3.34) \\
			BN & -0.012 & -0.022** & -0.002 & -0.004** & -0.006 & -0.002 & -0.003 & 0.001 & -0.011 & 0.002 \\
			& (-1.00) & (-2.13) & (-1.31) & (-2.57) & (-1.15) & (-0.32) & (-0.41) & (0.08) & (-0.78) & (0.16) \\
			\rowcolor[rgb]{ .906,  .902,  .902} \textit{(Pred. Sign)} & (+) & (+) & (+) & (+) & (+) & (+) & (+) & (+) & (+) & (+) \\
			\rowcolor[rgb]{ .906,  .902,  .902} $\Delta$DRET$\times$BN & 0.095 & -0.025 & 0.066*** & 0.052** & 0.127*** & 0.085** & 0.281*** & 0.130** & 0.391*** & 0.244*** \\
			\rowcolor[rgb]{ .906,  .902,  .902} & (0.66) & (-0.25) & (4.01) & (2.51) & (2.89) & (2.00) & (3.62) & (2.14) & (3.20) & (4.30) \\
			SIZE & 0.024** & 0.013 & -0.001 & -0.000 & -0.004 & 0.003 & -0.012* & 0.011 & -0.003 & -0.002 \\
			& (2.10) & (1.29) & (-0.79) & (-0.19) & (-1.26) & (0.86) & (-1.95) & (1.54) & (-0.29) & (-0.20) \\
			MTB & -0.001 & -0.003 & -0.000 & -0.000 & -0.000 & -0.001 & -0.000 & -0.003*** & 0.001 & -0.004** \\
			& (-0.47) & (-1.62) & (-0.07) & (-0.87) & (-0.16) & (-1.54) & (-0.29) & (-3.16) & (0.41) & (-2.40) \\
			LEV & -0.074 & 0.035 & -0.006 & -0.010 & -0.014 & -0.016 & -0.019 & 0.005 & 0.054 & -0.047 \\
			& (-1.30) & (0.63) & (-1.03) & (-1.63) & (-0.66) & (-0.94) & (-0.60) & (0.16) & (0.76) & (-0.97) \\
			EARN & 0.263 & 0.486*** & 0.008 & 0.001 & 0.097 & 0.051 & 0.007 & 0.152** & 0.003 & -0.074 \\
			& (1.33) & (4.91) & (0.33) & (0.06) & (1.58) & (1.30) & (0.07) & (2.41) & (0.02) & (-0.89) \\
			STD\_EARN & -0.155 & -0.335** & 0.021 & -0.021* & 0.049 & -0.162*** & 0.095 & -0.186** & 0.544** & 0.077 \\
			& (-0.89) & (-2.32) & (0.88) & (-1.76) & (0.67) & (-3.04) & (0.61) & (-1.98) & (2.07) & (0.48) \\
			BUSSEG & -0.006 & -0.015 & -0.000 & 0.001 & -0.003 & 0.007 & -0.001 & 0.002 & -0.017 & 0.039* \\
			& (-0.45) & (-0.82) & (-0.19) & (0.75) & (-0.51) & (1.15) & (-0.12) & (0.18) & (-0.85) & (1.70) \\
			GEOSEG & 0.019 & 0.010 & 0.002 & 0.003* & 0.002 & -0.002 & -0.006 & -0.006 & 0.005 & -0.036* \\
			& (1.59) & (0.67) & (1.41) & (1.85) & (0.37) & (-0.29) & (-0.67) & (-0.63) & (0.26) & (-1.78) \\
			AF & -0.013 & -0.024 & 0.001 & 0.010 & 0.018 & 0.006 & 0.053 & -0.009 & -0.100 & -0.057 \\
			& (-0.16) & (-0.43) & (0.08) & (0.96) & (0.50) & (0.32) & (1.02) & (-0.17) & (-1.26) & (-0.86) \\
			AFE & -0.020 & -0.085 & -0.011 & -0.009 & -0.016 & -0.017 & -0.141** & -0.081 & -0.140 & -0.142* \\
			& (-0.23) & (-0.88) & (-1.09) & (-1.05) & (-0.47) & (-0.48) & (-2.53) & (-1.58) & (-1.28) & (-1.88) \\
			Constant & -6.223*** & -6.067*** & -0.699*** & -0.700*** & -1.058*** & -1.089*** & -0.563*** & -0.684*** & -0.442*** & -0.330*** \\
			& (-66.29) & (-94.80) & (-69.04) & (-110.27) & (-35.28) & (-51.80) & (-11.01) & (-14.53) & (-4.52) & (-6.45) \\
			&   &   &   &   &   &   &   &   &   &  \\
			Observations & 38,881 & 35,134 & 38,881 & 35,134 & 38,881 & 35,134 & 38,881 & 35,134 & 38,881 & 35,134 \\
			Adjusted R-squared & 0.362 & 0.437 & 0.029 & 0.029 & 0.133 & 0.164 & 0.097 & 0.117 & 0.267 & 0.272 \\
			\bottomrule
			\bottomrule
		\end{tabular}%
	\end{center}
		\begin{footnotesize}
			\setcounter{equation}{0}
			\begin{equation}
				TEX_{i,t}=\beta_0+\beta_1\Delta DRET_{i,t-tlag}+\beta_2BN_{i,t-tlag}+\beta_3\Delta DRET_{i,t-tlag}\times 	BN_{i,t-tlag}+\sum\beta_nCONTROLS_{i,t}+\epsilon_{i,t}
			\end{equation}
			
			\noindent Table 4 presents the regression results of Equation (1) across high and low conditional conservatism subsamples. TEX represents a vector of textual properties. CONTROLS denotes a vector of control variables. See \hyperref[appc]{Appendix C} for variable definitions. All financial variables except returns are winsorized at 1\% and 99\% level. All regressions include firm and year-month fixed effects and standard errors are clustered at industry level identified by 4-digit SIC codes. ***, ** and * indicate significance at the 1\%, 5\% and 10\% levels in a two-tailed test.
		\end{footnotesize}
\end{table}%
\end{landscape}

%%%%%%%%%%%%%%%%%%%%%%%%% TABLE 5
\newpage
\begin{landscape}
	% Table generated by Excel2LaTeX from sheet 'T3'
\begin{table}[H] \label{T5}
	\begin{center}
		\tabcolsep=0.11cm
		\begin{tabular}{lcccc}
			\multicolumn{5}{c}{\textbf{Table 5. Narrative Conservatism and Unconditional Conservatism}} \\
			\toprule
			\toprule
			Dep. Variables & \multicolumn{2}{c}{TLAG} & \multicolumn{2}{c}{TONE} \\
			\cmidrule{2-5}
			& (1) & (2) & (3) & (4) \\
			\midrule
			Panel A: Intangible Assets & LOW & HIGH & LOW & HIGH \\
			\midrule
			%&   &   &   &  \\
			$\Delta$DRET & 1.975*** & 3.026*** & -1.205 & -2.647** \\
			& (11.64) & (9.89) & (-1.23) & (-2.07) \\
			BN & -0.032 & -0.130*** & -0.193 & -0.060 \\
			& (-1.13) & (-4.26) & (-1.17) & (-0.38) \\
			\rowcolor[rgb]{ .906,  .902,  .902} \textit{(Pred. Sign)} & (-) & (-) & (+) & (+) \\
			\rowcolor[rgb]{ .906,  .902,  .902} $\Delta$DRET$\times$BN & -3.181*** & -6.326*** & 1.044 & 5.773** \\
			\rowcolor[rgb]{ .906,  .902,  .902} & (-10.61) & (-13.28) & (0.82) & (2.42) \\
			Constant & -3.065*** & -3.588*** & -0.478 & -3.469 \\
			& (-3.58) & (-6.35) & (-0.06) & (-1.18) \\
			&   &   &   &  \\
			Observations & 29,136 & 31,806 & 29,136 & 31,806 \\
			Adjusted R-squared & 0.118 & 0.146 & 0.132 & 0.123 \\
			\midrule
			Panel B: R\&D Expenses & LOW & HIGH & LOW & HIGH \\
			\midrule
			%&   &   &   &  \\
			$\Delta$DRET & 1.651*** & 1.946*** & -0.209 & -1.566 \\
			& (6.85) & (7.52) & (-0.30) & (-1.33) \\
			BN & 0.011 & -0.025 & -0.149 & -0.058 \\
			& (0.26) & (-0.91) & (-1.20) & (-0.50) \\
			\rowcolor[rgb]{ .906,  .902,  .902} \textit{(Pred. Sign)} & (-) & (-) & (+) & (+) \\
			\rowcolor[rgb]{ .906,  .902,  .902} $\Delta$DRET$\times$BN & -2.426*** & -2.983*** & -0.325 & 2.432* \\
			\rowcolor[rgb]{ .906,  .902,  .902} & (-5.65) & (-7.03) & (-0.39) & (1.66) \\
			Constant & -2.520*** & -2.678*** & -1.751 & -5.212 \\
			& (-4.66) & (-5.07) & (-0.25) & (-1.43) \\
			&   &   &   &  \\
			Observations & 19,740 & 22,608 & 19,740 & 22,608 \\
			Adjusted R-squared & 0.106 & 0.143 & 0.184 & 0.115 \\
			\bottomrule
			\bottomrule
		\end{tabular}%
	\end{center}
\end{table}%
	% Table generated by Excel2LaTeX from sheet 'T3'
\begin{table}[H]
	\begin{center}
		\tabcolsep=0.11cm
		\begin{tabular}{lcccccccccc}
			\multicolumn{11}{c}{\textbf{Table 5. Narrative Conservatism and Unconditional Conservatism (Continued)}} \\
			\toprule
			\toprule
			Dep. Variables & \multicolumn{2}{c}{NW} & \multicolumn{2}{c}{N8K} & \multicolumn{2}{c}{NITEM} & \multicolumn{2}{c}{NEXHIBIT} & \multicolumn{2}{c}{NGRAPH} \\
			\cmidrule{2-11}
			& (5) & (6) & (7) & (8) & (9) & (10) & (11) & (12) & (13) & (14) \\
			\midrule
			Panel A: Intangible Assets & LOW & HIGH & LOW & HIGH & LOW & HIGH & LOW & HIGH & LOW & HIGH \\
			\midrule
			%&   &   &   &   &   &  &   &   &   &  \\
			$\Delta$DRET & 0.041 & -0.142 & -0.033*** & -0.042*** & -0.098*** & -0.053 & -0.087 & -0.195*** & -0.148** & -0.467*** \\
			& (0.48) & (-1.02) & (-2.74) & (-2.90) & (-2.62) & (-1.25) & (-1.54) & (-2.95) & (-2.17) & (-3.88) \\
			BN & -0.002 & -0.029* & -0.002 & -0.000 & -0.007 & -0.003 & -0.000 & -0.007 & -0.001 & -0.018 \\
			& (-0.13) & (-1.92) & (-1.10) & (-0.19) & (-0.83) & (-0.45) & (-0.04) & (-0.80) & (-0.07) & (-1.17) \\
			\rowcolor[rgb]{ .906,  .902,  .902} \textit{(Pred. Sign)} & (+) & (+) & (+) & (+) & (+) & (+) & (+) & (+) & (+) & (+) \\
			\rowcolor[rgb]{ .906,  .902,  .902} $\Delta$DRET$\times$BN & -0.042 & 0.059 & 0.049*** & 0.076*** & 0.118* & 0.048 & 0.135 & 0.272*** & 0.219** & 0.622*** \\
			\rowcolor[rgb]{ .906,  .902,  .902} & (-0.34) & (0.32) & (3.01) & (3.48) & (1.95) & (0.76) & (1.51) & (2.72) & (2.14) & (3.18) \\
			Constant & -6.439*** & -7.127*** & -0.692*** & -0.692*** & -0.745*** & -0.890*** & -0.314 & -0.456*** & 0.156* & -0.173 \\
			& (-20.69) & (-21.31) & (-94.98) & (-64.07) & (-11.47) & (-14.91) & (-1.50) & (-2.71) & (1.79) & (-1.36) \\
			&   &   &   &   &   &   &   &   &   &  \\
			Observations & 29,136 & 31,806 & 29,136 & 31,806 & 29,136 & 31,806 & 29,136 & 31,806 & 29,136 & 31,806 \\
			Adjusted R-squared & 0.385 & 0.315 & 0.022 & 0.036 & 0.144 & 0.133 & 0.113 & 0.088 & 0.257 & 0.282 \\
			\midrule
			Panel B: R\&D Expenses & LOW & HIGH & LOW & HIGH & LOW & HIGH & LOW & HIGH & LOW & HIGH\\
			\midrule
			%&   &   &   &   &   &   &   &   &   &  \\
			$\Delta$DRET & -0.068 & 0.005 & -0.054*** & -0.031** & -0.120*** & -0.007 & -0.137** & -0.047 & -0.050 & -0.348*** \\
			& (-0.69) & (0.06) & (-2.60) & (-2.55) & (-3.08) & (-0.23) & (-1.98) & (-1.00) & (-0.63) & (-4.77) \\
			BN & -0.017 & -0.005 & -0.008*** & -0.001 & -0.006 & 0.005 & -0.003 & 0.013 & 0.011 & -0.020 \\
			& (-1.23) & (-0.44) & (-3.60) & (-0.38) & (-0.84) & (1.02) & (-0.23) & (1.59) & (0.56) & (-1.53) \\
			\rowcolor[rgb]{ .906,  .902,  .902} \textit{(Pred. Sign)} & (+) & (+) & (+) & (+) & (+) & (+) & (+) & (+) & (+) & (+) \\
			\rowcolor[rgb]{ .906,  .902,  .902} $\Delta$DRET$\times$BN & 0.032 & -0.010 & 0.054* & 0.049*** & 0.137** & 0.043 & 0.177** & 0.197*** & 0.128* & 0.388*** \\
			\rowcolor[rgb]{ .906,  .902,  .902} & (0.24) & (-0.08) & (1.95) & (4.22) & (2.03) & (1.03) & (2.22) & (3.08) & (1.71) & (4.30) \\
			Constant & -7.250*** & -7.660*** & -0.657*** & -0.676*** & -0.795*** & -0.852*** & -0.476*** & -0.400** & 0.394** & -0.109 \\
			& (-8.77) & (-18.41) & (-25.93) & (-63.28) & (-10.34) & (-12.76) & (-3.74) & (-2.21) & (2.07) & (-1.01) \\
			&   &   &   &   &   &   &   &   &   &  \\
			Observations & 19,740 & 22,608 & 19,740 & 22,608 & 19,740 & 22,608 & 19,740 & 22,608 & 19,740 & 22,608 \\
			Adjusted R-squared & 0.491 & 0.355 & 0.005 & 0.009 & 0.156 & 0.130 & 0.129 & 0.092 & 0.255 & 0.253 \\
			\bottomrule
			\bottomrule
		\end{tabular}%
	\end{center}
		\begin{footnotesize}
			\setcounter{equation}{0}
			\begin{equation}
				TEX_{i,t}=\beta_0+\beta_1\Delta DRET_{i,t-tlag}+\beta_2BN_{i,t-tlag}+\beta_3\Delta DRET_{i,t-tlag}\times 	BN_{i,t-tlag}+\sum\beta_nCONTROLS_{i,t}+\epsilon_{i,t}
			\end{equation}
			
			\noindent Table 5 presents the regression results of Equation (1) across high and low intangible assets and R\&D expenses subsamples. TEX represents a vector of textual properties. CONTROLS denotes a vector of control variables. See \hyperref[appc]{Appendix C} for variable definitions. All financial variables except returns are winsorized at 1\% and 99\% level. All regressions include full set of control variables, firm and year-month fixed effects. Standard errors are clustered at industry level identified by 4-digit SIC codes. ***, ** and * indicate significance at the 1\%, 5\% and 10\% levels in a two-tailed test.
		\end{footnotesize}
\end{table}%
\end{landscape}

%%%%%%%%%%%%%%%%%%%%%%%%% TABLE 6 Panel A
\newpage
%\begin{landscape}
% Table generated by Excel2LaTeX from sheet 'T6'
\begin{table}[H]	\label{T6PA}%
	\begin{center}
		\begin{tabular}{lcccc}
			\multicolumn{5}{c}{\textbf{Table 6. Panel A. Narrative Conservatism in Quarterly Reports}} \\
			\midrule
			\midrule
			& (1) & (2) & (3) & (4) \\
			Dep. Variables & TONE & TONE & NW & NW \\
			\midrule
			%&   &   &   &  \\
			QRET & -0.371*** & 0.095 & -0.039*** & -0.040*** \\
			& (-2.78) & (0.69) & (-3.54) & (-3.54) \\
			NEG & -0.077 & -0.075 & -0.004 & -0.005 \\
			& (-1.59) & (-1.52) & (-0.95) & (-1.08) \\
			\rowcolor[rgb]{ .906,  .902,  .902} \textit{(Pred. Sign)} & (+) & (+) & (+) & (+) \\
			\rowcolor[rgb]{ .906,  .902,  .902} QRET$\times$NEG & 2.274*** & 1.191*** & 0.140*** & 0.094*** \\
			\rowcolor[rgb]{ .906,  .902,  .902} & (8.19) & (5.20) & (6.56) & (5.12) \\
			SIZE &   & 0.540*** &   & -0.027*** \\
			&   & (6.36) &   & (-3.25) \\
			MTB &   & 0.046*** &   & 0.005*** \\
			&   & (3.79) &   & (5.18) \\
			LEV &   & -1.212** &   & -0.293*** \\
			&   & (-2.48) &   & (-10.11) \\
			EARN &   & 14.674*** &   & 0.635*** \\
			&   & (5.54) &   & (3.80) \\
			STD\_EARN &   & -7.233*** &   & -0.654*** \\
			&   & (-4.68) &   & (-6.85) \\
			BUSSEG &   & 0.468** &   & -0.019 \\
			&   & (2.22) &   & (-1.50) \\
			GEOSEG &   & 0.319* &   & 0.020* \\
			&   & (1.82) &   & (1.81) \\
			AF &   & -3.316*** &   & -0.043 \\
			&   & (-4.40) &   & (-1.07) \\
			AFE &   & 3.339*** &   & 0.168*** \\
			&   & (4.60) &   & (3.02) \\
			Constant & -18.117*** & -21.970*** & -8.224*** & -8.082*** \\
			& (-38.84) & (-36.79) & (-267.21) & (-156.81) \\
			&   &   &   &  \\
			Observations & 116,156 & 116,156 & 116,156 & 116,156 \\
			Adjusted R-squared & 0.586 & 0.597 & 0.695 & 0.698 \\
			\bottomrule
			\bottomrule
		\end{tabular}%
	\end{center}
\begin{footnotesize}
	\setcounter{equation}{0}
	\begin{equation}
		TEX_{i,t}=\beta_0+\beta_1QRET_{i,t}+\beta_2NEG_{i,t}+\beta_3QRET_{i,t}\times NEG_{i,t}+\sum\beta_nCONTROLS_{i,t}+\epsilon_{i,t}
	\end{equation}
	
	\noindent Table 6 Panel A presents the regression results of Equation (1) using subsamples of MD\&A (Column 1 and 3) and NFS (Column 2 and 4) sections. TEX represents a vector of textual properties that consists of NW\_MDA, NW\_NFS, TONE\_MDA and TONE\_NFS. CONTROLS denotes a vector of control variables. See \hyperref[appc]{Appendix C} for variable definitions. All financial variables except returns are winsorized at 1\% and 99\% level. All regressions include firm and year-quarter fixed effects and standard errors are clustered at industry level identified by 4-digit SIC codes. ***, ** and * indicate significance at the 1\%, 5\% and 10\% levels in a two-tailed test.
\end{footnotesize}
\end{table}%
%\end{landscape}

%%%%%%%%%%%%%%%%%%%%%%%%% TABLE 6 Panel B
\newpage
%\begin{landscape}
% Table generated by Excel2LaTeX from sheet 'T6'
\begin{table}[H] \label{T6PB}%
	\begin{center}
		\begin{tabular}{lcccc}
			\multicolumn{5}{c}{\textbf{Table 6. Panel B. Narrative Conservatism 10-Q Sections}} \\
			\midrule
			\midrule
			Dep. Variables & \multicolumn{2}{c}{TONE} & \multicolumn{2}{c}{NW} \\
			\cmidrule{2-5}
			& (1) & (2) & (3) & (4) \\
			Section & MDA & NFS & MDA & NFS \\
			\midrule
			%&   &   &   &  \\
			QRET & 0.109 & 0.297 & -0.055*** & -0.033* \\
			& (0.64) & (1.15) & (-4.34) & (-1.70) \\
			NEG & -0.123** & 0.014 & -0.012*** & -0.005 \\
			& (-1.98) & (0.17) & (-3.05) & (-1.01) \\
			\rowcolor[rgb]{ .906,  .902,  .902} \textit{(Pred. Sign)} & (+) & (+) & (+) & (+) \\
			\rowcolor[rgb]{ .906,  .902,  .902} QRET$\times$NEG & 1.423*** & 0.882* & 0.102*** & 0.055* \\
			\rowcolor[rgb]{ .906,  .902,  .902} & (4.54) & (1.88) & (4.18) & (1.65) \\
			SIZE & 0.626*** & 0.900*** & -0.030*** & -0.013 \\
			& (4.26) & (5.14) & (-3.36) & (-1.01) \\
			MTB & 0.021 & 0.054** & 0.003** & 0.004*** \\
			& (1.12) & (2.21) & (2.41) & (3.28) \\
			LEV & -0.213 & -0.802 & -0.189*** & -0.362*** \\
			& (-0.33) & (-0.94) & (-5.32) & (-5.88) \\
			EARN & 17.163*** & 12.079*** & 0.470** & 0.693*** \\
			& (5.26) & (5.69) & (2.16) & (3.83) \\
			STD\_EARN & -8.090*** & -6.020** & -0.547*** & -0.816*** \\
			& (-4.64) & (-2.20) & (-3.35) & (-6.19) \\
			BUSSEG & -0.065 & -0.159 & -0.057*** & -0.031 \\
			& (-0.23) & (-0.45) & (-2.93) & (-1.58) \\
			GEOSEG & 0.052 & 0.999*** & 0.063*** & 0.036** \\
			& (0.16) & (2.61) & (3.01) & (1.98) \\
			AF & 1.979* & -0.343 & 0.140 & -0.073 \\
			& (1.86) & (-0.22) & (1.61) & (-0.95) \\
			AFE & 7.938*** & 4.137*** & 0.227*** & 0.243*** \\
			& (7.81) & (3.74) & (3.20) & (3.56) \\
			Constant & -7.264* & -12.393** & -7.167*** & -7.224*** \\
			& (-1.84) & (-2.57) & (-15.46) & (-18.08) \\
			&   &   &   &  \\
			Observations & 48,089 & 48,089 & 48,089 & 48,089 \\
			Adjusted R-squared & 0.559 & 0.579 & 0.734 & 0.816 \\
			\bottomrule
			\bottomrule
		\end{tabular}%
	\end{center}
\begin{footnotesize}
	\setcounter{equation}{0}
	\begin{equation}
		TEX_{i,t}=\beta_0+\beta_1QRET_{i,t}+\beta_2NEG_{i,t}+\beta_3QRET_{i,t}\times NEG_{i,t}+\sum\beta_nCONTROLS_{i,t}+\epsilon_{i,t}
	\end{equation}
	
	\noindent Table 6 Panel B presents the regression results of Equation (1) using subsamples of MD\&A (Column 1 and 3) and NFS (Column 2 and 4) sections. TEX represents a vector of textual properties that consists of NW\_MDA, NW\_NFS, TONE\_MDA and TONE\_NFS. CONTROLS denotes a vector of control variables. See \hyperref[appc]{Appendix C} for variable definitions. All financial variables except returns are winsorized at 1\% and 99\% level. All regressions include firm and year-quarter fixed effects and standard errors are clustered at industry level identified by 4-digit SIC codes. ***, ** and * indicate significance at the 1\%, 5\% and 10\% levels in a two-tailed test.
\end{footnotesize}
\end{table}%
%\end{landscape}


%%%%%%%%%%%%%%%%%%%%%%%%% TABLE 7
\newpage
\begin{landscape}
	% Table generated by Excel2LaTeX from sheet 'T3'
\begin{table}[H] \label{T7}
	\begin{center}
		\tabcolsep=0.3cm
		\begin{tabular}{lcccc}
			\multicolumn{5}{c}{\textbf{Table 7. Narrative Conservatism and Unconditional Conservatism}} \\
			\toprule
			\toprule
			Dep. Variables & \multicolumn{2}{c}{TLAG} & \multicolumn{2}{c}{TONE} \\
			\cmidrule{2-5}
			& (1) & (2) & (3) & (4) \\
			\midrule
			Panel A: Intangible Assets & LOW & HIGH & LOW & HIGH \\
			\midrule
			%&   &   &   &  \\
			$\Delta$DRET & 1.975*** & 3.026*** & -1.205 & -2.647** \\
			& (11.64) & (9.89) & (-1.23) & (-2.07) \\
			BN & -0.032 & -0.130*** & -0.193 & -0.060 \\
			& (-1.13) & (-4.26) & (-1.17) & (-0.38) \\
			\rowcolor[rgb]{ .906,  .902,  .902} \textit{(Pred. Sign)} & (-) & (-) & (+) & (+) \\
			\rowcolor[rgb]{ .906,  .902,  .902} $\Delta$DRET$\times$BN & -3.181*** & -6.326*** & 1.044 & 5.773** \\
			\rowcolor[rgb]{ .906,  .902,  .902} & (-10.61) & (-13.28) & (0.82) & (2.42) \\
			Constant & -3.065*** & -3.588*** & -0.478 & -3.469 \\
			& (-3.58) & (-6.35) & (-0.06) & (-1.18) \\
			&   &   &   &  \\
			Observations & 29,136 & 31,806 & 29,136 & 31,806 \\
			Adjusted R-squared & 0.118 & 0.146 & 0.132 & 0.123 \\
			\midrule
			Panel B: R\&D Expenses & LOW & HIGH & LOW & HIGH \\
			\midrule
			%&   &   &   &  \\
			$\Delta$DRET & 1.651*** & 1.946*** & -0.209 & -1.566 \\
			& (6.85) & (7.52) & (-0.30) & (-1.33) \\
			BN & 0.011 & -0.025 & -0.149 & -0.058 \\
			& (0.26) & (-0.91) & (-1.20) & (-0.50) \\
			\rowcolor[rgb]{ .906,  .902,  .902} \textit{(Pred. Sign)} & (-) & (-) & (+) & (+) \\
			\rowcolor[rgb]{ .906,  .902,  .902} $\Delta$DRET$\times$BN & -2.426*** & -2.983*** & -0.325 & 2.432* \\
			\rowcolor[rgb]{ .906,  .902,  .902} & (-5.65) & (-7.03) & (-0.39) & (1.66) \\
			Constant & -2.520*** & -2.678*** & -1.751 & -5.212 \\
			& (-4.66) & (-5.07) & (-0.25) & (-1.43) \\
			&   &   &   &  \\
			Observations & 19,740 & 22,608 & 19,740 & 22,608 \\
			Adjusted R-squared & 0.106 & 0.143 & 0.184 & 0.115 \\
			\bottomrule
			\bottomrule
		\end{tabular}%
	\end{center}
		\begin{footnotesize}
			\setcounter{equation}{0}
			\begin{equation}
				TEX_{i,t}=\beta_0+\beta_1\Delta DRET_{i,t-tlag}+\beta_2BN_{i,t-tlag}+\beta_3\Delta DRET_{i,t-tlag}\times 		BN_{i,t-tlag}+\sum\beta_nCONTROLS_{i,t}+\epsilon_{i,t}
			\end{equation}
			
			\noindent Table 7 presents the regression results of Equation (1) across high and low intangible assets and R\&D expenses subsamples. TEX represents a vector of textual properties. CONTROLS denotes a vector of control variables. See \hyperref[appc]{Appendix C} for variable definitions. All financial variables except returns are winsorized at 1\% and 99\% level. All regressions include full set of control variables, firm and year-month fixed effects. Standard errors are clustered at industry level identified by 4-digit SIC codes. ***, ** and * indicate significance at the 1\%, 5\% and 10\% levels in a two-tailed test.
		\end{footnotesize}
\end{table}%
	% Table generated by Excel2LaTeX from sheet 'T3'
\begin{table}[H]
	\begin{center}
		\tabcolsep=0.11cm
		\begin{tabular}{lcccccccccc}
			\multicolumn{11}{c}{\textbf{Table 7. Narrative Conservatism in Voluntary and Mandatory Disclosure (Continued)}} \\
			\toprule
			\toprule
			Dep. Variables & \multicolumn{2}{c}{NW} & \multicolumn{2}{c}{N8K} & \multicolumn{2}{c}{NITEM} & \multicolumn{2}{c}{NEXHIBIT} & \multicolumn{2}{c}{NGRAPH} \\
			\cmidrule{2-11}
			& (5) & (6) & (7) & (8) & (9) & (10) & (11) & (12) & (13) & (14) \\
			Disclosure Type & VD & MD & VD & MD & VD & MD & VD & MD & VD & MD \\
			\midrule
			%&   &   &   &   &   &  &   &   &   &  \\
			$\Delta$DRET & -0.156** & 0.039 & -0.063*** & -0.051 & -0.048*** & -0.020* & -0.092** & -0.017 & -0.153*** & 0.030 \\
			& (-2.37) & (0.31) & (-2.72) & (-1.08) & (-4.07) & (-1.65) & (-2.25) & (-0.18) & (-2.64) & (0.47) \\
			BN & -0.018** & 0.002 & -0.004 & -0.007 & -0.002 & -0.002 & -0.003 & 0.000 & -0.017 & 0.010 \\
			& (-2.14) & (0.13) & (-0.99) & (-1.08) & (-1.59) & (-1.59) & (-0.39) & (0.01) & (-1.61) & (1.02) \\
			\rowcolor[rgb]{ .906,  .902,  .902} \textit{(Pred. Sign)} & (+) & (+) & (+) & (+) & (+) & (+) & (+) & (+) & (+) & (+) \\
			\rowcolor[rgb]{ .906,  .902,  .902} $\Delta$DRET$\times$BN & 0.210** & 0.003 & 0.093*** & 0.045 & 0.070*** & 0.026 & 0.175*** & 0.050 & 0.133 & 0.031 \\
			\rowcolor[rgb]{ .906,  .902,  .902} & (2.18) & (0.02) & (2.91) & (0.75) & (5.44) & (1.60) & (2.86) & (0.47) & (1.61) & (0.42) \\
			SIZE & 0.011 & 0.035*** & 0.003 & -0.003 & -0.001 & -0.001 & 0.000 & 0.005 & 0.006 & -0.003 \\
			& (1.18) & (2.64) & (0.90) & (-0.63) & (-0.50) & (-0.88) & (0.04) & (0.45) & (0.56) & (-0.41) \\
			MTB & 0.000 & -0.004** & -0.000 & -0.001 & -0.000 & 0.000** & -0.002*** & -0.001 & -0.003** & -0.001 \\
			& (0.02) & (-2.20) & (-0.16) & (-0.75) & (-1.01) & (1.99) & (-2.67) & (-0.93) & (-2.02) & (-0.69) \\
			LEV & -0.102** & 0.073 & -0.033** & 0.004 & -0.012** & -0.001 & -0.021 & -0.026 & -0.004 & -0.008 \\
			& (-2.42) & (1.02) & (-2.30) & (0.16) & (-2.57) & (-0.22) & (-1.00) & (-0.51) & (-0.08) & (-0.22) \\
			EARN & 0.302*** & 0.270 & 0.047 & 0.103 & -0.003 & -0.009 & 0.109* & 0.054 & -0.110 & 0.054 \\
			& (2.72) & (1.42) & (1.20) & (1.34) & (-0.23) & (-0.99) & (1.94) & (0.44) & (-1.17) & (0.58) \\
			STD\_EARN & -0.254* & -0.021 & -0.096* & -0.078 & -0.004 & -0.018 & -0.014 & -0.255 & 0.373** & -0.136 \\
			& (-1.94) & (-0.08) & (-1.69) & (-0.81) & (-0.25) & (-0.91) & (-0.17) & (-1.34) & (2.17) & (-1.10) \\
			BUSSEG & -0.004 & -0.025 & 0.006 & -0.017** & 0.000 & 0.000 & 0.012* & -0.027* & -0.015 & 0.001 \\
			& (-0.26) & (-1.11) & (1.35) & (-2.02) & (0.28) & (0.10) & (1.71) & (-1.75) & (-0.69) & (0.11) \\
			GEOSEG & 0.008 & 0.004 & -0.004 & 0.003 & 0.002 & 0.003** & -0.022*** & 0.008 & -0.018 & -0.006 \\
			& (0.67) & (0.20) & (-0.87) & (0.33) & (1.28) & (2.54) & (-3.67) & (0.55) & (-0.92) & (-0.57) \\
			AF & -0.033 & 0.013 & 0.003 & 0.005 & 0.002 & 0.001 & 0.026 & 0.031 & -0.087 & -0.073 \\
			& (-0.43) & (0.18) & (0.13) & (0.15) & (0.17) & (0.09) & (0.74) & (0.37) & (-1.08) & (-1.57) \\
			AFE & 0.013 & -0.266** & 0.034 & -0.080 & -0.019** & 0.022** & 0.005 & -0.192** & -0.170* & -0.022 \\
			& (0.16) & (-2.05) & (1.17) & (-1.64) & (-2.33) & (2.19) & (0.12) & (-2.17) & (-1.77) & (-0.35) \\
			Constant & -6.786*** & -8.541*** & -0.889*** & -0.839*** & -0.687*** & -0.693*** & -0.436*** & -0.585*** & 0.000 & -0.020 \\
			& (-28.58) & (-14.52) & (-18.87) & (-10.34) & (-96.80) & (-130.77) & (-4.01) & (-2.98) & (0.00) & (-0.44) \\
			&   &   &   &   &   &   &   &   &   &  \\
			Observations & 53,460 & 21,900 & 53,460 & 21,900 & 53,460 & 21,900 & 53,460 & 21,900 & 53,460 & 21,900 \\
			Adjusted R-squared & 0.448 & 0.505 & 0.212 & 0.073 & 0.040 & -0.023 & 0.162 & 0.139 & 0.360 & 0.141 \\
			\bottomrule
			\bottomrule
		\end{tabular}%
	\end{center}
		\begin{footnotesize}
			\setcounter{equation}{0}
			\begin{equation}
				TEX_{i,t}=\beta_0+\beta_1\Delta DRET_{i,t-tlag}+\beta_2BN_{i,t-tlag}+\beta_3\Delta DRET_{i,t-tlag}\times 	BN_{i,t-tlag}+\sum\beta_nCONTROLS_{i,t}+\epsilon_{i,t}
			\end{equation}
			
			\noindent Table 7 presents the regression results of Equation (1) across voluntary and mandatory disclosure subsamples. TEX represents a vector of textual properties. CONTROLS denotes a vector of control variables. See \hyperref[appc]{Appendix C} for variable definitions. All financial variables except returns are winsorized at 1\% and 99\% level. All regressions include firm and year-month fixed effects and standard errors are clustered at industry level identified by 4-digit SIC codes. ***, ** and * indicate significance at the 1\%, 5\% and 10\% levels in a two-tailed test.
		\end{footnotesize}
\end{table}%
\end{landscape}


%%%%%%%%%%%%%%%%%%%%%%%%%% TABLE 7
%\newpage
%%\begin{landscape}
%% Table generated by Excel2LaTeX from sheet 'T3'
\begin{table}[H] \label{T7}
	\begin{center}
		\tabcolsep=0.3cm
		\begin{tabular}{lcccc}
			\multicolumn{5}{c}{\textbf{Table 7. Narrative Conservatism and Unconditional Conservatism}} \\
			\toprule
			\toprule
			Dep. Variables & \multicolumn{2}{c}{TLAG} & \multicolumn{2}{c}{TONE} \\
			\cmidrule{2-5}
			& (1) & (2) & (3) & (4) \\
			\midrule
			Panel A: Intangible Assets & LOW & HIGH & LOW & HIGH \\
			\midrule
			%&   &   &   &  \\
			$\Delta$DRET & 1.975*** & 3.026*** & -1.205 & -2.647** \\
			& (11.64) & (9.89) & (-1.23) & (-2.07) \\
			BN & -0.032 & -0.130*** & -0.193 & -0.060 \\
			& (-1.13) & (-4.26) & (-1.17) & (-0.38) \\
			\rowcolor[rgb]{ .906,  .902,  .902} \textit{(Pred. Sign)} & (-) & (-) & (+) & (+) \\
			\rowcolor[rgb]{ .906,  .902,  .902} $\Delta$DRET$\times$BN & -3.181*** & -6.326*** & 1.044 & 5.773** \\
			\rowcolor[rgb]{ .906,  .902,  .902} & (-10.61) & (-13.28) & (0.82) & (2.42) \\
			Constant & -3.065*** & -3.588*** & -0.478 & -3.469 \\
			& (-3.58) & (-6.35) & (-0.06) & (-1.18) \\
			&   &   &   &  \\
			Observations & 29,136 & 31,806 & 29,136 & 31,806 \\
			Adjusted R-squared & 0.118 & 0.146 & 0.132 & 0.123 \\
			\midrule
			Panel B: R\&D Expenses & LOW & HIGH & LOW & HIGH \\
			\midrule
			%&   &   &   &  \\
			$\Delta$DRET & 1.651*** & 1.946*** & -0.209 & -1.566 \\
			& (6.85) & (7.52) & (-0.30) & (-1.33) \\
			BN & 0.011 & -0.025 & -0.149 & -0.058 \\
			& (0.26) & (-0.91) & (-1.20) & (-0.50) \\
			\rowcolor[rgb]{ .906,  .902,  .902} \textit{(Pred. Sign)} & (-) & (-) & (+) & (+) \\
			\rowcolor[rgb]{ .906,  .902,  .902} $\Delta$DRET$\times$BN & -2.426*** & -2.983*** & -0.325 & 2.432* \\
			\rowcolor[rgb]{ .906,  .902,  .902} & (-5.65) & (-7.03) & (-0.39) & (1.66) \\
			Constant & -2.520*** & -2.678*** & -1.751 & -5.212 \\
			& (-4.66) & (-5.07) & (-0.25) & (-1.43) \\
			&   &   &   &  \\
			Observations & 19,740 & 22,608 & 19,740 & 22,608 \\
			Adjusted R-squared & 0.106 & 0.143 & 0.184 & 0.115 \\
			\bottomrule
			\bottomrule
		\end{tabular}%
	\end{center}
		\begin{footnotesize}
			\setcounter{equation}{0}
			\begin{equation}
				TEX_{i,t}=\beta_0+\beta_1\Delta DRET_{i,t-tlag}+\beta_2BN_{i,t-tlag}+\beta_3\Delta DRET_{i,t-tlag}\times 		BN_{i,t-tlag}+\sum\beta_nCONTROLS_{i,t}+\epsilon_{i,t}
			\end{equation}
			
			\noindent Table 7 presents the regression results of Equation (1) across high and low intangible assets and R\&D expenses subsamples. TEX represents a vector of textual properties. CONTROLS denotes a vector of control variables. See \hyperref[appc]{Appendix C} for variable definitions. All financial variables except returns are winsorized at 1\% and 99\% level. All regressions include full set of control variables, firm and year-month fixed effects. Standard errors are clustered at industry level identified by 4-digit SIC codes. ***, ** and * indicate significance at the 1\%, 5\% and 10\% levels in a two-tailed test.
		\end{footnotesize}
\end{table}%
%%\end{landscape}
%
%%%%%%%%%%%%%%%%%%%%%%%%%% TABLE 8
%\newpage
%%\begin{landscape}
%% Table generated by Excel2LaTeX from sheet 'T8PA'
\begin{table}[htbp] \label{T8}
  \centering
    \begin{tabular}{lcccccc}
    \multicolumn{7}{c}{\textbf{Table 8. Managerial Incentives and Narrative Conservatism}} \\
    \midrule
    \midrule
    Dep. Vars.& \multicolumn{2}{c}{NW} & \multicolumn{2}{c}{TONE} & \multicolumn{2}{c}{TLAG}\\
    \midrule
    \textbf{Panel A}  & (1) & (2) & (3) & (4) & (5) & (6) \\
    \multicolumn{1}{l}{\textbf{Option Value}} & LOW & HIGH & LOW & HIGH & LOW & HIGH \\
    \cmidrule{2-7}
    \multicolumn{1}{l}{QRET} & 0.041 & 0.104*** & 1.164*** & 0.545 & -0.224 & -0.151 \\
      & (0.98) & (2.66) & (3.48) & (1.38) & (-0.71) & (-0.54) \\
    \multicolumn{1}{l}{NEG} & 0.023 & -0.003 & -0.082 & -0.132 & 0.095 & -0.070 \\
      & (1.38) & (-0.26) & (-0.70) & (-1.01) & (0.97) & (-0.56) \\
%    \rowcolor[rgb]{ .933,  .925,  .882} Sign Prediction & - & - & + & + & + & + \\
    \rowcolor[rgb]{ .933,  .925,  .882} \multicolumn{1}{l}{QRET$\times$NEG} & -0.177** & -0.252*** & 1.615** & 1.584** & -0.859 & -0.693 \\
    \rowcolor[rgb]{ .933,  .925,  .882}  & (-2.24) & (-4.44) & (2.57) & (2.55) & (-1.63) & (-1.51) \\
%    \multicolumn{1}{l}{SIZE} & -0.007 & 0.026 & 1.316*** & 1.158*** & -0.325** & -0.023 \\
%      & (-0.32) & (1.39) & (5.49) & (5.58) & (-2.15) & (-0.19) \\
%    \multicolumn{1}{l}{MTB} & -0.003 & -0.007*** & 0.101*** & 0.124*** & -0.083*** & -0.006 \\
%      & (-0.80) & (-3.06) & (2.63) & (4.13) & (-3.06) & (-0.35) \\
%    \multicolumn{1}{l}{LEV} & 0.508*** & 0.298*** & -1.696 & -1.473 & 1.485 & 1.000 \\
%      & (5.30) & (3.27) & (-1.59) & (-1.28) & (1.62) & (1.60) \\
%    \multicolumn{1}{l}{Constant} & 8.185*** & 8.324*** & -20.044*** & -6.415 & 46.377*** & 40.805*** \\
%      & (57.72) & (13.16) & (-3.64) & (-1.35) & (27.95) & (11.75) \\
    \rowcolor[rgb]{ .933,  .925,  .882} \multicolumn{1}{l}{Diff. QRET$\times$NEG} & \multicolumn{2}{c}{0.076$^{\star\star\star}$} & \multicolumn{2}{c}{0.030} & \multicolumn{2}{c}{-0.166} \\
    \rowcolor[rgb]{ .933,  .925,  .882}  & \multicolumn{2}{c}{(2.02)} & \multicolumn{2}{c}{(0.14)} & \multicolumn{2}{c}{(-0.86)} \\
      &   &   &   &   &   &  \\
    \multicolumn{1}{l}{Observations} & 15,229 & 15,226 & 15,229 & 15,226 & 15,229 & 15,226 \\
    \multicolumn{1}{l}{Adjusted R-squared} & 0.443 & 0.493 & 0.551 & 0.610 & 0.540 & 0.588 \\
    \midrule
    \textbf{Panel B}  & (1) & (2) & (3) & (4) & (5) & (6) \\
    \multicolumn{1}{l}{\textbf{SEO}} & NO & YES & NO & YES & NO & YES \\
    \cmidrule{2-7}
    QRET & 0.060*** & 0.027 & -0.222 & 0.060 & -0.506*** & -0.153 \\
    & (3.70) & (1.52) & (-1.32) & (0.41) & (-2.89) & (-0.79) \\
    NEG & -0.002 & -0.001 & -0.104 & -0.073 & 0.048 & 0.050 \\
    & (-0.25) & (-0.11) & (-1.56) & (-0.82) & (1.00) & (0.84) \\
    \rowcolor[rgb]{ .933,  .925,  .882} QRET$\times$NEG & -0.153*** & -0.163*** & 2.448*** & 1.357*** & -0.510* & -0.415 \\
    \rowcolor[rgb]{ .933,  .925,  .882} & (-5.33) & (-4.25) & (8.25) & (3.16) & (-1.73) & (-1.49) \\
    \rowcolor[rgb]{ .933,  .925,  .882} \multicolumn{1}{l}{Diff. QRET$\times$NEG} & \multicolumn{2}{c}{0.009} & \multicolumn{2}{c}{1.091$^{\star\star\star}$} & \multicolumn{2}{c}{-0.095} \\
    \rowcolor[rgb]{ .933,  .925,  .882}  & \multicolumn{2}{c}{(1.07)} & \multicolumn{2}{c}{(5.99)} & \multicolumn{2}{c}{(-0.59)} \\
    &   &   &   &   &   &  \\
    Observations & 45,490 & 37,054 & 45,490 & 37,054 & 45,490 & 37,054 \\
    Adjusted R-squared & 0.696 & 0.687 & 0.552 & 0.623 & 0.634 & 0.674 \\
    \midrule
    \multicolumn{1}{l}{Year-quarter FE} & YES & YES & YES & YES & YES & YES \\
    \multicolumn{1}{l}{Firm FE} & YES & YES & YES & YES & YES & YES \\
    \multicolumn{1}{l}{Industry clustered SE} & YES & YES & YES & YES & YES & YES \\
    \multicolumn{1}{l}{Controls} & YES & YES & YES & YES & YES & YES \\
    \bottomrule
    \bottomrule
    \end{tabular}%
\end{table}%

%%\end{landscape}
%
%%%%%%%%%%%%%%%%%%%%%%%%%% TABLE 9
%\newpage
%%\begin{landscape}
%% Table generated by Excel2LaTeX from sheet 'T10'
\begin{table}[H]   \label{T9}
  \begin{center}
  	    \begin{tabular}{lcccccc}
  		\multicolumn{7}{c}{\textbf{Table 9. Narrative Conservatism, Intangible Assets and R\&D Expenses}} \\
  		\midrule
  		\midrule
  		Dep. Variables & \multicolumn{2}{c}{NW} & \multicolumn{2}{c}{TONE} & \multicolumn{2}{c}{TLAG} \\
  		& (1) & \multicolumn{1}{c}{(2)} & (3) & \multicolumn{1}{c}{(4)} & (5) & \multicolumn{1}{c}{(6)} \\
  		\midrule
  		\textbf{Panel A: Intangible Assets} & LOW & HIGH & LOW & HIGH & LOW & HIGH \\
  		\cmidrule{2-7}
%  		QRET & 0.002 & 0.037* & 0.566*** & 0.466 & -0.527*** & -0.304 \\
%  		& (0.20) & (1.95) & (3.53) & (1.55) & (-3.55) & (-1.27) \\
%  		NEG & 0.006* & 0.010*** & 0.015 & -0.134** & -0.023 & 0.051 \\
%  		& (1.70) & (2.66) & (0.21) & (-2.16) & (-0.44) & (0.89) \\
  		\rowcolor[rgb]{ .933,  .925,  .882} \textit{(Pred. Sign)} & (-) & (-) & (+) & (+) & (+) & (+) \\
  		\rowcolor[rgb]{ .933,  .925,  .882} QRET$\times$NEG & -0.024 & -0.068*** & 0.469 & 0.475 & -0.109 & -0.093 \\
  		\rowcolor[rgb]{ .933,  .925,  .882} & (-1.21) & (-2.71) & (1.50) & (1.08) & (-0.44) & (-0.24) \\
  		&   &   &   &   &   &  \\
  		Observations & 29,636 & 29,634 & 29,636 & 29,634 & 29,636 & 29,634 \\
  		Adjusted R-squared & 0.831 & 0.798 & 0.708 & 0.678 & 0.654 & 0.693 \\
  		\midrule
  		\textbf{Panel B: R\&D Expenses} & LOW & HIGH & LOW & HIGH & LOW & HIGH \\
  		\cmidrule{2-7}
  
%  		QRET & 0.015 & 0.055** & 0.659** & 0.525*** & -0.413** & -0.450* \\
%  		 & (0.75) & (2.18) & (2.52) & (2.68) & (-2.05) & (-1.70) \\
%  		NEG & 0.000 & 0.015 & -0.061 & -0.120 & 0.109 & -0.025 \\
%  		 & (0.05) & (1.64) & (-0.64) & (-1.20) & (1.50) & (-0.34) \\
  		\rowcolor[rgb]{ .933,  .925,  .882} \textit{(Pred. Sign)} & (-) & (-) & (+) & (+) & (+) & (+) \\
  		\rowcolor[rgb]{ .933,  .925,  .882} QRET$\times$NEG & -0.065 & -0.075** & 0.710 & 0.048 & 0.336 & -0.029 \\
  		\rowcolor[rgb]{ .933,  .925,  .882}  & (-1.56) & (-2.45) & (1.53) & (0.10) & (1.15) & (-0.06) \\
  		&   &   &   &   &   &  \\
  		Observations & 22,899 & 22,898 & 22,899 & 22,898 & 22,899 & 22,898 \\
  		Adjusted R-squared & 0.623 & 0.682 & 0.581 & 0.635 & 0.626 & 0.619 \\
  		
  		\bottomrule
  		\bottomrule
  	\end{tabular}%
  \end{center}
\begin{footnotesize}
	\setcounter{equation}{0}
	\begin{equation}
		TEX_{i,t}=\beta_0+\beta_1QRET_{i,t}+\beta_2NEG_{i,t}+\beta_3QRET_{i,t}\times NEG_{i,t}+\sum\beta_nCONTROLS_{i,t}+\epsilon_{i,t}
	\end{equation}
	
	\noindent Table 9 presents the regression results of Equation (1) using 10-Q subsamples of intangible assets (Panel A) and R\&D expenses (Panel B). TEX represents a vector of textual properties that consists of NW, TONE and TLAG. All regressions control for SIZE, MTB, LEV, EARN, STD\_RET, STD\_EARN, AGE, BUSSEG, GEOSEG, AFE and AF (untabulated). See \hyperref[appb]{Appendix B} for variable definitions. All financial variables except returns are winsorized at 1\% and 99\% level. All regressions include firm and time fixed effects and standard errors are clustered at industry level identified by 4-digit SIC codes. ***, ** and * indicate significance at the 1\%, 5\% and 10\% levels in a two-tailed test.
\end{footnotesize}
\end{table}%

%%\end{landscape}

\newpage
\section*{Appendix}
\subsection*{Appendix A: 10-Q and 8-K parsing}
\label{appa}
We develop a Python program to automatically parse, process and retrieve 10-K and 8-K filings from EDGAR database. Our algorithm consists of the following steps:

1. Download all quarterly master indexes from EDGAR using \textit{python-edgar}\footnote{Python-edgar package documentation available at \url{https://github.com/edouardswiac/python-edgar/blob/master/README.md}} package.

2. Filter all 10-Q and 8-K filings\footnote{Our analysis exclude amendments such as 10-Q/A and 8-K/A} from EDGAR master index files and obtain url of the \textit{filing detail} webpage\footnote{One example of filing detail webpage is available at \url{https://www.sec.gov/Archives/edgar/data/320193/000032019320000050/0000320193-20-000050-index.html}} for each of the 10-Q and 8-K filings. 

3. Extract (a) identification information\footnote{For example cik, accession number, reporting period, filing date and 8-K items etc.} and (b) url of report in HTM/TXT format\footnote{One example of report in HTM format is available at \url{https://www.sec.gov/Archives/edgar/data/320193/000032019320000050/a8-kq220203282020.htm}. We first search for url of main report in HTM format. If HTM format main report is not available, then we extract the url of TXT format full report. Each EDGAR filing can be accessed in three formats at maximum: regular text (*.txt), web pages (*.htm) and eXtensible Business Reporting Language, also known as XBRL (*.xml). Early filings in EDGAR are only in TXT format. Later filings extend to HTM format, and in 2009 SEC adopted the XBRL for all corporate filings \cite{secFinalRuleInteractive2009}. Therefore, current existing EDGAR filings all contain a TXT file, and depending on their filing date and company reporting policy they may or may not contain HTM or XML files. Normally all filings in XML format are also available in HTM format. We manually checked 100 random filings that are in XML format, and all of them are also available in HTM format with the same content. The TXT files usually contain not only the main report, but also all other additional filing materials (if any) such as graphics, exhibits and press release etc. However, the HTM files only contain the main report. We mainly focus on HTM files other than TXT files because the former naturally filters out less relevant information, and provides a cleaner textual content of the essential information. XML files are not parsed due to low tractability. } from the \textit{filing detail} webpage for each of the 10-Q and 8-K filings. 

4. Parse and cleanse\footnote{Cleansing steps are: (a) delete nondisplay section; (b) delete all tables that contains more than 4 numbers; and (c) delete all HTML tags} all 10-Q and 8-K filings with url of HTM/TXT format report, using \textit{beautiful soup}\footnote{Beautiful soup package documentation available at \url{https://www.crummy.com/software/BeautifulSoup/bs4/doc/}} package. 

5. Save all clean 10-Q and 8-K filings to local device. 

6. Perform word count on clean 10-Q and 8-K filings using LM dictionary\footnote{LM dictionary available at \url{https://sraf.nd.edu/textual-analysis/resources/\#LM\%20Sentiment\%20Word\%20Lists}}. 

All Python scripts and data are available online via \url{https://github.com/fengzhi22/narrative_conservatism}.

\subsection*{Appendix B: Textual Variable Definition}
\label{appb}
\begin{table}[H]
	\centering
	\begin{tabular}{lp{15cm}p{15cm}}
		\textbf{Variable} & \textbf{Definition} \\
		NW & Number of words, defined as the natural logarithm of one plus the count of total words (nw)\\
		nw & Raw count of total words\\
		TONE & Tone, defined as number of net positive words per thousand total words, calculated as total number of positive words minus the sum of total number of negative words and total number of negations, and multiply the previous result by one thousand\\
		TLAG & Time lag, defined as number of days elapsed between the news release date (CRSP entry date) and document filing date (EDGAR filing date)\\
		ABTONE & Abnormal tone, calculated as the residual of the cross-sectional expected tone model (Equation 3) in \citet*{huangToneManagement2014}\\
		N8K & Number of 8-Ks reported in one day\\
		NITEM & Number of 8-K items reported in one day\\
		
	\end{tabular}%
\end{table}%

\subsection*{Appendix C: Financial Variable Definition}
\label{appc}
\begin{table}[H]
	\centering
	\begin{tabular}{lp{15cm}p{15cm}}
		\textbf{Variable} & \textbf{Definition} \\
		
		EARN & Quarterly earnings, defined as quarterly earnings before extraordinary items (Compustat data item IBQ) scaled by beginning-of-quarter total assets (Compustat data item ATQ) \\
		$\Delta$EARN & Change in quarterly earnings, defined as current quarterly earnings minus one-quarter-lagged quarterly earnings \\
		LEV & Leverage ratio, defined as beginning-of-quarter short term debt (Compustat data item DLCQ) plus beginning-of-quarter long term debt (Compustat data item DLTTQ) scaled by beginning-of-quarter total assets (Compustat data item ATQ) \\
		MTB & Market-to-book ratio, defined as beginning-of-quarter market value of equity, calculated as common share price (Compustat data item PRCCQ) times common shares outstanding (Compustat data item CSHOQ) divided by beginning-of-quarter book value of equity (Compustat data item CEQQ) \\
		SIZE & Firm size, defined as the natural logarithm of market value of equity, calculated as natural logarithm of common share price (Compustat data item PRCCQ) times common shares outstanding (Compustat data item CSHOQ) \\
		QRET & Quarterly market-adjusted stock return, defined as buy-and-hold stock return (CRSP data item RET) over the fiscal quarter adjusted by the value-weighted stock return (CRSP data item VWRETD) over the same period \\
		DRET & Daily market-adjusted stock return, defined as daily buy-and-hold stock return (CRSP data item RET) adjusted by the daily value-weighted stock return (CRSP data item VWRETD)\\
		$\Delta$DRET & Change in daily market-adjusted stock return (DRET), defined as current daily market-adjusted stock return minus one-day-lagged daily market-adjusted stock return \\
		NEG & Indicator for negative quarterly return, which is set to 1 when market-adjusted stock return (QRET) is negative and 0 otherwise \\
		BN & Indicator for daily bad news, which is set to 1 (0) if the negative (positive) change in daily market-adjusted stock return ($\Delta$DRET) is three times larger than the firm’s average decrease (increase) in daily return over the calendar year.\\
		AF & Analyst forecast, defined as analysts' mean consensus forecast for one-year-ahead earnings per share, scaled by stock price per share at the end of the fiscal quarter (Compustat data item PRCCQ)\\
		AFE & Analyst forecast error, defined as I/B/E/S earnings per share minus the median of the most recent analysts' forecasts, deflated by stock price per share at the end of the fiscal quarter (Compustat data item PRCCQ)\\
		BUSSEG & Business segment, defined as the natural logarithm of one plus number of business segments, or one if item is missing from Compustat\\
		GEOSEG & Geographical segment, defined as the natural logarithm of one plus number of geographical segments, or one if item is missing from Compustat\\
		AGE & Firm age, defined as the natural logarithm of one plus number of days elapsed since the firm's first entry date in CRSP\\
		STD\_EARN & Standard deviation of quarterly earnings (EARN) over the last five quarters\\
		STD\_QRET & Standard deviation of monthly market-adjusted stock return over all months in the fiscal quarter\\
		LOSS & Indicator for loss, which is set to 1 when quaterly earnings (EARN) is negative and 0 otherwise\\
	\end{tabular}%
\end{table}%
\newpage
\subsection*{Appendix D: 8-K Item List}
\label{appd}
\input{../output/table/appd}
\newpage
\subsection*{Appendix E: 8-K Matching Cases}
We check whether the SEC filings and the market returns movements are related to the same corporate events, assuming market efficiency. First, we identify the firm-day with top ten largest changes in daily returns ($\Delta$DRET) upwards and downwards. Next, we read the 8-Ks matched to the news and see if the corporate events depicted in the 8-Ks are in line with the market movements both in terms of direction and magnitude. We find that the 8-K matching cases make economic sense overall. See selected 8-K matching cases below.
\label{appe}
\begin{center}
	\textbf{Good News}
\end{center}
\subsubsection*{Case 1}
Differential Brands Group Inc. (CIK = 844143) experienced a significant rise in market-adjusted daily stock returns ($\Delta$DRET = 5.14) on June 27 of 2018. On June 27 of 2018, the company filed an 8-K with ending reporting period on the same day, which contained Item 8.01: Other Events and Item 9.01: Financial Statements and Exhibits. This 8-K stated that ``On June 27, 2018, Differential Brands Group Inc. issued a press release announcing that it has entered into a definitive purchase agreement with Global Brands Group Holding Limited, a Hong Kong listed company (`GBG'), to acquire a significant part of GBG’s North American licensing business".
\subsubsection*{Case 2}
Karuna Therapeutics, Inc. (CIK = 1771917) experienced a significant rise in market-adjusted daily stock returns ($\Delta$DRET = 4.42) on November 18 of 2019. On November 18 of 2019, the company filed an 8-K with ending reporting period on the same day, which contained Item 8.01: Other Events and Item 9.01: Financial Statements and Exhibits. This 8-K contained a press release, which stated that ``Karuna Therapeutics, Inc. (Nasdaq: KRTX), a clinical-stage biopharmaceutical company committed to developing novel therapies with the potential to transform the lives of people with disabling and potentially fatal neuropsychiatric disorders and pain, today announced results from its Phase 2 clinical trial of KarXT for the treatment of acute psychosis in patients with schizophrenia. In the clinical trial, KarXT demonstrated a statistically significant and clinically meaningful 11.6 point mean reduction in total Positive and Negative Syndrome Scale (PANSS) score compared to placebo (p$<$0.0001) and also demonstrated good overall tolerability. A statistically significant reduction in the secondary endpoints of PANSS-Positive and PANSS-Negative scores were also observed (p$<$0.001)". 
\subsubsection*{Case 3}
Opexa Therapeutics, Inc. (CIK = 1069308) experienced a significant rise in market-adjusted daily stock returns ($\Delta$DRET = 3.34) on August 7 of 2009. August 7 of 2009, the company filed an 8-K with ending reporting period on the same day, which contained Item 1.01: Entry into a Material Definitive Agreement, Item 1.02: Termination of a Material Definitive Agreement and Item 9.01: Financial Statements and Exhibits. This 8-K stated that ``Effective August 6, 2009, Opexa Therapeutics, Inc., a company developing a novel T-cell immunotherapy for multiple sclerosis (MS), entered into an exclusive agreement with Novartis for the further development of Opexa’s novel stem cell technology. This technology, which has generated preliminary data showing the potential to generate monocyte derived islet cells from peripheral blood mononuclear cells, was in early preclinical development at Opexa". 
\subsubsection*{Case 4}
Amarin Corporation plc (CIK = 897448) experienced a significant rise in market-adjusted daily stock returns ($\Delta$DRET = 3.13) on September 24 of 2018. On September 24 of 2018, the company filed an 8-K with ending reporting period on the same day, which contained Item 8.01: Other Events and Item 9.01: Financial Statements and Exhibits. This 8-K only contained a press release, which stated that ``September 24, 2018 - Amarin Corporation plc (NASDAQ:AMRN), announced today topline results from the Vascepa® cardiovascular (CV) outcomes trial, REDUCE-IT™, a global study of 8,179 statin-treated adults with elevated CV risk. REDUCE-IT met its primary endpoint demonstrating an approximately 25\% relative risk reduction, to a high degree of statistical significance (p$<$0.001), in major adverse CV events (MACE) in the intent-to-treat patient population with use of Vascepa 4 grams/day as compared to placebo".
\subsubsection*{Case 5}
Avanir Pharmaceuticals (CIK = 858803) experienced a significant rise in market-adjusted daily stock returns ($\Delta$DRET = 3.06) on April 18 of 2007. On April 18 of 2007, the company filed an 8-K with ending reporting period on the same day, which contained Item 8.01: Other Events. This 8-K stated that ``On April 18, 2007, Avanir Pharmaceuticals (the `Company') announced top-line results from the Company's Phase III clinical trial evaluating the investigational drug Zenvia(TM) (dextromethorphan hydrobromide/quinidine sulfate (`DMQ')), an NMDA antagonist and sigma-1 agonist, in diabetic neuropathic pain".
\begin{center}
	\textbf{Bad News}
\end{center}
\subsubsection*{Case 1}
NovaBay Pharmaceuticals, Inc. (CIK = 1389545) experienced a significant drop in market-adjusted daily stock returns ($\Delta$DRET = -9.06) on June 11 of 2019. On June 17 of 2019, the company filed an 8-K with ending reporting period on June 11 of 2019, which contained Item 5.02: Departure of Directors or Certain Officers; Election of Directors; Appointment of Certain Officers: Compensatory Arrangements of Certain Officers and Item 9.01: Financial Statements and Exhibits. This 8-K stated that ``On June 17, 2019, NovaBay Pharmaceuticals, Inc. (the `Company') announced, effective as of June 11, 2019, the Board of Directors (the `Board') of the Company appointed Justin Hall and Jason Raleigh to the permanent positions of President and Chief Executive Officer and Chief Financial Officer, respectively".
\subsubsection*{Case 2}
Dynavax Technologies Corporation (CIK = 1029142) experienced a significant drop in market-adjusted daily stock returns ($\Delta$DRET = -5.79) on December 18 of 2008. On December 19 of 2008, the company filed an 8-K with ending reporting period on December 18 of 2008, which contained Item 1.02: Termination of a Material Definitive Agreement, Item 8.01: Other Events and Item 9.01: Financial Statements and Exhibits. This 8-K stated that ``On December 19, 2008, Dynavax Technologies Corporation (the `Company') announced the termination of an exclusive license and development collaboration agreement and a related manufacturing agreement (the `Collaboration Arrangement') with Merck \& Co., Inc. (`Merck') for HEPLISAV(TM), a Phase 3 hepatitis B virus vaccine". 
\subsubsection*{Case 3}
Proteostasis Therapeutics, Inc. (CIK = 1445283) experienced a significant drop in market-adjusted daily stock returns ($\Delta$DRET = -4.72) on October 19 of 2018. On October 24 of 2018, the company filed an 8-K with ending reporting period on October 22 of 2018, which contained Item 1.01: Entry into a Material Definitive Agreement, Item 8.01: Other Events and Item 9.01: Financial Statements and Exhibits. This 8-K stated that ``On October 23, 2018, Proteostasis Therapeutics, Inc. (the `Company') entered into an underwriting agreement (the `Underwriting Agreement') with Leerink Partners LLC and Piper Jaffray \& Co. as representatives of the several underwriters named therein (the `Underwriters'), relating to the underwritten public offering of 11,000,000 shares of the Company’s common stock, par value \$0.001 per share (the `Offering'). The price to the public in the Offering was \$6.75 per share". 
\subsubsection*{Case 4}
7th Level, Inc. (CIK = 920038) experienced a significant drop in market-adjusted daily stock returns ($\Delta$DRET = -4.27) on April 22 of 1998. On April 23 of 1998, the company filed an 8-K with ending reporting period on April 23 of 1998, which contained Item 5: Other events and Item 7: Financial statements and exhibits. This 8-K only contained a press release, which stated that ``...7th Level, Inc. (NASDAQ: SEVL) announced that the Company and privately-held Pulse Entertainment, Inc. of Los Angeles have decided not to proceed with their proposed merger.  Separately, 7th Level announced it has obtained commitments for a \$4.5 million bridge loan and a \$10 million private placement to finance the ramp up and rollout of 7th Level's revolutionary new line of technology products".
\subsubsection*{Case 5}
Atlantic Alliance Partnership Corp. (CIK = 1630940) experienced a significant drop in market-adjusted daily stock returns ($\Delta$DRET = -3.65) on November 3 of 2016. On November 8 of 2016, the company filed an 8-K with ending reporting period on November 7 of 2016, which contained Item 5.02: Departure of Directors or Certain Officers; Election of Directors; Appointment of Certain Officers: Compensatory Arrangements of Certain Officers. This 8-K stated that ``On November 7, 2016, Mr. Jonathan Goodwin resigned as the Chief Executive Officer and a director of Atlantic Alliance Partnership Corp. (the `Company'), and Mr. Waheed Alli resigned as the Chairman of the Company, each to pursue other professional interests. Such resignations were not the result of any disagreement with the Company. On November 7, 2016, the board of directors of the Company (the `Board') appointed Mr. Iain Abrahams as the Chief Executive Officer of the Company and Mr. Mark Klein (a director of the Company prior to such date) as the Chairman of the Company. Mr. Abrahams will continue to serve as a director of the Company. Mr. Daniel Winston has been appointed to serve on the audit committee of the Board in lieu of Mr. Abrahams".

\newpage
\setcounter{page}{1}
\section*{Online Appendix}

%%%%%%%%%%%%%%%%%%%%%%%%% Online Appendix TABLE 1
% Table generated by Excel2LaTeX from sheet 'T3'
\begin{table}[H] \label{OAT1}
	\begin{center}
		\tabcolsep=0.11cm
		\begin{tabular}{lcccc}
			\multicolumn{5}{c}{\textbf{Online Appendix. Table 1. }} \\
			\multicolumn{5}{c}{\textbf{Is 8-K Narrative Disclosure Conservative? (Restricted Sample)}} \\
			\toprule
			\toprule
			& (1) & (2) & (3) & (4) \\
			Dep. Variables & TLAG & TLAG & TONE & TONE \\
			\midrule
			&   &   &   &  \\
			$\Delta$DRET & 0.668*** & 0.708*** & -0.874 & -0.655 \\
			& (6.94) & (6.41) & (-0.76) & (-0.52) \\
			BN & -0.047*** & -0.049*** & -0.061 & -0.051 \\
			& (-3.06) & (-3.02) & (-0.52) & (-0.42) \\
			\rowcolor[rgb]{ .906,  .902,  .902} \textit{(Pred. Sign)} & (-) & (-) & (+) & (+) \\
			\rowcolor[rgb]{ .906,  .902,  .902} $\Delta$DRET$\times$BN & -1.303*** & -1.421*** & 2.460 & 2.096 \\
			\rowcolor[rgb]{ .906,  .902,  .902} & (-7.18) & (-5.87) & (1.22) & (1.05) \\
			SIZE &   & 0.006 &   & 0.100 \\
			&   & (0.59) &   & (0.98) \\
			MTB &   & 0.001 &   & -0.020 \\
			&   & (0.91) &   & (-1.10) \\
			LEV &   & -0.057 &   & -0.502 \\
			&   & (-1.30) &   & (-0.98) \\
			EARN &   & 0.377*** &   & 2.647* \\
			&   & (2.78) &   & (1.88) \\
			STD\_EARN &   & -0.132 &   & -2.315 \\
			&   & (-0.80) &   & (-1.29) \\
			BUSSEG &   & -0.008 &   & 0.070 \\
			&   & (-0.52) &   & (0.40) \\
			GEOSEG &   & 0.012 &   & 0.116 \\
			&   & (0.85) &   & (0.77) \\
			AF &   & 0.061 &   & -0.775 \\
			&   & (0.72) &   & (-1.15) \\
			AFE &   & 0.041 &   & 2.393** \\
			&   & (0.35) &   & (2.59) \\
			Constant & -0.695*** & -0.761*** & -2.929 & -2.473 \\
			& (-4.64) & (-5.08) & (-0.46) & (-0.39) \\
			&   &   &   &  \\
			Observations & 28,814 & 26,370 & 28,814 & 26,370 \\
			Adjusted R-squared & 0.174 & 0.174 & 0.219 & 0.223 \\
			\bottomrule
			\bottomrule
		\end{tabular}%
	\end{center}
\end{table}%
\begin{equation*}
\begin{split}
tone_{i,t}=\beta_0&+\beta_1EARN_{i,t}+\beta_2RET_{i,t}+\beta_3SIZE_{i,t}+\beta_4MTB_{i,t}+\beta_5STD\_EARN_{i,t}\\
&+\beta_6STD\_RET_{i,t}+\beta_7AGE_{i,t}+\beta_8BUSSEG_{i,t}+\beta_9GEOSEG_{i,t}+\beta_{10}LOSS_{i,t}\\
&+\beta_{11}\Delta EARN_{i,t}+\beta_{12}AFE_{i,t}+\beta_{13}AF_{i,t}+\epsilon_{i,t}
\end{split}
\end{equation*}

Online Appendix Table 1 presents regression results of the above Equation (Column 1) in comparison with the expected tone model results in \cite{huangToneManagement2014} (Column 2). Dependent variable $tone_{i,t}$ is defined as number of net positive words, calculated as total number of positive words minus the sum of total number of negative words and total number of negations. Independent variables are defined in \hyperref[appc]{Appendix C}. All financial variables except returns are winsorized at 1\% and 99\% level. ***, ** and * indicate significance at the 1\%, 5\% and 10\% levels in a two-tailed test. The coefficient of MTB in Column 1 is consistent with that in Column 2 in terms of sign, because in the expected tone model of \cite{huangToneManagement2014} the authors use book-to-market ratio instead of market-to-book ratio. 


%%%%%%%%%%%%%%%%%%%%%%%%% Online Appendix TABLE 2
\newpage
%\begin{landscape}
% Table generated by Excel2LaTeX from sheet 'OAT3'
\begin{table}[H] \label{oat2}
  \begin{center}
  	    \begin{tabular}{lcrrr}
  		\multicolumn{5}{c}{\textbf{Online Appendix. Table 2.}} \\
  		\multicolumn{5}{c}{\textbf{Panel A: Summary of Fiscal Yearly Regressions}} \\
  		\midrule
  		\midrule
  		Indep. Vars. & Prediction & Coeff. & S.E. & t-stats \\
  		\midrule
  		Intercept &   & -0.016 & 0.006 & -2.50 \\
  		NEG &   & 0.007 & 0.013 & 0.55 \\
  		RET & (+) & -0.051 & 0.031 & -1.66 \\
  		RET$\times$SIZE & (+) & 0.012 & 0.006 & 2.20 \\
  		RET$\times$MTB & (-) & -0.016 & 0.004 & -4.05 \\
  		RET$\times$LEV & (-) & 0.033 & 0.040 & 0.81 \\
  		RET$\times$NEG & (+) & 0.235 & 0.067 & 3.53 \\
  		RET$\times$NEG$\times$SIZE & (-) & -0.034 & 0.011 & -3.09 \\
  		RET$\times$NEG$\times$MTB & (+) & 0.020 & 0.006 & 3.25 \\
  		RET$\times$NEG$\times$LEV & (+) & -0.099 & 0.076 & -1.30 \\
  		SIZE &   & 0.004 & 0.001 & 4.20 \\
  		MTB &   & 0.001 & 0.001 & 1.00 \\
  		LEV &   & -0.013 & 0.008 & -1.61 \\
  		NEG$\times$SIZE &   & -0.001 & 0.002 & -0.48 \\
  		NEG$\times$MTB &   & 0.000 & 0.001 & -0.27 \\
  		NEG$\times$LEV &   & 0.002 & 0.015 & 0.10 \\
  		\bottomrule
  		\bottomrule
  	\end{tabular}%
  \end{center}
\end{table}%

\begin{table}[htbp]
	\begin{center}
		\begin{tabular}{lrrrrrrrrr}
			\multicolumn{10}{l}{\textbf{Panel B: Summary Statistics of C\_SCORE and G\_SCORE}}  \\
			\midrule
			\midrule
			& \multicolumn{1}{c}{mean} & \multicolumn{1}{c}{median} & \multicolumn{1}{c}{std. dev} & \multicolumn{1}{c}{max} & \multicolumn{1}{c}{min} & \multicolumn{1}{c}{p1} & \multicolumn{1}{c}{p25} & \multicolumn{1}{c}{p75} & \multicolumn{1}{c}{p99} \\
			\midrule
			C\_SCORE & 0.053 & 0.054 & 0.063 & 1.143 & -0.228 & -0.103 & 0.015 & 0.092 & 0.212 \\
			G\_SCORE & -0.008 & -0.008 & 0.031 & 0.188 & -1.314 & -0.091 & -0.023 & 0.009 & 0.067 \\
			\bottomrule
			\bottomrule
		\end{tabular}%
	\end{center}
		\begin{footnotesize}
			\setcounter{equation}{2}
			\begin{equation}
				\begin{split}
					EARN_{i,t} = \beta_0&+\beta_1NEG_{i,t}+\beta_2RET_{i,t}\\
					&+\beta_3RET_{i,t}\times SIZE_{i,t}+\beta_4RET_{i,t}\times MTB_{i,t}+\beta_5RET_{i,t}\times LEV_{i,t}+\beta_6RET_{i,t}\times NEG_{i,t}\\
					&+\beta_7RET_{i,t}\times NEG_{i,t}\times SIZE_{i,t}+\beta_8RET_{i,t}\times NEG_{i,t}\times MTB_{i,t}+\beta_9RET_{i,t}\times NEG_{i,t}\times 	LEV_{i,t}\\
					&+\beta_{10}SIZE_{i,t}+\beta_{11}MTB_{i,t}+\beta_{12}LEV_{i,t}\\
					&+\beta_{13}NEG_{i,t}\times SIZE_{i,t}+\beta_{14}NEG_{i,t}\times MTB_{i,t}+\beta_{15}NEG_{i,t}\times LEV_{i,t}+ \epsilon_{i,t}
				\end{split}
			\end{equation}
		
			\begin{equation}
				C\_SCORE_{i,t} = \beta_6+\beta_7SIZE_{i,t}+\beta_8MTB_{i,t}+\beta_9LEV_{i,t}
			\end{equation}
		
			\begin{equation}
				G\_SCORE_{i,t} = \beta_2+\beta_3SIZE_{i,t}+\beta_4MTB_{i,t}+\beta_5LEV_{i,t}
			\end{equation}
			
			\noindent Online Appendix Table 2 presents the key statistics in constructing C\_SCORE and G\_SCORE. Panel A presents the mean of coefficients, the mean of standard errors and the t-statistics obtained from 23 fiscal yearly regressions (Equation 3) using 10-Q sample from 1993 to 2015. C\_SCORE and G\_SCORE are calculated following Equation 4 and Equation 5 respectively. Panel B presents the summary statistics of C\_SCORE and G\_SCORE. See \hyperref[appb]{Appendix B} for variable definitions. All financial variables except returns are winsorized at 1\% and 99\% level. The mean and standard errors of coefficients and the summary statistics of C\_SCORE and G\_SCORE are consistent with \citeA{khanEstimationEmpiricalProperties2009} overall. 
		\end{footnotesize}
\end{table}%

\setcounter{equation}{2}
\begin{equation}
TEX_{i,t}=\beta_0+\beta_1\Delta DRET_{i,t-tlag}+\beta_2BN_{i,t-tlag}+\beta_3\Delta DRET_{i,t-tlag}\times BN_{i,t-tlag}+\beta_nControls_{i,t}+\epsilon_{i,t}
\end{equation}

Online Appendix Table 2 presents regression results of Equation (3) using restricted 8-K sample. All observations in restricted 8-K sample are subject to four (five) business day 8-K reporting deadline after (before) May 23rd 2004. TEX represents a vector of textual properties that consists of number of words (NW), tone (TONE) and reporting time lag (TLAG). \textit{Controls} denotes a vector of control variables including firm size (SIZE), market-to-book ratio (MTB) and leverage ratio (LEV). See \hyperref[appb]{Appendix B} and \hyperref[appc]{Appendix C} for textual and financial variable definitions. All financial variables except returns are winsorized at 1\% and 99\% level. Column 2, 4 and 6 include firm and time fixed effects and standard errors are clustered at industry level identified by 4-digit SIC codes. ***, ** and * indicate significance at the 1\%, 5\% and 10\% levels in a two-tailed test.
%\end{landscape} 

\end{document}

